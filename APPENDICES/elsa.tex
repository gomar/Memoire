%!TEX root = ../adrien_gomar_phd.tex

\label{app:elsa}

\emph{elsA}\footnote{\emph{elsA} stands for \underline{e}nsemble 
\underline{l}ogiciel pour la \underline{s}imulation en 
\underline{A}\'erodynamique} solver~\cite{Cambier2013} initially developed by ONERA in 1997.

ELSA EST UN SOLVEUR COPRESSIBLE

\section{Extension of the harmonic balance approach to
  turbomachinery computations}
\label{sec:turbomachinery_adaptation}


The time-domain harmonic balance method has been implemented 
by CERFACS in the
\emph{elsA} solver~\cite{Cambier2013} developed by ONERA. 
This code solves the RANS equations using a cell-centered
approach on multi-blocks structured meshes.  Using the HB method,
significant savings in CPU cost have been observed in various
applications such as rotor/stator interactions~\cite{JSicot2012}
and dynamic derivatives computation~\cite{CIHassan2011}. 


\mytodo{implicitation}

\section{Extension of the harmonic balance approach to
  turbomachinery computations}
\label{sec:turbomachinery_adaptation}

To efficiently apply the HB approach to turbomachinery
configurations, phase-lag boundary conditions~\cite{Erdos1977} are
used to cut down the mesh size by using a grid that spans only one
blade passage per row. The phase-lag boundary conditions are two-fold:
i) the azimuthal boundaries of a passage and ii) the blade row
interface which must handle different row pitches on either
sides. Furthermore, in the HB framework, each row captures the blade
passing frequency of the opposite row leading to different time
samplings solved in each row.

The phase-lag condition is based on the space-time periodicity of the
flow variables. It states that the flow in one passage~$\theta$ is the
same as the next passage~$\theta+\Delta\theta$ but at another
time~$t+\delta t$:
\begin{equation}
  W\left(\theta+\Delta\theta,t \right) = W\left(\theta,t+\delta t \right),
  \label{eq:choro}
\end{equation}
where $\Delta \theta$ is the pitch of the considered row.  The time
lag $\delta t$ can be expressed as the phase of a rotating wave
traveling at the same speed as the relative rotation speed of the
opposite row: $\delta t=\beta/\omega_\beta$.  The Inter-Blade Phase
Angle $\beta$ (IBPA) depends on each row blade count and relative
rotation speed. It is analytically given \citet{Gerolymos1991}.  The Fourier
transform of Eq.~\eqref{eq:choro} implies that the spectrum of the
flow in a passage is equal to the spectrum of the neighbor passage
modulated by a complex exponential depending on the IBPA:
\begin{equation}
  \label{eq:serfourphasetemps}
  \widehat{W}_k(x, r,  \theta+\theta_G)  = {\widehat{W}_k(x, r,
    \theta)e^{i k\beta}}.
\end{equation}
At the azimuthal boundaries, this modulation can be computed on the fly
in the HB framework as a sampling of the time period is always known
and it is straightforward to derive an analytic derivation in the time
domain (see Ref.~\cite{JSicot2012}). The blade row interface is more complex
as the different pitches and relative motion of the rows require to
duplicate the flow in the azimuthal direction using the phase-lag
periodicity. A time interpolation also occurs to take the
different time samplings into account and a non-abutting mesh
technique is applied as the mesh will unlikely have matching
cells. To remove spurious waves, an over-sampling and a filtering are
performed. 

\mytodo{extension to multi-freq}

\section{Adaptation to the ALE Formulation}
\label{app:elsa_ael}

The deformation speed $s_D$ is estimated by applying the HB time
derivative operator to the mesh points at all instants:
$s_D^*=D_t(X^*)$. This decomposition is exact when the
deformation of the mesh has less harmonics than those solved in the
HB computation~\cite{JDufour2009}. In the present case, the mesh
deformation follows a purely single frequency law (see
Eq.~\eqref{eq:harm_vib_displ_vector}), therefore all the HB computations will correctly
estimate the mesh deformation speed regardless of their harmonic content.


