%!TEX root = ../adrien_gomar_phd.tex

\label{app:wrong_multi_nlh}

Let us consider the specific example of $u^\prime$ taken as:
\begin{equation}
	u^\prime = (\widehat{u}_{-1} e^{-i t} + \widehat{u}_{1} e^{i t}) +
		(\widehat{u}_{-2} e^{-i \pi t} + \widehat{u}_{2} e^{i \pi t}).
\end{equation}
Namely, we consider equation~\eqref{eq:sm_nlh_decomposition_pert_multi}
with the specific harmonic 
pulsations: $\omega_1 = 1$ and $\omega_2 = \pi$.
The cross-term $u^\prime u^\prime$ is equal to:
\begin{equation}
	\begin{split}
		u^\prime u^\prime = 
			&(\widehat{u}_{-1})^2 e^{-i 2 t}
			+ (\widehat{u}_{1})^2 e^{i 2 t}
			+ (\widehat{u}_{-2})^2 e^{- i 2 \pi t}
			+ (\widehat{u}_{2})^2 e^{i 2 \pi t} \\
		&+ 2 \left[
				\widehat{u}_{-1} \widehat{u}_{-2} e^{i (-1 -\pi) t} 
				+ \widehat{u}_{-1} \widehat{u}_{2} e^{i (-1 + \pi) t}
				+ \widehat{u}_{1} \widehat{u}_{-2} e^{i (1 - \pi) t} 
				+ \widehat{u}_{1} \widehat{u}_{2} e^{i (1 + \pi) t} 
			 \right] \\
		&+ 2 \widehat{u}_{-1}\widehat{u}_{1}
			 	+ 2 \widehat{u}_{-2}\widehat{u}_{2}.
	\end{split}
\end{equation}
Thus,
\begin{equation}
	\begin{split}
		\overline{u^\prime u^\prime} &=
			  (\widehat{u}_{-1})^2 e^{-i 2 t}
			+ (\widehat{u}_{1})^2 e^{i 2 t}
			+ (\widehat{u}_{-2})^2 e^{- i 2 \pi t}
			+ (\widehat{u}_{2})^2 e^{i 2 \pi t} \\
		&+ 2 \left[
				\widehat{u}_{-1} \widehat{u}_{-2} e^{i (-1 -\pi) t} 
				+ \widehat{u}_{-1} \widehat{u}_{2} e^{i (-1 + \pi) t}
				+ \widehat{u}_{1} \widehat{u}_{-2} e^{i (1 - \pi) t} 
				+ \widehat{u}_{1} \widehat{u}_{2} e^{i (1 + \pi) t} 
			 \right] \\
		&+ 2 \widehat{u}_{-1}\widehat{u}_{1}
			 	+ 2 \widehat{u}_{-2}\widehat{u}_{2} \diff t.
	\end{split}
\end{equation}