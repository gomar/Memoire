%!TEX root = ../../adrien_gomar_phd.tex

The Harmonic Balance Technique proposed by \citet{Hall2002}
is a step further the non-linear frequency domain method. Instead
of using the fast Fourier transform to cast back the equations
to the time domain at each pseudo-iteration step, 
the equations are mathematically derived to be directly
computed into the time-domain.
To explain the method, we will use the general form of 
the viscous Burger's equation as defined in
Eq.~\ref{eq:sm_nonlinear_convection_residual}.

\subsection{Mono-frequential formulation}

Following the same approach as the non-linear frequency domain approach,
one consider that both $u$ and $R$ are periodic
in time with respect to period $T = 2 \pi / \omega$
and can be written using a Fourier series:
\begin{equation}
	\begin{split}
		u &= \sum_{k=-\infty}^{\infty} \widehat{u}_k e^{i k \omega t}, \\
		R(t) &= \sum_{k=-\infty}^{\infty} \widehat{R}_k e^{i k \omega t}.
		\label{eq:sm_hall_dft}
	\end{split}
\end{equation}
Injecting Eq.~\ref{eq:sm_hall_dft} in 
Eq.~\ref{eq:sm_nonlinear_convection_residual}, and considering
the orthogonality of the the complex exponentials:
\begin{equation}
	i k \omega \widehat{u}_k + \widehat{R}_k = 0, \: k \in [-N, N].
	\label{eq:sm_hall_frequential_eq}
\end{equation}

In the same way as one uses Fourier coefficients to
evaluate the temporal signal,
one can reconstruct the Fourier coefficients using
temporal evaluations taken at evenly spaced timelevels
sampling the period $T = 2 \pi / \omega$ using the forward
Fourier transform. Moreover, 
according to the Nyquist-Shannon~\cite{Shannon1949} sampling theorem, 
at least $2N+1$ timelevels are needed to capture $N$ frequencies,
leading to:
\begin{equation}
	\widehat{u}_k = \frac{1}{2N+1} 
	\sum_{n=0}^{2N} u_n^\star e^{-i k \omega t_n}.
\end{equation}
If $E$ denotes the matrix composed of the elements 
$(E)_{k,n} = e^{-i (k - N) \omega t_n} / 2N+1$, one can write $\widehat{u}_k$
and $\widehat{R}_k$ as:
\begin{equation}
	\begin{split}
		\widehat{u}_k &= E u^\star \\
		\widehat{R}_k &= E R^\star,
	\end{split}
	\label{eq:sm_matrix_fourier_operator}
\end{equation}
where $u^\star$ and $R^\star$ 
denote the vectors formed of all the evaluations of respectively $u$
and $R$,
made at $2N+1$ timelevels uniformly sampling the period of interest. 
$E$ can thus be named the Fourier matrix.
Note that conversely, using the inverse Fourier matrix $E^{-1}$:
\begin{equation}
	\begin{split}
		u^\star &= E^{-1} \widehat{u}_k \\
		R^\star &= E^{-1} \widehat{R}_k.
	\end{split}
\end{equation}

Injecting the matrix formulation of 
Eq.~\ref{eq:sm_matrix_fourier_operator} in 
Eq.~\ref{eq:sm_hall_frequential_eq}
gives:
\begin{equation}
	i K \omega E u^\star + E R^\star = 0,
\end{equation}
where $K$ is a diagonal matrix formed of all the $k \in [-N, N]$.
Note that first, the matrix formulation encompass all harmonics
$k \in [-N, N]$ and second, it does not require the
orthogonality of the complex exponentials.
Now multiplying the equation by the inverse Fourier matrix $E^{-1}$:
\begin{equation}
	i \omega E^{-1} K E u^\star + R^\star = 0,
\end{equation}
where $R^\star$ can now be substituted:
\begin{equation}
		i \omega E^{-1} K E u^\star + 
		\displaystyle \frac{\partial}{\partial x}
		\frac{(u^\star)^2}{2} = 0
\end{equation}
What happened here is that instead of developing $R(t)$
in the frequency domain, which is tedious, this term is kept
as it through all the development process. 
Since $R(t)$ only includes spatial derivatives, no non-linear
terms
arise by using the Fourier decomposition. Hence, multiplying it
by the Fourier matrix leads to the unity matrix. 
Finally $R(t)$ is simply evaluated at $2N+1$ timelevels.

This approach is really close to the non-linear frequency domain method.
The reader might observe that we are introducing the discrete Fourier
transform and its inverse. This is close to the fast Fourier transform
and it's inverse, as proposed by \citet{McMullen2001}. However,
as the development is on the equations and not during the time loop,
we get $2N+1$ steady equations, by definition in the time
domain, that are coupled by a source term.
The source term appears as a spectral operator defined as:
\begin{equation}
	D_t = i \omega E^{-1} K E.
	\label{eq:sm_hb_mono_source_term_matrix}
\end{equation}
The main difference with the nonlinear frequency domain approach
is that the source term is known at the very first iteration and does
not change, meaning that we do not spend time computing a costly
fast fourier transform and its inverse at each time-step. 
\todo{not so costly !! cite mc mullen}

\citet{Gopinath2005} provide an analytical formulation of the
source term defined in Eq.~\ref{eq:sm_hb_mono_source_term_matrix}.
It is a matrix operator whose elements are defined as:
\begin{equation}
  (D_t)_{k, n} =
  \begin{cases}
    \frac{\pi}{T}(-1)^{k-n}\csc\left(\frac{\pi
        (k-n)}{2N+1}\right) &, \, k\neq n,\\
    0 &, \, k=n.
  \end{cases}
  \label{eq:sm_hb_mono_source_term_analytic}
\end{equation}
Finally, adding a pseudo-time ($\tau$) derivative to 
time march the equations to the steady state, 
the mono-frequential formulation of 
Eq.~\ref{eq:sm_nonlinear_convection_conservative} in the harmonic
balance framework is given by:
\begin{equation}
	\fbox{$
	\displaystyle \frac{\partial u^\star}{\partial \tau} + 
	D_t (u^\star) + 
	\displaystyle \frac{\partial}{\partial x}
		\frac{(u^\star)^2}{2} = 0
	$}
\end{equation}
with $D_t$ defined using Eq.~\ref{eq:sm_hb_mono_source_term_analytic}
and $u^\star = [u(t_0), \ldots, u(t_{2N+1})]$.


\subsection{Multi-frequential formulation}

\citet{Gopinath2007} and \citet{Ekici2007} 
extended the harmonic balance approach to
a multi-frequential formulation. To do so, they considered
a Fourier matrix defined as:
\begin{equation}
	(E)_{k,n} = \frac{1}{2N+1} e^{-i \omega_{k-N} t_n},
\end{equation}
where $N$ is the chosen number of frequencies.
Note that replacing $\omega_{k-N}$ by $(k - N) \omega$ gives
the mono-frequential inverse Fourier matrix back. 
This can be justified mathematically: in the
framework of the almost-periodic functions~\cite{Besicovitch1932},
such a function (which is composed of several
frequencies non necessarily harmonically related) can be approximated
by an almost-periodic
discrete Fourier transform. If $f(t)$ is an almost periodic function,
\citet{Besicovitch1932} proves that it can be approximated as:
\begin{equation}
	f(t) \approx \sum_{k=-N}^{N} \widehat{f}_k 
	e^{i \omega_k t}.
\end{equation}
This allows the use of the multi-frequential Fourier matrix as defined
above. However, in the multi-frequential case, the inverse Fourier matrix
$E^{-1}$ is not known \textit{a priori} 
and has to be numerically computed. Actually, as demonstrated by 
\citet{Gopinath2007}, it is easier to expressed $E^{-1}$ analytically,
compute its temporal derivative (that is hence analytical too) 
and inverse numerically $E^{-1}$ to obtain $E$.

Using the same process as for the mono-frequential formulation,
Eq.~\ref{eq:sm_nonlinear_convection_residual} becomes:
\begin{equation}
	i E^{-1} P E u^\star + R^\star = 0,
\end{equation}
where $P$ is a diagonal matrix formed of all the pulsations $\omega_k$.
\todo{montrer la diff avec mono-freq}

Finally, adding a pseudo-time ($\tau$) derivative 
to time-march the equation to the steady state,
the multi-frequential formulation of 
Eq.~\ref{eq:sm_nonlinear_convection_conservative} in the harmonic
balance framework reads:
\begin{equation}
	\fbox{$
	\displaystyle \frac{\partial u^\star}{\partial \tau} +
	D_t (u^\star) + 
	\displaystyle \frac{\partial}{\partial x}
		\frac{(u^\star)^2}{2} = 0
	$}
\end{equation}
with $D_t$ defined as:
\begin{equation}
	D_t = i E^{-1} P E
\end{equation}
and $u^\star = [u(t_0), \ldots, u(t_{2N})]$.

\paragraph{Condition number and convergence}
For the mono-frequential formulation, the use of a uniform
sampling of the timelevels has the good property of being
well conditioned.
Before going into detail, let us recall the definition of the
condition number $\kappa$ of a matrix $E$:
\begin{equation}
	\kappa (E) = \kappa (E^{-1}) = \| E \| \cdot \| E^{-1} \|, \quad
    \kappa(E) \geq 1,
\end{equation}
where $\| \cdot \|$ denotes a matrix norm.  Considering the resolution
of $A x = b$, if $A$ is invertible and if $\delta A$, $\delta x$ and
$\delta b$ are the numerical errors associated with the computation of
$A$, $x$ and, respectively, then:
\begin{equation}
   (A + \delta A)(x + \delta x) = b + \delta b.
   \label{eq:error_reso}
\end{equation}
By definition, the condition number sets an upper bound for 
the error made on~$x$:
\begin{equation}
   \frac{\| \delta x \|}{\| x \|} \leq \kappa(A)\left[\frac{\| \delta A \|}{\| A \|} + \frac{\| \delta b \|}{\| b \|} \right].
   \label{eq:conditonnig_amp}
\end{equation}
The error on the iterative resolution of the governing equations can
therefore be amplified by the harmonic balance source term. 
This amplification is
led by the condition number of the almost-periodic Fourier matrix. This
also means that if the errors are small but the condition number is
high, and vice-versa, the computation can diverge too. However, the
errors can not be \emph{a priori} controlled, thus the need to
minimize the condition number.

In the case of periodic-flows, the Fourier matrix is well-conditioned: the
uniform sampling for harmonically related frequencies leads to a
condition number equal to~$1$, which is the theoretical lower bound
for the condition number.  This is linked to the orthogonality of the
complex exponential family.  On the other hand, when the frequencies are arbitrary, it is usually
impossible to choose a uniform set of time instants over which the
almost-periodic Fourier matrix~$E$ is well conditioned. In fact, it is common for uniformly-sampled
sinusoids at two or more frequencies to be nearly linearly dependent,
which causes them not to be orthogonal, leading to the
ill-conditioning encountered in practice. This issue will be discussed
further in Sec.\todo{sec conditinnning}

\subsection{Extension to the Navier-Stokes equations}
Equivalent to the non-linear frequency domain method, the
considered equation is really close to the finite-volume
semi-discrete form of the Navier-Stokes equations. Therefore,
nothing particular as to be made to apply the harmonic balance approach
to the Navier-Stokes equations.