%!TEX root = ../../adrien_gomar_phd.tex

The Harmonic Balance Technique proposed by \citet{Hall2002}
is a step further the non-linear frequency domain method. Instead
of using the fast Fourier transform to cast back the equations
to the time domain at each pseudo-iteration step, 
the equations are mathematically derived to 
computed into the time-domain.
To explain the method, we will use the general form of 
the viscous Burger's equation defined in 
Eq.~\ref{eq:sm_nonlinear_convection_residual}.

\subsection{Mono-frequential formulation}

Following the same approach as the non-linear frequency domain approach,
one consider that both $u$ and $R$ are periodic
in time with respect to period $T = 2 \pi / \omega$
and can be written using a Fourier series:
\begin{equation}
	\begin{split}
		u &= \sum_{k=-\infty}^{\infty} \widehat{u}_k e^{i k \omega t} \\
		R(t) &= \sum_{k=-\infty}^{\infty} \widehat{R}_k e^{i k \omega t}
		\label{eq:sm_hall_dft}
	\end{split}
\end{equation}
Injecting Eq.~\ref{eq:sm_hall_dft} in 
Eq.~\ref{eq:sm_nonlinear_convection_residual}, and considering
the orthogonality of the the complex exponentials:
\begin{equation}
	i k \omega \widehat{u}_k + \widehat{R}_k = 0, \: k \in [-N, N].
	\label{eq:sm_hall_frequential_eq}
\end{equation}

In the same way as one uses Fourier coefficients to
evaluate the temporal signal,
one can reconstruct the Fourier coefficients using
temporal evaluations taken at evenly spaced timelevels
sampling the period $T = 2 \pi / \omega$ using the forward
Fourier transform. Moreover, 
according to the Nyquist-Shannon~\cite{Shannon1949} sampling theorem, 
at least $2N+1$ timelevels are needed to capture $N$ frequencies,
leading to:
\begin{equation}
	\widehat{u}_k = \frac{1}{2N+1} 
	\sum_{n=0}^{2N} u_n^\star e^{-i k \omega t_n}.
\end{equation}
If $E$ denotes the matrix composed of the elements 
$E_{k,n} = e^{-i k \omega t_n} / 2N+1$, one can write $\widehat{u}_k$
and $\widehat{R}_k$ as:
\begin{equation}
	\begin{split}
		\widehat{u}_k &= E u^\star \\
		\widehat{R}_k &= E R^\star,
	\end{split}
	\label{eq:sm_matrix_fourier_operator}
\end{equation}
where $u^\star$ and $R^\star$ 
denote the vectors formed of all the evaluations of respectively $u$
and $R$,
made at $2N+1$ timelevels uniformly sampling the period of interest. 
$E$ can thus be named the inverse Fourier operator.
Note that conversely, using the Fourier operator $E^{-1}$:
\begin{equation}
	\begin{split}
		u^\star &= E^{-1} \widehat{u}_k \\
		R^\star &= E^{-1} \widehat{R}_k.
	\end{split}
\end{equation}

Injecting the matrix formulation of 
Eq.~\ref{eq:sm_matrix_fourier_operator} in 
Eq.~\ref{eq:sm_hall_frequential_eq}
gives:
\begin{equation}
	i K \omega E u^\star + E R^\star = 0,
\end{equation}
where $K$ is a diagonal matrix formed of all the $k \in [-N, N]$.
Now multiplying the equation by the Fourier operator $E^{-1}$:
\begin{equation}
	i \omega E^{-1} K E u^\star + R^\star = 0,
\end{equation}
where $R^\star$ can now be substituted:
\begin{equation}
	\fbox{$
		i \omega E^{-1} K E u^\star + 
		\displaystyle \frac{\partial}{\partial x}
		\frac{(u^\star)^2}{2} = 0
	$}
\end{equation}
What happened here is that instead of developing $R(t)$
in the frequency domain, which is tedious, this term is kept
as it is through all the development process. 
Since $R(t)$ only includes spatial derivatives, no non-linear
terms
rise from the Fourier decomposition. Hence, multiplying it
by the Fourier operator leads to the unity matrix. 
Finally $R(t)$ is simply evaluated at $2N+1$ timelevels.

This approach is really close to the non-linear frequency domain method.
The reader might observe that we are introducing the discrete Fourier
transform and its inverse. This is close to the fast Fourier transform
and it's inverse, as proposed by \citet{McMullen2001}. However,
as the development is on the equations and not during the time loop,
we get $2N+1$ steady equations that are coupled by a source term.
This source term appears as a spectral operator defined as:
\begin{equation}
	D_t = \frac{\partial E^{-1}}{\partial t} E = i \omega E^{-1} K E.
	\label{eq:sm_hb_mono_source_term_matrix}
\end{equation}
The main difference with the nonlinear frequency domain approach
is that the source term is known at the very first iteration and does
not change, meaning that we do not spend time computing a costly
fast fourier transform and its inverse at each time-step.

\citet{Gopinath2005} provide an analytical formulation of the
source term defined in Eq.~\ref{eq:sm_hb_mono_source_term_matrix}.
It a matrix operator whose elements are defined as:
\begin{equation}
  (D_t)_{l, j} =
  \begin{cases}
    \frac{\pi}{T}(-1)^{l-j}\csc\left(\frac{\pi
        (l-j)}{2N+1}\right) &, \, l\neq j,\\
    0 &, \, l=j.
  \end{cases}
\end{equation}
