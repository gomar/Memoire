%!TEX root = ../../adrien_gomar_phd.tex

The HB method has been originally
 proposed by \citet{Hall2002}.
It is a step further the non-linear frequency domain method. Instead
of using the fast Fourier transform to cast back the equations
to the time domain at each pseudo-iteration step, 
the equations are mathematically derived to be directly
computed into the time-domain.
To explain the method, we will again use the general form of 
the non-viscous Burger's equation as defined in
Eq.~\eqref{eq:sm_nonlinear_convection_residual}.


\subsection{Mono-frequential formulation}

Following the same approach as the non-linear frequency domain one,
consider that both $u$ and $R$ are periodic
in time with respect to period $T = 2 \pi / \omega$
and can be written using a Fourier series:
\begin{equation}
	\begin{split}
		u(t) &= \sum_{k=-\infty}^{\infty} \widehat{u}_k e^{i k \omega t}, \\
		R(t) &= \sum_{k=-\infty}^{\infty} \widehat{R}_k e^{i k \omega t}.
		\label{eq:sm_hall_dft}
	\end{split}
\end{equation}
Injecting Eq.~\eqref{eq:sm_hall_dft} in 
Eq.~\eqref{eq:sm_nonlinear_convection_residual}, and considering
the orthogonality of the complex exponentials:
\begin{equation}
	i k \omega \widehat{u}_k + \widehat{R}_k = 0, \: k \in [-N, N].
	\label{eq:sm_hall_frequential_eq}
\end{equation}

In the same way as one uses Fourier coefficients to
evaluate the temporal signal,
one can reconstruct the Fourier coefficients using
temporal evaluations. These are taken at evenly spaced timelevels
sampling the period $T = 2 \pi / \omega$ using the forward
Fourier transform. Moreover, 
according to the Nyquist-Shannon~\cite{Shannon1949} sampling theorem, 
at least $2N$ timelevels are needed to capture $N$ frequencies. Actually
$2N+1$ timelevels are used to prevent from odd-even decoupling as
demonstrated by \citet{Weide2005}. $\widehat{u}_k$ can thus
be expressed in function of $u(t)$:
\begin{equation}
	\widehat{u}_k = \frac{1}{2N+1} 
	\sum_{n=0}^{2N} u(t_n) e^{-i k \omega t_n}.
\end{equation}
If $E$ denotes the matrix composed of the elements 
$(E)_{k,n} = e^{-i (k - N) \omega t_n} / 2N+1$, one can write $\widehat{u}_k$
and $\widehat{R}_k$ as:
\begin{equation}
	\begin{split}
		\widehat{u}_k &= E u^\star, \\
		\widehat{R}_k &= E R^\star.
	\end{split}
	\label{eq:sm_matrix_fourier_operator}
\end{equation}
where $u^\star$ and $R^\star$ 
denote the vectors formed of all the evaluations of respectively $u$
and $R$,
made at $2N+1$ timelevels uniformly sampling the period of interest:
\begin{equation}
	\begin{split}
		u^\star &= [u(t_0), \ldots, u(t_{2N})], \\
		R^\star &= [R(t_0), \ldots, R(t_{2N})].
	\end{split}
\end{equation}
$E$ can thus be named the Fourier matrix.
Note that conversely, using the inverse Fourier matrix $E^{-1}$:
\begin{equation}
	\begin{split}
		u^\star &= E^{-1} \widehat{u}_k \\
		R^\star &= E^{-1} \widehat{R}_k.
	\end{split}
\end{equation}
Injecting the matrix formulation of 
Eq.~\eqref{eq:sm_matrix_fourier_operator} in 
Eq.~\eqref{eq:sm_hall_frequential_eq}
gives:
\begin{equation}
	i \omega K E u^\star + E R^\star = 0,
\end{equation}
where $K$ is a diagonal matrix formed of all the $k \in [-N, N]$.
Note that first, the matrix formulation encompass all harmonics
$k \in [-N, N]$ and second, it does not require the
orthogonality of the complex exponentials.
Now multiplying the equation by the inverse Fourier matrix $E^{-1}$:
\begin{equation}
	i \omega E^{-1} K E u^\star + R^\star = 0,
	\label{eq:sm_hb_matrix_form_mono}
\end{equation}
where $R^\star$ can now be substituted:
\begin{equation}
		i \omega E^{-1} K E u^\star + 
		\displaystyle \frac{\partial}{\partial x}
		\frac{(u^\star)^2}{2} = 0.
\end{equation}
What happened here is that instead of developing $R(t)$
in the frequency domain as made in the NLH approach,
which is tedious, this term is kept
in this form through all the development process. 
Since $R(t)$ only includes spatial derivatives, no non-linear
terms
arise by using the Fourier decomposition. Thus, multiplying it
by the Fourier matrix leads to the unity matrix. 
$R(t)$ is then simply evaluated at $2N+1$ timelevels.

This approach is really close to the NLFD method.
The reader might observe that we are introducing the discrete Fourier
transform and its inverse. This is close to the fast Fourier transform
and its inverse, as proposed by \citet{McMullen2001}. However,
as the development is on the equations and not during the time loop,
we get $2N+1$ steady equations, by definition in the time
domain, that are coupled by a source term.
The main difference with the nonlinear frequency domain approach
is that the source term is known at the first iteration and does
not change, meaning that we do not spend time computing a
fast Fourier transform and its inverse at each time-step.
The source term appears as a spectral operator defined as:
\begin{equation}
	D_t = i \omega E^{-1} K E.
	\label{eq:sm_hb_mono_source_term_matrix}
\end{equation}

\citet{Gopinath2005} provide an analytical formulation of the
source term defined in Eq.~\eqref{eq:sm_hb_mono_source_term_matrix}.
It is a matrix operator whose elements are defined as:
\begin{equation}
  (D_t)_{k, n} =
  \begin{cases}
    \frac{\pi}{T}(-1)^{k-n}\csc\left(\frac{\pi
        (k-n)}{2N+1}\right) &, \, k\neq n,\\
    0 &, \, k=n.
  \end{cases}
  \label{eq:sm_hb_mono_source_term_analytic}
\end{equation}
Finally, adding a pseudo-time ($\tau$) derivative to 
time march the equations to the steady state, 
the mono-frequential formulation of 
Eq.~\eqref{eq:sm_nonlinear_convection_conservative} in the harmonic
balance framework is given by:
\begin{equation}
	\fbox{$
	\displaystyle \frac{\partial u^\star}{\partial \tau} + 
	D_t (u^\star) + 
	\displaystyle \frac{\partial}{\partial x}
		\frac{(u^\star)^2}{2} = 0.
	$}
\end{equation}
with $D_t$ defined using Eq.~\eqref{eq:sm_hb_mono_source_term_analytic}.

% \citet{Hall2002} first presented what they call
% the Harmonic Balance Technique (HBT) referred here
% as the mono-frequential Harmonic Balance (HB). This is 
% the time-domain equivalent of the NonLinear Harmonic method
% presented by \citet{He1998}, except that the term
% $u^\prime u^\prime$ is not neglected in the
% current approach. It is validated on 
% the harmonic vibration response of a front 
% stage transonic compressor rotor.

% \citet{Gopinath2005} called the mono-frequential HB,
% the Time Spectral Method (TSM), analyzed its stability and
% showed that the spectral operator can be analytically derived.
% \citet{Weide2005} avoided the odd-even decoupling by choosing 
% an odd number of time-levels
% and applied the method to a $1-1$ scaled NASA Stage $35$ compressor 
% and
% compared the results to a classical time-marching scheme.

% \citet{JDufour2009} highlight the benefits of using a 
% non-linear approach for oscillating-flap simulations
% compared to linearized approaches. A one-harmonic HB simulation
% gives results comparable to an expensive time-marching simulation.

\subsection{Multi-frequential formulation}
\citet{Gopinath2007} and \citet{Ekici2007} 
extended the harmonic balance approach to
a multi-frequential formulation. To do so, they considered
a Fourier matrix defined as:
\begin{equation}
	(E)_{k,n} = \frac{1}{2N+1} e^{-i \omega_{k-N} t_n},
\end{equation}
where $N$ is the chosen number of frequencies.
Note that replacing $\omega_{k-N}$ by $(k - N) \omega$ gives
the mono-frequential inverse Fourier matrix back. 
This can be justified mathematically: in the
framework of the almost-periodic functions~\cite{Besicovitch1932},
such a function (which is composed of multiple
frequencies non necessarily harmonically related) can be approximated
by an almost-periodic
discrete Fourier transform. If $f(t)$ is an almost periodic function,
\citet{Besicovitch1932} proves that it can be approximated as:
\begin{equation}
	f(t) \approx \sum_{k=-N}^{N} \widehat{f}_k 
	e^{i \omega_k t}.
\end{equation}
This allows the use of the multi-frequential Fourier matrix as defined
above. However, in the multi-frequential case, the inverse Fourier matrix
$E^{-1}$ is not known \textit{a priori} 
and has to be numerically computed. Actually, as demonstrated by 
\citet{Gopinath2007}, it is easier to expressed $E^{-1}$ analytically,
compute its temporal derivative (that is hence analytical too) 
and inverse it numerically to obtain $E$.

Using the same process as for the mono-frequential formulation,
Eq.~\eqref{eq:sm_nonlinear_convection_residual} becomes:
\begin{equation}
	i E^{-1} P E u^\star + R^\star = 0,
\end{equation}
where $P$ is a diagonal matrix formed of all the pulsations $\omega_k$.
This is really close to the mono-frequential formulation given
in Eq.~\eqref{eq:sm_hb_matrix_form_mono}.
Finally, adding a pseudo-time ($\tau$) derivative 
to time-march the equation to the steady state,
the multi-frequential formulation of 
Eq.~\eqref{eq:sm_nonlinear_convection_conservative} in the harmonic
balance framework reads:
\begin{equation}
	\fbox{$
	\displaystyle \frac{\partial u^\star}{\partial \tau} +
	D_t (u^\star) + 
	\displaystyle \frac{\partial}{\partial x}
		\frac{(u^\star)^2}{2} = 0.
	$}
\end{equation}
with $D_t$ defined as:
\begin{equation}
	D_t = i E^{-1} P E,
\end{equation}
and again $u^\star = [u(t_0), \ldots, u(t_{2N})]$ 
and $R^\star = [R(t_0), \ldots, R(t_{2N})]$.

For the mono-frequential formulation, the use of a uniform
sampling of the timelevels has the good property of being
well conditioned.
Before go into details, let us recall the definition of the
condition number $\kappa$ of a matrix $E$:
\begin{equation}
	\kappa (E) = \kappa (E^{-1}) = \| E \| \cdot \| E^{-1} \|, \quad
    \kappa(E) \geq 1,
\end{equation}
where $\| \cdot \|$ denotes a matrix norm.  Considering the resolution
of the system of equation
$A x = b$, if $A$ is invertible and if $\delta A$, $\delta x$ and
$\delta b$ are the numerical errors associated with the computation of
$A$, $x$ and $b$, respectively, then:
\begin{equation}
   (A + \delta A)(x + \delta x) = b + \delta b.
   \label{eq:error_reso}
\end{equation}
By definition, the condition number sets an upper bound for 
the error made on~$x$:
\begin{equation}
   \frac{\| \delta x \|}{\| x \|} \leq 
   \kappa(A)\left[\frac{\| \delta A \|}{\| A \|} + 
   \frac{\| \delta b \|}{\| b \|} \right].
   \label{eq:conditonnig_amp}
\end{equation}
The error on the iterative resolution of the governing equations can
therefore be amplified by the harmonic balance source term. 
This amplification is
led by the condition number of the almost-periodic Fourier matrix. This
also means that if the error is small but the condition number is
high, and vice-versa, the computation can diverge too. However, the
error can not be \emph{a priori} controlled, thus the need to
minimize the condition number.

In the mono-frequential formulation, 
the Fourier matrix is well-conditioned: the
uniform sampling for harmonically related frequencies leads to a
condition number equal to~$1$, which is the theoretical lower bound
for the condition number.  This is linked to the orthogonality of the
complex exponential family.  On the other hand, when the frequencies are arbitrary, it is usually
impossible to choose a uniform set of time instants over which the
almost-periodic Fourier matrix~$E$ is well conditioned. In fact, it is common for uniformly-sampled
sinusoids at two or more frequencies to be nearly linearly dependent,
which causes them not to be orthogonal, leading to the
ill-conditioning encountered in practice. This issue will be discussed
further in Sec.\mytodo{sec conditinnning} and an innovative
solution will be proposed.

\subsection{Extensions}

\paragraph{Navier-Stokes equations}
Equivalent to the non-linear frequency domain method, the
considered equation is really close to the finite-volume
semi-discrete form of the Navier-Stokes equations. Therefore,
nothing particular as to be made to apply the harmonic balance approach
to the Navier-Stokes equations.
This shows its advantage over the NLFD and particularly over the NLH method.
\citet{JSicot2008} developed an implicit
version of the mono-frequential harmonic
balance approach, improving its implementation
capability into existing industrial CFD codes.


\paragraph{Turbomachinery computations}
\citet{Gopinath2007} and \citet{Ekici2007}
first introduced the muti-frequential harmonic balance
method. Both of them applied the method to
a two-dimensional multistage compressor called
configuration D. Note that \citet{Ekici2007} use
$3N+1$ timelevels to improve the condition number
of the source term while \citet{Gopinath2007}
stays with $2N+1$ timelevels.
\citet{Ekici2008a} applied the multi-frequential method
to the effect of wake passing on the vibration of
a turbine blade. Note that the stator is modeled
by an unsteady wake injection but not computed.
In this computation, $2N+1$ timelevels are considered.
\citet{JSicot2012} adapted the harmonic balance 
to phase lag periodic 
conditions to reduce the computational domain. 
A row coupling strategy is set up, it involves 
a time and space interpolation at row interfaces.
A filter is applied to remove spurious waves.
\citet{JSicot2013} applied the multi-frequential 
method to a $3$D
$3.5$ stage industrial compressor and to the
aeroelasticity of a contra-rotating fan, proving
the maturity of the method.

\paragraph{Aeroelastic simulations}
\citet{Thomas2002a} uses the method to
determine the limit-cycle oscillation solution
of a transonic airfoil configuration using the
Euler equations and \citet{Thomas2004b} extended
it to the viscous Navier-Stokes equations.
\citet{Huang2013} applied the mono-frequential
HB method to the flutter prediction of the 
$11\textsuperscript{th}$ 
standard configuration for aeroelasticity~\cite{Fransson1999}.
They show that with only one harmonic, the local
harmonic response of the fluid is superimposed
to the results of a time-marching simulation.
\todo{moi}
\citet{JSicot2013} applied the multi-frequential 
method to a $3$D
$3.5$ stage industrial compressor and to the
aeroelasticity of a contra-rotating fan, proving
the maturity of the method.
\citet{JGuedeney2013} introduce non-uniform 
timelevels to minimize the condition number of multi-frequential
computations. 
The effect of the condition number on the stability
of the computations is assessed along with algorithms
to optimize the choice of the timelevels.
The method is then applied
to a $1.5$ stage subsonic compressor.
Note that a part of this work has been done in this
thesis and will be detailed latter on.


\paragraph{Transient problems}
\citet{Mavriplis2012} extended the method to 
an hybrid polynomial/harmonic balance approaches. 
It allows to use the method for maneuver simulations, 
where a part of the simulation exhibits a physical transient.
The method is also extended to overlapping mesh grids.

\paragraph{Sampling for multi-frequential formulation}
\citet{Gopinath2007} and \citet{Ekici2007}
first introduced the muti-frequential harmonic balance
method. Both of them applied the method to
a two-dimensional multistage compressor called
configuration D. Note that \citet{Ekici2007} use
$3N+1$ timelevels to improve the condition number
of the source term while \citet{Gopinath2007}
stays with $2N+1$ timelevels.
\citet{Ekici2008a} applied the multi-frequential method
to the effect of wake passing on the vibration of
a turbine blade. Note that the stator is modeled
by an unsteady wake injection but not computed.
In this computation, $2N+1$ timelevels are considered.
\citet{JGuedeney2013} introduce non-uniform 
timelevels to minimize the condition number of multi-frequential
computations. 
The effect of the condition number on the stability
of the computations is assessed along with algorithms
to optimize the choice of the timelevels.
The method is then applied
to a $1.5$ stage subsonic compressor.
Note that a part of this work has been done in this
thesis and will be detailed latter on.


\paragraph{Gradient-based method to determine the frequency}
With the same approach as \citet{McMullen2002}, \citet{Gopinath2006}
developed a gradient-based method to estimate the frequency of a 
vortex shedding behind a cylinder and a NACA0012 airfoil 
at high angle of attack using the harmonic balance approach.

\paragraph{Optimum shape design}
\citet{Thomas2005b} used an automatic 
differentiation software tool to derive an adjoint code
from a harmonic balance code and validated this approach
on a representative model problem.

\paragraph{Adaptive method}
\citet{Maple2004} presented an adaptive harmonic
balance approach. The number of harmonics is increased
if the energy of the last harmonic divided by the cumulative
sum of the energy of each harmonic is larger than a 
given threshold. During the first iterations, only
a low number of harmonic is kept. Then, when the flow
is almost converged, the adaptive harmonic balance
approach is used. This ensures that higher order harmonics
are not injected at the first iterations, when the
flow is not physical. A $86\%$ reduction in time (and
in memory footprint) is seen compared to a resolved (converged in
terms of harmonics $N$) harmonic
balance computation. This has to be compared to
the $2$ factor speed-up observed by \citet{Mosahebi2013}
with an adaptive NLFD approach.

\subsection{Cost of the method}
As mentioned before, the cost of the method is linked to
the number of timelevels simulated.
In fact, each new timelevel corresponds to an additional steady computation.
Thus, if $2N+1$ timelevels are considered and if $\mathdollar_{\text{RANS}}$ 
denotes the CPU and memory cost of
one steady computation, the cost of the HB method can be 
approximated by:
\begin{equation}
	\mathdollar_{\text{HB}} = (2N+1) \cdot \mathdollar_{\text{RANS}}.
\end{equation}
Note that \citet{Ekici2007,Ekici2008a} use $3N+1$
timelevels or more due to solve the bad conditioning of the
source term. Thus, in that
case, the cost is bigger and scales with the chosen number
of timelevels.