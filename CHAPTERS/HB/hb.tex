%!TEX root = ../../adrien_gomar_phd.tex

Originally proposed by \citet{Hall2002}, the Harmonic Balance Techniques
relies on a simple observation: to develop spectral methods, and in
particular the NonLinear Harmonic method, one has made use of the Fourier
transform to efficiently represent an unsteady signal with a Fourier series.
However, the method needs to be resolved in the frequency domain meaning
that all the numerical techniques should be adapted: the numerical schemes,
the turbulent model and so on. The smart idea 
proposed by \citet{Hall2002} is to
cast back the steady frequential equations to the time domain
by using an inverse Fourier transform.

Let us write Eq.~\ref{eq:sm_nonlinear_convection_conservative} 
in a more general form:
\begin{equation}
	\frac{\partial u}{\partial t} + R (u) = 0,
	\label{eq:sm_nonlinear_convection_residual}
\end{equation}
with
\begin{equation}
	R(u) = \frac{\partial}{\partial x} \left( 
	\frac{u^2}{2} \right).
\end{equation}

\subsection{Mono-frequential formulation}
A finite Fourier series truncated at order $N$ is defined as:
\begin{equation}
	u = \sum_{k=-N}^{N} 
	\widehat{u}_k e^{i k \omega t}.
	\label{eq:sm_hall_dft}
\end{equation}
Injecting Eq.~\ref{eq:sm_hall_dft} in 
Eq.~\ref{eq:sm_nonlinear_convection_residual}, and since
the complex exponentials form an orthogonal basis, 
this leads to:
\begin{equation}
	i k \omega \widehat{u}_k + \widehat{R}_k = 0, \: k \in [-N, N].
	\label{eq:sm_hall_frequential_eq}
\end{equation}
Obviously, the term $\widehat{R}_k$ is the hardest to develop
due to the nonlinearities. Here, the Harmonic Balance approach
is clever. 
Instead of developing this term in the frequency domain
using the Fourier coefficients $\widehat{u}_k$ of $u$,
one can assume that $R(u)$ can be decomposed using its own Fourier series
truncated at order $N$:
\begin{equation}
	R(u) = \sum_{k=-N}^{N} 
	\widehat{R}_k e^{i k \omega t}.
\end{equation}
This is an approximation as due to the nonlinearities, 
terms of order greater than $N$ appears, consequently, 
these are neglected in the present
formulation . However, the cross coupling terms corresponding
of the term $u^\prime u^\prime$ of the NonLinear Harmonic 
approach are not neglected here and
the terms of order greater
than $N$ are already neglected as $u$ is only evaluated using 
at most $N$ harmonics. This is thus coherent.

In the same way as one uses Fourier coefficients to
evaluate the temporal signal,
one can reconstruct the Fourier coefficients using
temporal evaluations taken at evenly spaced timelevels
sampling the period $T = 2 \pi / \omega$ using the forward
Fourier transform. Moreover, 
according to the Nyquist-Shannon~\cite{Shannon1949} sampling theorem, 
at least $2N+1$ timelevels are needed to capture $N$ frequencies,
leading to:
\begin{equation}
	\widehat{u}_k = \frac{1}{2N+1} 
	\sum_{n=0}^{2N} u_n^\star e^{-i k \omega t_n}.
\end{equation}
If $E$ denotes the matrix composed of the elements 
$E_{k,n} = e^{-i k \omega t_n} / 2N+1$, one can write $\widehat{u}_k$
and $\widehat{R}_k$ as:
\begin{equation}
	\begin{split}
		\widehat{u}_k &= E u^\star \\
		\widehat{R}_k &= E R^\star,
	\end{split}
\end{equation}
where $u^\star$ and $R^\star$ 
denote the vectors formed of all the evaluations of respectively $u$
and $R$,
made at the $2N+1$ timelevels. $E$ is thus the inverse Fourier operator.
Note that conversely, using the Fourier operator $E^{-1}$:
\begin{equation}
	\begin{split}
		u^\star &= E^{-1} \widehat{u}_k \\
		R^\star &= E^{-1} \widehat{R}_k.
	\end{split}
\end{equation}
Injecting the matrix formulation in Eq.~\ref{eq:sm_hall_frequential_eq}
gives:
\begin{equation}
	i K \omega E u^\star + E R^\star = 0,
\end{equation}
where $K$ is a diagonal matrix formed of all the $k \in [-N, N]$
Now multiplying the equation by the Fourier operator $E^{-1}$:
\begin{equation}
	i \omega E^{-1} K E u^\star + R^\star = 0,
\end{equation}
where $R^\star$ can now be substituted:
\begin{equation}
	\fbox{$
		i \omega E^{-1} K E u^\star + 
		\displaystyle \frac{\partial}{\partial x}
		\frac{(u^\star)^2}{2} = 0
	$}
\end{equation}
What happened here is that instead of developing $R(u)$
in the frequency domain, which is tedious, this term is kept
as it is through all the development process. 
Since $R(u)$ only includes spatial derivatives, no additional terms
rise from the Fourier decomposition. Hence, multiplying it
by the Fourier operator leads to the unity matrix. It is
simply evaluated at $2N+1$ timelevels.

The result is $2N+1$ steady equations coupled by a
source term that appears as a spectral
operator defined as:
\begin{equation}
	D_t = i \omega E^{-1} K E
\end{equation}
