%!TEX root = ../../adrien_gomar_phd.tex

A large amount of flows are time-periodic in a turbomachines.
In fact, consider a stage of a turbomachine, as for instance
a urbine stator-rotor configuration as shown 
in Fig.~\ref{fig:sm_unsteady_turbomachine}. 
\begin{figure}[htbp]
  \centering
  \includegraphics*[width=0.4\textwidth]{unsteady_turbomachine.pdf}
  \caption{Main unsteady effects present in a turbomachinery stage. Here, a turbine stator-rotor
  configuration is shown.}
  \label{fig:sm_unsteady_turbomachine}
\end{figure}

Due to the
viscosity effects acting on the stator blades, 
a wake is generated behind it and 
impinges the rotor row. In opposite, the flow field
generated around the rotor can literally go back up
to the stator row. In fact, in absence of compressibility
effects, the acoustic fluctuations can go backwards leading
to the potential effects. Moreover as, the blade have a relative velocity
the field that is created in a row is perceived as unsteady in the opposite row
ones it crosses the stator-rotor interface. This unsteadiness can be
correlated to the so-called Blade Passing Frequency (BPF) defined as:
\begin{equation}
	f = \frac{\Omega_{rel} B_{opp}}{2 \pi},
\end{equation}
where $\Omega_{rel}$ is the relative speed difference 
and $B_{opp}$ the number of blades of the opposite row.
At first order, the unsteady effects presented here drive
most of the time-dependent field in a turbomachine. This 
is off course an approximation, but we will see at the end
of this chapter that the range of unsteady periodic
flow phenomenon in a turbomachines is predominant.

However, a classical time-marching scheme has no knowledge
of the periodic nature of the field in a turbomachine, hence
the idea that efficient algorithm can be build by taking advantage 
of the periodic feature. These algorithms can be called
the spectral methods.

