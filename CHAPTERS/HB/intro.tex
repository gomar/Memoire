%!TEX root = ../../adrien_gomar_phd.tex

The computational power available today in
the research centers and in the industry
is so big that large eddy simulation
becomes possible for industrial configurations.
This is actually needed as high-fidelity
simulations help the turbomachinery community understand
the complex nature of flows that develop 
within these components,
allowing breakthrough ideas.
However, even if high fidelity approaches
are today within reach,
they will always be room for fast reliable
computations. 
In fact, on a daily basis, engineers need
to run lots of simulations to test new designs.
In this framework, large eddy simulation is
too demanding to be used for design purposes.

Today, in most companies, steady Reynolds-Averaged
Navier-Stokes (RANS) based solver are used on a daily basis.
For instance, this tool 
helped building the new $3$D-shape
of the forthcoming CFM-LEAP engine
depicted in Fig.~\ref{fig:sm_leap}.
\begin{figure}[htbp]
  \centering
  \includegraphics*[width=0.40\textwidth]{leap.jpg}
  \caption{$3$D-shape fan blades of the forthcoming CFM-LEAP engine.}
  \label{fig:sm_leap}
\end{figure}
Some further improvements are made possible by the use
of the unsteady RANS computations.
However, in the industry, unsteady computations
are still too expensive to be used on a daily basis.
Based on a simple idea, spectral methods are 
able to reproduce the unsteady field to engineering
accuracy, for a cost proportional to the cost of a
steady computation.

In turbomachines, a large amount of flows is time-periodic.
In fact, consider a stage of a turbomachine, as for instance
a turbine stator-rotor configuration as shown 
in Fig.~\ref{fig:sm_unsteady_turbomachine}. 
\begin{figure}[htbp]
  \centering
  \includegraphics*[width=0.4\textwidth]{unsteady_turbomachine.pdf}
  \caption{Main unsteady effects present in a turbomachinery stage. Here, a turbine stator-rotor
  configuration is shown.}
  \label{fig:sm_unsteady_turbomachine}
\end{figure}
Due to the
viscosity effects acting on the stator blades, 
a wake is generated behind it and 
impinges the rotor row. In opposite, the flow field
generated around the rotor can literally go back up
to the stator row. In fact
the acoustic fluctuations can go backwards yielding
the potential effects. Moreover as, the rows have a 
rotation speed difference,
the field that is created in one row is perceived as unsteady in the opposite 
row frame of reference. This unsteadiness can be
correlated with the so-called Blade Passing Frequency (BPF) defined as:
\begin{equation}
	f = \frac{\Omega_{rel} B_{opp}}{2 \pi},
\end{equation}
where $f$ is the BPF, $\Omega_{rel}$ the relative speed difference 
and $B_{opp}$ the number of blades in the opposite row.
At first order, the unsteady effects presented here drive
most of the time-dependent field in a turbomachine. This 
is of course an approximation, but we will see at the end
of this chapter that the range of unsteady periodic
flow phenomenon in a CROR is large.

The problem with classical time-marching scheme is 
that it has no knowledge
of the periodic nature of the field, thus
the idea that efficient algorithm can be build by taking advantage 
of this periodicity. Hence the spectral methods that are
presented above.

