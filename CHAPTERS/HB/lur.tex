%!TEX root = ../../adrien_gomar_phd.tex

\citet{Verdon1984} originally developed unsteady linearized 
method in the framework of potential flows. Latter on, \citet{Hall1989}
extended the linearized method to Euler equations and
\citet{Clark2000} applied it on the Reynolds-Averaged Navier-Stokes equations.
The unsteady linearized method relies on a decomposition of the variables
into a steady part and a small-disturbance unsteady component:
\begin{equation}
	u = \overline{u} + u^\prime,
	\label{eq:sm_lur_decomposition}
\end{equation}
where $u^\prime$ is considered to be small. In his thesis,
\citet{Hall1987} defines small to be less than $10\%$ of the
steady flow.
Injecting Eq.~\ref{eq:sm_lur_decomposition} into 
Eq.~\ref{eq:sm_nonlinear_convection_conservative} leads to:
\begin{equation}
	\frac{\partial u^\prime}{\partial t} + 
	\frac{1}{2}\frac{\partial}{\partial x} \left[
	\overline{u}^2 + 2 \overline{u} u^\prime + u^\prime u^\prime \right] = 
	0.
	\label{eq:sm_lur_step_1}
\end{equation}
By means of linearization, i.e. collecting terms
of equal order and neglecting terms of higher order, 
Eq.~\ref{eq:sm_lur_step_1} can be split
into a steady equation:
\begin{equation}
	\frac{\partial \overline{u}^2}{\partial x} = 0,
\end{equation}
and an unsteady first order perturbation equation:
\begin{equation}
	\frac{\partial u^\prime}{\partial t} +
	\frac{\partial}{\partial x} \left[
	\overline{u} u^\prime \right] = 
	0.
\end{equation}

\paragraph{Mono-frequential formulation}
Now, assuming that the velocity perturbation is periodic in 
time with period $T = 2\pi / \omega$, the unsteady perturbations 
can be decomposed into a Fourier series:
