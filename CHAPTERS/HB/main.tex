%!TEX root = ../../adrien_gomar_phd.tex
\chapter{Spectral methods}
\label{cha:spectral_methods}

\chabstract{The spectral methods are presented in this chapter.
Actually four are presented: the Linearized 
Unsteady Reynolds-averaged Navier-Stokes (LUR), 
the NonLinear Harmonic (NLH), 
the NonLinear Frequency Domain (NLFD) 
and the Harmonic Balance (HB) method. The LUR
method comes from a linearization of the governing equation
while the three others are build to take into account for the 
non-linearities. The NLH, NLFD and HB methods
rely on the decomposition of the variable of interest
in Fourier series. By truncating these at order $N$,
$2N+1$ steady equations coupled by
a source term are developed. 
Emphasis is put on the development
of multi-frequential formulation and its
mathematical background to allow multi-stage
configurations.
The applicability of 
these methods is demonstrated in the literature
through analytical test cases, $2$D/$3$D academic 
turbomachine configurations,
industrial transonic multi-stage applications, 
aeroelastic configurations and even unsteady
optimization problems. The cost of the method
is almost $2N+1$ times the cost of a steady
computation with $N$ being the number of computed harmonics.}

% ================
% = INTRODUCTION =
% ================
\section{Introduction}
\label{sec:sm_intro}
%!TEX root = ../../../adrien_gomar_phd.tex

For an aircraft in steady flight conditions, 
lift balances weight and 
thrust balances drag. This explains why engineers try
indefinitely to reduce weight while increasing
thrust. A trade-off between those two is to work
on the propulsive efficiency of the engine. In this
section, general information on propulsion
that leads to the concepts of propeller and
contra-rotating open rotor are given.

\subsection{Thrust equation}
\label{sub:cror_thrust}
Consider the conservative equation of momentum
\begin{equation}
	\frac{\partial \rho \vec{V}}{\partial t} 
	+ \nabla \cdot (\rho \vec{V} \otimes \vec{V} + p \mathbb{I} - \vec{\vec{\Sigma}}_v) = 0,
\end{equation}
where $\rho$ is the density, $\vec{V}$ the velocity vector, $p$ the pressure and
$\vec{\vec{\Sigma}}_v$ the viscous stress terms.
Consider two closed domains $\Sigma$ and $\Sigma^\prime$ as
shown in Fig.~\ref{fig:cror_control_volume}.
\begin{figure}[htp]
  \centering
  \includegraphics*[width=0.30\textwidth]{control_volume.pdf}
  \caption{Domains used for the application of the momentum equation.}
  \label{fig:cror_control_volume}
\end{figure}
The domain $\Sigma^\prime$ represents a fluid domain outside from the
engine encompassed by the solid domain $\Sigma$.
Taking a steady state hypothesis, one can write
\begin{equation}
	\oint_{\Sigma} \left(\rho \vec{V} \otimes \vec{V} + 
	                       p \mathbb{I} - 
	                       \vec{\vec{\Sigma}}_v \right) \cdot \vec{n} \diff S
    =
   	\oint_{\Sigma^\prime} \left(\rho \vec{V} \otimes \vec{V} + 
	                       p \mathbb{I} - 
	                       \vec{\vec{\Sigma}}_v \right) \cdot \vec{n} \diff S,
\end{equation} 
where $\vec{n}$ is the normal vector.
As $\Sigma^\prime$ is an arbitrary domain, we can take it sufficiently
away from the engine so that $\vec{\vec{\Sigma}}_v$ becomes zero (\emph{i.e.}
viscosity stress terms are null).
Moreover, 
\begin{equation}
	\oint_{\Sigma} \left(\rho \vec{V} \otimes \vec{V} \right) \cdot \vec{n} \diff S = 0,
\end{equation}
since the surface is solid ($\vec{V} = \vec{0}$ on wall). 
If $\vec{F}$ denotes the resultant forces acting on $\Sigma$
\begin{equation}
	\vec{F} = \oint_{\Sigma} \left(p \mathbb{I} - 
	\vec{\vec{\Sigma}}_v \right) \cdot \vec{n} \diff S,
\end{equation}
then
\begin{equation}
	\vec{F} = \oint_{\Sigma^\prime} \left(\rho \vec{V} \otimes \vec{V} +
	p \mathbb{I} \right) \cdot \vec{n} \diff S.
\end{equation}
Assuming that $\Sigma^\prime$ is a stream tube, and projecting the equation
onto the $x$-axis gives the formula for the thrust $F_x$
\begin{equation}
	F_x = \dot{m} V_{out} + p_{out} S_{out}
	- \dot{m} V_{in} - p_{in} S_{in},
\end{equation}
using the notation of Fig.~\ref{fig:cror_control_volume}.

Far downstream of the engine $S_{in} = S_{out}$ and
considering that we have an adapted nozzle ($p_{in} = p_{out}$),
the thrust $F_x$ can be written as
\begin{equation}
	\fbox{$
	F_x = \dot{m} (V_{out} - V_{in}) = \dot{m} \Delta V_x
	$}
	\label{eq:cror_thrust}
\end{equation}
where $\dot{m}$ is the mass-flow rate going through the
propeller and $\Delta V_x$ is
the increment of axial velocity. From this simple equation,
one can see that to increase the thrust $F_x$, there are two parameters:
the mass-flow and the axial velocity increment.

\subsection{Global propulsive efficiency}
\label{sub:cror_efficiency}

The global propulsive efficiency $\eta$ measures the 
success in converting a mechanical power into a
propulsive power. It results from the combination
of the kinetic efficiency $\eta_{K}$ and the propulsive efficiency
$\eta_{PR}$
\begin{equation}
	\eta = \eta_{K} \times \eta_{PR}.
\end{equation}
This is schematically represented in Fig.~\ref{fig:cror_efficiency}.
\begin{figure}[htp]
  \centering
  \includegraphics*[width=0.40\textwidth]{efficiency.pdf}
  \caption{Efficiency relations from mechanical power to propulsive power.}
  \label{fig:cror_efficiency}
\end{figure}

\paragraph{Kinetic efficiency}
The kinetic efficiency measures the success in converting the mechanical
power $P_m$ into a kinetic power $P_k$
\begin{equation}
	\eta_K = \frac{P_k}{P_m}.
\end{equation}

The mechanical power delivered as input
can be computed through the first thermodynamic principle. In fact, in absence
of heat exchange, the mechanical power $P_m$ can be estimated as
\begin{equation}
	P_m = \dot{m} (h_{i_{out}} - h_{i_{in}}),
\end{equation}
where $h_i$ is the total enthalpy and subscript $in$ and $out$ are
the input and output, respectively, of the propulsion system as represented
in Fig.~\ref{fig:cror_control_volume}.
The kinetic power $P_k$ is given by
\begin{equation}
	P_k = \dot{m} \left(\frac{1}{2} V^2_{out} -
	\frac{1}{2} V^2_{in} \right).
\end{equation}
This leads to a kinetic efficiency that can be expressed as
\begin{equation}
	\eta_{K} = \frac{V^2_{out} - V^2_{in}}{2 (h_{i_{out}} - h_{i_{in}})}
\end{equation}

\paragraph{Propulsive efficiency}
The propulsive efficiency $\eta_{PR}$ measures the success
in creating a propulsive power $P_{pr}$ from a
kinetic power $P_k$
\begin{equation}
	\eta_{PR} = \frac{P_{pr}}{P_k}.
\end{equation}
The propulsive power is computed using the thrust $F_x$
\begin{equation}
	P_{pr} = F_x \times V_{\infty},
\end{equation}
where $V_{\infty}$ is the free-stream velocity.
Finally, if the free-stream velocity is the inlet velocity $V_{in}$
and the inlet and outlet velocities are purely axial
\begin{equation}
	\fbox{$
	\eta_{PR} = \displaystyle \frac{1}{1 + \frac{V_{out} - V_{in}}{2 V_{in}}}
	$}
	\label{eq:cror_propulsive_efficiency}
\end{equation}
This formula means that the most efficient engine produces
a very small velocity increment.

\subsection{Toward propeller engines}
\label{sub:cror_toward_propeller}

One way to improve the environmental footprint of
airplanes engines is to increase the propulsive efficiency
by reducing the kinetic power needed to drive the engine.
Doing so while maintaining the thrust can be achieved through
a higher mass-flow rate. Two new concepts are thus derived from
this simple statement: the High ByPass-Ratio (HBPR) which
is basically a turbofan with a larger fan exhaust, and the
propeller whose mass-flow rate is not limited
by the architecture, as the blades are not within a nacelle.
In the following section, the propeller engine will be detailed
and the drawbacks of such an architecture will be highlighted to
motivate the use
of a second propeller row, yielding the contra-rotating open rotor
architecture.




% ====================
% = STATE OF THE ART =
% ====================
\section{State of the art}
\label{sec:sm_state_of_the_art}
%!TEX root = ../../adrien_gomar_phd.tex

There is a large variety of spectral methods that exists in the
literature. The most important will be presented in this section.
As the development of these approaches on the Navier-Stokes equations
can be tedious, the following will only concentrate on the simplest
nonlinear equation, the advection equation for the velocity
defined as:
\begin{equation}
	\frac{\partial u}{\partial t} + 
	u \frac{\partial u}{\partial x} = 
	0.
	\label{eq:sm_nonlinear_convection}
\end{equation}
This equation can be formulated in a conservative manner for simplicity:
\begin{equation}
	\frac{\partial u}{\partial t} + 
	\frac{\partial}{\partial x} \left( \frac{u^2}{2} \right) = 
	0.
	\label{eq:sm_nonlinear_convection_conservative}
\end{equation}

The reader is referred to the cited papers for a detailed description
of the following methods and in particular, their
development for the Navier-Stokes equations. 

In total four spectral methods are presented, 
the Linearized Unsteady Reynolds averaged
Navier-Stokes, the NonLinear Harmonic method, the NonLinear Frequency Domain
method and the Harmonic Balance method.  
These names are chosen here
for simplicity but the reader might find in the literature many more
appellations. When it is the case, an effort will be made to synthesize
these to give the reader a good overview of the type of spectral methods available. 
Moreover, special emphasis will be put on the harmonic
balance method as it is the approach used in this thesis.


\subsection{The \underline{L}inearized 
\underline{U}nsteady \underline{R}eynolds-averaged Navier-Stokes method (LUR)}
\label{sub:sm_lur}
%!TEX root = ../../adrien_gomar_phd.tex

TO FILL

\subsection{The \underline{N}on\underline{L}inear 
\underline{H}armonic method (NLH)}
\label{sub:sm_nlh}
%!TEX root = ../../adrien_gomar_phd.tex

Originally developed by \citet{He1998} and \citet{Ning1998},
the NonLinear Harmonic method
relies on a decomposition of the conservative variables into a
time-averaged part plus an unsteady perturbation:
\begin{equation}
	u = \overline{u} + u^\prime,
	\label{eq:sm_nlh_decomposition}
\end{equation}
where $\overline{.}$ denotes the time-averaging operator and
$.^\prime$ the corresponding unsteady perturbation.
By injecting Eq.~\ref{eq:sm_nlh_decomposition} into
Eq.~\ref{eq:sm_nonlinear_convection_conservative}, one gets:
\begin{equation}
	\frac{\partial u^\prime}{\partial t} + 
	\frac{1}{2}\frac{\partial}{\partial x} \left[
	\overline{u}^2 + 2 \overline{u} u^\prime + u^\prime u^\prime \right] = 
	0.
	\label{eq:sm_nlh_step_1}
\end{equation}
The time-averaged equation can be obtained by time-averaging
equation~\ref{eq:sm_nlh_step_1}:
\begin{equation}
	(\overline{\ref{eq:sm_nlh_step_1}})
	\Leftrightarrow
	\frac{\partial}{\partial x}
	\left[\overline{u}^2 + 
	\overline{u^\prime u^\prime}\right] =
	0,
	\label{eq:sm_nlh_step_2}
\end{equation}
The term $\overline{u^\prime u^\prime}$
appears due to the non-linearities of the considered equation. It
is called the nonlinear stress terms 
(or the deterministic stress terms) as a reference to 
the Reynolds stress terms. 
The equations for the unsteady perturbations is then obtained by keeping
the first order terms of the unsteady equation~\ref{eq:sm_nlh_step_1}.
This means that the term $u^\prime u^\prime$ is neglected and leads
to:
\begin{equation}
	\frac{\partial u^\prime}{\partial t} + 
	\frac{\partial}{\partial x} \left[\overline{u} u^\prime \right] = 
	0.
\end{equation}

\paragraph{Mono-frequential formulation}
For now on, no assumption has been made neither on the velocity $u$,
nor on its time-averaged part and unsteady perturbation part.
Now, assuming that the velocity perturbation 
is periodic in time with period
$T=2 \pi / \omega$,
the unsteady fluctuations can be decomposed into 
a Fourier series:
\begin{equation}
	u^\prime = \sum_{k=-\infty \atop k \neq 0}^{\infty} 
	\widehat{u}_k e^{i \omega k t}.
	\label{eq:sm_nlh_decomposition_pert}
\end{equation}
Hence, since the complex exponentials forms 
an orthogonal basis, we have for all harmonics 
$-\infty \leq k \leq \infty, \; k \neq 0$:
\begin{equation}
	i \omega k \widehat{u}_k + 
	\frac{\partial}{\partial x} \left[ \overline{u} \widehat{u}_k\right] =
	0.
\end{equation}
One can notice that the time-averaged part has been removed from
the Fourier series through $k \neq 0$.
Each harmonic equation represents now a steady equation as no temporal
derivative is present anymore.

The term $\overline{u^\prime u^\prime}$ remains in the time-averaged
equation and needs to be computed. It can be 
directly worked out when the harmonics are known:
\begin{equation}
	\begin{split}
		u^\prime u^\prime &= 
		\left[
			\sum_{k=-\infty \atop k \neq 0}^{\infty} \widehat{u}_k e^{i \omega k t} 
		\right]
		\left[
			\sum_{k=-\infty \atop k \neq 0}^{\infty} \widehat{u}_k e^{i \omega k t} 
		\right] \\
		&= \sum_{k=-\infty \atop k \neq 0}^{\infty} (\widehat{u}_k)^2
		   e^{i 2 \omega k t} +
		   2 \sum_{j,k=-\infty \atop j \neq k \neq 0}^{\infty} 
		   \widehat{u}_k \widehat{u}_j e^{i \omega (j + k) t} \\
	\end{split}
\end{equation}
Thus,
\begin{equation}
	\begin{split}
		\overline{u^\prime u^\prime} &= 
		\frac{1}{T} \int_{t=0}^{T} \left[ 
			\sum_{k=-\infty \atop k \neq 0}^{\infty} (\widehat{u}_k)^2
		   	e^{i 2 \omega k t} +
		   	2 \sum_{j,k=-\infty \atop j \neq k \neq 0}^{\infty} 
		   	\widehat{u}_k \widehat{u}_j e^{i \omega (j + k) t} 
		\right] dt\\
		&= \frac{2}{T} \int_{t=0}^{T} \sum_{j,k=-\infty \atop j \neq k \neq 0}^{\infty} 
		   	\widehat{u}_k \widehat{u}_j 
		   	e^{i \omega (j + k) t} dt \\
		&= \frac{2}{T} \int_{t=0}^{T} 
			\sum_{k=-\infty \atop k \neq 0}^{\infty} 
			\widehat{u}_k \widehat{u}_{-k}  dt.
	\end{split}
\end{equation}
As $\widehat{u}_k$ and $\widehat{u}_{-k}$ are complex conjugates,
finally $\overline{u^\prime u^\prime}$ is equal to:
\begin{equation}
	\overline{u^\prime u^\prime} = 
	2 \sum_{k=-\infty \atop k \neq 0}^{\infty} |\widehat{u}_k|^2.
	\label{eq:sm_nlh_deterministic_stress_terms}
\end{equation}
This last equation depends only on the computed harmonics, meaning
that no term is modelled. Moreover, this term couples the
time-average solution with the unsteady perturbations. This is this
terms that is \todo{LUR}

Finally, as computing an infinite number of harmonics is not feasible,
the number of harmonics is truncated at order $N$. 
This is a fare assumption as most
of the physical flows have a finite unsteady spectrum. This
is for sure a reduce order approach. The goal of the spectral
methods being to have a compact representation of the unsteady time
signals. As for a mesh grid convergence, the number of harmonics $N$
is increased until the unsteady representation of the signal is
converged for the variable of interest. The discussion on the
convergence of spectral methods will be detailed later on in this 
thesis.

To summarize, the NonLinear Harmonic
method applied to Eq.~\ref{eq:sm_nonlinear_convection_conservative},
gives $2N + 1$ equations. A pseudo-time ($\tau$) derivative is
added to march the equations in pseudo-time to the steady-state 
solutions of all the harmonics:
\begin{equation}
	\fbox{$
	\begin{cases}
		\displaystyle \frac{\partial \overline{u}}{\partial \tau} + 
		\frac{\partial}{\partial x}
			\left[\overline{u}^2 + 
			\overline{u^\prime u^\prime}\right] &=
			0, \\
		\displaystyle \frac{\partial \widehat{u}_k}{\partial \tau} + 
		i \omega k \widehat{u}_k + 
			\frac{\partial}{\partial x} 
			\left[ \overline{u} \widehat{u}_k\right] &= 
			0, \: k \in [-N, N], \: k \neq 0,
	\end{cases}
	$}
	\label{eq:sm_nlh_subset_eq}
\end{equation}
coupled by the deterministic stress term $\overline{u^\prime u^\prime}$
defined in Eq.~\ref{eq:sm_nlh_deterministic_stress_terms}.
Three hypothesis lies under these equations:
\begin{itemize}
	\item the unsteady perturbations are periodic in time
	with period $T= 2 \pi / \omega$, 
	meaning that they can be decomposed into a Fourier series,
	\item the high-order cross-coupling terms $u^\prime u^\prime$
	are neglected,
	\item the number of harmonics is set to $N$.
\end{itemize}

\paragraph{Multi-frequential formulation}

In \citet{He2002}, the method is extended to a multi-frequential
perturbation. Instead of writing them
using a Fourier series as defined in Eq.~\ref{eq:sm_nlh_decomposition_pert},
these are written using a sum of harmonics each of which
having a frequency $\omega_k$:
\begin{equation}
	u^\prime = \sum_{k=-N \atop k \neq 0}^{N} 
	\widehat{u}_k e^{i \omega_k t}.
	\label{eq:sm_nlh_decomposition_pert_multi}
\end{equation}
Note that the term $k \omega$ in Eq.~\ref{eq:sm_nlh_decomposition_pert}
is now $\omega_k$ meaning that frequencies can be chosen
arbitrarily.
The derivation of the equations is kept the same and the following
$2N+1$ subset of equations are given:
\begin{equation}
	\fbox{$
	\begin{cases}
		\displaystyle
		\frac{\partial \overline{u}}{\partial \tau} +
		\frac{\partial}{\partial x}
			\left[\overline{u}^2 + 
			\overline{u^\prime u^\prime}\right] &=
			0, \\
		\displaystyle
		\frac{\partial \widehat{u}_k}{\partial \tau} + 
		i \omega_k \widehat{u}_k + 
			\frac{\partial}{\partial x} 
			\left[ \overline{u} \widehat{u}_k\right] &= 
			0, \: k \in [-N, N], \: k \neq 0.
	\end{cases}
	$}
	\label{eq:sm_nlh_subset_eq_multi}
\end{equation}
However, as the complex exponentials do not form
an orthogonal basis, writing Eq.~\ref{eq:sm_nlh_subset_eq_multi}
for each harmonic $k \in [-N, N], \: k \neq 0$ is mathematically
not true. In \citet{He2002}, they argue that the terms
are collected for each harmonic. 
The same development is made by \citet{Vilmin2006}.

The coupling deterministic stress term is evaluated using the
same equation as for the mono-frequential formulation.
In the multi-frequential formulation, 
the equation Eq.~\ref{eq:sm_nlh_deterministic_stress_terms}
is generally not true.
In fact, in the mono-frequential formulation, the term
\begin{equation}
	\frac{1}{T} \int_{t=0}^{T} (\widehat{u}_k)^2
		e^{i 2 \omega k t} dt
\end{equation}
vanishes for each $k$ as the integral of the
exponential $e^{i 2 \omega k t}$ with respect to $t$
is given by $e^{i 2 \omega k t} / 2 i \omega k$ that is
periodic with period $T$. However, in the multi-frequential
formulation, for some choice of frequencies, the period of all
of these may be difficult or even impossible to define. It
seems that mathematical justifications should be given
to be able to evaluate the deterministic stress term 
using Eq.~\ref{eq:sm_nlh_deterministic_stress_terms}.

\paragraph{Clocking effects}
In \citet{He2002}, the nonlinear harmonic method is extended to
compute all clocking position in one computation. Before
go into details of how this is done, let us explain what is
the clocking effect.
\begin{figure}[htbp]
  \centering 
    \subfigure{\includegraphics[width=.3\textwidth]{CLOCKING_EFFECT_1.pdf}}
    \subfigure{\includegraphics[width=.3\textwidth]{CLOCKING_EFFECT_2.pdf}}
    \subfigure{\includegraphics[width=.3\textwidth]{CLOCKING_EFFECT_3.pdf}}
  \caption{Different clocking positions for a stator/rotor/stator
  configuration.}
  \label{fig:sm_nlh_clocking_effect}
\end{figure}
Fig.~\ref{fig:sm_nlh_clocking_effect} displays three subfigures showing three
different clocking position in a stator/rotor/stator configuration.
As both stator are fixed, their relative position is of 
prior interest. The wake that is generated behind the first stator
is cut by the rotor blades but however almost convected up to 
the stator row. The stator being fixed, the wake generated
behind the first stator is seen stationary by the second stator.
Hence, the importance of their relative position. Here, the
first clocking position does not provide unsteadiness to the
last stator while the third clocking position gives the highest
level of unsteadiness in the stator. \todo{papier qui justifie
l'importance du calcul du clocking}

The brute force to compute the clocking effect on a
configuration is to consider all relative positions. This means
that the geometry of the stator should be rotated for each new 
clocking position. The innovative thinking proposed in 
\citet{He2002} is to consider the clocking effect as a steady wave.
In fact, as both stator are fixed, a steady perturbation
generated behind the first stator is still steady in the second stator.
In terms of frequencies, a steady perturbation is a perturbations 
whose frequency is zero. In \citet{He2002} and \cite{Vilmin2009}, 
a perturbation with a very small frequency (close to machine precision)
is computed. The clocking effect can then be evaluated by
post-processing the Fourier coefficient of the zero frequency mode.

Recently, the clocking effects \citet{Vilmin2013a}
\todo{à cmopléter}

\paragraph{Extension to the Navier-Stokes equations}

\paragraph{Applications}
The NonLinear Harmonic method 
has been succesfully implemented
in the NUMECA-Fine/Turbo solver by 
\citet{Vilmin2006, Vilmin2007, Vilmin2009, Vilmin2013a}.\todo{a améliorer + ajout frozen turbulence}

\subsection{The \underline{N}on\underline{L}inear 
\underline{F}requency \underline{D}omain method (NLFD)}
\label{sub:sm_nlfd}
%!TEX root = ../../adrien_gomar_phd.tex

\subsection{Mono-frequential formulation}

Originally proposed by \citet{McMullen2001}, the NLFD
method relies on a simple observation: to develop spectral methods, and in
particular the NLH method, one has made use of the Fourier
series to efficiently represent an unsteady signal.
This representation is then used to compute the unsteady
equation by converting it into $2N+1$ steady equations coupled by a 
source term where $N$ is small.
However, the method needs to be resolved in the frequency domain meaning
that all the numerical techniques should be adapted: the numerical schemes,
the turbulent models and so on. The smart idea 
proposed by \citet{McMullen2001} is to
make use of the fast Fourier Transform and its inverse to
allow it to be easily implemented into a classical time-domain code.
To explain the development of this method, let us first 
write Eq.~\eqref{eq:sm_nonlinear_convection_conservative} 
in a more general form:
\begin{equation}
	(\ref{eq:sm_nonlinear_convection_conservative})
	\Leftrightarrow
	\frac{\partial u}{\partial t} + R = 0,
	\label{eq:sm_nonlinear_convection_residual}
\end{equation}
with
\begin{equation}
	R = \frac{\partial}{\partial x} \left( 
	\frac{u^2}{2} \right).
\end{equation}
Let us now consider that both $u$ and $R$ are periodic
in time with respect to period $T = 2 \pi / \omega$
and can be written using a Fourier series:
\begin{equation}
	\begin{split}
		u(t) &= \sum_{k=-\infty}^{\infty} \widehat{u}_k e^{i k \omega t}, \\
		R(t) &= \sum_{k=-\infty}^{\infty} \widehat{R}_k e^{i k \omega t}.
	\end{split}
\end{equation}
Note that decomposing $R(t)$ into a Fourier series is equivalent
to use the Fourier decomposition of $u(t)$ and express
$R(t)$ using the Fourier coefficients $\widehat{u}_k$ of $u$
since the cross-terms that may arise are also expressed 
using the same complex exponentials. This comes from the fact
that multiplying a complex exponential with a complex exponential
just forms a new complex exponential at the power of the sum of the
two.
Injecting these decompositions into 
Eq.~\eqref{eq:sm_nonlinear_convection_residual} and taking into account
for the orthogonality of the complex exponentials:
\begin{equation}
	i k \omega \widehat{u}_k + \widehat{R}_k = 0, \: k \in [-\infty, \infty].
\end{equation}
As previously, only a small number of harmonics $N$ is kept and 
a pseudo-time ($\tau$) derivative is added to march the equations
in pseudo-time to the steady-state solutions of all the harmonics:
\begin{equation}
	\fbox{$
	\displaystyle \frac{\partial \widehat{u}_k}{\partial \tau} + 
	i k \omega \widehat{u}_k + \widehat{R}_k = 0, \: k \in [-N, N].
	$}
	\label{eq:sm_nlfd_subset_eq}
\end{equation}
The fact that $R(t)$ is expressed using its own Fourier series 
makes it simpler to implement 
as it avoid developing its expression using 
the complex coefficients $\widehat{u}_k$. 
However, $\widehat{R}_k$ must be evaluated. To do so, as depicted
in Fig.~\ref{fig:nlfd_principle}, \citet{McMullen2001}
propose to use an Inverse Fast-Fourier Transform (IFFT) to get
$u(t)$ from $\widehat{u}_k$. Then the considered governing equations
are used to evaluate $R(t)$ which leads to $\widehat{R}_k$
through a Fast-Fourier Transform (FFT). Finally, the next iteration value 
$\widehat{u}_k$
is evaluated by adding $\widehat{R}_k$ and 
the corresponding temporal derivative $i k \omega \widehat{u}_k$. All
harmonics are coupled through the IFFT and FFT operations
that needs all of the former to compute the counterpart temporal signal,
hence the coupling. Moreover, 
in the non-viscous Burger's equation framework, 
the term $u^\prime u^\prime$ is not neglected anymore compared to the
NLH approach and the computation of the deterministic stress is encompass
by the FFT and IFFT operations.
\begin{figure}[htbp]
  \centering
  \includegraphics*[width=0.50\textwidth]{nlfd_principle.pdf}
  \caption{Simplified diagram of the computation of $\widehat{R}_k$ from $\widehat{u}_k$
  for the non-linear frequency domain method.}
  \label{fig:nlfd_principle}
\end{figure}

\subsection{Extension to the Navier-Stokes equations}
The Navier-Stokes equations can be written in finite-volume,
semi-discrete form as:
\begin{equation}
	V \frac{\partial W}{\partial t} + R(W) = 0,
	\label{eq:navier_stokes_fv_sd}
\end{equation}
where $V$ is the volume of the cell and $W$
the vector of conservative variables.
This formulation is similar to
Eq.~\eqref{eq:sm_nonlinear_convection_residual} meaning that
nothing particular has to be made to derive this approach for
the Navier-Stokes equations. This is indeed attractive as the
method can be applied almost directly, except for the FFT and IFFT
step that should be added into the pseudo time loop.

\subsection{Time line}
\begin{figure}[htbp]
  \centering
  \includegraphics*[scale=0.6]{timeline_nlfd.pdf}
  \caption{Time line for the non-linear frequency domain method.}
  \label{fig:timeline_nlfd}
\end{figure}
Figure~\ref{fig:timeline_nlfd} shows the time line of the
development of the NLFD method.
The method was first introduced by \citet{McMullen2001}
and applied to the two-dimensional
Navier-Stokes equations. It has been validated against an
unsteady channel flow that admits an analytical 
solution~\cite{Merkle1987} and applied to a cylinder
vortex shedding. This could be done as the frequency of the
vortex shedding was known \emph{a priori} from experimental
and numerical data. However, for a given cylinder, it is not
possible to know this frequency. This is why
 \citet{McMullen2002, McMullen2006a}
proposed a gradient based method for determining the frequency
of a periodic phenomena where the frequency is unknown
\emph{a priori}. They argue that the frequency domain formulation
helps forming a gradient operator to find the period $T$ based
on the minimization of the residual of the unsteady equations.
They applied their algorithm to find the frequency of the vortex
shedding around a cylinder, and found it with a $3.5\%$ accuracy
compared to experimental data.
\citet{Nadarajah2003} compared an optimum shape design 
strategy for 
airfoils undergoing a change 
in angle of attack as a function of time 
using both a classical time-marching scheme
and the NLFD scheme within the Euler equations
framework. It is shown that the NLFD method
gives the same accuracy for the gradient and the optimum with only 
three time-instants ($N=1$)
compared to $23$ time-instants needed for 
the time-marching approach.
\citet{McMullen2006} evaluated all the acceleration techniques
that can be used with the NLFD method such as multigrid, local
time-stepping, residual averaging, coarse grid spectral viscosity,
proving that a steady-state convergence rate can be obtained.
Moreover, they compared the efficiency of the
NLFD method to a classical time-marching
scheme and showed that, for the same accuracy of the
solution, a one order to two orders of magnitude gain can
be obtained using the NLFD approach on a pitching airfoil.
\citet{Nadarajah2007} extended their optimum shape design strategy
using the NLFD to the three-dimensional Navier-Stokes equations.
A wing undergoing a change 
in angle of attack as a function of time is computed and
it is demonstrated that
five instants (namely two harmonics) is sufficient to provide
accurate results.
\citet{Kachra2008} extended the approach to the strong coupling of
aero-elasticity within the two-dimensional Euler equations framework.
They demonstrate that with a one-harmonic NLFD computation, the
flutter boundary of a NACA$64A010$ airfoil is correctly predicted.
This leads to a gain of one order of magnitude compared to a classical
time-marching procedure. Both the fluid dynamics and the structural equations
are solved using the NLFD approach. Moreover they quantitatively estimate the
cost of the FFT and IFFT done at each time-loop of the NLFD approach to be
approximately $2\%$ of the cost of one iteration, which is negligible.
\citet{Tatossian2011} extended the approach of \citet{Nadarajah2007}
to the aerodynamic shape optimization of hovering rotor blades
in the Euler framework.
First the NLFD approach is validated
against the Caradonna–Tung experimental blade.
The capability of 
their shape optimization process
to redesign this blade is assessed and gives a proof
of concept.
\citet{Mosahebi2013} implemented an adaptive NLFD approach named
the p-NLFD. Based on the energy of the last mode compared
to the whole spectrum, the number of harmonics
is increased if a fixed threshold is not reached.
A speed-up of $2$ in terms of CPU and
memory reduction is observed for the case of a
vortex-shedding behind a cylinder.




\subsection{Cost of the method}
The NLFD method is close to the NLH approach in terms of number
of equation solved. However, at each time-step a fast Fourier transform
is performed to cast back the harmonics into the time-domain in order
to compute the residual $R(t)$. \citet{McMullen2006} argue
that the cost of the fast Fourier transform is less than the cost of 
the spatial derivatives, and is thus negligible. This is 
confirmed by \citet{Kachra2008}.
Based on this affirmation, one can say that equivalent to the NLH
approach, if $\mathdollar_{\text{RANS}}$ 
denotes the CPU and memory cost of
one steady computation, the cost of the NLFD method can be 
approximated by:
\begin{equation}
	\mathdollar_{\text{NLFD}} = (2N+1) \cdot \mathdollar_{\text{RANS}}.
\end{equation}
This evaluation of the cost is confirmed by \citet{McMullen2002}.

% ====================
% = HARMONIC BALANCE =
% ====================
\section{The \underline{H}armonic \underline{B}alance method (HB)}
\label{sec:sm_hb}
%!TEX root = ../../adrien_gomar_phd.tex

The Harmonic Balance Technique proposed by \citet{Hall2002}
is a step further the non-linear frequency domain method. Instead
of using the fast Fourier transform to cast back the equations
to the time domain at each pseudo-iteration step, 
the equations are mathematically derived to 
computed into the time-domain.
To explain the method, we will use the general form of 
the viscous Burger's equation defined in 
Eq.~\ref{eq:sm_nonlinear_convection_residual}.

\subsection{Mono-frequential formulation}

Following the same approach as the non-linear frequency domain approach,
one consider that both $u$ and $R$ are periodic
in time with respect to period $T = 2 \pi / \omega$
and can be written using a Fourier series:
\begin{equation}
	\begin{split}
		u &= \sum_{k=-\infty}^{\infty} \widehat{u}_k e^{i k \omega t} \\
		R(t) &= \sum_{k=-\infty}^{\infty} \widehat{R}_k e^{i k \omega t}
		\label{eq:sm_hall_dft}
	\end{split}
\end{equation}
Injecting Eq.~\ref{eq:sm_hall_dft} in 
Eq.~\ref{eq:sm_nonlinear_convection_residual}, and considering
the orthogonality of the the complex exponentials:
\begin{equation}
	i k \omega \widehat{u}_k + \widehat{R}_k = 0, \: k \in [-N, N].
	\label{eq:sm_hall_frequential_eq}
\end{equation}

In the same way as one uses Fourier coefficients to
evaluate the temporal signal,
one can reconstruct the Fourier coefficients using
temporal evaluations taken at evenly spaced timelevels
sampling the period $T = 2 \pi / \omega$ using the forward
Fourier transform. Moreover, 
according to the Nyquist-Shannon~\cite{Shannon1949} sampling theorem, 
at least $2N+1$ timelevels are needed to capture $N$ frequencies,
leading to:
\begin{equation}
	\widehat{u}_k = \frac{1}{2N+1} 
	\sum_{n=0}^{2N} u_n^\star e^{-i k \omega t_n}.
\end{equation}
If $E$ denotes the matrix composed of the elements 
$E_{k,n} = e^{-i k \omega t_n} / 2N+1$, one can write $\widehat{u}_k$
and $\widehat{R}_k$ as:
\begin{equation}
	\begin{split}
		\widehat{u}_k &= E u^\star \\
		\widehat{R}_k &= E R^\star,
	\end{split}
	\label{eq:sm_matrix_fourier_operator}
\end{equation}
where $u^\star$ and $R^\star$ 
denote the vectors formed of all the evaluations of respectively $u$
and $R$,
made at $2N+1$ timelevels uniformly sampling the period of interest. 
$E$ can thus be named the inverse Fourier operator.
Note that conversely, using the Fourier operator $E^{-1}$:
\begin{equation}
	\begin{split}
		u^\star &= E^{-1} \widehat{u}_k \\
		R^\star &= E^{-1} \widehat{R}_k.
	\end{split}
\end{equation}

Injecting the matrix formulation of 
Eq.~\ref{eq:sm_matrix_fourier_operator} in 
Eq.~\ref{eq:sm_hall_frequential_eq}
gives:
\begin{equation}
	i K \omega E u^\star + E R^\star = 0,
\end{equation}
where $K$ is a diagonal matrix formed of all the $k \in [-N, N]$.
Now multiplying the equation by the Fourier operator $E^{-1}$:
\begin{equation}
	i \omega E^{-1} K E u^\star + R^\star = 0,
\end{equation}
where $R^\star$ can now be substituted:
\begin{equation}
	\fbox{$
		i \omega E^{-1} K E u^\star + 
		\displaystyle \frac{\partial}{\partial x}
		\frac{(u^\star)^2}{2} = 0
	$}
\end{equation}
What happened here is that instead of developing $R(t)$
in the frequency domain, which is tedious, this term is kept
as it is through all the development process. 
Since $R(t)$ only includes spatial derivatives, no non-linear
terms
rise from the Fourier decomposition. Hence, multiplying it
by the Fourier operator leads to the unity matrix. 
Finally $R(t)$ is simply evaluated at $2N+1$ timelevels.

This approach is really close to the non-linear frequency domain method.
The reader might observe that we are introducing the discrete Fourier
transform and its inverse. This is close to the fast Fourier transform
and it's inverse, as proposed by \citet{McMullen2001}. However,
as the development is on the equations and not during the time loop,
we get $2N+1$ steady equations that are coupled by a source term.
This source term appears as a spectral operator defined as:
\begin{equation}
	D_t = \frac{\partial E^{-1}}{\partial t} E = i \omega E^{-1} K E.
	\label{eq:sm_hb_mono_source_term_matrix}
\end{equation}
The main difference with the nonlinear frequency domain approach
is that the source term is known at the very first iteration and does
not change, meaning that we do not spend time computing a costly
fast fourier transform and its inverse at each time-step.

\citet{Gopinath2005} provide an analytical formulation of the
source term defined in Eq.~\ref{eq:sm_hb_mono_source_term_matrix}.
It a matrix operator whose elements are defined as:
\begin{equation}
  (D_t)_{l, j} =
  \begin{cases}
    \frac{\pi}{T}(-1)^{l-j}\csc\left(\frac{\pi
        (l-j)}{2N+1}\right) &, \, l\neq j,\\
    0 &, \, l=j.
  \end{cases}
\end{equation}


% ===================================
% = Periodic flows in turbomachines =
% ===================================
\section{Periodic flows in turbomachines}
\label{sec:sm_hudson}
%!TEX root = ../../adrien_gomar_phd.tex

As said in the introduction, the relative
velocity between two consecutive rows induces
unsteadinesses that can be correlated with the 
Blade Passing Frequency (BPF). A 

\chconclu{Four spectral methods have been presented in this chapter.
The main mathematical development have been demonstrated and 
the weakness/hypothesis of each method has been highlighted.
The large literature available on these methods shows that
they are ready for industrial, numerically hard
unsteady applications.}
