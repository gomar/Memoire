%!TEX root = ../../adrien_gomar_phd.tex

\paragraph{Mono-frequential formulation}

As seen before on Sec.~\ref{sub:sm_nlh}, the expansion of 
a non-linear unsteady equation in the frequency domain
can be tedious. This is even more true when the
considered equations are the Navier-Stokes equations
or any large set of non-linear equations. In
fact, the numerical schemes, the turbulence model and so
on, must be derived in the frequency domain which might be
hard to do. The NonLinear Frequency Domain approach is
proposed by \citet{McMullen2001} to overcome this issue.

To explain the development of this method, let us first 
write Eq.~\ref{eq:sm_nonlinear_convection_conservative} 
in a more general form:
\begin{equation}
	\frac{\partial u}{\partial t} + R (t) = 0,
	\label{eq:sm_nonlinear_convection_residual}
\end{equation}
with
\begin{equation}
	R(t) = \frac{\partial}{\partial x} \left( 
	\frac{u^2}{2} \right).
\end{equation}
Let us now consider that both $u$ and $R$ are periodic
in time with respect to period $T = 2 \pi / \omega$
and can be written using a Fourier series:
\begin{equation}
	\begin{split}
		u &= \sum_{k=-\infty}^{\infty} \widehat{u}_k e^{i k \omega t} \\
		R(t) &= \sum_{k=-\infty}^{\infty} \widehat{R}_k e^{i k \omega t}
	\end{split}
\end{equation}
Injecting these decompositions into 
Eq.~\ref{eq:sm_nonlinear_convection_residual} and taking into account
for the orthogonality of the complex exponentials leads to:
\begin{equation}
	i k \omega \widehat{u}_k + R_k = 0, \: k \in [-\infty, \infty]
\end{equation}
As previously, only a small number of harmonic $N$ is kept and 
a pseudo-time ($\tau$) derivative is added to march the equations
in pseudo-time to the steady-state solutions of all the harmonics:
\begin{equation}
	\fbox{$
	\displaystyle \frac{\partial \widehat{u}_k}{\partial \tau} + 
	i k \omega \widehat{u}_k + \widehat{R}_k = 0, \: k \in [-N, N]
	$}
	\label{eq:sm_nlfd_subset_eq}
\end{equation}
The fact that $R(t)$ is expressed using its own Fourier series 
makes it simpler to implement 
as it avoid developing its expression using $\widehat{u}_k^j$.
However, $\widehat{R}_k$ must be evaluated. To do so, as depicted
in Fig.~\ref{fig:nlfd_principle}, \citet{McMullen2001}
propose to use an Inverse Fast-Fourier Transform (IFFT) to get
$u(t)$ from $\widehat{u}_k^j$. Then the considered governing equation
is used to evaluate $R(t)$ which leads to $\widehat{R}_k$
through a Fast-Fourier Transform (FFT). Finally, $\widehat{u}_k^{j+1}$
is evaluated by ading the corresponding temporal derivative. All
harmonics are coupled through the IFFT and FFT operations
that needs all of them to compute the temporal signal.
\begin{figure}[htbp]
  \centering
  \includegraphics*[width=0.50\textwidth]{nlfd_principle.pdf}
  \caption{The computation of $\widehat{R}_k$ from $\widehat{u}_k$
  for the NonLinear Frequency Domain method}
  \label{fig:nlfd_principle}
\end{figure}

\paragraph{Extension to the Navier-Stokes equations}
The Navier-Stokes equations can be written in finite-volume,
semi-discrete form as:
\begin{equation}
	V \frac{\partial W}{\partial t} + R(W) = 0.
	\label{eq:navier_stokes_fv_sd}
\end{equation}
This formulation is exactly the same as the one of 
Eq.~\ref{eq:sm_nonlinear_convection_residual} meaning that
nothing particular as to be made to apply this method to
the Navier-Stokes equations. This is indeed very attractive as the
method can be applied almost directly, except for the FFT and IFFT
part that should be added into the time loop.

\paragraph{Applications}

\paragraph{Time line}
\begin{figure}[htbp]
  \centering
  \includegraphics*[width=0.40\textwidth]{timeline_nlfd.pdf}
  \caption{Time line for the NonLinear Frequency Domain method}
  \label{fig:timeline_nlfd}
\end{figure}
Figure~\ref{fig:timeline_nlfd} shows the time line of the
development of the NonLinear Frequency Domain method.
The method was first introduced by \citet{McMullen2001}
at Stanford University and applied on the two-dimensional
Navier-Stokes equations. It has been validated against an
unsteady channel flow that admits an analytical 
solution~\cite{Merkle1987} and applied to a cylinder
vortex shedding. This could be done as the frequency of the
vortex shedding was known \textit{a priori} from experimental
and numerical data. However, for a given cylinder, it is not
possible to know this frequency. This is why \citet{McMullen2002}
proposed a gradient based method for determining the period.
They applied their algorithm to find the frequency of the vortex
shedding around a cylinder.

\paragraph{Cost of the method}
The method 