%!TEX root = ../../adrien_gomar_phd.tex

\begin{figure}[htbp]
  \centering
  \includegraphics*[width=0.40\textwidth]{timeline_nlfd.pdf}
  \caption{Time line for the NonLinear Frequency Domain method}
  \label{fig:timeline_nlfd}
\end{figure}
Figure~\ref{fig:timeline_nlfd} shows the time line of the
development of the NonLinear Frequency Domain method.
The method was first introduced by \citet{McMullen2001}
at Stanford University on the two-dimensional 
Navier-Stokes equations.

\paragraph{Mono-frequential formulation}

As seen before on Sec.~\ref{sub:sm_nlh}, the expansion of 
a non-linear unsteady equation in the frequency domain
can be tedious. This is even more true when the
equations considered are the Navier-Stokes equations. In
fact, the numerical schemes, the turbulence model and so
on, must be derived in the frequency domain which might be
hard. To overcome this issue, \citet{McMullen2001} propose
the NonLinear Frequency Domain approach.

To develop this method, let us first 
write Eq.~\ref{eq:sm_nonlinear_convection_conservative} 
in a more general form:
\begin{equation}
	\frac{\partial u}{\partial t} + R (t) = 0,
	\label{eq:sm_nonlinear_convection_residual}
\end{equation}
with
\begin{equation}
	R(t) = \frac{\partial}{\partial x} \left( 
	\frac{u^2}{2} \right).
\end{equation}
Let us now consider that both $u$ and $R(t)$ are periodic
in time and can be written using a Fourier series:
\begin{equation}
	\begin{split}
		u &= \sum_{k=-\infty}^{\infty} \widehat{u}_k e^{i k \omega t} \\
		R(t) &= \sum_{k=-\infty}^{\infty} \widehat{R}_k e^{i k \omega t}
	\end{split}
\end{equation}
Injecting these decompositions into 
Eq.~\ref{eq:sm_nonlinear_convection_residual} and taking into account
for the orthogonality of the complex exponentials leads to:
\begin{equation}
	i k \omega \widehat{u}_k + R_k = 0, \: k \in [-\infty, \infty]
\end{equation}
As previously, only a small number of harmonic $N$ is kept.
Finally:
\begin{equation}
	\fbox{$
	i k \omega \widehat{u}_k + \widehat{R}_k = 0, \: k \in [-N, N]
	$}
	\label{eq:sm_nlfd_subset_eq}
\end{equation}
The fact that $R(t)$ is expressed using its own Fourier series is 
simpler as it avoid developing its expression using $\widehat{u}_k$.
However, $\widehat{R}_k$ must be evaluated. To do so, as depicted
in Fig.~\ref{fig:nlfd_principle}, \citet{McMullen2001}
propose to use an Inverse Fast-Fourier Transform (IFFT) to get
$u(t)$ from $\widehat{u}_k$. Then the considered governing equation
is used to evaluate $R(t)$ which leads to $\widehat{R}_k$
through a Fast-Fourier Transform (FFT). Finally, $\widehat{u}_k^{j+1}$
is evaluated by ading the corresponding temporal derivative. All
harmonics are coupled through the IFFT and FFT operations
that needs all of them to compute the temporal signal.
\begin{figure}[htbp]
  \centering
  \includegraphics*[width=0.50\textwidth]{nlfd_principle.pdf}
  \caption{The computation of $\widehat{R}_k$ from $\widehat{u}_k$
  for the NonLinear Frequency Domain method}
  \label{fig:nlfd_principle_1}
\end{figure}

\paragraph{Extension to the Navier-Stokes equations}
The Navier-Stokes equations can be written in finite-volume,
semi-discrete form as:
\begin{equation}
	V \frac{\partial W}{\partial t} + R(W) = 0.
	\label{eq:navier_stokes_fv_sd}
\end{equation}
This formulation is exactly the same as the one of 
Eq.~\ref{eq:sm_nonlinear_convection_residual} meaning that
nothing particular as to be made to apply this method to
the Navier-Stokes equations. This is indeed very attractive as the
method can be applied almost directly, except for the FFT and IFFT
part that should be added in the time loop.

\paragraph{Cost of the method}
The method 

\paragraph{Applications}
