%!TEX root = ../../adrien_gomar_phd.tex

Originally developed by \citet{He1998} and \citet{Ning1998},
the NLH method
relies on a decomposition of the conservative variables into a
time-averaged part plus an unsteady perturbation:
\begin{equation}
	u = \overline{u} + u^\prime,
	\label{eq:sm_nlh_decomposition}
\end{equation}
where $\overline{.}$ denotes the time-averaging operator and
$.^\prime$ its unsteady perturbation counterpart.
By injecting Eq.~\ref{eq:sm_nlh_decomposition} into
Eq.~\ref{eq:sm_nonlinear_convection_conservative}, one gets:
\begin{equation}
	\frac{\partial u^\prime}{\partial t} + 
	\frac{1}{2}\frac{\partial}{\partial x} \left[
	\overline{u}^2 + 2 \overline{u} u^\prime + u^\prime u^\prime \right] = 
	0.
	\label{eq:sm_nlh_step_1}
\end{equation}
The time-averaged equation can be obtained by time-averaging
equation~\ref{eq:sm_nlh_step_1}:
\begin{equation}
	(\overline{\ref{eq:sm_nlh_step_1}})
	\Leftrightarrow
	\frac{\partial}{\partial x}
	\left[\overline{u}^2 + 
	\overline{u^\prime u^\prime}\right] =
	0,
	\label{eq:sm_nlh_step_2}
\end{equation}
The term $\overline{u^\prime u^\prime}$
appears due to the non-linearities of the considered equation. It
is called the nonlinear 
(or the deterministic) stress terms as a reference to 
the Reynolds stress terms. 
The equation for the unsteady perturbation is then obtained by keeping
the first order terms of the unsteady equation~\ref{eq:sm_nlh_step_1}.
This means that the term $u^\prime u^\prime$ is neglected and leads
to:
\begin{equation}
	\frac{\partial u^\prime}{\partial t} + 
	\frac{\partial}{\partial x} \left[\overline{u} u^\prime \right] = 
	0.
\end{equation}

\paragraph{Mono-frequential formulation}
For now on, no assumption has been made neither on the velocity $u$,
nor on its time-averaged part and unsteady perturbation part.
Now, assuming that the velocity perturbation 
is periodic in time with period
$T=2 \pi / \omega$,
the unsteady perturbation can be decomposed into 
a Fourier series:
\begin{equation}
	u^\prime = \sum_{k=-\infty \atop k \neq 0}^{\infty} 
	\widehat{u}_k e^{i \omega k t}.
	\label{eq:sm_nlh_decomposition_pert}
\end{equation}
Since the complex exponentials family forms 
an orthogonal basis, we have for all harmonics 
$-\infty \leq k \leq \infty, \; k \neq 0$:
\begin{equation}
	i \omega k \widehat{u}_k + 
	\frac{\partial}{\partial x} \left[ \overline{u} \widehat{u}_k\right] =
	0.
\end{equation}
One can notice that the time-averaged part has been removed from
the Fourier series through $k \neq 0$ as it is computed 
separately in Eq.~\ref{eq:sm_nlh_step_2}.
Each harmonic equation represents now a steady equation as no temporal
derivative is present anymore.

The term $\overline{u^\prime u^\prime}$ remains in the time-averaged
equation and needs to be computed. It can be 
directly worked out when the harmonics are known:
\begin{equation}
	\begin{split}
		u^\prime u^\prime &= 
		\left[
			\sum_{k=-\infty \atop k \neq 0}^{\infty} \widehat{u}_k e^{i \omega k t} 
		\right]
		\left[
			\sum_{k=-\infty \atop k \neq 0}^{\infty} \widehat{u}_k e^{i \omega k t} 
		\right] \\
		&= \sum_{k=-\infty \atop k \neq 0}^{\infty} (\widehat{u}_k)^2
		   e^{i 2 \omega k t} +
		   2 \sum_{j,k=-\infty \atop j \neq k \neq 0}^{\infty} 
		   \widehat{u}_k \widehat{u}_j e^{i \omega (k + j) t}.
	\end{split}
\end{equation}
Thus,
\begin{equation}
	\begin{split}
		\overline{u^\prime u^\prime} &= 
		\frac{1}{T} \int_{t=0}^{T} \left[ 
			\sum_{k=-\infty \atop k \neq 0}^{\infty} (\widehat{u}_k)^2
		   	e^{i 2 \omega k t} +
		   	2 \sum_{j,k=-\infty \atop j \neq k \neq 0}^{\infty} 
		   	\widehat{u}_k \widehat{u}_j e^{i \omega (k + j) t} 
		\right] dt\\
		&= \frac{2}{T} \int_{t=0}^{T} \sum_{j,k=-\infty \atop j \neq k \neq 0}^{\infty} 
		   	\widehat{u}_k \widehat{u}_j 
		   	e^{i \omega (k + j) t} dt \\
		&= \frac{2}{T} \int_{t=0}^{T} 
			\sum_{k=-\infty \atop k \neq 0}^{\infty} 
			\widehat{u}_k \widehat{u}_{-k}  dt.
	\end{split}
\end{equation}
As $\widehat{u}_k$ and $\widehat{u}_{-k}$ are complex conjugates,
finally $\overline{u^\prime u^\prime}$ is equal to:
\begin{equation}
	\overline{u^\prime u^\prime} = 
	2 \sum_{k=-\infty \atop k \neq 0}^{\infty} |\widehat{u}_k|^2.
	\label{eq:sm_nlh_deterministic_stress_terms}
\end{equation}
This last equation depends only on the computed harmonics, meaning
that no term is modeled. Moreover, this term couples the
time-average solution with the unsteady perturbation. This is this
term that is neglected in the linearized method seen in 
Sec.~\ref{sub:sm_lur}. It takes into account for the 
non-linearity of the considered equation.

Finally, as computing an infinite number of harmonics is 
numerically not feasible,
it is truncated at order $N$. 
This is a fare assumption as most
of the physical flows have a finite unsteady spectrum. This
is for sure a reduce order approach. However, the goal of spectral
methods is to have a compact representation of the unsteady time
signals. As for a mesh grid convergence, the number of harmonics $N$
is increased until the unsteady representation of the signal is
converged for the variable of interest. The discussion on the
convergence of spectral methods will be detailed later on in this 
thesis \todo{ref chapitre}.

To summarize, the NLH
method applied to Eq.~\ref{eq:sm_nonlinear_convection_conservative},
gives $2N$ perturbation equations and one time
averaged equation making $2N+1$ equations in total. 
A pseudo-time ($\tau$) derivative is
added to march the equations in pseudo-time to the steady-state 
solution of all the harmonics:
\begin{equation}
	\fbox{$
	\begin{cases}
		\displaystyle \frac{\partial \overline{u}}{\partial \tau} + 
		\frac{\partial}{\partial x}
			\left[\overline{u}^2 + 
			\overline{u^\prime u^\prime}\right] &=
			0, \\
		\displaystyle \frac{\partial \widehat{u}_k}{\partial \tau} + 
		i \omega k \widehat{u}_k + 
			\frac{\partial}{\partial x} 
			\left[ \overline{u} \widehat{u}_k\right] &= 
			0, \: k \in [-N, N], \: k \neq 0.
	\end{cases}
	$}
	\label{eq:sm_nlh_subset_eq}
\end{equation}
The equations are coupled by the deterministic 
stress term $\overline{u^\prime u^\prime}$
defined in Eq.~\ref{eq:sm_nlh_deterministic_stress_terms}.
The term $u^\prime u^\prime$ is neglected in this formulation.

\paragraph{Multi-frequential formulation}

\citet{He2002} extended the method to a multi-frequential
formulation. Instead of writing the perturbation
using a Fourier series as defined in Eq.~\ref{eq:sm_nlh_decomposition_pert},
these are written using a sum of harmonics each of which
having a frequency $\omega_k$:
\begin{equation}
	u^\prime = \sum_{k=-N \atop k \neq 0}^{N} 
	\widehat{u}_k e^{i \omega_k t}.
	\label{eq:sm_nlh_decomposition_pert_multi}
\end{equation}
Note that the term $k \omega$ in Eq.~\ref{eq:sm_nlh_decomposition_pert}
is now $\omega_k$ meaning that the frequencies can be chosen
arbitrarily.
The derivation of the equations is kept the same and the following
$2N+1$ subset of equations is finally obtained:
\begin{equation}
	\fbox{$
	\begin{cases}
		\displaystyle
		\frac{\partial \overline{u}}{\partial \tau} +
		\frac{\partial}{\partial x}
			\left[\overline{u}^2 + 
			\overline{u^\prime u^\prime}\right] &=
			0, \\
		\displaystyle
		\frac{\partial \widehat{u}_k}{\partial \tau} + 
		i \omega_k \widehat{u}_k + 
			\frac{\partial}{\partial x} 
			\left[ \overline{u} \widehat{u}_k\right] &= 
			0, \: k \in [-N, N], \: k \neq 0.
	\end{cases}
	$}
	\label{eq:sm_nlh_subset_eq_multi}
\end{equation}
Note that the difference with the mono-frequential formulation
is thin.
However, as the complex exponentials do not form
an orthogonal basis, writing Eq.~\ref{eq:sm_nlh_subset_eq_multi}
for each harmonic $k \in [-N, N], \: k \neq 0$ is mathematically
not true. \citet{He2002} argued that the terms
are collected for each harmonic. 
The same development is made by \citet{Vilmin2006}.

The coupling deterministic stress term is evaluated using the
same equation as for the mono-frequential formulation.
However, in the multi-frequential formulation, 
the equation Eq.~\ref{eq:sm_nlh_deterministic_stress_terms}
is generally not true.
In fact, in the mono-frequential formulation, the term
\begin{equation}
	\frac{1}{T} \int_{t=0}^{T} (\widehat{u}_k)^2
		e^{i 2 \omega k t} dt.
	\label{eq:sm_nlh_int_deterministic}
\end{equation}
vanishes for each $k$ as the integral of the
exponential $e^{i 2 \omega k t}$ with respect to $t$
is given by $e^{i 2 \omega k t} / i 2 \omega k$ that is
periodic with period $T$ meaning that the integral in 
Eq.~\ref{eq:sm_nlh_int_deterministic} is equal to zero
for the mono-frequential formulation. 
However, in the multi-frequential
formulation, for some choice of frequencies, the period of all
of these may be difficult or even impossible to define. It
seems that mathematical justifications should be given
to be able to evaluate the deterministic stress term 
using Eq.~\ref{eq:sm_nlh_deterministic_stress_terms}.

\paragraph{Clocking effects}
\citet{He2002} extended the nonlinear harmonic method to
the computation of all clocking positions in one computation. Before
go into details of how this is done, let us explain what is
the clocking effect, sometimes also referred to as the indexing effect.
\begin{figure}[htbp]
  \centering 
  \includegraphics[width=.5\textwidth]{clocking_effect.pdf}
  \caption{Different clocking positions for a stator/rotor/stator
  configuration.}
  \label{fig:sm_nlh_clocking_effect}
\end{figure}
Fig.~\ref{fig:sm_nlh_clocking_effect} shows six
different clocking positions of the first stator
in a stator/rotor/stator configuration.
As both stator are fixed, their relative position is of 
prior interest. In fact, the wake that is generated behind the first stator
is cut by the rotor blades and transmitted to 
the second stator row. The stators being fixed, the wake generated
behind the first stator is seen as a stationary wave in the second stator.
Hence, the importance of their relative position. For instance,
\citet{Huber1996} showed that
on their 1.5 turbine stage, the variation of efficiency due to clocking
position was equal to $0.8\%$ of efficiency, showing the
importance of the clocking effect.

The brute force to compute the clocking effect on a
configuration is to consider all relative positions. This means
that the geometry of the stator should be rotated for each new 
clocking position and hence a new unsteady computation should be 
run. The innovative procedure proposed by 
\citet{He2002} is to consider the clocking effect as a steady wave.
In fact, as both stator are fixed, a steady perturbation
generated behind the first stator is still steady in the second stator.
In terms of frequencies, a steady perturbation
can be assimilated to a zero frequency mode. 
In \citet{He2002} and \citet{Vilmin2009}, 
a perturbation with a zero frequency
is thus additionally computed. The clocking effect can then be evaluated by
post-processing the Fourier coefficient of the zero frequency mode.
Recently, the computation of clocking effects on
arbitrary configurations has been made possible
by \citet{Vilmin2013a}.

\paragraph{Extension to the Navier-Stokes equations}
As shown above, as the development of the NLH
method is made in the frequency domain, applying the method to
complex equations can be difficult. For the Navier-Stokes equations,
this step is particularly hard. Nevertheless, several authors have
done this and the reader is referred to the following papers
for more information~\cite{He1998, Chen2001, He2002, Vilmin2006}.

\paragraph{Time line}
\begin{figure}[htbp]
  \centering
  \includegraphics*[scale=0.6]{timeline_nlh.pdf}
  \caption{Time line for the non-linear harmonic method.}
  \label{fig:timeline_nlh}
\end{figure}
Figure~\ref{fig:timeline_nlh} shows the time line of the
development of the NLH method. 
Remember that black filled
rectangles highlight multi-frequential formulation papers.
\citet{He1998}
and \citet{Ning1998} first introduced the method.
\citet{He1998} developed the method for the 
two-dimensional Reynolds-Averaged Navier-Stokes equations. 
While only one harmonic is
kept in the applications, the method is presented for multiple
harmonics. It is validated on an unsteady boundary layer
on a flat plate, a transonic diffuser with oscillating back pressure
and on an oscillating transonic compressor cascade.
The results with only one-harmonic ($N=1$) give a
much better agreement that steady method does compared
to a classical time-marching scheme. However these results
are not superimposed to the time-marching ones.
\citet{Ning1998} developed the method on the two-dimensional
Euler equations with moving grids 
and applied it on a transonic unsteady
channel flow and on two vibrating cascade test cases
showing a good agreement with a classical non-linear
time-marching scheme. Emphasis is put on the capability
of the non-linear harmonic method to correctly capture
the unsteadiness in presence of strong non-linearities
compared to linearized methods (presented in 
Sec.~\ref{sub:sm_lur}).
\citet{Chen2001} extended the method to the three-dimensional
Navier-Stokes equations. Stage configurations are treated and
at the interface of the stage, the perturbation is exchanged using
azimuthal Fourier transform and in opposite,
the time-average field and the deterministic stresses
are flux-averaged like in a mixing-plane approach.
Still the method is mono-frequential as only
one stage is considered, each row seeing the
opposite blade passing frequency. The method is
finally applied to a flat plat and a $2$D/$3$D
DLR compressor stage. $N=5$ harmonics are needed
to accurately capture the wake at the interface of
the compressor stage.
\citet{He2002} extended the method
to take into account for the clocking effects. The frequencies
can be chosen arbitrarily allowing its application on multi-stage
turbomachines. The clocking effect is considered as a zero-frequency
temporal wave which is actually a spatial wave. The clocking
effect is then just analyzed through a post-processing procedure.
The method is applied on a $2.5$-stage transonic
compressor. The results are analyzed in terms of clocking effects
but not compared to neither classical time marching 
computations nor analytical simulations.
\citet{Vilmin2006} implemented the NLH method into
the commercial code Fine/Turbo. They extended the rotor-stator
interface to a non-matching join sliding mesh interface which
leads to the continuity of the unsteady flow field at the interface.
The method is validated on analytical test cases 
(oscillating $1$D flow, oscillating $2$D Couette flow and
the unsteady boundary layer test case of \citet{He1998}), 
on the $2$D DLR compressor stage used by \citet{Chen2001}
and on a $3$D industrial radial turbine.
Finally
a $4$-stage industrial transonic compressor is simulated to prove the 
maturity of the method. The effect of mesh refinement is highlighted
on the convergence of the results for a fixed number of harmonic
NLH computations. They show that approximately $40$ grid points are 
needed per wavelength to accurately capture them using the NLH approach.
\citet{Vilmin2007} extended the method to thermally perfect gas and
applied it on the $3$D industrial radial turbine they used
earlier~\cite{Vilmin2007} and compared their
results to a classical time-marching simulation and to the previous
NLH computations made within a perfect gas framework.
\citet{Vilmin2009} implemented and validated the clocking computation method
proposed by \citet{He2002}. They validated their approach on a model problem
(basically an annular cylindrical duct split in two domains with an
input tangential distortion)
and applied it
on a $2$D $6.5$-stage compressor and the Aachen $3$D $1.5$ 
stage axial turbine.
\citet{Vilmin2013a} extended the row interface to take into
account in the third rotor for perturbation 
coming from the first rotor in an
arbitrary rotor/rotor/rotor configurations.
The method is validated on a three-cylinder test case, applied
to a CROR with a mounted pylon, a stator-rotor-stator axial 
turbomachine and finally
an impeller/bladed diffuser/volute radial compressor.

\paragraph{Cost of the method}
Compared to the LUR method, the number of equations to solve is 
not constant here. In fact, if $N$ denotes the number of harmonics
computed in total (sum of each harmonic of each perturbation)
and if $\mathdollar_{\text{RANS}}$ 
denotes the CPU and memory cost of
one steady computation, $2N$ harmonic equations and 
one time-average equation
are solved, thus:
\begin{equation}
	\mathdollar_{\text{NLH}} = (2N+1) \cdot \mathdollar_{\text{RANS}}.
\end{equation}
However, \citet{Vilmin2006} do not apply the NLH formulation
to the turbulent equation (the one equation of \citet{Spalart1992}),
since five equations are solved using the NLH approach and one not,
the cost becomes:
\begin{equation}
	\mathdollar_{\text{NLH}} = \frac{5 \cdot (2N+1) + 1}{6} \cdot \mathdollar_{\text{RANS}}.
\end{equation}
This is another assumption as the turbulent field in a wake,
for instance, is seen unsteady in the opposite row frame
of reference~\cite{Lakshminarayana1980}.
