%!TEX root = ../../adrien_gomar_phd.tex

There is a large variety of spectral methods that exists in the
literature. The most important will be presented in this section.
As the development of these approaches on the Navier-Stokes equations
can be tedious, the following will only concentrate on the simplest
nonlinear equation, the advection equation for the velocity
defined as:
\begin{equation}
	\frac{\partial u}{\partial t} + 
	u \frac{\partial u}{\partial x} = 
	0.
	\label{eq:sm_nonlinear_convection}
\end{equation}
This equation can be formulated in a conservative manner for simplicity:
\begin{equation}
	\frac{\partial u}{\partial t} + 
	\frac{\partial}{\partial x} \left( \frac{u^2}{2} \right) = 
	0.
	\label{eq:sm_nonlinear_convection_conservative}
\end{equation}

The reader is referred to the cited papers for a detailed description
of the following methods and in particular, their
development for the Navier-Stokes equations. 

In total four spectral methods are presented, 
the Linearized Unsteady Reynolds averaged
Navier-Stokes, the NonLinear Harmonic method, the NonLinear Frequency Domain
method and the Harmonic Balance method.  
These names are chosen here
for simplicity but the reader might find in the literature many more
appellations. When it is the case, an effort will be made to synthesize
these to give the reader a good overview of the type of spectral methods available. 
Moreover, special emphasis will be put on the harmonic
balance method as it is the approach used in this thesis.
