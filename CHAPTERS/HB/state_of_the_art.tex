%!TEX root = ../../adrien_gomar_phd.tex

There is a large variety of spectral methods that exists in the
literature. The most important will be presented in this section.
As the development of these approaches on the Navier-Stokes equations
can be tedious, the following will only concentrate on the simplest
nonlinear equation, the non-viscous Burger's equation defined as:
\begin{equation}
	\frac{\partial u}{\partial t} + 
	u \frac{\partial u}{\partial x} = 
	0.
	\label{eq:sm_nonlinear_convection}
\end{equation}
This equation can be formulated in a conservative manner for simplicity:
\begin{equation}
	\frac{\partial u}{\partial t} + 
	\frac{\partial}{\partial x} \left( \frac{u^2}{2} \right) = 
	0.
	\label{eq:sm_nonlinear_convection_conservative}
\end{equation}

In total four spectral methods are presented, 
the \underline{L}inearized \underline{U}nsteady 
\underline{R}eynolds averaged
Navier-Stokes method (LUR), 
the \underline{N}on\underline{L}inear 
\underline{H}armonic method (NLH), the \underline{N}onLinear 
\underline{F}requency \underline{D}omain
method (NLFD) and the \underline{H}armonic \underline{B}alance 
method (HB).
These names are chosen here
for simplicity but the reader might find in the literature many more
appellations. When it is the case, an effort will be made to synthesize
these to give the reader a good overview of the type of spectral methods available and their differences. 
Moreover, special emphasis will be put on the harmonic
balance method as it is the approach used in this thesis.
Since the methods will be presented for simplicity on the non-viscous
Burger's equation, the reader is referred to 
the cited papers for a detailed description of 
application of the spectral methods
to Computational Fluid Dynamics. 
