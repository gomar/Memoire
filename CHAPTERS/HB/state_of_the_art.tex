%!TEX root = ../../adrien_gomar_phd.tex

There is a large variety of Fourier-based methods that exists in the
literature. 
The most important Fourier-based methods will be presented in this section.
In total four Fourier-based methods are presented, 
the \underline{L}inearized \underline{U}nsteady 
\underline{R}eynolds averaged
Navier-Stokes method (LUR), 
the \underline{N}on-\underline{L}inear 
\underline{H}armonic method (NLH), the \underline{N}on-\underline{L}inear 
\underline{F}requency \underline{D}omain
method (NLFD) and the \underline{H}armonic \underline{B}alance 
method (HB).
These names are chosen here
for clarity but the reader might find in the literature more
appellations. When this is the case, an effort will be made to synthesize
these appellations to give the reader a good 
overview of the type of Fourier-based methods that exists in the literature
and their differences.

As the development of these approaches on the Navier-Stokes equations
can be tedious, the following developments 
will only concentrate on the simplest
non-linear partial differential equation, 
the non-viscous Burger's equation defined as:
\begin{equation}
	\frac{\partial u}{\partial t} + 
	u \frac{\partial u}{\partial x} = 
	0.
	\label{eq:sm_nonlinear_convection}
\end{equation}
This equation can be formulated in a conservative manner for simplicity:
\begin{equation}
	\frac{\partial u}{\partial t} + 
	\frac{\partial}{\partial x} \left( \frac{u^2}{2} \right) = 
	0.
	\label{eq:sm_nonlinear_convection_conservative}
\end{equation}
For each method, when possible
\begin{itemize} \itemsep0pt \parskip0pt
 	\item the mono-frequential and multi-frequential 
 	formulations will be detailed,
 	\item some special extensions, if they exist, will be explained
 	\item since the development of the methods will be 
 	presented on the non-viscous 
 	Burger's equation, 
	the basic idea to extend the method to the Navier-Stokes
	equations will be detailed along with the relevant papers,
	that describe it,
	\item finally, an estimation of the cost of the method 
	compared to an equivalent steady computation will be detailed.
\end{itemize}
