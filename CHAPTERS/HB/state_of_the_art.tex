%!TEX root = ../../adrien_gomar_phd.tex

There is a large variety of spectral methods that exists in the
literature. 
As the development of these approaches on the Navier-Stokes equations
can be tedious, the following developments 
will only concentrate on the simplest
nonlinear partial differential equation, 
the non-viscous Burger's equation defined as:
\begin{equation}
	\frac{\partial u}{\partial t} + 
	u \frac{\partial u}{\partial x} = 
	0.
	\label{eq:sm_nonlinear_convection}
\end{equation}
This equation can be formulated in a conservative manner for simplicity:
\begin{equation}
	\frac{\partial u}{\partial t} + 
	\frac{\partial}{\partial x} \left( \frac{u^2}{2} \right) = 
	0.
	\label{eq:sm_nonlinear_convection_conservative}
\end{equation}



The most important spectral methods will be presented in this section.
In total four spectral methods are presented, 
the \underline{L}inearized \underline{U}nsteady 
\underline{R}eynolds averaged
Navier-Stokes method (LUR), 
the \underline{N}on\underline{L}inear 
\underline{H}armonic method (NLH), the \underline{N}onLinear 
\underline{F}requency \underline{D}omain
method (NLFD) and the \underline{H}armonic \underline{B}alance 
method (HB).
These names are chosen here
for clarity but the reader might find in the literature many more
appellations. When this is the case, an effort will be made to synthesize
these appellations to give the reader a good 
overview of the type of spectral methods that exist in the literature
and their differences.
For each method, when possible,
\begin{itemize}
 	\item the mono-frequential and multi-frequential 
 	formulations will be detailed,
 	\item some special features, if they exist, will be explained
 	\item since the development of the methods will be 
 	presented on the non-viscous 
 	Burger's equation, 
	the basic idea to extend the method to the Navier-Stokes
	equations will be detailed along with the relevant papers,
	that describe it,
	\item the time line of the development 
	of the method will be given,
	\item finally, an estimation of the cost of the method 
	compared to an equivalent steady computation will be detailed.
\end{itemize}
\todo{comment diminuer la taille de l'interligne de itemize ?}
For the time line, a diagram will be given for clarity, note
that the black filled blocks correspond to multi-frequential
formulations while the white filled ones stand for mono-frequential
formulations.
