%!TEX root = ../main.tex
\section{choro} % (fold)
\label{sec:choro}

Without loss of generality one can
consider two rows (labeled $1$ and $2$) with a 
rotation rate equal to respectively $\Omega_1$ and $\Omega_2$ as
depicted in Fig.~\ref{fig:chorochronicity}.
$A$ and $B$ are two fixed observer in their respective reference
of frame. The cylindrical coordinate system is chosen.

\begin{figure}[htbp]
  \centering
  \includegraphics*[width=0.60\textwidth]{./CHOROCHRONICITY.pdf}
  \caption{Inter-blade dephasing}
  \label{fig:chorochronicity}
\end{figure}

The translation on the theta axis due to the different rotation rates can be 
written as:
\begin{equation}
    X_A (t + \delta t) = X_A (t) +  \Omega_1 \delta t \cdot \vec{e_\theta},
    \label{eq:choro_pos_A}
\end{equation}
and
\begin{equation}
    X_B (t + \delta t) = X_B (t) +  \Omega_2 \delta t \cdot \vec{e_\theta}.
    \label{eq:choro_pos_B}
\end{equation}

By subtracting Eq.~\eqref{eq:choro_pos_B} and Eq.~\eqref{eq:choro_pos_A}
one gets:
\begin{equation}
    X_A (t + \delta t) - X_A (t) = X_B (t + \delta t) - X_B (t) + (\Omega_2 - \Omega_1) \delta t \cdot \vec{e_\theta}.
\end{equation}



% section choro (end)