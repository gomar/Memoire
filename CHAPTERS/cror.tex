%!TEX root = ../main.tex

\chapter{Flow around contra-rotating open rotors: general information} % (fold)
\label{cha:cror}

\section{Introduction} % (fold)
\label{sec:introduction}
augmentation rendement moteur 
=> augmentation Overall Pressure Ration (par température d'entrée turbine 
ou mise en cascade de compresseurs)
ou augmentation du ByPass Ratio
CROR + AEL is justified

% section introduction (end)

\section{What is aeroelasticity ?} % (fold)
\label{sec:what_is_aeroelasticity_}

Aeroelasticity relies on the interaction of three forces:
the aerodynamic forces $\mathcal{A}$,
inertial forces $\mathcal{I}$ and elastic forces $\mathcal{E}$.
as depicted in Fig.~\ref{fig:COLLAR_TRIANGLE}.
\begin{figure}[htbp]
  \centering
  \includegraphics*[width=0.40\textwidth]{COLLAR_TRIANGLE.pdf}
  \caption{Collar's aeroelastic triangle}
  \label{fig:COLLAR_TRIANGLE}
\end{figure}
This diagram shows the relation between aerodynamic forces $\mathcal{A}$,
inertial forces $\mathcal{I}$ and elastic forces $\mathcal{E}$.
The forces are linked two by two to form: Static Aeroelasticity ($\mathcal{A} + \mathcal{E}$),
Structural Dynamics ($\mathcal{E} + \mathcal{I}$) and Flight Mechanics ($\mathcal{A} + \mathcal{I}$).
The interaction between these three forces can cause several issues such as:
\begin{itemize}
	\item divergence, which is a static aeroelastic phenomenon.
	It occurs when, for example, an airfoil has an axial stiffness that increases.
	Under aerodynamic loads, the trailing edge is stiffer than the leading
	edge, hence the fastest deflection of the leading edge which increases the
	angle of attack and thus the aerodynamic loads. This is a vicious circle that
	leads to the failure of the airfoil.
	\item flutter
	\item buffeting
	\item 
\end{itemize}

% section what_is_aeroelasticity_ (end)

\section{Aerodynamic of a Contra-Rotating Open Rotor} % (fold)
\label{sec:aerodynamic_of_a_contra_rotating_open_rotor}

In this section, we will detailed the different phenomena
arising in a Contra-Rotating Open Rotor.

\subsection{Introduction} % (fold)
\label{sub:introduction}

The Contra-Rotating Open Rotor denoted CROR in this thesis, sometimes
called Open Rotor or Unduct Fan, is an engine composed of 

% subsection introduction (end)

\subsection{Velocity triangle} % (fold)
\label{sub:velocity_triangle}

The CROR is designed to have no swirl at his outlet. In fact,
in a propeller, the main source of losses in terms
of efficiency comes from the swirl created downstream of
the rotor as mentioned above. This energy is wasted as the tangential component
of the velocity can not be used to generate thrust.
The basic idea behind the CROR technology is to put a
second contra rotating row that will transform this
tangential velocity to an axial velocity.
\begin{figure}[htbp]
  \centering
  \includegraphics*[width=0.40\textwidth]{VELOCITY_TRIANGLE_CROR}
  \caption{Velocity triangle for the CROR configuration.}
  \label{fig:VELOCITY_TRIANGLE_CROR}
\end{figure}

Figure~\ref{fig:VELOCITY_TRIANGLE_CROR} shows the velocity triangle for
a CROR configuration. Three vectors are depicted: the absolute velocity
$V$, the relative velocity $W$ and the rotation speed $U$. Between
the front rotor and the rear rotor, the rotation speed is inversed.
This





% subsection velocity_triangle (end)


% section aerodynamic_of_a_contra_rotating_open_rotor (end)

% chapter cror (end)
