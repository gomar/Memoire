%!TEX root = ../main.tex
\chapter{Generalities}
\label{generalities}

\lhead{Chapitre ??. \emph{Generalities}}

\section{Equation solved} % (fold)
\label{sec:equation_solved}

The Unsteady Reynolds-Averaged Navier-Stokes (U-RANS) equations in
integral form are given by
\begin{equation}
   \int_\Omega \frac{\partial W}{\partial t} dV + \oint_{\partial
     \Omega} \overrightarrow{F} \cdot \overrightarrow{N} ds = 0,
   \label{eq:intNS}
\end{equation} 
where $\overrightarrow{F}$~is the flux across $\partial \Omega$ and
$W$~is the vector of the conservative unknowns (conservative variables
and turbulent variables).  Assuming $\Omega$ is a
control volume, the semi-discrete finite-volume form of the
U-RANS equations is obtained from Eq.~\eqref{eq:intNS}:
\begin{equation}
   \frac{d}{dt} \left(V  \overline{W}\right) + R \left( \overline{W}
   \right) = 0,
   \label{eq:semiDiscNS}
\end{equation} 
with $V$~the volume of the cell~$\Omega$, $R$~the residual resulting
from the discretization of the fluxes and the source terms (including
the turbulent equations), and $\overline{W}$ the mean of the
unknowns over the control volume.  In the following, the over line
symbol~$\overline{\cdot}$ is dropped out for clarity.

\section{Fluid / Structure Interaction}

\subsection{Weak Coupling Approach}

The weak coupling approach~\cite{Rougeault2003} is a one way
  coupling from structure to fluid: First, a modal identification
of the structure is carried out. Then the fluid response to the
harmonic prescribed motion of the structure modes is simulated,
where the harmonic motion of the geometry is ensured by a mesh
deformation technique, based on a structural analogy method
implementing linear elastic elements. Finally, knowing the unsteady
pressure load, a stability study can be performed in the frequency
domain.

\subsection{Linear Modal Structure Model}

\subsubsection{Governing Equations}
Once the modal basis $\Phi$ is identified, either by mean of a Finite
Element model or an experimental identification, the equation of
structure dynamics under aerodynamic load $F_A$ reads:
\begin{equation}
  \label{eq:2}
  M\ddot{q}+D\dot{q}+Kq-\Phi^\top F_A(t)=0, \quad x=\Phi q.
\end{equation}
The weak coupling approach assumes the linearity of the response of
the fluid with respect to the displacement of the structure. Therefore
small displacements are assumed and the so-called Generalized
Aerodynamic Forces (GAF) are linearized, which adds aerodynamic
stiffness~$K_A$ and damping~$D_A$:
\begin{equation}
  \label{eq:4}
  \Phi^\top F_A(t) = D_A\dot{q}+K_Aq.
\end{equation}
In order to estimate the unsteady aerodynamic forces $F_A(t)$,
  a fluid simulation is run with a prescribed harmonic motion of the
  structure:
\begin{equation}
  \label{eq:6}
  q(t)=\cos(\omega t).
\end{equation}
A stability analysis is then performed in the frequency domain:
\begin{equation}
  \label{eq:5}
  q=\hat{q}e^{p t}\Rightarrow\left(
    p^2M + p(D-D_A) + (K-K_A)
  \right)\hat{q}=0,
\end{equation}
where the Laplace variable $p$ is of the form
$p=i\omega(1+i\alpha)$. Finally, considering only weakly damped or
amplified modes (i.e. $|\alpha| \ll 1$), the damping of the
fluid/structure coupled system reads $\alpha=-\Re e(p)/\Im m(p)$.

\subsubsection{Single Passage Reduction for Turbomachinery Computations}

As the blade row is rotating, the stiffness of the blades is increased
and gyroscopic terms are added. Equation~\eqref{eq:2} becomes
\begin{equation}
  \label{eq:gyr}
  M\ddot{q}+(D+D_G)\dot{q}+(K+K_G)q-\Phi^\top F_A(t)=0, \quad x=\Phi q,
\end{equation}
where $D_G$ is the skew-symmetric gyroscopic damping matrix and $K_G$ is
the gyroscopic matrix of deflection for inclusion of centrifugal
elements for instance.  The disk being flexible, the blades do not vibrate
independently of each other. The cyclic symmetry leads to complex
vibration modes, which can be seen as rotating waves traveling at an
integer multiple $n_d$ of the rotation speed \cite{Lane:1956fk}. $n_d$
is called a nodal diameter. Opposite nodal diameters have the same
vibration mode propagating in opposite directions. Therefore their
respective modes are complex conjugate.


% section aeroelasticity_module (end)

% section equation_solved (end)
