%!TEX root = ../main.tex

montrer l'essence des méthodes spectrales avec une equation toute simple

depuis les années 1990 on essaye d emodeliser les instationnarite en TBM:
methode des deterministic stress puis LUR et enfin les methodes harmoniques.

\chapter{Spectral methods} % (fold)
\label{cha:spectral_methods}

\section{Introduction} % (fold)
\label{sec:sm_introduction}
% section introduction (end)

\section{State of the art} % (fold)
\label{sec:sm_state_of_the_art}

There is a large variety of spectral methods that exists in the
literature. The most important will be presented in this section.
As the development of these approaches on the Navier-Stokes equations
can be tedious, the following will only concentrate on the simplest
nonlinear equation, the advection equation for the velocity
defined as:
\begin{equation}
	\frac{\partial u}{\partial t} + 
	u \frac{\partial u}{\partial x} = 
	0.
	\label{eq:sm_nonlinear_convection}
\end{equation}
This equation can be formulated in a conservative manner for simplicity:
\begin{equation}
	\frac{\partial u}{\partial t} + 
	\frac{\partial}{\partial x} \left( \frac{u^2}{2} \right) = 
	0.
	\label{eq:sm_nonlinear_convection_conservative}
\end{equation}

\subsection{NonLinear Harmonic method} % (fold)
\label{sub:sm_nonlinear_harmonic_method}

Originally developed by \citet{He1998} and \citet{Ning1998}, the 
NonLinear Harmonic method
relies on a decomposition of the conservative variables into a
time-averaged part plus an unsteady perturbation:
\begin{equation}
	u = \overline{u} + u^\prime,
	\label{eq:sm_nlh_decomposition}
\end{equation}
where $\overline{.}$ denotes the time-averaging operator and
$.^\prime$ the corresponding unsteady perturbation.
By injecting Eq.~\ref{eq:sm_nlh_decomposition} into
Eq.~\ref{eq:sm_nonlinear_convection_conservative}:
\begin{equation}
	\frac{\partial \left( \overline{u} + u^\prime \right)}{\partial t} + 
	\frac{\partial}{\partial x} \left[\frac{
		\left( \overline{u} + u^\prime\right)
		\left( \overline{u} + u^\prime\right)}{2}\right] = 
	0,
\end{equation}
giving:
\begin{equation}
	\frac{\partial u^\prime}{\partial t} + 
	\frac{1}{2}\frac{\partial}{\partial x} \left[
	\overline{u}^2 + 2 \overline{u} u^\prime + u^\prime u^\prime \right] = 
	0.
	\label{eq:sm_nlh_step_1}
\end{equation}
The time-averaged equation can be obtained by time-averaging
equation~\ref{eq:sm_nlh_step_1}:
\begin{equation}
	(\overline{\ref{eq:sm_nlh_step_1}})
	\Leftrightarrow
	\frac{\partial}{\partial x}
	\left[\overline{u}^2 + 
	\overline{u^\prime u^\prime}\right] =
	0,
	\label{eq:sm_nlh_step_2}
\end{equation}
The term $\overline{u^\prime u^\prime}$
appears due to the nonlinearities of the considered equation. It
is called the nonlinear stress terms 
(or the deterministic stress terms) as a reference to 
the Reynolds stress terms. 
The equations for the unsteady perturbations is then obtained by keeping
the first order terms of the unsteady equation~\ref{eq:sm_nlh_step_1}
leading to:
\begin{equation}
	\frac{\partial u^\prime}{\partial t} + 
	\frac{\partial}{\partial x} \left[\overline{u} u^\prime \right] = 
	0.
\end{equation}
For now on, no assumption has been made on the velocity $u$.
Assuming that the velocity unsteadiness 
are periodic in time with period
$T=2 \pi / \omega$,
the unsteady fluctuations can be decomposed into 
a Fourier series:
\begin{equation}
	u^\prime = \sum_{k=-\infty \atop k \neq 0}^{\infty} 
	\widehat{u}_k e^{i \omega k t}.
	\label{eq:sm_nlh_decomposition_pert}
\end{equation}
Hence, since the complex exponentials forms 
an exponential basis, we have for all harmonics 
$-\infty \leq k \leq \infty, \; k \neq 0$:
\begin{equation}
	i \omega k \widehat{u}_k + 
	\frac{\partial}{\partial x} \left[ \overline{u} \widehat{u}_k\right] =
	0.
\end{equation}
The number of harmonics is truncated at order $N$ to reduce the
number of computations to run. This is a fare assumption as most
of the physical flows do not have an infinite spectrum. This
is for sure a reduce order approach. The goal of the spectral
methods being to have a compact representation of the unsteady time
signals. To close the equations, the term 
$\overline{u^\prime u^\prime}$ needs to be computed. It can be 
directly worked out when the harmonics are known:
\begin{equation}
	\begin{split}
		u^\prime u^\prime &= 
		\left[
			\sum_{k=-N \atop k \neq 0}^{N} \widehat{u}_k e^{i \omega k t} 
		\right]
		\left[
			\sum_{k=-N \atop k \neq 0}^{N} \widehat{u}_k e^{i \omega k t} 
		\right] \\
		&= \sum_{k=-N \atop k \neq 0}^{N} (\widehat{u}_k)^2
		   e^{i 2 \omega k t} +
		   2 \sum_{j,k=-N \atop j \neq k \neq 0}^{N} 
		   \widehat{u}_k \widehat{u}_j e^{i \omega (j + k) t} \\
	\end{split}
\end{equation}
Thus,
\begin{equation}
	\begin{split}
		\overline{u^\prime u^\prime} &= 
		\frac{1}{T} \int_{t=0}^{T} \left[ 
			\sum_{k=-N \atop k \neq 0}^{N} (\widehat{u}_k)^2
		   	e^{i 2 \omega k t} +
		   	2 \sum_{j,k=-N \atop j \neq k \neq 0}^{N} 
		   	\widehat{u}_k \widehat{u}_j e^{i \omega (j + k) t} 
		\right] dt\\
		&= \frac{2}{T} \int_{t=0}^{T} \sum_{j,k=-N \atop j \neq k \neq 0}^{N} 
		   	\widehat{u}_k \widehat{u}_j 
		   	e^{i \omega (j + k) t} dt \\
		&= \frac{2}{T} \int_{t=0}^{T} 
			\sum_{k=-N \atop k \neq 0}^{N} 
			\widehat{u}_k \widehat{u}_{-k}  dt.
	\end{split}
\end{equation}
As $\widehat{u}_k$ and $\widehat{u}_{-k}$ are complex conjugates,
finally $\overline{u^\prime u^\prime}$ is equal to:
\begin{equation}
	\overline{u^\prime u^\prime} = 
	2 \sum_{k=-N \atop k \neq 0}^{N} |\widehat{u}_k|^2.
\end{equation}
This last equation only depends on the computed harmonics, meaning
that no term is modelled.

To summarize, in the nonlinear harmonic method presented here,
three hypothesis are kept:
\begin{itemize}
	\item the unsteady perturbations are periodic in time, 
	meaning that they can be decomposed using a Fourier series,
	\item the high-order cross-coupling terms $u^\prime u^\prime$
	are neglected,
	\item the number of harmonics is set to $N$.
\end{itemize}


% subsection nonlinear_harmonic_method (end)

\subsection{Harmonic Balance Technique of Hall et al.} % (fold)
\label{sub:harmonic_balance_technique_of_hall}

\citet{Hall2002} developed a time-domain formulation for spectral
methods. Instead of decomposing the $u$ variable into a time-averaged
part and unsteady fluctuations, as in Eq.~\ref{eq:sm_nlh_decomposition},
the variables of interest are decomposed using a Fourier series:
\begin{equation}
	u (t) = \sum_{k=-N}^{N} \widehat{u}_k e^{i k \omega t}.
	\label{eq:sm_hbt_decomposition}
\end{equation}
This can be written using a matrix formulation:
\begin{equation}
	u (t) = 
\end{equation}


% subsection harmonic_balance_technique_of_hall (end)

% section state_of_the_art (end)

% chapter spectral_methods (end)