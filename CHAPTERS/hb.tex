%!TEX root = ../main.tex

montrer l'essence des méthodes spectrales avec une equation toute simple

depuis les années 1990 on essaye d emodeliser les instationnarite en TBM:
methode des deterministic stress puis LUR et enfin les methodes harmoniques.

\chapter{Spectral methods} % (fold)
\label{cha:spectral_methods}

\section{Introduction} % (fold)
\label{sec:sm_introduction}
% section introduction (end)

\section{State of the art} % (fold)
\label{sec:sm_state_of_the_art}

There is a large variety of spectral methods that exists in the
literature. The most important will be presented in this section.
As the development of these approaches on the Navier-Stokes equations
can be tedious, the following will only concentrate on the simplest
nonlinear equation, the advection equation for the velocity
defined as:
\begin{equation}
	\frac{\partial u}{\partial t} + 
	u \frac{\partial u}{\partial x} = 
	0.
	\label{eq:sm_nonlinear_convection}
\end{equation}
This equation can be formulated in a conservative manner for simplicity:
\begin{equation}
	\frac{\partial u}{\partial t} + 
	\frac{\partial}{\partial x} \left( \frac{u^2}{2} \right) = 
	0.
	\label{eq:sm_nonlinear_convection_conservative}
\end{equation}

\subsection{NonLinear Harmonic method} % (fold)
\label{sub:sm_nonlinear_harmonic_method}

Originally developed by \citet{He1998} and \citet{Ning1998}, the 
NonLinear Harmonic method
relies on a decomposition of the conservative variables into a
time-averaged part plus an unsteady perturbation:
\begin{equation}
	u = \overline{u} + u^\prime,
	\label{eq:sm_nlh_decomposition}
\end{equation}
where $\overline{.}$ denotes the time-averaging operator and
$.^\prime$ the corresponding unsteady perturbation.
By injecting Eq.~\ref{eq:sm_nlh_decomposition} into
Eq.~\ref{eq:sm_nonlinear_convection_conservative}, one gets:
\begin{equation}
	\frac{\partial \left( \overline{u} + u^\prime \right)}{\partial t} + 
	\frac{\partial}{\partial x} \left[\frac{
		\left( \overline{u} + u^\prime\right)
		\left( \overline{u} + u^\prime\right)}{2}\right] = 
	0.
	\label{eq:sm_nlh_step_1}
\end{equation}
The time-averaged equation can be obtained by time-averaging
equation~\ref{eq:sm_nlh_step_1}:
\begin{equation}
	(\overline{\ref{eq:sm_nlh_step_1}})
	\Leftrightarrow
	\frac{\partial \overline{u} \, \overline{u}}{\partial x} +
	\frac{\partial \overline{u^\prime u^\prime}}{\partial x} =
	0
	\label{eq:sm_nlh_step_2}
\end{equation}

The equations for the unsteady perturbations can be obtained by
the difference of Eq.~\ref{eq:sm_nlh_step_1}
and Eq.~\ref{eq:sm_nlh_step_2}:
\begin{equation}
	(\ref{eq:sm_nlh_step_1}) - (\overline{\ref{eq:sm_nlh_step_1}})
	\Leftrightarrow
	\frac{\partial u^\prime}{\partial t} +
	\frac{1}{2} \frac{\partial}{\partial x} \left(
		2 u^\prime \overline{u} + 
		u^\prime u^\prime - 
		\overline{u^\prime u^\prime}\right) =
	0
	\label{eq:sm_nlh_step_3}
\end{equation}
The term $\frac{\partial \overline{u^\prime u^\prime}}{\partial x}$
appears due to the nonlinearities of the considered equation. It
is called the nonlinear stress terms 
(or the deterministic stress terms) as a reference to 
the Reynolds stress terms. 

For now on, no assumption has been made on the velocity $u$.
% subsection nonlinear_harmonic_method (end)

\subsection{Harmonic Balance Technique of Hall et al.} % (fold)
\label{sub:harmonic_balance_technique_of_hall}

\citet{Hall2002} developed a time-domain formulation for spectral
methods. Instead of decomposing the $u$ variable into a time-averaged
part and unsteady fluctuations, as in Eq.~\ref{eq:sm_nlh_decomposition},
the variables of interest are decomposed using a Fourier series:
\begin{equation}
	u (t) = \sum_{k=-N}^{N} \widehat{u}_k e^{i k \omega t}.
	\label{eq:sm_hbt_decomposition}
\end{equation}
This can be written using a matrix formulation:
\begin{equation}
	u (t) = 
\end{equation}


% subsection harmonic_balance_technique_of_hall (end)

% section state_of_the_art (end)

% chapter spectral_methods (end)