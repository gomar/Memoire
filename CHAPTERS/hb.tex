%!TEX root = ../main.tex

montrer l'essence des méthodes spectrales avec une equation toute simple

depuis les années 1990 on essaye d emodeliser les instationnarite en TBM:
methode des deterministic stress puis LUR et enfin les methodes harmoniques.

\chapter{Spectral methods} % (fold)
\label{cha:spectral_methods}

\section{Introduction} % (fold)
\label{sec:introduction}

It is of common knowledge that the flow in a turbomachinery
is inherently unsteady

% section introduction (end)

\section{State of the art} % (fold)
\label{sec:state_of_the_art}

Since the pioneer work of \citet{He1998}, several research group have
worked on spectral methods worldwide.
Basically, these methods rely on the Fourier decomposition.
In fact, assuming that the flow in a turbomachine is periodic,
the variables can be decomposed into Fourier series. By selecting
only a small number of harmonics to represent the variables,
we have a compact way of writing the variables and hence computing them

\begin{equation}
	V \frac{\partial W}{\partial t} + R(W) = 0
	\label{eq:navier_stokes_discrete}
\end{equation}

\subsection{Nonlinear Harmonic method} % (fold)
\label{sub:nonlinear_harmonic_method}

In their original article \citet{He1998} decompose
the main variables of the flow into a temporal mean
part and unsteady perturbations:
\begin{equation}
	W(t) = \bar{W} + W^\prime
	\label{eq:nlh_decompose}
\end{equation}
where $\bar{.}$ is the temporal mean operator and
$W^\prime$ are the unsteady perturbations.
The unsteady perturbations are defined as
\begin{equation}
	W^\prime = \sum_{j=1}^{N_p} W^\prime_j,
	\label{eq:nlh_w_prime}
\end{equation}
where $N_p$ denotes the number of perturbations.
Then, the perturbations are assumed to be periodic and hence
decomposed using a Fourier series:
\begin{equation}
	W^\prime_j = \sum_{k=1}^{N} \widehat{W}_k e ^{i k \omega_j t}.
	\label{eq:nlh_fourier}
\end{equation}
Even though, in the original paper of \citet{He1998} the only use one 
perturbation ($N_p=1$) with one harmonic ($N=1$) the method
is developed for multiple perturbations and multiple harmonics.

Finally, the vector of conservative variable can be written as:
\begin{equation}
	W(t) = \bar{W} + \sum_{j=1}^{N_p} \sum_{k=1}^{N} \widehat{W}_k e ^{i k \omega_j t}.
\end{equation}
It is then substituted in the Euler equations.

% subsection nonlinear_harmonic_method (end)

\subsection{NonLinear Frequency Domain method} % (fold)
\label{sub:nonlinear_frequency_domain_method}

Originally proposed by \citet{McMullen2001}, the NLFD aims at
keeping a time domain solver to avoid developing a new
frequency domain solver. In fact, it is not easy to derive 
the turbulence model in the frequency domain and hence limits the
use of such an algorithm in an industrial CFD code.


% subsection nonlinear_frequency_domain_method (end)

% section state_of_the_art (end)

% chapter spectral_methods (end)
