%!TEX root = ../main.tex

montrer l'essence des méthodes spectrales avec une equation toute simple

depuis les années 1990 on essaye d emodeliser les instationnarite en TBM:
methode des deterministic stress puis LUR et enfin les methodes harmoniques.

\chapter{Spectral methods} % (fold)
\label{cha:spectral_methods}

\section{Introduction} % (fold)
\label{sec:sm_introduction}
% section introduction (end)

\section{State of the art} % (fold)
\label{sec:sm_state_of_the_art}

There is a large variety of spectral methods that exists in the
literature. The most important will be presented in this section.
As the development of these approaches on the Navier-Stokes equations
can be tedious, the following will only concentrate on the simplest
nonlinear equation, the advection equation for the velocity
defined as:
\begin{equation}
	\frac{\partial u}{\partial t} + 
	u \frac{\partial u}{\partial x} = 
	0.
	\label{eq:sm_nonlinear_convection}
\end{equation}
This equation can be formulated in a conservative manner for simplicity:
\begin{equation}
	\frac{\partial u}{\partial t} + 
	\frac{\partial}{\partial x} \left( \frac{u^2}{2} \right) = 
	0.
	\label{eq:sm_nonlinear_convection_conservative}
\end{equation}

\subsection{NonLinear Harmonic method} % (fold)
\label{sub:sm_nonlinear_harmonic_method}

Originally developed by \citet{He1998} and \citet{Ning1998}, the method
relies on a decomposition of the conservative variables into a
time-averaged part plus an unsteady perturbation, here for the velocity
$u$:
\begin{equation}
	u = \overline{u} + u^\prime,
	\label{eq:sm_nlh_decomposition}
\end{equation}
where $\overline{.}$ denotes the time-averaging operator and
$.^\prime$ its unsteady perturbation.
By injecting Eq.~\ref{eq:sm_nlh_decomposition} into
Eq.~\ref{eq:sm_nonlinear_convection_conservative}, one gets:
\begin{equation}
	\frac{\partial \left( \overline{u} + u^\prime \right)}{\partial t} + 
	\frac{\partial}{\partial x} \left[\frac{
		\left( \overline{u} + u^\prime\right)
		\left( \overline{u} + u^\prime\right)}{2}\right] = 
	0.
	\label{eq:sm_nlh_step_1}
\end{equation}
The time-averaged equation can be obtained by time-averaging
equation Eq.~\ref{eq:sm_nlh_step_1}:
\begin{equation}
	(\overline{\ref{eq:sm_nlh_step_1}})
	\Leftrightarrow
	\frac{\partial \overline{u} \, \overline{u}}{\partial x} +
	\frac{\partial \overline{u^\prime u^\prime}}{\partial x} =
	0
	\label{eq:sm_nlh_step_2}
\end{equation}
The term $\frac{\partial \overline{u^\prime u^\prime}}{\partial x}$
appears due to the nonlinearities of the considered equation, it
is called the unsteady stress term as a reference to 
the Reynolds stress terms.

The equations for the unsteady perturbations can be obtained by
the difference of Eq.~\ref{eq:sm_nlh_step_1}
and Eq.~\ref{eq:sm_nlh_step_2}:
\begin{equation}
	(\ref{eq:sm_nlh_step_1}) - (\overline{\ref{eq:sm_nlh_step_1}})
	\Leftrightarrow
	\frac{\partial u^\prime}{\partial t} +
	\frac{1}{2} \frac{\partial}{\partial x} \left(
		2 u^\prime \overline{u} + 
		u^\prime u^\prime - 
		\overline{u^\prime u^\prime}\right) =
	0
	\label{eq:sm_nlh_step_3}
\end{equation}

For now on, no assumption has been made on the flow. The innovative step
is to consider the unsteady perturbation $u^\prime$ to be formed of
multiple disturbances:
\begin{equation}
	u^\prime = \sum_{i=1}^{N_p} u_i^\prime,
	\label{eq:sm_nlh_mp_disturbances_1}
\end{equation}
where $N_p$ is the total number of perturbations, and to consider that these
are periodic in time, 
meaning that we can decomposed them with their Fourier series:
\begin{equation}
	u^\prime = \sum_{i=1}^{N_p} \sum_{i=1}^{N_p},
	\label{eq:sm_nlh_mp_disturbances_2}
\end{equation}

% subsection nonlinear_harmonic_method (end)



% section state_of_the_art (end)

% chapter spectral_methods (end)
