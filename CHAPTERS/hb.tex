%!TEX root = ../main.tex
\section{Almost-Periodic flows} % (fold)
\label{sec:almost_periodic_flows}

\subsubsection{Mapping on a set of arbitrary frequencies}
\label{sec:mapping-an-arbitrary}

If the flow variables are composed of non-harmonically related
frequencies (\emph{i.e.} the flow spectrum has high-energy
discrete-frequency modes), the flow regime can be termed as
almost-periodic~\cite{Besicovitch:1954qy}. Instead of a regular
Fourier series, the U-RANS equations are projected on a set of
complex exponentials with arbitrary angular frequencies~$\omega_k$.
The conservative variables and the residuals are then approximated by
\begin{equation}
   W(t) \approx \sum_{k=-N}^{N} \widehat{W}_k e^{i \omega_k t},\quad
   R(t) \approx \sum_{k=-N}^{N} \widehat{R}_k e^{i \omega_k t},
   \label{eq:fourierW}
\end{equation}
where $\widehat{W}_k$ and $\widehat{R}_k$ are the coefficients of the
almost-periodic Fourier series for the frequency $f_k = \omega_k/2\pi$.
% As in section~\ref{sec:periodic-flows}, the crossed terms need to be attended and a linearization to the first order is done. 
% Using the previous example and transposing it into a multifrequential context, $u$ and $v$ are now expressed as
% \begin{equation}
%  u = \sum_{k=-N}^{N} \hat{u}_k e^{i\omega_k t},\quad v = \sum_{l=-N}^{N} \hat{v}_k e^{i\omega_l t},
% \end{equation}
% and the product $uv$ is approximated by
% \begin{equation}
%  uv \approx \sum_{k=-N}^{N} \hat{u}_k\hat{v}_k e^{i\omega_k t}.
% \end{equation}
% Every frequency composition which may arise from the product $uv$ is thus neglected.\newline
Injecting this decomposition in Eq.~\eqref{eq:semiDiscNS} yields
\begin{equation}
    \sum_{k=-N}^{N} \left(i \omega_k V \widehat{W}_k + \widehat{R}_k
    \right) e^{i \omega_k t} =0.
   \label{eq:fourierNS_2}
\end{equation}
Sampling in time onto a set of $2 N + 1$ time levels to solve
Eq.~\eqref{eq:fourierNS_2}, the following matrix formulation is
obtained:
\begin{equation}
   A^{-1} \cdot \left(i V P\widehat{W}^{\star} + \widehat{R}^{\star} \right) = 0,
   \label{eq:matrixFourierNS}
\end{equation}
where the almost-periodic inverse discrete Fourier transform (IDFT)
matrix reads:
\begin{equation}
   A^{-1} = 
   \begin{bmatrix}
      exp(i\omega_{-N}t_{0})  & \cdots & exp(i\omega_{0}t_{0})    & \cdots  & exp(i\omega_{N}t_{0}) \\
      \vdots                  &       & \vdots                  &       & \vdots                \\
      exp(i\omega_{-N}t_{k})  & \cdots & exp(i\omega_{0}t_{k})    & \cdots  & exp(i\omega_{N}t_{k}) \\
      \vdots                  &       & \vdots                  &       & \vdots                \\
      exp(i\omega_{-N}t_{2N}) & \cdots & exp(i\omega_{0}t_{2N}) & \cdots  & exp(i\omega_{N}t_{2N})
      \end{bmatrix},
   \label{MatriceA}
\end{equation}
with $\omega_0=0$, $t_0=0$, $\omega_{-N} = -\omega_{N}$ and
\begin{equation}
  \begin{split}
   P &= diag\left(-\omega_N,\ldots,\omega_0,\ldots,\omega_N \right),\\
   \widehat{W}^{\star} & = \left[\widehat{W}_{-N},\ldots,\widehat{W}_0,\ldots,\widehat{W}_N \right]^\top,\\
   \widehat{R}^{\star} & = \left[\widehat{R}_{-N},\ldots,\widehat{R}_0,\ldots,\widehat{R}_N \right]^\top.
  \end{split}
\end{equation}
As opposed to the case of periodic flow, the arbitrary complex
exponentials family does not form, \emph{a priori}, an orthogonal
basis. % Hence the matrix
% formulation.

Knowing a time sampling that allows $A^{-1}$ to be invertible,
the almost-periodic Fourier coefficients can be approximated thanks to
\begin{equation}
  \begin{cases}
    \widehat{W}^{\star} = A W^{\star}, & \text{ with } W^{\star} = \left[ W\left( t_0
      \right),\ldots,W\left( t_i \right),\ldots,W\left( t_{2N} \right)
    \right]^\top,\\
    \widehat{R}^{\star} = A R^{\star}, & \text{ with } R^{\star} = \left[ R\left( t_0
      \right),\ldots,R\left( t_i \right),\ldots,R\left( t_{2N} \right)
    \right]^\top.
  \end{cases}
   \label{eq:idftW}
\end{equation}
Equation~\eqref{eq:matrixFourierNS} thus becomes
\begin{equation}
   iVA^{-1}PA + R^{\star} = 
   V D_t [ W^{\star}] + R^{\star} = 0,
   \label{eq:matrixFourierNS_2}
\end{equation}
where the multiple-frequency HB time-derivative operator $D_t[.] = i
A^{-1} P A$, the HB source term, can not be easily derived
analytically, and has to be numerically computed. This must be 
real matrix, however the authors were not able to prove it mathematically.
Nonetheless, numerical experiments tends to confirm this assertion. Indeed, 
the magnitude of the ratio of the real part over the imaginary part is around $10^{15}$.
The remaining value of the imaginary numbers may then be attributed to rounding errors.
%appears as a source term that
%represents a high-order formulation of the initial time derivative in
%Eq.~\eqref{eq:semiDiscNS}.  It is equivalent to the single frequency
%operator Eq.~\eqref{eq:dt} if $\omega_k=k\omega$.
% Again, a pseudo-time derivative $\tau^*$ is added to Eq.~\eqref{eq:matrixFourierNS}
% in order to time march the equations to the steady-state solutions:
% % 
% \begin{equation}
%    V \frac{\partial W^{\star}}{\partial \tau^*} + V D_t [ W^{\star}] + R^{\star} = 
%    V \frac{\partial W^{\star}}{\partial \tau^*} + V D_t [ W^{\star}] + R^{\star} = 0.
%    \label{eq:matrixFourierNS_3}
% \end{equation}
% 

%As the exact formulation of the matrix $A^{-1}$ can not be derived
%analytically, the source term $D_t[.]$ has to be numerically computed.
At this step of the derivation of the method, the time sampling $[t_0,
\ldots, t_{2N}]$ remains to be specified.

\subsubsection{Condition number and convergence}
\label{sec:condition_number}

Kundert~\emph{et al.}~\cite{Kundert1988} show that the condition
number of $A$, and thus $A^{-1}$, has a salient role in the
convergence of Harmonic Balance computations. The condition number of
the almost-periodic DFT matrix~$A$ is defined as
\begin{equation}
   \kappa (A) = \kappa (A^{-1}) = \| A \| \cdot \| A^{-1} \|, \quad
   \kappa(A) \geq 1,
   \label{eq:condition_number}
\end{equation}
where $\| \cdot \|$ denotes a matrix norm.  Considering the resolution
of $A x = b$, if $A$ is invertible and if $\delta A$, $\delta x$ and
$\delta b$ are the numerical errors associated with the computation of
$A$, $x$ and $b$, respectively, then
\begin{equation}
   (A + \delta A)(x + \delta x) = b + \delta b.
   \label{eq:error_reso}
\end{equation}
Therefore, the condition number sets an upper bound for the error made on~$x$:
\begin{equation}
   \frac{\| \delta x \|}{\| x \|} \leq \kappa(A)\left[\frac{\| \delta A \|}{\| A \|} + \frac{\| \delta b \|}{\| b \|} \right].
   \label{eq:conditonnig_amp}
\end{equation}
The error on the iterative resolution of the U-RANS equations can
therefore be amplified by the HB source term. This amplification is
led by the condition number of the almost-periodic DFT matrix. This
also means that if the errors are small but the condition number is
high, and vice-versa, the computation can diverge too. However, the
errors can not be \emph{a priori} controlled, thus the need to
minimize the condition number.

In the case of periodic-flows, the DFT matrix is well-conditioned: the
uniform sampling for harmonically related frequencies leads to a
condition number equal to~$1$, which is the theoretical lower bound
for the condition number.  This is linked to the orthogonality of the
complex exponential family.  On the other hand, when the frequencies are arbitrary, it is usually
impossible to choose a uniform set of time instants over which the
almost-periodic DFT matrix~$A$ is well conditioned. In fact, it is common for uniformly-sampled
sinusoids at two or more frequencies to be nearly linearly dependent,
which causes them not to be orthogonal, leading to the
ill-conditioning encountered in practice. As the frequency set is
chosen by the user, the only degrees of freedom left to get a
well-conditioned matrix are the time levels.  The following section
describes two algorithms to find a non-uniform time sampling that
minimizes the almost-periodic DFT matrix condition number.

% section almost_periodic_flows (end)