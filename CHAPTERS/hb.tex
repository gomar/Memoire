%!TEX root = ../main.tex
\section{Almost-Periodic flows} % (fold)
\label{sec:almost_periodic_flows}

\subsection{Mapping on a set of arbitrary frequencies}
\label{sec:mapping-an-arbitrary}

If the flow variables are composed of non-harmonically related
frequencies (\emph{i.e.} the flow spectrum has high-energy
discrete-frequency modes), the flow regime can be termed as
almost-periodic~\cite{Besicovitch:1954qy}. Instead of a regular
Fourier series, the U-RANS equations are projected on a set of
complex exponentials with arbitrary angular frequencies~$\omega_k$.
The conservative variables and the residuals are then approximated by
\begin{equation}
   W(t) \approx \sum_{k=-N}^{N} \widehat{W}_k e^{i \omega_k t},\quad
   R(t) \approx \sum_{k=-N}^{N} \widehat{R}_k e^{i \omega_k t},
   \label{eq:fourierW}
\end{equation}
where $\widehat{W}_k$ and $\widehat{R}_k$ are the coefficients of the
almost-periodic Fourier series for the frequency $f_k = \omega_k/2\pi$.
% As in section~\ref{sec:periodic-flows}, the crossed terms need to be attended and a linearization to the first order is done. 
% Using the previous example and transposing it into a multifrequential context, $u$ and $v$ are now expressed as
% \begin{equation}
%  u = \sum_{k=-N}^{N} \hat{u}_k e^{i\omega_k t},\quad v = \sum_{l=-N}^{N} \hat{v}_k e^{i\omega_l t},
% \end{equation}
% and the product $uv$ is approximated by
% \begin{equation}
%  uv \approx \sum_{k=-N}^{N} \hat{u}_k\hat{v}_k e^{i\omega_k t}.
% \end{equation}
% Every frequency composition which may arise from the product $uv$ is thus neglected.\newline
Injecting this decomposition in Eq.~\eqref{eq:semiDiscNS} yields
\begin{equation}
    \sum_{k=-N}^{N} \left(i \omega_k V \widehat{W}_k + \widehat{R}_k
    \right) e^{i \omega_k t} =0.
   \label{eq:fourierNS_2}
\end{equation}
Sampling in time onto a set of $2 N + 1$ time levels to solve
Eq.~\eqref{eq:fourierNS_2}, the following matrix formulation is
obtained:
\begin{equation}
   A^{-1} \cdot \left(i V P\widehat{W}^{\star} + \widehat{R}^{\star} \right) = 0,
   \label{eq:matrixFourierNS}
\end{equation}
where the almost-periodic inverse discrete Fourier transform (IDFT)
matrix reads:
\begin{equation}
   A^{-1} = 
   \begin{bmatrix}
      exp(i\omega_{-N}t_{0})  & \cdots & exp(i\omega_{0}t_{0})    & \cdots  & exp(i\omega_{N}t_{0}) \\
      \vdots                  &       & \vdots                  &       & \vdots                \\
      exp(i\omega_{-N}t_{k})  & \cdots & exp(i\omega_{0}t_{k})    & \cdots  & exp(i\omega_{N}t_{k}) \\
      \vdots                  &       & \vdots                  &       & \vdots                \\
      exp(i\omega_{-N}t_{2N}) & \cdots & exp(i\omega_{0}t_{2N}) & \cdots  & exp(i\omega_{N}t_{2N})
      \end{bmatrix},
   \label{MatriceA}
\end{equation}
with $\omega_0=0$, $t_0=0$, $\omega_{-N} = -\omega_{N}$ and
\begin{equation}
  \begin{split}
   P &= diag\left(-\omega_N,\ldots,\omega_0,\ldots,\omega_N \right),\\
   \widehat{W}^{\star} & = \left[\widehat{W}_{-N},\ldots,\widehat{W}_0,\ldots,\widehat{W}_N \right]^\top,\\
   \widehat{R}^{\star} & = \left[\widehat{R}_{-N},\ldots,\widehat{R}_0,\ldots,\widehat{R}_N \right]^\top.
  \end{split}
\end{equation}
As opposed to the case of periodic flow, the arbitrary complex
exponentials family does not form, \emph{a priori}, an orthogonal
basis. % Hence the matrix
% formulation.

Knowing a time sampling that allows $A^{-1}$ to be invertible,
the almost-periodic Fourier coefficients can be approximated thanks to
\begin{equation}
  \begin{cases}
    \widehat{W}^{\star} = A W^{\star}, & \text{ with } W^{\star} = \left[ W\left( t_0
      \right),\ldots,W\left( t_i \right),\ldots,W\left( t_{2N} \right)
    \right]^\top,\\
    \widehat{R}^{\star} = A R^{\star}, & \text{ with } R^{\star} = \left[ R\left( t_0
      \right),\ldots,R\left( t_i \right),\ldots,R\left( t_{2N} \right)
    \right]^\top.
  \end{cases}
   \label{eq:idftW}
\end{equation}
Equation~\eqref{eq:matrixFourierNS} thus becomes
\begin{equation}
   iVA^{-1}PA + R^{\star} = 
   V D_t [ W^{\star}] + R^{\star} = 0,
   \label{eq:matrixFourierNS_2}
\end{equation}
where the multiple-frequency HB time-derivative operator $D_t[.] = i
A^{-1} P A$, the HB source term, can not be easily derived
analytically, and has to be numerically computed. This must be 
real matrix, however the authors were not able to prove it mathematically.
Nonetheless, numerical experiments tends to confirm this assertion. Indeed, 
the magnitude of the ratio of the real part over the imaginary part is around $10^{15}$.
The remaining value of the imaginary numbers may then be attributed to rounding errors.
%appears as a source term that
%represents a high-order formulation of the initial time derivative in
%Eq.~\eqref{eq:semiDiscNS}.  It is equivalent to the single frequency
%operator Eq.~\eqref{eq:dt} if $\omega_k=k\omega$.
% Again, a pseudo-time derivative $\tau^*$ is added to Eq.~\eqref{eq:matrixFourierNS}
% in order to time march the equations to the steady-state solutions:
% % 
% \begin{equation}
%    V \frac{\partial W^{\star}}{\partial \tau^*} + V D_t [ W^{\star}] + R^{\star} = 
%    V \frac{\partial W^{\star}}{\partial \tau^*} + V D_t [ W^{\star}] + R^{\star} = 0.
%    \label{eq:matrixFourierNS_3}
% \end{equation}
% 

%As the exact formulation of the matrix $A^{-1}$ can not be derived
%analytically, the source term $D_t[.]$ has to be numerically computed.
At this step of the derivation of the method, the time sampling $[t_0,
\ldots, t_{2N}]$ remains to be specified.

\subsubsection{Condition number and convergence}
\label{sec:condition_number}

Kundert~\emph{et al.}~\cite{Kundert1988} show that the condition
number of $A$, and thus $A^{-1}$, has a salient role in the
convergence of Harmonic Balance computations. The condition number of
the almost-periodic DFT matrix~$A$ is defined as
\begin{equation}
   \kappa (A) = \kappa (A^{-1}) = \| A \| \cdot \| A^{-1} \|, \quad
   \kappa(A) \geq 1,
   \label{eq:condition_number}
\end{equation}
where $\| \cdot \|$ denotes a matrix norm.  Considering the resolution
of $A x = b$, if $A$ is invertible and if $\delta A$, $\delta x$ and
$\delta b$ are the numerical errors associated with the computation of
$A$, $x$ and $b$, respectively, then
\begin{equation}
   (A + \delta A)(x + \delta x) = b + \delta b.
   \label{eq:error_reso}
\end{equation}
Therefore, the condition number sets an upper bound for the error made on~$x$:
\begin{equation}
   \frac{\| \delta x \|}{\| x \|} \leq \kappa(A)\left[\frac{\| \delta A \|}{\| A \|} + \frac{\| \delta b \|}{\| b \|} \right].
   \label{eq:conditonnig_amp}
\end{equation}
The error on the iterative resolution of the U-RANS equations can
therefore be amplified by the HB source term. This amplification is
led by the condition number of the almost-periodic DFT matrix. This
also means that if the errors are small but the condition number is
high, and vice-versa, the computation can diverge too. However, the
errors can not be \emph{a priori} controlled, thus the need to
minimize the condition number.

In the case of periodic-flows, the DFT matrix is well-conditioned: the
uniform sampling for harmonically related frequencies leads to a
condition number equal to~$1$, which is the theoretical lower bound
for the condition number.  This is linked to the orthogonality of the
complex exponential family.  On the other hand, when the frequencies are arbitrary, it is usually
impossible to choose a uniform set of time instants over which the
almost-periodic DFT matrix~$A$ is well conditioned. In fact, it is common for uniformly-sampled
sinusoids at two or more frequencies to be nearly linearly dependent,
which causes them not to be orthogonal, leading to the
ill-conditioning encountered in practice. As the frequency set is
chosen by the user, the only degrees of freedom left to get a
well-conditioned matrix are the time levels.  The following section
describes two algorithms to find a non-uniform time sampling that
minimizes the almost-periodic DFT matrix condition number.


Under the assumption that all unsteady phenomena in a blade row during
stable operation are periodic and can be correlated with the rotation
rate $\Omega$ of the shaft, the dominant frequencies are those created
by the passage of the neighboring blades. In a multi-row turbomachine,
a blade row sandwiched between the upstream and downstream rows is
subjected to wake and potential effects. In practical turbomachines,
the blade counts of neighboring rows are generally different and
coprime. Consequently, a sandwiched blade row resolves various
combinations of the frequencies, which are additions and/or
subtractions of multiples of the blade passing frequencies: according
to Tyler and Sofrin~\cite{Tyler1962}, the $k$\textsuperscript{th}
frequency in the blade row $j$ is given by
\begin{equation}
  \omega_k^{rowj} = \sum_{i=1}^{nRows} n_{k,i}B_i(\Omega_i-\Omega_j).
   \label{eq:calcFreq}
\end{equation}
Here, $B_i$ and $\Omega_i$ are respectively the blade count and the
rotation rate of the $i$\textsuperscript{th} blade row, $n_{k,i}$
is the $k$\textsuperscript{th} set of $nRows$ integers driving the frequency combinations. It must
be noted that only the blade rows that are mobile relative to the
considered $j$ one contribute to its temporal frequencies and that
every blade row solves its own set of frequencies and thus its own set
of time levels.  To set up a HB computation for a multistage
configuration, it is of course impossible to use each and every 
possible $n_{k,i}$, and the user has to choose which frequency
combinations will appear in the computation of each row.

In the literature, Gopinath~\emph{et al.}~\cite{gopinath2007three} and
Ekici \& Hall~\cite{Ekici2007} assessed their implementation of the
Harmonic Balance on a 2D multi-stage compressor (namely configuration
D). It is composed of a rotor sandwiched by two stators having
 32, 40 and 50~blades, respectively. Various combinations of the stators
BPFs are considered, but always with evenly-spaced time levels sampling the
largest period.  While Gopinath~\emph{et al.} use $2N+1$ samples,
Ekici \& Hall over-sample this period with $3N+1$ time levels. This
leads to a rectangular $(2N+1)\times(3N+1)$ almost-periodic Fourier
Matrix and requires the computation of its Moore-Penrose
pseudo-inverse. The chosen frequencies and the \emph{a posteriori}
associated condition numbers of the above references are given
Tab.~\ref{tab:literature_multistage}.  For $N=4$, the $3N+1$ instants
oversampling approach of Ekici \& Hall efficiently reduces the
condition number. But for this case, the use of evenly-spaced time
levels is sufficient as the condition number seems to be small enough
for the considered magnitude of unsteadiness.  However, such an
approach fails when dealing with more widely-separated frequencies as
illustrated in the present contribution. Moreover, using an oversampling increases
the CPU cost and the required memory as the number of steady computations
to solve simultaneously is higher. These two reasons highlight the
need for a non-uniform HB method as proposed in the current paper.
\begin{table}[htbp]
  \centering
  \begin{tabular}{|r|*{6}{c|}}
    \hline
    & \multicolumn{2}{c|}{Frequencies} & \multicolumn{4}{c|}{$\kappa(A)$} \\
    \cline{2-7}
    & $n$ S$1$ & $n$ S$2$ & EQUI $2N+1$ &  EQUI $3N+1$ & APFT & OPT  \\
    \hline
    \hline
    $N=2$ & 1 & 0 & $\mathbf{3.79}$ & $3.00$ & $1.72$ & $1.08$ \\
    Ref.~\cite{gopinath2007three}& 0 & 1 & & & & \\
    \hline
    \hline
    & 1 & 0 &  & & & \\
    $N=3$& 0 & 1 &$5.40$ & $\mathbf{3.84}$ & $1.71$ & $1.00$ \\
    Ref.~\cite{Ekici2007}& 1 & 1 & & & &\\
    \hline
    \hline
    & 1 & 0 & & & & \\
    $N=4$& 0 & 1 & $\mathbf{11.25}$ & $2.07$ & $3.46$ & $1.13$ \\
    Ref.~\cite{gopinath2007three}& 1 & 1 & & & &\\
    & 1 & -1 & & & &\\
    \hline
    \hline
    & 1 & 0 & & & &\\
    & 0 & 1 & & & &\\
    & 1 & 1 & & & &\\
    $N=7$ & 1 & -1 & $\mathbf{16.66}$ & $14.61$ & $12.95$ & $1.00$ \\
    Ref.~\cite{gopinath2007three}& 2 & 0 & & & &\\
    & 2 & -1 & & & &\\
    & 2 & 1 & & & &\\
    \hline
  \end{tabular}
  \caption{Frequency combinations and associated condition number of
    computations made in the literature.}
  \label{tab:literature_multistage}
\end{table} 

\subsection{Boundary conditions for sector reduction}
\label{sec:bound-cond-sect}

Section ??? showed how to contain the problem
size by reducing the time span over which the solution is sought. In
the following sections, it is explained how to cut down the mesh size
by using a grid that spans only one blade passage per row.

\subsubsection{Phase-lagged azimuthal boundary conditions}
\label{sec:BC}

In a single blade passage computation of a multi-row configuration,
the phase-lag condition~\cite{Erdos1977} needs to be used to take the
space-time periodicity into account.  It states that the flow in one
blade passage~$\theta$ is the same as next blade
passage~$\theta+\Delta\theta$ but at another time~$t+\delta t$:
\begin{equation}
  W\left(\theta+\Delta\theta,t \right) = W\left(\theta,t+\delta t \right),
  \label{eq:choro}
\end{equation}
where $\Delta \theta$ is the pitch of the considered row. Assuming
that every temporal lag is associated with a rotating wave of rotational
speed $\omega_k$, the constant time lag can be expressed as
\begin{equation}
  \delta t = \frac{\beta_k}{\omega_k},\quad\forall k,
\end{equation}
where 
\begin{equation}
  \begin{split}
  \beta_k &= 2\pi \text{sign}(\omega_k)\left(
    1-\frac{1}{B_j}
    \sum_{i\neq j} n_{k,i}B_i \right), 
  \end{split}
\end{equation}
the $n_{k,i}$ being the integers specified for the computation of the
frequencies from Eq.~\eqref{eq:calcFreq}, $B_i$ the number of
blades in row $i$ and subscript $j$ denoting the current row.

The phase-lag condition was adapted to the time-domain HB by
Gopinath~\emph{et al.}~\cite{Gopinath2005}. The derivation starts with
the almost-periodic Fourier transform of Eq.~\eqref{eq:choro}:
\begin{equation}
  \sum_{k=-N}^{N} \widehat{W}_k\left(\theta+\Delta\theta,t \right)
  e^{i\omega_k t} = \sum_{k=-N}^{N} \widehat{W}_k\left(\theta,t
  \right)e^{i\omega_k \delta t} e^{i\omega_k t}.
  \label{eq:choroFourier_2}
\end{equation}
Thus, the flow spectrum from one blade passage is equal to that of the
next blade passage modulated by the inter-blade phase angle $\beta_k$:
\begin{equation}
  \widehat{W}_k\left( \theta+\Delta\theta,t \right) =
  \widehat{W}_k\left( \theta,t \right) e^{i\omega_k \delta t} = 
  \widehat{W}_k\left( \theta,t \right) e^{i\beta_k}.
  \label{eq:tmp}
\end{equation}
Using the same notation as previously, the following matrix
formulation is obtained:
\begin{equation}
  W^{\star} = A^{-1}MAW^{\star}\left( \theta \right),
  \label{eq:choroMatrix}
\end{equation}
where
\begin{equation}
  M = diag \left( -\beta_N,\ldots,\beta_0,\ldots,\beta_N \right),
\end{equation}
and $A^{-1}$ is given by Eq.~\eqref{MatriceA}.

\subsubsection{Stage coupling}
\label{sec:stageCoupling}

Each blade row has its own frequency set and therefore its own time
sampling. Therefore, the $n$\textsuperscript{th} time level in the
$j$\textsuperscript{th} and $(j+1)$\textsuperscript{th} rows do not
necessarily match the same physical time.  Consequently, at the
interface between adjacent blade rows, the flow field on the donor
side needs to be generated for all the time levels of the receiver
side using a spectral interpolation. A non-abutting join interface is
used to perform the spatial communications between the two rows
\cite{Lerat1996}. In order to account for the pitch difference and
relative motion, a duplication of the flow is carried out in the
azimuthal direction using the phase-lag periodicity. Moreover, as
described in Ref.~\cite{Sicot2012}, the time levels at the interface
are oversampled and filtered to prevent aliasing.

% section almost_periodic_flows (end)