%!TEX root = ../main.tex
\chapter{Harmonic balance methods} % (fold)
\label{cha:harmonic_balance_methods}

\section{Periodic flows} % (fold)
\label{sec:periodic_flows}

If the mean flow variables $W$~are periodic in time of period $T =
2\pi/\omega$, so are the residuals $R(W)$ and the Fourier series of
Eq.~\eqref{eq:semiDiscNS} is
\begin{equation}
  \label{eq:seriefour}
  \sum_{k=-\infty}^\infty \left(ik\omega
    V\widehat{W}_k+\widehat{R}_k\right)e^{ik\omega t}=0,
\end{equation}
where $\widehat{W}_k$ and $\widehat{R}_k$ are the Fourier coefficients
of $W$ and $R$ corresponding to the mode~$k$:
\begin{equation}
   W(t) = \sum_{k=-\infty}^{\infty} \widehat{W}_k e^{i k\omega t},\quad
   R(t) = \sum_{k=-\infty}^{\infty} \widehat{R}_k e^{i k\omega t}.
   \label{eq:fourierWsf}
\end{equation}
The complex exponential family forming an orthogonal basis, the only
way for Eq.~\eqref{eq:seriefour} to be true is that the weight of
every mode $k$ is zero, which leads to an infinite number of steady
equations in the frequency domain:
\begin{equation}
  \label{eq:orthodelim}
  ik\omega V\widehat{W}_k+\widehat{R}_k=0, \quad \quad \forall k \in
  \mathbb{Z}.
\end{equation}
McMullen~\emph{et al.}~\cite{McMullen2001,McMullen2002,McMullen2006}
solve a subset of these equations up to mode~$N$, $-N\leq k\leq N$,
yielding the Non-Linear Frequency Domain (NLFD) method.

%As the present method has to be implemented in a time-domain solver,
%these equations are cast back into the time domain by mean of an
%Inverse Discrete Fourier Transform (IDFT) and the Time Spectral
%Method~(TSM)~\cite{Gopinath2005} is retrived.
The principle of the time-domain Harmonic Balance approach, sometimes
referred to as Time Spectral Method (TSM)~\cite{Gopinath2005,
  Sicot2008}, is to use an Inverse Discrete Fourier Transform (IDFT)
to cast the equations back into the time domain.  The IDFT then
induces linear relations between Fourier coefficients $\widehat{W}_k$
and a uniform sampling of $W$ at $2N+1$ instants in the period:
\begin{equation}
  W_n=\sum_{k=-N}^N\widehat{W}_k\exp(i\omega n\Delta t),\quad \quad 0 \leq n < 2N+1,
\end{equation}
with $W_n \equiv W(n \Delta t)$ and $\Delta t=T/(2N+1)$. This leads to
a new system of $2N+1$~mathematically steady equations coupled by a
source term:
\begin{equation}
  \label{eq:hbttime}
  R(W_n)+VD_t(W_n)=0, \quad \quad 0 \leq n < 2N+1.
\end{equation}
The source term $VD_t(W_n)$ appears as a high-order formulation of the
initial time derivative in Eq.~\eqref{eq:semiDiscNS}. This new time
operator connects all the time levels and can be expressed
analytically as
\begin{equation}
\label{eq:dt}
  D_t(W_n)=\sum_{m=-N}^{N} d_m W_{n+m},
\end{equation}
with
\begin{equation}
  d_m=
  \begin{cases}
    \frac{\pi}{T}(-1)^{m+1}\csc\left(\frac{\pi
        m}{2N+1}\right) &, \, m\neq 0,\\
    0 &, \, m=0.
  \end{cases}
\end{equation}
This equation clearly states that the source term is real for periodic flows.
A similar derivation can be made for an even number of instants, but
it is proved in Ref.~\cite{Weide2005} that it can lead to a numerically unstable odd-even
decoupling. % and as a consequence, the method can become
%unstable. % Time-dependent boundary conditions could also benefit from
% such a derivation, but this is not an issue for external aerodynamic
% applications and has not been done yet.

A pseudo-time ($\tau_n$) derivative is added to
Eqs.~\eqref{eq:hbttime} to march the equations in pseudo-time to the
steady-state solutions of all the instants:
\begin{equation}
  \label{eq:pseudohbttime}
  V\frac{\partial W_n}{\partial\tau_n} + R(W_n)+VD_t(W_n)=0, \quad \quad
  0 \leq n < 2N+1.
\end{equation}
This time step is defined locally in a given cell and can
  be different for all the HB instants. For stability reasons, its
computation is modified~\cite{Weide2005} to take into account
the additional source term,
\begin{equation}
  \label{eq:stabdeltat}
  \Delta\tau_n=\text{CFL}\frac{V}{\|\xi_n\|+\omega NV}.
\end{equation}
The extra term $\omega NV$ is added to the spectral radius $\|\xi_n\|$ to
restrict the time step.  Equation~\eqref{eq:stabdeltat} implies that a
high frequency and/or a high number of harmonics~$N$ can considerably
restrict the time step, especially for explicit Runge Kutta time
integration scheme, as mentioned in~\cite{Hall2002}. % Actually, it has been
% observed~\cite{Hall2002} that the convergence of the method slows down
% for increasing~$N$\todo{compacter}. All the cited references use
% explicit schemes, such as Runge-Kutta, to carry out the pseudo-time
% integration. Their limit stability criteria on CFL numbers is very
% sensitive to such a restriction.
%Conversely, implicit schemes are more
%stable and allow larger CFL numbers. 
Several implicit schemes, which are theoretically unconditionally stable and thus allow larger
CFL number, have been derived for the HB method: Krylov-space based
methods are used
in~\cite{FLD:FLD2111,woodgate09:_implic_harmon_balan_solver_for}, and
Antheaume~\emph{et
  al.}~\cite{antheaume11:_implic_time_spect_method_for} propose a
point Jacobi algorithm. The present paper uses the block-Jacobi
algorithm derived in Ref.~\cite{Sicot2008} to improve robustness and
efficiency.

This time-domain harmonic balance method has been implemented in the
\emph{elsA} solver~\cite{cambier2012} developed by ONERA and
CERFACS. This code solves the RANS equations using a cell-centered
approach on multi-blocks structured meshes.  Using the HB method, significant savings in
CPU cost have been observed in various applications such as dynamic
derivatives computation~\cite{Hassan2011},
aeroelasticity~\cite{Dufour2010} and rotor/stator
interactions~\cite{Sicot2012}. However, this approach is limited to
periodic flows (\emph{i.e.} a single fundamental frequency) and is
unfit when the main frequencies of the system are not integers multiple
of each other (such as multi-stage turbomachines for instance). The single-frequency HB method is therefore
extended to the case where the flow is not periodic in time but is
almost periodic.

% section periodic_flows (end)

% chapter harmonic_balance_methods (end)