%!TEX root = ../../../adrien_gomar_phd.tex
\chapter{Advantages of Fourier-based time methods}
\label{cha:advantages}

\chabstract{In this chapter, the advantage of using the 
harmonic balance operator to estimate the 
temporal derivative is assessed. First, an analytical
function, whose derivative is known, is chosen to
compare the accuracy of the estimation of the derivative.
The harmonic balance operator is compared to
two finite-difference schemes. It is shown that 
the HB operator is spectral accurate. The same
function is then used along with the linear advection toy problem
to ensure that this property is kept when iteratively solving
an equation. Finally, a function composed of two separated frequencies
is tested and it is shown that the multi-frequential
is more adapted to this configuration.}

\minitoc
\newpage


\section{Comparison of the harmonic balance operator and finite difference schemes}
\label{sec:hb_operator}
%!TEX root = ../../../adrien_gomar_phd.tex

To assess the capability of harmonic balance operator define 
in Eq.~\eqref{eq:sm_hb_mono_source_term_analytic} for the mono-frequential
formulation or in Eq.~\eqref{eq:sm_multi_spectral_operator} 
its multi-frequential equivalent to
capture time-periodic fields, it is tested on a pure
harmonic signal of the form
\begin{equation}
    \label{eq:sum_sin}
    u(t) = \cos(\omega t) + \sin(2 \omega t) +
    \cos(3 \omega t) + \sin(4 \omega t) + \cos(5 \omega t),
\end{equation}
where $\omega = 2 \pi f$ and $f$ is the temporal frequency of
the phenomenon.
The analytical time-derivative is then:
\begin{equation}
    \label{eq:sum_sin_deriv}
    \frac{\partial u}{\partial t} = 
    \omega\left[ -\sin(\omega t) + 
    2\cos(2 \omega t) -
    3\sin(3 \omega t) + 
    4\cos(4 \omega t) -
    5\sin(5 \omega t)\right]
\end{equation}
This derivative will be compared to the estimation
given by the HB operator defined in Eq.~\eqref{eq:sm_hb_mono_source_term_analytic}
or Eq.~\eqref{eq:sm_multi_spectral_operator}, depending 
on the formulation,
and to two finite difference schemes,
a second-order centered scheme:
\begin{equation}
    \frac{\partial u}{\partial t} (t=t_q) \approx 
    \frac{u^{q+1} - u^{q-1}}{2 \Delta t},
    \label{eq:hb_op_center2}
\end{equation}
and a fourth-order centered scheme:
\begin{equation}
    \frac{\partial u}{\partial t} (t=t_q) \approx 
    \frac{-u^{q+2} + 8 u^{q+1} - 8 u^{q-1} + u^{q-2}}{12\Delta t}.
    \label{eq:hb_op_center4}
\end{equation}

The application of these three schemes applied 
to the estimation of the time-derivative of signal Eq.~\eqref{eq:sum_sin}
is shown
in Fig.~\ref{fig:hb_operator_sample}.
\begin{figure}[htbp]
  \centering
  \subfigure[$7$ samples]{\includegraphics[width=.4\textwidth]{HB_OPERATOR_PAPER_S7.pdf}}
  \subfigure[$11$ samples]{\includegraphics[width=.4\textwidth]{HB_OPERATOR_PAPER_S11.pdf}}
  \subfigure[$15$ samples]{\includegraphics[width=.4\textwidth]{HB_OPERATOR_PAPER_S15.pdf}}
  \subfigure[$19$ samples]{\includegraphics[width=.4\textwidth]{HB_OPERATOR_PAPER_S19.pdf}}
  \caption{Time-derivative estimation by the harmonic balance operator,
  the 2\textsuperscript{nd} order and 4\textsuperscript{th} finite difference schemes.}
  \label{fig:hb_operator_sample}
\end{figure}
Four samples
are tested: $7$, $11$, $15$ and $19$. This corresponds
to respectively $3$, $5$, $7$ and $9$ harmonics
for the HB operator. One can see that, the more the number
of samples, the better the prediction of the time-derivative.
As expected, the 4\textsuperscript{th} order finite difference
scheme does a better job than the 2\textsuperscript{nd} order.
For $19$ samples, the 4\textsuperscript{th} order finite difference
scheme almost fits the analytical solution. In opposite,
the HB operator prediction is superimpose with the analytical solution
with $11$ samples, i.e. $N=5$ harmonics. After that, increasing the
number of harmonics (or samples, recall that the number of samples
in the HB is equal to $2N+1$ with $N$ the number of harmonics)
does not improve the solution as it is already superimposed with
the analytical results.

To quantitatively analyze the results, the 
$\mathcal{L}_2$-norm of the absolute error between the analytic
derivative and the estimation given by the different schemes
is plotted in Fig.~\ref{fig:hb_operator_error}.
\begin{figure}[htb]
  \centering
   \includegraphics[width=.55\textwidth]{HB_OPERATOR_ERROR.pdf}
   \caption{$\mathcal{L}_2$-norm of the error for each time-derivative
   schemes.}
  \label{fig:hb_operator_error}
\end{figure}
One can observe the classical converging slope
equal to the order of the finite-difference schemes for the
$2$\textsuperscript{nd} and the $4$\textsuperscript{th}
order finite difference schemes. 
Adding more samples constantly improves the estimation of 
the time-derivative.
Conversely, as a spectral operator, the HB time-derivative scheme 
has a different behavior. When the Nyquist-Shannon 
criteria~\cite{Nyquist1928, Shannon1949} is not
satisfied, the error is high as Fig.~\ref{fig:hb_operator_sample}
suggest. But as soon as this criteria is satisfied, the error
drastically decreases by about $15$ order of magnitude.
Actually, on pure
harmonic signals, the HB operator is exact as soon as the time
sampling satisfies the Shannon criteria. The $\approx 10^{-13}$
remaining error is due to round-off errors.
From this study, it can be infer that the behavior
of the harmonic balance operator is not monotonic
and needs thus a deeper investigation.

\section{Periodic convection of a sum of sine functions}
\label{sec:sum_sine}
%!TEX root = ../../../adrien_gomar_phd.tex

In this second example, the linear advection toy problem 
defined in Sec.~\ref{sec:toy_convection} is used with
a perturbation 
in the form of a finite sum of sine functions, similar to the one used
in Sec.~\ref{sec:hb_operator},
applied at the left boundary:
\begin{equation}
    u_l(t) = \cos(\omega t) + \sin(2 \omega t) +
    \cos(3 \omega t) + \sin(4 \omega t) + \cos(5 \omega t).
    \label{eq:sum_injected_fct}
\end{equation}
Harmonic balance computations are run with 1 to 10~harmonics.
For each computation, we show spatial distributions of the solution
at three time instances, namely, $t=0$, $t=T/3$ and $t=2T/3$.
Since these instances are not necessarily used in the HB discretization,
a temporal interpolation is performed.
To do so, the frequency content of the HB solution is used
together with an inverse Fourier transform on the time-vector
$[0, T/3, 2T/3]$.
Figure~\ref{fig:inj_sine_results} depicts the results of HB computations
using 1 to 6~harmonics. The analytical solution is also reported for comparison.

\begin{figure}[htb]
  \centering
  \subfigure[$N=1$]{\includegraphics[width=.35\textwidth]{convection_sin_N1.pdf}}
  \subfigure[$N=2$]{\includegraphics[width=.35\textwidth]{convection_sin_N2.pdf}}
  \subfigure[$N=3$]{\includegraphics[width=.35\textwidth]{convection_sin_N3.pdf}}
  \subfigure[$N=4$]{\includegraphics[width=.35\textwidth]{convection_sin_N4.pdf}}
  \subfigure[$N=5$]{\includegraphics[width=.35\textwidth]{convection_sin_N5.pdf}}
  \subfigure[$N=6$]{\includegraphics[width=.35\textwidth]{convection_sin_N6.pdf}}
  \caption{Linear advection of a sum of sine functions: 
  numerical solutions at different time instances for different numbers of harmonics.}
  \label{fig:inj_sine_results}
\end{figure}

The accuracy of the solution 
improves with the number of harmonics,
until it reaches the frequency content
of the injected signal, i.e. 5~harmonics.
After that, the results of the HB computations are
superimposed with the analytical solution. 

The $\mathcal{L}_2$-norm of the error as a function of the number of harmonics
is shown in Fig.~\ref{fig:conv_sum_sine}. Two results are displayed:
one for the reference mesh (2,000 grid points) and one for
a refined mesh (4,000 grid points).
\begin{figure}[htb]
  \centering
  \includegraphics[width=.4\textwidth]{convection_sin_error.pdf}
  \caption{Linear advection of a sum of sine functions: convergence of the HB method error.}
  \label{fig:conv_sum_sine}
\end{figure}
The convergence of the HB computations is slow  for
$N \leq 4$. However, when the number of harmonics composing
the injected function is reached ($N=5$), the error is minimum and computing
more harmonics does not change the error.
It seems that the convergence rate 
of Fourier-based time methods is inherently linked to the spectrum of the
temporal phenomenon that one wants to capture. Here a finite discrete spectrum composed of only five harmonics
is imposed.
The value of the plateau obtained 
after $N=5$ is representative of the error introduced by the different
discretizations. In fact, refining the mesh changes this value
without modifying the error levels of the lower harmonics points.
Note that the error is the true residual, meaning
that the computation is compared to the analytical result.
This is why the only way to have a zero machine value like in
Fig.~\ref{fig:hb_operator_error}, 
would be to have an infinite number of grid points and
pseudo-iterations.

The temporal discrete Fourier transform
of the computational results is compared to the
analytical results in Fig.~\ref{fig:dft_sin}.
\begin{figure}[htb]
  \centering
  \includegraphics[width=.4\textwidth]{convection_sin_dft.pdf}
  \caption{Linear advection of a sum of sine functions: 
  discrete Fourier transform.}
  \label{fig:dft_sin}
\end{figure}
When the number of harmonics grows in the spectral computations,
the Fourier transform gets closer to the analytical solution.
When the whole frequency content of the injected 
function is contained in the HB solution, 
the numerical results are superimposed with the analytical ones.
For intermediate sampling frequencies, as for 
instance the three-harmonics HB computation, 
the resolved harmonics have higher amplitudes 
than the exact one, since they compensate for harmonics that are not resolved.

When the number of harmonics composing the spectrum of the
computed signal is reached, the computational results are superposed
with the analytical ones to with plotting accuracy, namely we obtain spectral accuracy.
This is the main advantage of Fourier-based time methods: when the
signal has a narrow spectrum, as it is the case for the sum
of sine function used here, the
convergence can be very fast compared to a classical time-marching scheme.

\FloatBarrier


\section{Capturing a multiple-frequency signal}
\label{sec:adv_multifreq}
%!TEX root = ../../../adrien_gomar_phd.tex

\mytodo{Reponse forcee => multiple de freq de revolution, + mettre ratio frequence
calcul dream AEL}

In Section~\ref{sec:sm_hb_multi}, the multi-frequential harmonic
balance approach has been presented. In this method,
the frequencies can be chosen arbitrarily. This becomes particularly
interesting when dealing with signal/ flow field composed of segregated
frequencies. For instance, let us consider the linear advection toy problem
as defined in Sec.~\ref{sec:toy_convection} with 
a perturbation 
in the form of a sum of two sine functions,
applied at the left boundary:
\begin{equation}
    u_l(t) = \sin(\omega t) + \sin(22 \omega t).
    \label{eq:multifreq_inj_func}
\end{equation}

\subsection{Toward aeroelasticity of contra-rotating open rotors}
The reason why we studied the multi-frequential harmonic balance
approach for the aeroelasticity of contra-rotating open rotors
is that it is a problem where the major frequencies are most
likely to be segregated as shown previously in Chap.~\ref{cha:cror} and
Chap.~\ref{cha:ael}. In this framework, the multi-frequential
harmonic balance approach looks very promising according to the results
seen in this Chapter.

\subsection{Using a mono-frequential approach}

Obviously, computing the advection of such a signal using
a classical time-marching scheme would require to discretize the
smaller period. The largest frequency
(here $f_2 = 22$~Hz) acts as a bottleneck as the time-step will be chosen
according to this frequency. The cost scales thus with the ratio of $f_2 / f_1 = 22$. This
means that compared to a computation where only $f_1$ or only $f_2$
is involved, the cost will be multiplied by~ \mbox{$(2 \times 22 + 1) / (2 \time 1+1) = 9$}.

This holds true when computing the solution with the mono-frequential
harmonic balance approach. In fact, the frequencies can not be chosen arbitrarily.
Therefore, to compute such a configuration, a $N=22$ harmonic computation
will be needed to be spectral accurate. To emphasize that, mono-frequential
HB computations are run with 1 to 25~harmonics.
As made in the Section~\ref{sec:sum_sine}, 
for six chosen computations of the~25 computations, 
we show spatial distributions of the solution
at three time instances, namely, $t=0$, $t=T/3$ and $t=2T/3$.
It is shown in Fig.~\ref{fig:inj_multifreq_tsm}.
\begin{figure}[htb]
  \centering
  \subfigure[$N=1$]{\includegraphics[width=.35\textwidth]{convection_multifreq_tsm_N1.pdf}}
  \subfigure[$N=5$]{\includegraphics[width=.35\textwidth]{convection_multifreq_tsm_N5.pdf}}
  \subfigure[$N=11$]{\includegraphics[width=.35\textwidth]{convection_multifreq_tsm_N11.pdf}}
  \subfigure[$N=16$]{\includegraphics[width=.35\textwidth]{convection_multifreq_tsm_N16.pdf}}
  \subfigure[$N=22$]{\includegraphics[width=.35\textwidth]{convection_multifreq_tsm_N22.pdf}}
  \subfigure[$N=23$]{\includegraphics[width=.35\textwidth]{convection_multifreq_tsm_N23.pdf}}
  \caption{Linear advection of a sum of two segregated sine functions: 
  numerical solutions at different time instances for different numbers of harmonics.}
  \label{fig:inj_multifreq_tsm}
\end{figure}
Again, the accuracy in capturing the injected function
improves with the number of harmonics,
until it reaches the frequency content
of the injected signal, i.e. 22~harmonics.
After that, the results of the HB computations are
superimposed with the analytical solution. 
The problem, with such a segregation of frequencies, is that 
the mono-frequential version suffers the same
problems as a classical time-marching scheme in terms of 
computational cost.

To quantitatively analyze the results,
the discrete $\mathcal{L}_2$-norm of the error 
in time is computed over all the time instances
at each grid points over the domain.
Then, the average in space is computed.
It is shown in Fig.~\ref{fig:conv_multifreq_tsm} for the
mono-frequential HB computations ranging from one to~25
harmonics.
\begin{figure}[htb]
  \centering
  \includegraphics[width=.5\textwidth]{convection_multifreq_error.pdf}
  \caption{Linear advection of a sum of two segregated sine functions: convergence of the mono-frequential HB method error.}
  \label{fig:conv_multifreq_tsm}
\end{figure}
When the number of harmonics
used to compute the solution is higher than the content of the spectrum,
then the error decreases drastically. The spectral accuracy is retrieved
but only starting at $N=22$.
In fact, similar as in Sec.~\ref{sec:sum_sine},
the injected function is indefinitely differential and periodic
leading an infinite convergence slope. We can observe a slight local convergence
for the $N=5$ harmonics HB computation. This is due to the fortunate 
capture of the low-frequency pattern of the injected function.

\subsection{Using a multi-frequential approach}

One of the advantage of the multi-frequential HB method introduced in Sec.~\ref{sec:sm_hb_multi}
and used in this thesis is that it can take arbitrary frequencies into account.
In the case of an injected signal with a large frequency segregation, the
benefit might be tremendous. Let us consider again the signal defined in 
Eq.~\eqref{eq:multifreq_inj_func} and compute one HB simulation using 
$f_1=1$~Hz and $f_2=22$~Hz as input frequencies. This gives a computation
of two coupled calculation
that is nine times faster than the $N=22$ converged mono-frequential
HB computation.
Again
we show spatial distributions of the solution
at three time instances, namely, $t=0$, $t=T/3$ and $t=2T/3$
in Fig.~\ref{fig:inj_multifreq_hb}.
\begin{figure}[htb]
  \centering
  \includegraphics[width=.5\textwidth]{convection_multifreq_hbt_N2.pdf}
  \caption{Linear advection of a sum of two segregated sine functions: 
  numerical solutions at different time instances for different numbers of harmonics using the
  multi-frequential harmonic balance method.}
  \label{fig:inj_multifreq_hb}
\end{figure}
With only two input frequencies, the multi-frequential
HB solution is superimposed with the analytical solution.
Moreover, the $\mathcal{L}2$-norm of the error is 
exactly the same as the one of the $N=22$ mono-frequential
approach.
\mytodo{conclusion partielle: c'est trop cool !! -> rapprocher de Campbell}


\chconclu{}
