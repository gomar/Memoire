%!TEX root = ../../../adrien_gomar_phd.tex

As previously stated, the main unsteady
phenomena encountered in CROR can be correlated with the blade passing
frequency (see Sec.~\ref{sec:cror_unsteady}).
In addition to that, the aeroelastic phenomenon
studied here, namely blade flutter sensibility simulation, has a vibration frequency that
is imposed (forced movement simulation) and depends on the structural proper modes
of the structure (see Sec.~\todo{sec blade flutter}).
In general, these are not harmonically related nor
of the same order of magnitude. Hence the use of the
multi-frequential formulation of the HB approach. 

The condition number $\kappa$ of a matrix $E$ is defined as:
\begin{equation}
  \kappa (E) = \kappa (E^{-1}) = \| E \| \cdot \| E^{-1} \|, \quad
    \kappa(E) \geq 1,
\end{equation}
where $\| \cdot \|$ denotes a matrix norm. Considering the resolution
of the system of equation
$A x = b$, if $A$ is invertible and if $\delta A$, $\delta x$ and
$\delta b$ are the numerical errors associated with the computation of
$A$, $x$ and $b$, respectively, then:
\begin{equation}
   (A + \delta A)(x + \delta x) = b + \delta b.
   \label{eq:error_reso}
\end{equation}
By definition, the condition number sets an upper bound for 
the error made on~$x$:
\begin{equation}
   \frac{\| \delta x \|}{\| x \|} \leq 
   \kappa(A)\left[\frac{\| \delta A \|}{\| A \|} + 
   \frac{\| \delta b \|}{\| b \|} \right].
   \label{eq:conditonnig_amp}
\end{equation}
By transposing this to our problem, one can say that
the error on the iterative resolution of the governing equations can
therefore be amplified by the harmonic balance source term.
This amplification is
led by the condition number of the DFT matrix $E$. 


In the mono-frequential formulation, the logical sampling is the uniform one
which has the good property of providing
a well conditioned DFT matrix $E$. In fact, $E$ is orthogonal giving 
the smallest condition number $\kappa (E) = 1$
In the multi-frequential framework,
the condition number of the DFT matrix $E$ is not always unity and
varies under frequencies and time instances change~\cite{Kundert1988}. 
The frequencies
being imposed by the problem that is simulated,
the only degrees of freedom left to control the condition
number are the time instances. 
Moreover, the amplitude of the unsteadinesses, represented by $\delta x$
can not be \emph{a priori} controlled as this is ruled by the flow physics. 
Therefore, the condition
number must be minimized using the time instances.

All variations of the HB approach proposed in the literature rely on 
a uniform time sampling of the longest period of interest 
(though the number of samples can differ). 
This uniform time sampling can raise stability issues.
To emphasis this, let us consider two independent frequencies $f_1$
and $f_2$ that plays the role, respectively, of the blade passing frequency and
the vibration frequency. The two frequencies are arbitrarily chosen between~1
and $10,000$~Hz and the corresponding
condition number of the DFT matrix $E$
using evenly space time instances is shown in Fig.~\ref{fig:algo_equi_assessment}.
100~points are used to discretize each frequency interval giving a frequency step
of 100~Hz.
The problem being symmetric in $(f_1, f_2)$, so are the results.
Moreover, the structure of the results seems to indicate that only the
ratio of $f_2$ over $f_1$ is important as the $y=-x$ pattern of the curve
seems to duplicates in the $y=x$ direction, meaning that only the former
is important.

Almost half of the set of frequencies have a DFT matrix $E$
whose condition number is superior to ten.
The minimum values are obtained with harmonically related couple
of frequencies. In fact, the white zones in Fig.~\ref{fig:algo_equi_assessment}
are the regions where $f_2 = n f_1$ with $n \in \mathbb{N}$ or $1/n \in \mathbb{N}$.
Elsewhere, the condition number is large and grows exponentially.
\begin{figure}[htb]
  \centering
  \includegraphics*[width=0.5\textwidth]{algo_equi_assessment.pdf}
  \caption{Condition number of the discrete Fourier transform matrix $E$
  using two independent frequencies and evenly space time instances.}
  \label{fig:algo_equi_assessment}
\end{figure}
To highlight this, the minimum, maximum, mean and 
standard deviation (noted $\sigma$) values of the
previous example are summarized in Tab.~\ref{tab:hb_algo_equi}.
\begin{table}[htb]
  \ra{1.3} 
  \centering
  \begin{tabular}{cccc}
    \toprule
    min & max & mean & $\sigma$ \\
    \midrule
    $1.0$ & $9.4\e{16}$ & $1.5\e{14}$ & $2.8\e{15}$ \\
    \bottomrule
  \end{tabular}
  \caption{Condition number of the discrete Fourier transform matrix $E$: 
  statistics for two independent frequencies using evenly spaced time instances.}
  \label{tab:hb_algo_equi}
\end{table}  
The values of the maximum, mean and standard deviation are tremendous
and the standard deviation is greater than the mean
($\sigma > $ mean) preventing the blindly use of such a sampling strategy for 
HB computations.
In numerical methods, it is common to deal with ill-conditioned
problems. However, we will show below that HB results are 
very sensitive to the condition number of the DFT matrix $E$.
To do so, the linear advection toy problem
presented in Chap.~\ref{cha:toy_problems},
is used with varying condition number and input unsteadinesses.
