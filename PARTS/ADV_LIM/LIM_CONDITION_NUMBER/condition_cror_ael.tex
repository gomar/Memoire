%!TEX root = ../../../adrien_gomar_phd.tex

At the beginning of this thesis, we have seen that the main unsteady
phenomena encountered in CROR can be correlated with the blade passing
frequency (see Sec.~\ref{sec:cror_unsteady}).
In addition to that, the aeroelastic phenomenon
studied here, namely blade flutter, has a vibration frequency that
depends on the structural properties of the blade (see Sec.~\todo{sec blade flutter}). 
In general, these two
frequencies are not equal. Hence the use of the
multi-frequential formulation of the HB approach. 
In this multi-frequential framework,
the condition number of the DFT matrix $E$ is not always unity and
varies under frequencies and time instances change. The frequencies
being imposed by the problem that is simulated,
the only degrees of freedom left to control the condition
number are the time instances. 

The simpler choice of time instances are the evenly spaced ones on
the largest period. However, this choice leads to large condition number.
To emphasis this, let us consider two independent frequencies $f_1$
and $f_2$ that plays the role, respectively, of the blade passing frequency and
the vibration frequency. The two frequencies are arbitrarily chosen between~1
and $10,000$ Hz and the corresponding
condition number of the DFT matrix $E$
using evenly space time instances is shown in Fig.~\ref{fig:algo_equi_assessment}.
100~points are used to sample each frequency interval.
The problem being symmetric in $(f_1, f_2)$, so are the results.
Almost a half of the choice of frequencies have an DFT matrix $E$
whose condition number is superior to ten.
The minimum values are obtained with harmonically related couple
of frequencies. In fact, the white zones in Fig.~\ref{fig:algo_equi_assessment}
are the slope $n f_1$ where $n \in \mathbb{N}$ or $1/n \in \mathbb{N}$.
Elsewhere, the condition number is large. 
\begin{figure}[htb]
  \centering
  \includegraphics*[width=0.5\textwidth]{algo_equi_assessment.pdf}
  \caption{Condition number of the discrete Fourier transform matrix $E$
  using two independent frequencies and evenly space time instances.}
  \label{fig:algo_equi_assessment}
\end{figure}

Actually, the values of the condition number grows exponentially. To highlight
this, the minimum, maximum, mean and standard deviation (noted $\sigma$) values of the
previous example are summarized in Tab.~\ref{tab:hb_algo_equi}. 
The values of the mean and standard deviation are tremendous.

\begin{table}[htb]
  \ra{1.3} 
  \centering
  \begin{tabular}{cccc}
    \toprule
    min & max & mean & $\sigma$ \\
    \midrule
    $1.0$ & $9.4\e{16}$ & $1.5\e{14}$ & $2.8\e{15}$ \\
    \bottomrule
  \end{tabular}
  \caption{Condition number of the discrete Fourier transform matrix $E$: 
  statistics for two independent frequencies using evenly spaced time instances.}
  \label{tab:hb_algo_equi}
\end{table} 

To highlight numerically the issue, the two toy problems
presented in Chap.~\ref{cha:toy_problems},
are used with varying condition number.
