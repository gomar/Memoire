%!TEX root = ../../../adrien_gomar_phd.tex

In the turbomachinery literature, \citet{Gopinath2007} and
\citet{Ekici2007} assessed their implementation of the
HB on a 2D multi-stage compressor. 
It is composed of a rotor sandwiched by two stators having
32, 40 and 50~blades, respectively. Various combinations of the stators
BPFs are considered, but always with evenly-spaced time instances sampling the
largest period.  While \citet{Gopinath2007} use $2N+1$ samples,
\citet{Ekici2007} over-sample this period with $3N+1$ time instances. This
leads to a rectangular $(2N+1)\times(3N+1)$ almost-periodic Fourier
Matrix and is thus more CPU and memory demanding. 
The chosen frequencies and the \emph{a posteriori}
associated condition numbers of the above references are given
Tab.~\ref{tab:literature_multistage}.  For $N=4$, the $3N+1$ instants
oversampling approach of \citet{Ekici2007} efficiently reduces the
condition number. But for this case, the use of evenly-spaced time
instances is sufficient as the condition number seems to be small enough
for the considered magnitude of unsteadiness.  However, such an
approach fails when dealing with more widely-separated frequencies as
illustrated in the present contribution. 
Moreover, using an oversampling increases
the CPU cost and the required memory as the number of steady computations
to solve simultaneously is higher. These two reasons explain the
need for a non-uniform HB method as proposed by \citet{JGuedeney2013}
and an efficient algorithm to choose the time instances as presented here.
\begin{table}[htb]
  \centering
  \begin{tabular}{rcccc}
    \toprule
    \multicolumn{1}{c}{Reference} & \multicolumn{4}{c}{$\kappa(E)$} \\
    \multicolumn{1}{c}{and \# harmonics} & EVE $(2N+1)$ & EVE $(3N+1)$ & APFT & OPT \\
    \midrule
    \citet{Gopinath2007} ($N=2$) & $\mathbf{3.79}$ & $3.00$ & $1.72$ & $1.08$ \\
    \citet{Ekici2007} ($N=3$) & $5.40$ & $\mathbf{3.84}$ & $1.71$ & $1.00$ \\
    \citet{Gopinath2007} ($N=4$) & $\mathbf{11.25}$ & $2.07$ & $3.46$ & $1.13$ \\
    \citet{Gopinath2007} ($N=7$) & $\mathbf{16.66}$ & $14.61$ & $12.95$ & $1.00$ \\
    \bottomrule
  \end{tabular}
  \caption{Literature review of the condition number used in multi-frequential
  harmonic balance computations.}
  \label{tab:literature_multistage}
\end{table}
