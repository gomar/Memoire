%!TEX root = ../../../adrien_gomar_phd.tex

A pure harmonic signal is imposed at the left boundary condition
of the linear advection equation model problem:
\begin{equation}
   u_l (t) = 1 + \sin \left(2 \pi f t\right).
\end{equation}
The time instances of the harmonic balance
computations are chosen to reach varying condition numbers
such that $1 \leq \kappa (E) \leq 10$.  
The minimal condition number
$\kappa(E) = 1$ is obtained with evenly spaced time instances.
In fact, as the injected function is mono-frequential, 
the theoretical lower of the condition number is obtained by using evenly
spaced time instances.

As shown in the previous section, the condition number 
can, by definition, amplify the error made
on the iterative resolution of the linear advection equation.
This is illustrated in Fig.~\ref{fig:condition_number_local_amp} which 
shows the evolution of the results with a varying condition number.
The amplitude of the
sinusoidal function is either under or over-estimated when
$\kappa (E) \neq 1$. However, 
the higher the condition number, the worse the accuracy in capturing
the amplitude of the injected function.
\begin{figure}[htbp]
  \centering
  \includegraphics*[width=0.50\textwidth]{condition_number_local_amp.pdf}
  \caption{Linear advection of a sinusoidal function: numerical steady-state 
  solutions at $t=0$ for varying condition number.}
  \label{fig:condition_number_local_amp}
\end{figure}
