%!TEX root = ../../../adrien_gomar_phd.tex
\paragraph{Advection equation model problem}

A pure harmonic signal is imposed at the left boundary condition
of the linear advection equation problem presented 
in Sec.~\ref{sec:presentation_advection}
\begin{equation}
   u_l (t) = 1 + \sin \left(2 \pi f t\right).
\end{equation}
The minimal condition number
$\kappa(E) = 1$ is obtained with evenly spaced time instances.
In fact, as the injected function is mono-frequential, 
the theoretical lower bound of the condition number is obtained by using evenly
spaced time instances.
The time instances of the harmonic balance
computations are chosen to reach varying condition numbers
such that $1 \leq \kappa (E) \leq 10$.  

As shown in the previous section, the condition number 
can, by definition, amplify the error made
on the iterative resolution of the linear advection equation.
This is illustrated in Fig.~\ref{fig:condition_number_local_amp} which 
shows the evolution of the results with a varying condition number.
The amplitude of the
sinusoidal function is either under or over-estimated when
$\kappa (E) \neq 1$. However, 
the higher the condition number, the worse the accuracy in capturing
the amplitude of the injected function. Moreover, when $\kappa(E) \geq 6$,
the shape of the solution is even inverted which would lead to
bad conclusions if analyzed as it.
\begin{figure}[htp]
  \centering
  \includegraphics*[width=0.50\textwidth]{condition_number_local_amp.pdf}
  \caption{Linear advection of a sinusoidal function: numerical steady-state 
  solutions at $t=0$ for varying condition number.}
  \label{fig:condition_number_local_amp}
\end{figure}

\paragraph{Channel flow problem}
The previous example was based on the advection equation which
has the good property of having an analytical solution to 
interpret the results. The results show that the harmonic
balance solutions are very sensitive to the condition number.
To further analyze the condition number issue,
we consider the unsteady channel flow with 
oscillating back-pressure as described in 
Sec.~\ref{sec:channel_flow_problem}.
It is computed with a single
frequency oscillating back-pressure 
at the outlet: $f_1 = 3$~Hz, the second
frequency having a zero amplitude: $a_2= 0$
\begin{equation}
   p_{outlet} (t) = p_m \left[ 1 + a_1 \sin \left(2 \pi f_1 t\right) \right].
\end{equation}
This helps understanding the behavior of the HB source term
within the Navier--Stokes equations framework.

As the oscillating back-pressure is composed of only one frequency,
it is mono-frequential.
Thus, by using evenly-spaced time instances, the condition
number of the source term is unity $\kappa (E) = 1$. 
To highlight the issue related to the condition number,
the time instances are chosen to reach varying condition numbers
such that $1 \leq \kappa (E) \leq 3.43$.  
Two frequencies are
specified for the HB computation: $f_1$ and its first harmonic
$2f_1$. 


\begin{figure}[htp]
  \centering
  \subfigure[$a_1 = 0.01$]{\includegraphics[width=.45\textwidth]{CANAL2_RESIDUAL_VS_CONDITIONNING_AMP001_PPT.pdf}}
  \subfigure[$a_1 = 0.05$]{\includegraphics[width=.45\textwidth]{CANAL2_RESIDUAL_VS_CONDITIONNING_AMP005_PPT.pdf}}
  \caption{Relation between the condition number $\kappa (E)$ and the convergence of the solution.}
  \label{fig:canal_residual_vs_conditionning}
\end{figure}
The results in Fig.~\ref{fig:canal_residual_vs_conditionning} show that
for a condition number $\kappa (E) \geq 2.24$ and wave input amplitude
$a_1 = 0.05$, the computation diverges. However, the computations with
the same condition numbers but with a smaller input amplitude $a_1 = 0.01$
converge. In fact, the condition number amplifies the errors made
during the iterative process. When the input waves have a smaller
amplitude, the iterative errors are slighter.

The problem is that for a given configuration, the 
amplitude of the computed unsteady phenomena can not
be \emph{a priori} known. 

Varying condition number values have been found in the literature.
Precisely, \citet{Gopinath2007} and
\citet{Ekici2007} assessed their implementation of the
multi-frequential HB method on a 2D multi-stage compressor. 
It is composed of a rotor sandwiched by two stators having
32, 40 and 50~blades, respectively. Various combinations of the stators
blade passing frequencies are considered, 
but always with evenly-spaced time instances sampling the
largest period.  While \citet{Gopinath2007} use $2N+1$ samples (noted EVE $2N+1$),
\citet{Ekici2007} over-sample this period with $3N+1$ time instances
(noted EVE $3N+1$). This
leads to a rectangular $(2N+1)\times(3N+1)$ almost-periodic Fourier
matrix that gives thus HB computations that are more CPU and memory demanding
(see Sec.~\ref{sec:sm_hb_cost}). 
The chosen frequencies and the \emph{a posteriori}
associated condition numbers of the above references are given in
Tab.~\ref{tab:literature_multistage}.  
In bold text is highlighted the condition number used in the
original computations. The values vary between $2.07$
and $16.66$ which can lead to divergence with some
high amplitude unsteadinesses.
For $N=4$, the
oversampling approach of \citet{Ekici2007}, using $3N+1$ 
time instances, efficiently reduces the
condition number. But for this case, the use of evenly-spaced time
instances is sufficient as the condition number seems to be small enough
for the considered magnitude of unsteadiness.
\begin{table}[htp]
  \ra{1.3} \centering
  \begin{tabular}{rcc}
    \toprule
    \multicolumn{1}{c}{Reference} & \multicolumn{2}{c}{$\kappa(E)$} \\
    \multicolumn{1}{c}{and \# harmonics} & EVE $(2N+1)$ & EVE $(3N+1)$ \\
    \midrule
    \citet{Gopinath2007} ($N=2$) & $\mathbf{3.79}$ & $3.00$ \\
    \citet{Ekici2007} ($N=3$) & $5.40$ & $\mathbf{3.84}$ \\
    \citet{Gopinath2007} ($N=4$) & $\mathbf{11.25}$ & $2.07$ \\
    \citet{Gopinath2007} ($N=7$) & $\mathbf{16.66}$ & $14.61$ \\
    \bottomrule
  \end{tabular}
  \caption{Literature review of the condition number used in multi-frequential
  harmonic balance computations.}
  \label{tab:literature_multistage}
\end{table}

However, such an approach fails when dealing with configurations where:
\begin{itemize} \itemsep0pt \parskip0pt
  \item the amplitude of unsteadinesses is large as for instance
  large oscillations of a blade,
  \item the frequencies are widely segregated. In fact, as shown previously
  in Sec.~\ref{sec:condition_cror_ael}, the more segregated the frequencies, the
  higher the condition number using a uniform time sampling. This condition number
  can be tremendous ($\kappa (E) \gg 100$) preventing the use of such
  an approach for given configurations.
\end{itemize}

Moreover, as the amplitude of unsteadinesses plays a crucial role in
the amplification done by the condition number, the only
way to ensure that a simulation will converge is 
to minimize the condition number.
Therefore, the section below will be dedicated to
the development of algorithms to achieve this goal.
