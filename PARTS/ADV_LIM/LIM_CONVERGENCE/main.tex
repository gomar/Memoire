%!TEX root = ../../../adrien_gomar_phd.tex

\chapter{Convergence of Fourier-based 
time methods for turbomachinery wake passing problems}
\label{cha:limitations_convergence}

\defcitealias{JGomar2013}{{\small \emph{A. Gomar}, Q. Bouvy, F. Sicot, G. Dufour, P. Cinnella, and B. Fran\c cois. Convergence of Fourier-based time methods for turbomachinery wake passing problems. 
\emph{Journal of Computational Physics}, submitted in December 2013}}

\chabstract{Theoretical results about the convergence of spectral methods 
(see \emph{e.g.} \citet{Canuto2006}
for a comprehensive review) predict convergence of the numerical 
solution starting from a given number of harmonics, provided 
that the approximated function satisfies some regularity 
requirements~\cite{Zygmund1959}. 
Nevertheless, this number of harmonics is configuration-dependent 
and hardly predictable. Studies on the convergence of 
Fourier-based time methods for turbomachinery simulations 
have been previously reported in the literature, but with scattered results. 
For instance, using a frequency-domain approach, 
\citet{Vilmin2006} obtain accurate solutions 
using 5~harmonics for a compressor stage and 3~harmonics for a 
centripetal turbine stage. For a transonic compressor stage with 
forced blade vibration, \citet{ekici2010} use 
up to 7~harmonics with a time-domain harmonic balance approach. Finally, for a 
subsonic compressor stage, \citet{JSicot2012} report 
that 4~harmonics is the minimal requirement to properly capture wake interactions.
The goal of the present chapter is to analyze the
convergence of Fourier-based time method, 
with focus on turbomachinery applications. 
In this aim, the HB method is applied to 
the linear advection equation 
supplemented with unsteady boundary conditions 
of different degrees of smoothness,
to highlight the impact of solution regularity 
on HB convergence. 
Then, a model problem representative of a turbomachinery wake-passing 
configuration is set up, and different
error measures are introduced to compare the numerical and analytical solutions. 
These error measures allows finally to define a prediction tool, 
which will be applied to contra-rotating open rotor simulations
in Chapters~\ref{cha:dream_ls_isolated} and~\ref{cha:dream_hs_isolated}.
This work has been submitted in
\begin{quote}
	\citetalias{JGomar2013}
\end{quote}}


\newpage

\section{Periodic problems with an infinite Fourier spectrum}
\label{sec:rectangular_fct}
%!TEX root = ../../../adrien_gomar_phd.tex

In fluid dynamics, the simplest model 
representative of a shock wave
is the step function.
The periodic step function over the period $T=1/c$ is defined as:
\begin{equation}
    u_l(t) = 
    \begin{cases}
        0, & \text{if } 0 \leq t < \frac{T}{2}, \\
        1, & \text{if } \frac{T}{2} \leq t < T.
    \end{cases}
    \label{eq:inject_step}
\end{equation}

\begin{figure}[htp]
  \centering
  \subfigure[$N=1$]{\includegraphics[width=.35\textwidth]{convection_step_N1.pdf}}
  \subfigure[$N=2$]{\includegraphics[width=.35\textwidth]{convection_step_N2.pdf}}
  \subfigure[$N=3$]{\includegraphics[width=.35\textwidth]{convection_step_N3.pdf}}
  \subfigure[$N=4$]{\includegraphics[width=.35\textwidth]{convection_step_N4.pdf}}
  \subfigure[$N=5$]{\includegraphics[width=.35\textwidth]{convection_step_N5.pdf}}
  \subfigure[$N=6$]{\includegraphics[width=.35\textwidth]{convection_step_N6.pdf}}
  \caption{Linear advection of a rectangular function: 
  numerical solutions at different time instances for different numbers of harmonics.}

  \label{fig:inj_step_results}
\end{figure}

Figure~\ref{fig:inj_step_results} depicts the results of HB computations
using one to six harmonics at different time instances. The convergence rate 
is slow, and for the six-harmonics HB computation the
shape of the rectangular function is still barely captured. 
The well-known \citet{Gibbs1899}
phenomenon is observed, which is a typical drawback 
of Fourier-based methods applied to discontinuous problems, 
see e.g. \citet{Canuto2006}.

\begin{figure}[htp]
  \centering
  \includegraphics[width=.5\textwidth]{convection_step_error.pdf}
  \caption{Linear advection of a rectangular function: convergence of the HB method error.}
  \label{fig:conv_step}
\end{figure}
As for the previous case, the $\mathcal{L}_2$-norm 
of the error is depicted in Fig.~\ref{fig:conv_step}. 
The convergence of the sum of sine functions, that has been studied in Sec.~\ref{sec:sum_sine},
is added for comparison.
The convergence rate is dramatically different from the previous one: 
the error decreases slowly when more harmonics are introduced, 
but the exact solution is never reached, 
unless an infinite number of harmonics is considered.

The discrete Fourier transform of the results
is computed and compared to the analytical result in Fig.~\ref{fig:dft_step}.
For this case, the spectrum in not finite and cannot be captured accurately
with a finite number of samples.

\begin{figure}[htp]
  \centering
  \includegraphics[width=.5\textwidth]{convection_step_dft.pdf}
  \caption{Linear advection of a rectangular function: 
  discrete Fourier transform.}
  \label{fig:dft_step}
\end{figure}

\section{Toward turbomachinery wakes}
\label{sec:wake_fct}
%!TEX root = ../../../adrien_gomar_phd.tex

Consider for simplicity a turbomachinery stage composed of two rotors,
as for instance a CROR configuration.
A wake is shed behind
the upstream and the downstream rotor. 
It is stationary in the frame of reference attached to the upstream wheel.
\begin{figure}[htp]
    \centering\includegraphics[width=.35\textwidth]{cror_wakes.pdf}
  \caption{Characteristic wakes in a CROR configuration.}
  \label{fig:rotor-stator}
\end{figure}
However, when it crosses the rotor-rotor interface,
the wake becomes unsteady in the frame of reference of the second wheel. 
Thus, an upstream steady spatial distortion becomes unsteady in
the downstream row.

Thus, in the downstream reference frame, wakes coming 
from the upstream wheel can be represented, 
to a first approximation, as the periodic 
advection of a Gaussian function from the inter-wheel interface.

We consider again the linear advection problem defined in Sec.~\ref{sec:linear}, 
with $u_l$ now taken equal to a Gaussian function following the
\citet{Lakshminarayana1980} similarity law
\begin{equation}
    u_l (t) = u_m \left[1 - 
        \Delta u \cdot e^{
          -0.693 \left(\frac{2 c t}{L_x L} \right) ^ 2}\right],
\end{equation}
The full width at half maximum $L$ of the wake is set to 10\% of the domain size, 
$u_m$ is set to $c$ and $\Delta u$ to 10\% of $u_m$.

\begin{figure}[htp]
  \centering
  \subfigure[$N=1$]{\includegraphics[width=.35\textwidth]{convection_wake_N1.pdf}}
  \subfigure[$N=2$]{\includegraphics[width=.35\textwidth]{convection_wake_N2.pdf}}
  \subfigure[$N=3$]{\includegraphics[width=.35\textwidth]{convection_wake_N3.pdf}}
  \subfigure[$N=4$]{\includegraphics[width=.35\textwidth]{convection_wake_N4.pdf}}
  \subfigure[$N=5$]{\includegraphics[width=.35\textwidth]{convection_wake_N5.pdf}}
  \subfigure[$N=6$]{\includegraphics[width=.35\textwidth]{convection_wake_N6.pdf}}
  \caption{Linear advection of a Gaussian function representing a turbomachinery wake: 
  numerical solutions at different time instances for different numbers of harmonics.}
  \label{fig:inj_wake_results}
\end{figure}
Figure~\ref{fig:inj_wake_results} depicts the HB
computations for one to six harmonics. The numerical solution convergences
to the exact Gaussian function starting from $N=6$ harmonics.
When the number of harmonics is
too small, the width and the depth of the wake are badly approximated
by the method, and the solution exhibits some spurious oscillations. 

\begin{figure}[htp]
  \centering
  \includegraphics[width=.5\textwidth]{convection_wake_error.pdf}
  \caption{Linear advection of a Gaussian function representing a 
  turbomachinery wake: convergence of the HB method error.}
  \label{fig:conv_wake}
\end{figure}
Figure~\ref{fig:conv_wake} shows the quantitative convergence of 
the $\mathcal{L}_2$ error. The
convergence curves for the two functions studied in the previous sections
are also reported for comparison.
The error follows now a nearly exponential convergence.
\begin{figure}[htp]
  \centering
  \includegraphics[width=.5\textwidth]{convection_wake_dft.pdf}
  \caption{Linear advection of a Gaussian function representing a turbomachinery wake: 
  discrete Fourier transform.}
  \label{fig:dft_wake}
\end{figure}
The discrete Fourier transform of the results is
depicted against the analytical result in Fig.~\ref{fig:dft_wake}.
The $N=2$ and $N=4$ computations badly capture the amplitudes of the
resolved harmonics.
Starting from $N=6$, some of the lower 
frequencies are correctly captured, whereas high frequencies are
always under-estimated.
This improves when further harmonics are added to the computation.

For a better understanding of the HB convergence behavior, 
we consider the spectral content of the Gaussian wake model. 
Precisely, the Fourier transform $\widehat{g}$ of a Gaussian function $g$
defined as
\begin{equation}
    g(x) = A e^{-\alpha x^2},
    \label{eq:simple_gaussian_function}
\end{equation}
where $A$ and $\alpha$ are constants, is
\begin{equation}
    \widehat{g}(f) = A^\prime e^{-\alpha^\prime f^2},
    \label{eq:fourier_transform_gaussian}
\end{equation}
where
\begin{equation}
  \begin{cases}
    A^\prime=A \sqrt{\frac{\pi}{\alpha}},\\
    \alpha^\prime = \frac{\pi^2}{\alpha}.
  \end{cases}
\end{equation}

For the similarity law of Lakshminarayana and Davino, 
$\alpha$ and $\alpha^\prime$ can be identified as
\begin{equation}
    \alpha =  0.693 \left( \frac{2}{L} \right)^2, \quad
    \alpha^\prime =  \frac{1}{0.693} \left( \frac{\pi L}{2} \right)^2.
    \label{eq:gaussian_params_laksh}
\end{equation}
The exponential factor of the wake law~$\alpha$ is inversely
proportional to its Fourier counter-part~$\alpha'$, meaning that their
width will vary in opposite way: the thinner the wake, the wider its
spectrum and \emph{vice-versa}.

The convergence rate is inherently linked to
the spectrum of the considered unsteady signal.
As for the present case we know the analytical wake spectrum,
we define the theoretical truncation error as the ratio of
the energy contained in the unresolved part 
of the spectrum to the overall energy content of the full spectrum
\begin{equation}
    \varepsilon_{th}(f) = \sqrt{\frac{
        \int_f^\infty | \widehat{g}(\zeta)|^2 \diff \zeta
      }{
        \int_0^\infty | \widehat{g}(\zeta)|^2 \diff \zeta
      }}.
    \label{eq:def_truncation_error}
\end{equation}
Introducing the error function defined as
\begin{equation}
    \erf(x) = \frac{2}{\sqrt{\pi}} \int_0^x e^{-t^2} \diff t,
\end{equation}
and the complementary error function defined as
\begin{equation}
    \erfc(x) = 1 - \erf(x),
\end{equation}
then
\begin{align}
    \int_0^\infty | \widehat{g}(\zeta)|^2 \diff \zeta 
    &= \frac{1}{2} \int_{- \infty}^\infty | \widehat{g}(\zeta)|^2 \diff \zeta \\
    &= \frac{A^{\prime 2}}{2} \sqrt{\frac{\pi}{2 \alpha^\prime}},
\end{align}
and
\begin{equation}
    \int_f^\infty | \widehat{g}(\zeta)|^2 \diff \zeta = 
      \frac{A^{\prime 2}}{2} \sqrt{\frac{\pi}{2 \alpha^\prime}} \erfc (\sqrt{2 \alpha^\prime} f).
\end{equation}
The theoretical truncation error can then be written as
\begin{equation}
    \varepsilon_{th}(f, L) = \sqrt{\erfc (\sqrt{2 \alpha^\prime(L) } f)}.
    \label{eq:analytical_conv}
\end{equation}
One can notice from Eq.~\eqref{eq:analytical_conv} that the 
truncation error does not depend on the wake deficit $\Delta u$ 
but only on the wake width $L$.

\begin{figure}[htp]
    \centering\includegraphics[width=.5\textwidth]{ANALYTICAL_ERROR_PPT.pdf}
  \caption{Theoretical truncation error of the Lakshminarayana and Davino wake law.}
  \label{fig:analytic_error_paper}
\end{figure}
Eq.~\eqref{eq:analytical_conv} is depicted in
Fig.~\ref{fig:analytic_error_paper}. 
It can be seen that the wider the spectrum,
the higher the number of harmonics needed to
reach a certain level of error. 
Moreover, for a thin wake width (\emph{e.g.} 2\% of the pitch)
the number of harmonics required to capture it with a truncation 
error of 10\% is up to 25~harmonics.
In the limit of $L \to 0$, the wake becomes a Dirac function
which represents the worst possible case.
In the preceding example, the Gaussian function had a width
of 10\% which, according to Eq.~\eqref{eq:analytical_conv},
is captured by using $N=7$ harmonics for a target 10\% error.

\section{Application to a model turbomachinery configuration}
\label{sec:model_tbm}
%!TEX root = ../../../adrien_gomar_phd.tex

The model turbomachinery toy problem defined
in Sec.\ref{sec:model_tbm} is used here to 
understand the convergence properties of HB
computations for turbomachinery configurations
and by extrapolation to CROR applications.

\subsection{Spectral convergence study}
The primary interest in this section is the wake capturing capabilities of the 
Fourier-based time method in the rotating part. 
To analyze this, two error measures are defined and
evaluated. 

\subsubsection{Spatial-spectrum based error measure}
\label{sec:crit_1}
Initially, we propose an error measure ($\varepsilon_1$) based 
on the loss of signal energy
induced by the harmonic method at the interface. 
In fact, in the stator part, the wake is steady and is thus not
filtered by the HB operator. 
Conversely, in the rotor part, the steady wake becomes
unsteady due to the relative speed difference between the
stator and the rotor. However, only a finite number of harmonics~$N$
is used to describe the unsteady field, hence the filtering.

The first error quantification is set up to quantify this filtering 
by using only spatial information and is defined as the $\mathcal{L}_2$-norm 
applied on the 
difference between the rotor and the stator spectra.
It is equivalent to the analytical truncation error 
defined in Eq.~\eqref{eq:def_truncation_error}. 
Indeed, the error is described as the ratio of the unresolved energy 
in the rotor block
to the energy of the full spectrum, 
\emph{e.g.} that of the stator block:
\begin{equation}
    \varepsilon_1(N) = \sqrt{
    \frac{\sum_{f=1}^{f_{max}} | \widehat{s}^{~\theta}_N (f) - 
      \widehat{r}^{~\theta}_N (f)|^2}{ 
    \sum_{f=1}^{f_{max}} | \widehat{s}^{~\theta}_N (f)|^2}},
    \label{eq:def_crit_1}
\end{equation} 
where $\widehat{s}^{~\theta}_N$ denotes the spatial Fourier transform (indicated by
the $\widehat{\vphantom{s}.}$ operator) of the azimuthal extraction (shown
by superscript $\theta$) of the result of a HB simulation using $N$~harmonics,
in the stator; $\widehat{r}$ denotes the spectrum of 
the signal transferred to the rotor.
The higher frequency present in the spectrum is dictated 
by the spatial discretization. Thus, $f_{max} = 1 / 2\Delta \theta_{cells}$, 
using the notations of Eq.~\eqref{eq:az_spatial_discretization_1}.
As the azimuthal cell size is similar in both blocks, 
the same sampling is used leading to the same 
frequencies in both stator and rotor spectra.
Details of the algorithm used to compute $\varepsilon_1$ 
are given in \ref{app:epsilon_1_steps}.

The filtering introduced by the HB approach 
acts primarily on the time resolution. 
For under-resolved HB computations, a dissipation error is observed.
This dissipation is not spatially uniform and gives rise to
dispersion errors on the spatial spectrum and to spurious
high-frequencies as shown in 
Fig.~\ref{fig:spatial_crit} for HB computations $N=2$ to $N=10$.
These effects vanish when the HB computations converge
\emph{i.e.} for $N \geq 10$.
Therefore, the spectrum of the unresolved spurious frequencies 
is imposed to have a zero amplitude value to compute
$\varepsilon_1$.
\begin{figure}[htp]
  \centering
  \subfigure[$N=2$]{
  \includegraphics[width=.5\textwidth]{cut_wake_W0490_TSM_N002_adim.pdf}
  \includegraphics[width=.5\textwidth]{fft_fwake_1D_e490_N02.pdf}}
  \subfigure[$N=5$]{
  \includegraphics[width=.5\textwidth]{cut_wake_W0490_TSM_N005_adim.pdf}
  \includegraphics[width=.5\textwidth]{fft_fwake_1D_e490_N05.pdf}}
  \subfigure[$N=10$]{
  \includegraphics[width=.5\textwidth]{cut_wake_W0490_TSM_N010_adim.pdf}
  \includegraphics[width=.5\textwidth]{fft_fwake_1D_e490_N10.pdf}}
  \subfigure[$N=20$]{
  \includegraphics[width=.5\textwidth]{cut_wake_W0490_TSM_N020_adim.pdf}
  \includegraphics[width=.5\textwidth]{fft_fwake_1D_e490_N20.pdf}}
  \caption{Wake of $L=5\%$ width extracted in stator and rotor 
  blocks. Signal and spatial Fourier analysis for different computations.}
  \label{fig:spatial_crit}
\end{figure}

The azimuthal velocity distributions (left hand-side) and the corresponding spatial
spectra (right hand-side)
are presented in Fig.~\ref{fig:spatial_crit} 
for a relative wake thickness of~5\% with respect to the pitch 
and for HB computations using $N=2$, 5, 10 and 18, respectively.
For the stator, the azimuthal distribution follows a 
Gaussian function as expected. On the contrary, 
the rotor distribution is aliased by the HB discretization 
and exhibits spurious oscillations that tend to disappear
when the number of harmonics used in the computation 
increases.
For $N=10$, some oscillations are still present, 
but the wake captured in the moving block begins to 
converge to that leaving the upstream block.

Inspection of the spectra suggests the same conclusions.
The amplitude of $\widehat{\rho U}$ 
improves when increasing the number of harmonics.
As previously mentioned, for under-resolved HB computations,
a dispersion error is introduced and spurious high-frequencies appear 
in the spatial spectra as shown in Fig.~\ref{fig:spatial_crit}
for $N=2$ to $N=10$.
For $N=20$, the spectrum of the rotor 
block matches that of the stator block.
This is consistent with the theoretical analysis, in which more than 
$N=10$ harmonics are needed to capture the wake with less than $20\%$ 
of error for this particular
wake width (see Fig.~\ref{fig:analytic_error_paper}).

In summary, for this wake thickness, the temporal filtering 
on a simulation involving less than ten harmonics is too harsh and leads 
to a significant amount of unresolved energy, 
which deteriorates the numerical representation
of the wake.

For a more quantitative analysis, we compute the error measure
$\varepsilon_1$ for each computation ranging over different 
wake thicknesses and numbers of harmonics. 
Results are summarized in Fig.~\ref{fig:crit_1_3d}.
\begin{figure}[htp]
    \centering\includegraphics[width=.6\textwidth]{error_crit_1.pdf}
  \caption{Evaluation of the error due to the wake 
  capturing using the first error quantification $\varepsilon_1$.}
  \label{fig:crit_1_3d}
\end{figure}
Because it quantifies the unresolved energy in 
comparison to the resolved energy, $\varepsilon_1$ 
exhibits a behavior similar to that of 
the theoretical error $\varepsilon_{th}$ for a Gaussian function 
(Fig.~\ref{fig:analytic_error_paper}).
The iso-error contours have a similar shape 
as the analytical ones. 
The conclusions are equivalent: the truncation error decreases with 
the wake thickness and with the number of harmonics used to capture the wake.
Nevertheless, for thicker wakes and higher numbers of harmonics, 
the error measure $\varepsilon_1$ is over-estimated. 
For instance, around $N=15$ and for $L=25\%$,
$\varepsilon_1 \approx 10^{-2}$ whereas the theoretical error $\varepsilon_{th}$
is less than $10^{-4}$. The error 
measure $\varepsilon_1$ does not represent a 
realistic measure, because of the spatial 
Fourier transform performed to compute 
the error, as discussed in the following.

As can be seen in Fig.~\ref{fig:ST_discrepancies}, 
the Fourier transform of the spatial signal in the stator block tends to a plateau. 
The thicker the wake, 
the lower the frequency for which the plateau appears: 
approximately 15~harmonics for $L=10\%$
(see Fig.~\ref{fig:ST_discrepancies_a}) and 
6~harmonics for $L=25\%$ (see Fig.~\ref{fig:ST_discrepancies_b}).
Actually, for a $N$-harmonic HB computation, the spectrum is 
explicitly filtered in the moving block leading to an amplitude 
equal to zero above the $N^{th}$ harmonic. 
Therefore, when the HB computations are converged, the difference between the spatial 
spectra in the stator and in the rotor block is driven by the plateau present 
in the spatial spectrum of the stator block.
\begin{figure}[htp]
  \begin{center}
  \subfigure[$L=10\%$, $N=3$]{
    \includegraphics[width=.45\textwidth]{PB_FFT_AZI_10_3}\label{fig:ST_discrepancies_a}}
  \subfigure[$L=25\%$, $N=15$]{
    \includegraphics[width=.45\textwidth]{PB_FFT_AZI_25_15}\label{fig:ST_discrepancies_b}}
  \end{center}
  \caption{Discrepancies between spatial and temporal spectra.}
  \label{fig:ST_discrepancies}
\end{figure}

In fact, this behavior is linked to the windowing of the signal on 
a bounded interval, the pitch. To highlight that, the influence of 
a modification on the inlet boundary condition is analyzed.
The inlet wake distortion used in the model turbomachinery configuration is 
originally based on the analytical Lakshminarayana and Davino 
Gaussian law (see Eq.~\eqref{eq:similarity}). However, 
this law is discretized and imposed on a bounded interval 
that spans the angular pitch. As the relative thickness 
increases, the inlet condition diverges from the analytical 
Gaussian law for which the angular pitch is theoretically 
infinite. This is shown in Fig.~\ref{fig:inlet_law_fft} 
through the spectra of three Gaussian laws. The relative 
thickness of the laws are modified through the size 
of the pitch $\Delta \theta$. The multiplication by a factor $100$ 
of the pitch leads to a disappearance of the plateau 
in the spectrum, which accurately matches with the 
Fourier transform of a Gaussian function. 
\begin{figure}[htp]
  \centering
  \subfigure[$\Delta \theta = L$]{\includegraphics[width=.7\textwidth]{PB_FFT_AZI_DEMO_1}}
  \subfigure[$\Delta \theta = 10L$]{\includegraphics[width=.7\textwidth]{PB_FFT_AZI_DEMO_10}}
  \subfigure[$\Delta \theta = 100L$]{\includegraphics[width=.7\textwidth]{PB_FFT_AZI_DEMO_100}}
  \caption{Evolution of the spectrum of the inlet boundary condition for different angular pitch.}
  \label{fig:inlet_law_fft}
\end{figure}

To sum up, a plateau appears in the spatial spectrum of the
stator block. This plateau is explicitly filtered in the
rotor block above the $N^{th}$ harmonic, leading to an over-estimation of the 
first error measure. This over-estimation drives the error value
for higher number of harmonics and thicker wakes.

\FloatBarrier

\subsubsection{Spatial/Time duality error measure}
To get a more realistic error measure, we take 
again into account the energy loss
through the interface, but based on a spatial/time duality. 
As this loss of energy is 
precisely related to the filtering 
introduced on the temporal signal by the HB approach, the second 
error quantification $\varepsilon_2$ addresses the result on 
the temporal information. 

Near the interface of the blocks, consider a fixed observer in
the rotor frame of reference. This observer sees an unsteady 
wake passing as the blocks have a relative speed difference.
The first error quantification has shown the 
influence of the number of harmonics on the spatial signal 
in the rotor block. The error quantification will now
point that this spatial influence is due to a temporal filtering done by
the HB approach.

Following the same notation as in Eq.~\eqref{eq:def_crit_1}, 
the second error measure is written as:
\begin{equation}
    \varepsilon_2(N) = \sqrt{
    \frac{\sum_{f=1}^{f_{max}} | \widehat{s}^{~\theta}_N (f) - 
      \widehat{r}^{~t}_N (f)|^2}{ 
    \sum_{f=1}^{f_{max}} | \widehat{s}^{~\theta}_N (f)|^2}},
    \label{eq:def_crit_2}
\end{equation}
where superscript $t$ denotes the temporal version of
the Fourier transform.
By definition, $\varepsilon_2$
quantifies the matching between a spatial signal
and a temporal information.
Again, the error is described as the unresolved energy 
in the rotor block, 
divided by the energy of the full spectrum, 
e.g. that of the stator block. 
For $\varepsilon_1$, the amplitude 
of the harmonics above the $N^{th}$ one was imposed to zero. 
On the contrary, for $\varepsilon_2$, the temporal spectrum 
in the rotor block is, 
by essence null above the $N^{th}$ harmonic, as the filtering 
acts on temporal values. 
Details of the algorithm used to compute $\varepsilon_2$ are given in \ref{app:epsilon_2_steps}.

\begin{figure}[htp]
\centering
  \subfigure[$N=5$]{
  \includegraphics[width=.48\textwidth]{interp_wake_W0490_TSM_N005.pdf}
  \label{fig:temp_signal_a}}
  \subfigure[$N=10$]{
  \includegraphics[width=.48\textwidth]{interp_wake_W0490_TSM_N010.pdf}}
  \subfigure[$N=15$]{
  \includegraphics[width=.48\textwidth]{interp_wake_W0490_TSM_N015.pdf}
  \label{fig:temp_signal_c}}
  \caption{Temporal signal seen at loc~1 and loc~2 for a $L=5\%$ wake width.}
  \label{fig:temp_signal}
\end{figure}
Figure~\ref{fig:temp_signal} shows time signals
extracted at two different azimuthal positions at 
the interface of the rotor block, named loc~1 and loc~2. 
The small phase shift between the two 
signals is due to the space lag between the two points, 
and is the same for any choice of the number of 
harmonics used in the computation. On the contrary, 
differences in terms of amplitude are only due 
to the use of an insufficient number of harmonics: 
as the number of modes used for the time 
approximation is increased from $N=5$ to $N=15$, 
the amplitude of the space-shifted signals 
tends to converge to the same value, and 
spurious oscillations tend to disappear. Therefore, in the following,
only loc~1 will be considered.

Fig.~\ref{fig:dualite_crit} describes the space and 
time spectra of the axial momentum $\rho U$ at loc~1, 
for computations using $N=2$, 5, $10$ and $20$ 
harmonics and for a wake width of $L=5\%$.
The spatial spectrum contains the whole wavelength 
content associated to the incoming wake; 
on the contrary, due to the filtering introduced 
by the HB approach, the time spectrum is composed of only $N$ harmonics.
\begin{figure}[htp]
\centering
\subfigure[$N=2$]{\includegraphics[width=.45\textwidth]{SpcTme_Dualite_0490_02}}
\subfigure[$N=5$]{\includegraphics[width=.45\textwidth]{SpcTme_Dualite_0490_05}}
\subfigure[$N=10$]{\includegraphics[width=.45\textwidth]{SpcTme_Dualite_0490_10}}
\subfigure[$N=20$]{\includegraphics[width=.45\textwidth]{SpcTme_Dualite_0490_20}}
\caption{Spatial/time duality for a $L=5\%$ wake width.}
\label{fig:dualite_crit}
\end{figure}

For computations using less than 10 time harmonics, 
time spectra are truncated, and the amplitude of 
$\rho U$ differs from that of the corresponding mode in the spatial spectrum.

As the number of time harmonics is increased, 
the amplitude of lower harmonics becomes closer 
and closer to that of the corresponding harmonic 
in the reference signal, and errors move toward 
the higher resolved harmonics. For $N=20$, 
the amplitudes of the 20~resolved harmonics are 
similar for both the time and space spectra.

In summary, the preceding analysis shows that, 
for under-resolved HB computations, the time 
signal is affected by both amplitude and phase errors, 
since the energy content is redistributed incorrectly 
among the resolved harmonics.

To quantify this error, we apply the error measure~\eqref{eq:def_crit_2}
to HB computations of the model turbomachinery 
problem corresponding to different choices 
of the wake thickness and different numbers of 
harmonics. Results are presented in Fig.~\ref{fig:crit_2_3d}.
The $\varepsilon_2$ error map is qualitatively 
and quantitatively similar to the $\varepsilon_1$ 
discussed in the previous Section. 
Again, the truncation error measured using $\varepsilon_2$ 
for thick wakes and high numbers of harmonics 
does not follow the trend observed for the 
theoretical error $\varepsilon_{th}$, 
due to the spatial filtering introduced at the 
interface by the phase-lag condition.

The preceding analysis shows that, for HB computations 
that are well converged in terms in harmonics, 
the spatial spectrum in the stator and the 
time spectrum in the rotor block tend to match, 
except for additional spatial errors introduced
by the use of an azimuthal Fourier transform on a 
bounded interval, which confirms the 
validity of the error measure defined in Eq.~\eqref{eq:def_crit_2}.
\begin{figure}[htp]
   \centering \includegraphics[width=.6\textwidth]{error_crit_2.pdf}
  \caption{Evaluation of the error due to the wake 
  capturing using the second error quantification ($\varepsilon_2$).}
  \label{fig:crit_2_3d}
\end{figure}

\subsection{Comparison with the theoretical error measure}
\label{sub:comp_w_analytic}


The preceding results show that approximated truncation error 
measures computed for the model turbomachinery problem 
using a nonlinear flow model (Euler equations) 
exhibit trends, with respect to the wake thickness 
and number of HB harmonics, in close agreement with the 
theoretical error measure derived in Section~\ref{sec:turbomachine_wake} 
for a Gaussian function. 
Figure~\ref{fig:error_comp_curves} compares the 
different error measures for HB simulations of 
advected wakes of varying thickness versus 
the number of harmonics used for the time discretization. 
This corresponds to horizontal cuts of Figs~\ref{fig:analytic_error_paper}, 
\ref{fig:crit_1_3d} and~\ref{fig:crit_2_3d}. 
For number of harmonics higher than the cutoff 
harmonic used in the phase-lag condition 
the three error measures are seen to give 
results in very close agreement. After that value, 
both the $\varepsilon_1$ and $\varepsilon_2$ error 
measures applied to the model turbomachinery problem 
exhibit a plateau.
The preceding remarks suggest the idea that, 
since all error measure provide similar results, 
at least up to numbers of harmonics of interest for 
practical applicative problems, an a priori 
estimate of the number of harmonics required 
to achieve a given error level could be 
obtained by using the theoretical error measure 
Eq.~\eqref{eq:analytical_conv}, if a quick 
estimate of the wake thickness characteristic 
of a given turbomachinery problem is available. 
In the next Section, we show that a reasonable 
estimate can be obtained from a preliminary steady 
computation based on the mixing plane interface condition.
\begin{figure}[htp]
  \centering
  \subfigure[$L = 2 \%$]{\includegraphics[width=.46\textwidth]{RB_MP_0200_error_logY}}\quad
  \subfigure[$L = 5 \%$]{\includegraphics[width=.46\textwidth]{RB_MP_0490_error_logY}}\quad
  \subfigure[$L = 10 \%$]{\includegraphics[width=.46\textwidth]{RB_MP_0965_error_logY}}\quad
  \subfigure[$L = 15 \%$]{\includegraphics[width=.46\textwidth]{RB_MP_1520_error_logY}}\quad
  \caption{Truncation, computed and analytical errors for four wake width.}
  \label{fig:error_comp_curves}
\end{figure}

\subsection{Toward an priori error estimate}
In order to define an \emph{a priori} error measure 
that can be used to estimate the number of 
harmonics required to achieve a reasonable 
convergence of the HB method, we suggest to 
evaluate the wake thickness by using a preliminary 
mixing plane steady computation. Indeed, if 
potential effects due to the downstream row can 
be neglected, the spatial information at the interface 
in the stator block, essentially due to the incoming 
wakes, can be captured without taking into account 
the relative motion between the wheels, \emph{i.e.} 
by means of a mixing plane computation. 
Given the approximated azimuthal distribution at 
the stator interface, we consider the cumulative
energy content of the signal up to a given frequency $f$ 
(or, equivalently, to a given harmonic $N=f/f_1$ where $f_1$ is the
frequency value of the considered unsteadiness). 
The cumulative energy is defined as:
\begin{equation}
    E(f) = \frac{
      \int_0^f | \widehat{g}(\zeta)|^2 \diff \zeta
    }{
      \int_0^\infty | \widehat{g}(\zeta)|^2 \diff \zeta
    },
\end{equation}
where $\widehat{g}$ is the spectrum of the quantity of interest.
By comparison with Eq.~\eqref{eq:def_truncation_error},
the relation between the relative accumulated energy $E$
and the truncation error $\varepsilon_{mxp}$:
\begin{equation}
    E(f) = 1 - \varepsilon_{mxp}^2 (f).
    \label{eq:correspond_E_error}
\end{equation}


Note that this last error measure is based only on 
the amount of unresolved energy that is left 
in a computation if the spatial signal is 
truncated at a given cutoff frequency $f$, 
and does not require any information from the rotor
block, but it depends only on the characteristics 
of the incoming wake.

To check if the new error measure represents an 
accurate estimate of the truncation error of 
an HB simulation, we carry out again a 
parametric study of the error versus different 
wake thicknesses and numbers of harmonics 
(equivalently, cutoff frequencies), and compare 
the results to those of the \emph{a posteriori} error measures 
obtained for the model turbomachinery problem and for the 
theoretical error $\varepsilon_{th}$. 
Results corresponding to $\varepsilon_{mxp}$ are 
superposed to the corresponding curves in Fig.~\ref{fig:error_comp_curves}. 
The \emph{a priori} error measure ($\varepsilon_{mxp}$) matches 
the theoretical estimate ($\varepsilon_{th}$)
and the \emph{a posteriori} measures ($\varepsilon_1$, $\varepsilon_2$)
over a wide range of harmonics. Similarly to the \emph{a posteriori}
errors $\varepsilon_1$ and $\varepsilon_2$, the \emph{a priori} error
exhibits a plateau for high $N$ and high wake thicknesses, 
due to the application of the Fourier transform on a bounded interval. 
We also stress the close agreement between 
$\varepsilon_{mxp}$ and $\varepsilon_{th}$: specifically, 
estimates of the number of harmonics needed to capture 99\% 
of the cumulative energy (equivalently, to get a 
truncation error equal to 10\%) are identical for 
all error measures.

% \section*{Summary}

% We have seen that the convergence rate 
% of Fourier-based time methods and hence the harmonic balance approach, 
% in terms of harmonics required to describe the solution 
% with a given level of accuracy, depends on the spectral content of the 
% solution itself: Fourier-based time methods are particularly efficient 
% for flow problems characterized by a narrow Fourier 
% spectrum. We show that main source of unsteadiness in 
% turbomachinery flows is due to the relative motion of wakes 
% generated by a given blade row with respect to the downstream row. 
% Statistically speaking, the passing wakes are seen by the downstream 
% row as an azimuthally advected periodic Gaussian pulse, 
% characterized by its relative thickness compared to the pitch 
% in between two subsequent blades and by the velocity deficit 
% associated to it. We show that the narrower the wake, the larger 
% its Fourier spectrum, and the slower the convergence of Fourier-based time methods.
% In order to achieve \emph{a priori} estimates of the number of 
% harmonics required to accurately solve a given turbomachinery 
% problem, we introduce two error measures based on the relative 
% thickness of the passing wakes. It is shown that, for practical 
% purposes, these can be preliminarily estimated by running a 
% companion steady simulation of the turbomachinery stage. The 
% steady simulation is post-processed to extract information about 
% the spanwise distribution of wake thickness, and an error criterion 
% is used to estimate the number of harmonics required to resolve 99\% 
% of the energy content associated to the velocity signal. The 
% preliminary step has a negligible cost compared to the overall 
% simulation, since the steady computation is used to initialize 
% the unsteady run, and extraction of wake characteristics takes 
% less than a minute on a single processor. 
% The proposed methodology represents an efficient and reliable 
% operational tool to guide the choice of the number of harmonics 
% for a given turbomachinery problem, and to evaluate beforehand the 
% interest of applying or not a Fourier-based time integration scheme 
% instead of a classical time-marching scheme.
