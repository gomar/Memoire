%!TEX root = ../../../adrien_gomar_phd.tex

\chapter{Convergence of Fourier-based 
time methods for turbomachinery wake passing problems}
\label{cha:limitations_convergence}

\chabstract{Efficiency of the HB method results from a trade-off between accuracy and 
costs requirements.
On one hand, the accuracy of Fourier-based time methods depends on the number of harmonics
used to represent the frequency content of the time signal; on the other
hand, computational costs and memory consumption of the computations also scale
with the number of harmonics. Theoretical results about the convergence of spectral methods 
(see e.g. Canuto~\emph{et al.}~\cite{Canuto2006}
for a comprehensive review) predict convergence of the numerical 
solution starting from a given number of harmonics, provided 
that the approximated function satisfies some regularity 
requirements~\cite{Zygmund1959}. 
Nevertheless, this number of harmonics is configuration-dependent 
and hardly predictable. In this work, we focus on turbomachinery 
configurations, which involve flow across a series of fixed and 
rotating bladed wheels. Studies on the convergence of 
Fourier-based time methods for turbomachinery simulations 
have been previously reported in the literature, but with scattered results. 
For instance, using a frequency-domain approach, 
Vilmin~\emph{et al.}~\cite{Vilmin2006} obtain accurate solutions 
using 5~harmonics for a compressor stage and 3~harmonics for a 
centripetal turbine stage. For a transonic compressor stage with 
forced blade vibration, Ekici~\emph{et al.}~\cite{ekici2010} use 
up to 7~harmonics with a time-domain harmonic balance approach. Finally, for a 
subsonic compressor stage, Sicot~\emph{et al.}~\cite{JSicot2012} report 
that 4~harmonics is the minimal requirement to properly capture wake interactions
as illustrated in Fig.~\ref{fig:cme2}. 
\begin{figure}[htb]
  \centering
  \subfigure[reference unsteady si\-mu\-lation]{\includegraphics[width=.3\textwidth]{uransentropie_nb.jpg}}
  \subfigure[HB $N=2$]{\includegraphics[width=.3\textwidth]{n2_entropie_nb.jpg}}
  \subfigure[HB $N=3$]{\includegraphics[width=.3\textwidth]{n3_entropie_nb.jpg}}
  \subfigure[HB $N=4$]{\includegraphics[width=.3\textwidth]{n4_entropie_nb.jpg}}
  \subfigure[HB $N=5$]{\includegraphics[width=.3\textwidth]{n5_entropie_nb.jpg}}
  \subfigure[HB $N=6$]{\includegraphics[width=.3\textwidth]{n6_entropie_nb.jpg}}
  \caption{Convergence of harmonic balance computations for a rotor/stator configuration from \citet{JSicot2012}.}
  \label{fig:cme2}
\end{figure}

The preceding examples show that no consensus exists in the literature 
concerning the number of harmonics needed to achieve convergence,
even for similar configurations.
The goal of the present paper is twofold: to analyze the
convergence of Fourier-based time method, with focus on turbomachinery applications, 
and to provide a criterion for the minimal number of harmonics 
required to achieve a specified accuracy level. The paper is organized as follows: first,
we recall the design principles of the time-domain harmonic balance 
approach and theoretical results about the convergence of Fourier-based methods.
Second, the HB method is applied to the linear advection equation 
supplemented with unsteady boundary conditions of different degrees of smoothness,
to highlight the impact of solution regularity on HB convergence. 
Third, a model problem representative of a turbomachinery wake-passing 
configuration is set up, and different
error measures are introduced to compare the numerical and analytical solutions. 
These error measures allows finally to define a prediction tool, 
which is applied to contra-rotating open rotor simulations.
}

\minitoc
\newpage

\section{Linear advection of a periodic perturbation}
\label{sec:convergence_advection}
\input{PARTS/ADV_LIM/LIM_CONVERGENCE/linear_advection}

\section{Application to a model turbomachinery configuration}
\label{sec:rotating_blocks}
%!TEX root = ../../../adrien_gomar_phd.tex

The model turbomachinery toy problem defined
in Sec.\ref{sec:model_tbm} is used here to 
understand the convergence properties of HB
computations for turbomachinery configurations
and by extrapolation to CROR applications.

\subsection{Spectral convergence study}
The primary interest in this section is the wake capturing capabilities of the 
Fourier-based time method in the rotating part. 
To analyze this, two error measures are defined and
evaluated. 

\subsubsection{Spatial-spectrum based error measure}
\label{sec:crit_1}
Initially, we propose an error measure ($\varepsilon_1$) based 
on the loss of signal energy
induced by the harmonic method at the interface. 
In fact, in the stator part, the wake is steady and is thus not
filtered by the HB operator. 
Conversely, in the rotor part, the steady wake becomes
unsteady due to the relative speed difference between the
stator and the rotor. However, only a finite number of harmonics~$N$
is used to describe the unsteady field, hence the filtering.

The first error quantification is set up to quantify this filtering 
by using only spatial information and is defined as the $\mathcal{L}_2$-norm 
applied on the 
difference between the rotor and the stator spectra.
It is equivalent to the analytical truncation error 
defined in Eq.~\eqref{eq:def_truncation_error}. 
Indeed, the error is described as the ratio of the unresolved energy 
in the rotor block
to the energy of the full spectrum, 
\emph{e.g.} that of the stator block:
\begin{equation}
    \varepsilon_1(N) = \sqrt{
    \frac{\sum_{f=1}^{f_{max}} | \widehat{s}^{~\theta}_N (f) - 
      \widehat{r}^{~\theta}_N (f)|^2}{ 
    \sum_{f=1}^{f_{max}} | \widehat{s}^{~\theta}_N (f)|^2}},
    \label{eq:def_crit_1}
\end{equation} 
where $\widehat{s}^{~\theta}_N$ denotes the spatial Fourier transform (indicated by
the $\widehat{\vphantom{s}.}$ operator) of the azimuthal extraction (shown
by superscript $\theta$) of the result of a HB simulation using $N$~harmonics,
in the stator; $\widehat{r}$ denotes the spectrum of 
the signal transferred to the rotor.
The higher frequency present in the spectrum is dictated 
by the spatial discretization. Thus, $f_{max} = 1 / 2\Delta \theta_{cells}$, 
using the notations of Eq.~\eqref{eq:az_spatial_discretization_1}.
As the azimuthal cell size is similar in both blocks, 
the same sampling is used leading to the same 
frequencies in both stator and rotor spectra.
Details of the algorithm used to compute $\varepsilon_1$ 
are given in \ref{app:epsilon_1_steps}.

The filtering introduced by the HB approach 
acts primarily on the time resolution. 
For under-resolved HB computations, a dissipation error is observed.
This dissipation is not spatially uniform and gives rise to
dispersion errors on the spatial spectrum and to spurious
high-frequencies as shown in 
Fig.~\ref{fig:spatial_crit} for HB computations $N=2$ to $N=10$.
These effects vanish when the HB computations converge
\emph{i.e.} for $N \geq 10$.
Therefore, the spectrum of the unresolved spurious frequencies 
is imposed to have a zero amplitude value to compute
$\varepsilon_1$.
\begin{figure}[htp]
  \centering
  \subfigure[$N=2$]{
  \includegraphics[width=.5\textwidth]{cut_wake_W0490_TSM_N002_adim.pdf}
  \includegraphics[width=.5\textwidth]{fft_fwake_1D_e490_N02.pdf}}
  \subfigure[$N=5$]{
  \includegraphics[width=.5\textwidth]{cut_wake_W0490_TSM_N005_adim.pdf}
  \includegraphics[width=.5\textwidth]{fft_fwake_1D_e490_N05.pdf}}
  \subfigure[$N=10$]{
  \includegraphics[width=.5\textwidth]{cut_wake_W0490_TSM_N010_adim.pdf}
  \includegraphics[width=.5\textwidth]{fft_fwake_1D_e490_N10.pdf}}
  \subfigure[$N=20$]{
  \includegraphics[width=.5\textwidth]{cut_wake_W0490_TSM_N020_adim.pdf}
  \includegraphics[width=.5\textwidth]{fft_fwake_1D_e490_N20.pdf}}
  \caption{Wake of $L=5\%$ width extracted in stator and rotor 
  blocks. Signal and spatial Fourier analysis for different computations.}
  \label{fig:spatial_crit}
\end{figure}

The azimuthal velocity distributions (left hand-side) and the corresponding spatial
spectra (right hand-side)
are presented in Fig.~\ref{fig:spatial_crit} 
for a relative wake thickness of~5\% with respect to the pitch 
and for HB computations using $N=2$, 5, 10 and 18, respectively.
For the stator, the azimuthal distribution follows a 
Gaussian function as expected. On the contrary, 
the rotor distribution is aliased by the HB discretization 
and exhibits spurious oscillations that tend to disappear
when the number of harmonics used in the computation 
increases.
For $N=10$, some oscillations are still present, 
but the wake captured in the moving block begins to 
converge to that leaving the upstream block.

Inspection of the spectra suggests the same conclusions.
The amplitude of $\widehat{\rho U}$ 
improves when increasing the number of harmonics.
As previously mentioned, for under-resolved HB computations,
a dispersion error is introduced and spurious high-frequencies appear 
in the spatial spectra as shown in Fig.~\ref{fig:spatial_crit}
for $N=2$ to $N=10$.
For $N=20$, the spectrum of the rotor 
block matches that of the stator block.
This is consistent with the theoretical analysis, in which more than 
$N=10$ harmonics are needed to capture the wake with less than $20\%$ 
of error for this particular
wake width (see Fig.~\ref{fig:analytic_error_paper}).

In summary, for this wake thickness, the temporal filtering 
on a simulation involving less than ten harmonics is too harsh and leads 
to a significant amount of unresolved energy, 
which deteriorates the numerical representation
of the wake.

For a more quantitative analysis, we compute the error measure
$\varepsilon_1$ for each computation ranging over different 
wake thicknesses and numbers of harmonics. 
Results are summarized in Fig.~\ref{fig:crit_1_3d}.
\begin{figure}[htp]
    \centering\includegraphics[width=.6\textwidth]{error_crit_1.pdf}
  \caption{Evaluation of the error due to the wake 
  capturing using the first error quantification $\varepsilon_1$.}
  \label{fig:crit_1_3d}
\end{figure}
Because it quantifies the unresolved energy in 
comparison to the resolved energy, $\varepsilon_1$ 
exhibits a behavior similar to that of 
the theoretical error $\varepsilon_{th}$ for a Gaussian function 
(Fig.~\ref{fig:analytic_error_paper}).
The iso-error contours have a similar shape 
as the analytical ones. 
The conclusions are equivalent: the truncation error decreases with 
the wake thickness and with the number of harmonics used to capture the wake.
Nevertheless, for thicker wakes and higher numbers of harmonics, 
the error measure $\varepsilon_1$ is over-estimated. 
For instance, around $N=15$ and for $L=25\%$,
$\varepsilon_1 \approx 10^{-2}$ whereas the theoretical error $\varepsilon_{th}$
is less than $10^{-4}$. The error 
measure $\varepsilon_1$ does not represent a 
realistic measure, because of the spatial 
Fourier transform performed to compute 
the error, as discussed in the following.

As can be seen in Fig.~\ref{fig:ST_discrepancies}, 
the Fourier transform of the spatial signal in the stator block tends to a plateau. 
The thicker the wake, 
the lower the frequency for which the plateau appears: 
approximately 15~harmonics for $L=10\%$
(see Fig.~\ref{fig:ST_discrepancies_a}) and 
6~harmonics for $L=25\%$ (see Fig.~\ref{fig:ST_discrepancies_b}).
Actually, for a $N$-harmonic HB computation, the spectrum is 
explicitly filtered in the moving block leading to an amplitude 
equal to zero above the $N^{th}$ harmonic. 
Therefore, when the HB computations are converged, the difference between the spatial 
spectra in the stator and in the rotor block is driven by the plateau present 
in the spatial spectrum of the stator block.
\begin{figure}[htp]
  \begin{center}
  \subfigure[$L=10\%$, $N=3$]{
    \includegraphics[width=.45\textwidth]{PB_FFT_AZI_10_3}\label{fig:ST_discrepancies_a}}
  \subfigure[$L=25\%$, $N=15$]{
    \includegraphics[width=.45\textwidth]{PB_FFT_AZI_25_15}\label{fig:ST_discrepancies_b}}
  \end{center}
  \caption{Discrepancies between spatial and temporal spectra.}
  \label{fig:ST_discrepancies}
\end{figure}

In fact, this behavior is linked to the windowing of the signal on 
a bounded interval, the pitch. To highlight that, the influence of 
a modification on the inlet boundary condition is analyzed.
The inlet wake distortion used in the model turbomachinery configuration is 
originally based on the analytical Lakshminarayana and Davino 
Gaussian law (see Eq.~\eqref{eq:similarity}). However, 
this law is discretized and imposed on a bounded interval 
that spans the angular pitch. As the relative thickness 
increases, the inlet condition diverges from the analytical 
Gaussian law for which the angular pitch is theoretically 
infinite. This is shown in Fig.~\ref{fig:inlet_law_fft} 
through the spectra of three Gaussian laws. The relative 
thickness of the laws are modified through the size 
of the pitch $\Delta \theta$. The multiplication by a factor $100$ 
of the pitch leads to a disappearance of the plateau 
in the spectrum, which accurately matches with the 
Fourier transform of a Gaussian function. 
\begin{figure}[htp]
  \centering
  \subfigure[$\Delta \theta = L$]{\includegraphics[width=.7\textwidth]{PB_FFT_AZI_DEMO_1}}
  \subfigure[$\Delta \theta = 10L$]{\includegraphics[width=.7\textwidth]{PB_FFT_AZI_DEMO_10}}
  \subfigure[$\Delta \theta = 100L$]{\includegraphics[width=.7\textwidth]{PB_FFT_AZI_DEMO_100}}
  \caption{Evolution of the spectrum of the inlet boundary condition for different angular pitch.}
  \label{fig:inlet_law_fft}
\end{figure}

To sum up, a plateau appears in the spatial spectrum of the
stator block. This plateau is explicitly filtered in the
rotor block above the $N^{th}$ harmonic, leading to an over-estimation of the 
first error measure. This over-estimation drives the error value
for higher number of harmonics and thicker wakes.

\FloatBarrier

\subsubsection{Spatial/Time duality error measure}
To get a more realistic error measure, we take 
again into account the energy loss
through the interface, but based on a spatial/time duality. 
As this loss of energy is 
precisely related to the filtering 
introduced on the temporal signal by the HB approach, the second 
error quantification $\varepsilon_2$ addresses the result on 
the temporal information. 

Near the interface of the blocks, consider a fixed observer in
the rotor frame of reference. This observer sees an unsteady 
wake passing as the blocks have a relative speed difference.
The first error quantification has shown the 
influence of the number of harmonics on the spatial signal 
in the rotor block. The error quantification will now
point that this spatial influence is due to a temporal filtering done by
the HB approach.

Following the same notation as in Eq.~\eqref{eq:def_crit_1}, 
the second error measure is written as:
\begin{equation}
    \varepsilon_2(N) = \sqrt{
    \frac{\sum_{f=1}^{f_{max}} | \widehat{s}^{~\theta}_N (f) - 
      \widehat{r}^{~t}_N (f)|^2}{ 
    \sum_{f=1}^{f_{max}} | \widehat{s}^{~\theta}_N (f)|^2}},
    \label{eq:def_crit_2}
\end{equation}
where superscript $t$ denotes the temporal version of
the Fourier transform.
By definition, $\varepsilon_2$
quantifies the matching between a spatial signal
and a temporal information.
Again, the error is described as the unresolved energy 
in the rotor block, 
divided by the energy of the full spectrum, 
e.g. that of the stator block. 
For $\varepsilon_1$, the amplitude 
of the harmonics above the $N^{th}$ one was imposed to zero. 
On the contrary, for $\varepsilon_2$, the temporal spectrum 
in the rotor block is, 
by essence null above the $N^{th}$ harmonic, as the filtering 
acts on temporal values. 
Details of the algorithm used to compute $\varepsilon_2$ are given in \ref{app:epsilon_2_steps}.

\begin{figure}[htp]
\centering
  \subfigure[$N=5$]{
  \includegraphics[width=.48\textwidth]{interp_wake_W0490_TSM_N005.pdf}
  \label{fig:temp_signal_a}}
  \subfigure[$N=10$]{
  \includegraphics[width=.48\textwidth]{interp_wake_W0490_TSM_N010.pdf}}
  \subfigure[$N=15$]{
  \includegraphics[width=.48\textwidth]{interp_wake_W0490_TSM_N015.pdf}
  \label{fig:temp_signal_c}}
  \caption{Temporal signal seen at loc~1 and loc~2 for a $L=5\%$ wake width.}
  \label{fig:temp_signal}
\end{figure}
Figure~\ref{fig:temp_signal} shows time signals
extracted at two different azimuthal positions at 
the interface of the rotor block, named loc~1 and loc~2. 
The small phase shift between the two 
signals is due to the space lag between the two points, 
and is the same for any choice of the number of 
harmonics used in the computation. On the contrary, 
differences in terms of amplitude are only due 
to the use of an insufficient number of harmonics: 
as the number of modes used for the time 
approximation is increased from $N=5$ to $N=15$, 
the amplitude of the space-shifted signals 
tends to converge to the same value, and 
spurious oscillations tend to disappear. Therefore, in the following,
only loc~1 will be considered.

Fig.~\ref{fig:dualite_crit} describes the space and 
time spectra of the axial momentum $\rho U$ at loc~1, 
for computations using $N=2$, 5, $10$ and $20$ 
harmonics and for a wake width of $L=5\%$.
The spatial spectrum contains the whole wavelength 
content associated to the incoming wake; 
on the contrary, due to the filtering introduced 
by the HB approach, the time spectrum is composed of only $N$ harmonics.
\begin{figure}[htp]
\centering
\subfigure[$N=2$]{\includegraphics[width=.45\textwidth]{SpcTme_Dualite_0490_02}}
\subfigure[$N=5$]{\includegraphics[width=.45\textwidth]{SpcTme_Dualite_0490_05}}
\subfigure[$N=10$]{\includegraphics[width=.45\textwidth]{SpcTme_Dualite_0490_10}}
\subfigure[$N=20$]{\includegraphics[width=.45\textwidth]{SpcTme_Dualite_0490_20}}
\caption{Spatial/time duality for a $L=5\%$ wake width.}
\label{fig:dualite_crit}
\end{figure}

For computations using less than 10 time harmonics, 
time spectra are truncated, and the amplitude of 
$\rho U$ differs from that of the corresponding mode in the spatial spectrum.

As the number of time harmonics is increased, 
the amplitude of lower harmonics becomes closer 
and closer to that of the corresponding harmonic 
in the reference signal, and errors move toward 
the higher resolved harmonics. For $N=20$, 
the amplitudes of the 20~resolved harmonics are 
similar for both the time and space spectra.

In summary, the preceding analysis shows that, 
for under-resolved HB computations, the time 
signal is affected by both amplitude and phase errors, 
since the energy content is redistributed incorrectly 
among the resolved harmonics.

To quantify this error, we apply the error measure~\eqref{eq:def_crit_2}
to HB computations of the model turbomachinery 
problem corresponding to different choices 
of the wake thickness and different numbers of 
harmonics. Results are presented in Fig.~\ref{fig:crit_2_3d}.
The $\varepsilon_2$ error map is qualitatively 
and quantitatively similar to the $\varepsilon_1$ 
discussed in the previous Section. 
Again, the truncation error measured using $\varepsilon_2$ 
for thick wakes and high numbers of harmonics 
does not follow the trend observed for the 
theoretical error $\varepsilon_{th}$, 
due to the spatial filtering introduced at the 
interface by the phase-lag condition.

The preceding analysis shows that, for HB computations 
that are well converged in terms in harmonics, 
the spatial spectrum in the stator and the 
time spectrum in the rotor block tend to match, 
except for additional spatial errors introduced
by the use of an azimuthal Fourier transform on a 
bounded interval, which confirms the 
validity of the error measure defined in Eq.~\eqref{eq:def_crit_2}.
\begin{figure}[htp]
   \centering \includegraphics[width=.6\textwidth]{error_crit_2.pdf}
  \caption{Evaluation of the error due to the wake 
  capturing using the second error quantification ($\varepsilon_2$).}
  \label{fig:crit_2_3d}
\end{figure}

\subsection{Comparison with the theoretical error measure}
\label{sub:comp_w_analytic}


The preceding results show that approximated truncation error 
measures computed for the model turbomachinery problem 
using a nonlinear flow model (Euler equations) 
exhibit trends, with respect to the wake thickness 
and number of HB harmonics, in close agreement with the 
theoretical error measure derived in Section~\ref{sec:turbomachine_wake} 
for a Gaussian function. 
Figure~\ref{fig:error_comp_curves} compares the 
different error measures for HB simulations of 
advected wakes of varying thickness versus 
the number of harmonics used for the time discretization. 
This corresponds to horizontal cuts of Figs~\ref{fig:analytic_error_paper}, 
\ref{fig:crit_1_3d} and~\ref{fig:crit_2_3d}. 
For number of harmonics higher than the cutoff 
harmonic used in the phase-lag condition 
the three error measures are seen to give 
results in very close agreement. After that value, 
both the $\varepsilon_1$ and $\varepsilon_2$ error 
measures applied to the model turbomachinery problem 
exhibit a plateau.
The preceding remarks suggest the idea that, 
since all error measure provide similar results, 
at least up to numbers of harmonics of interest for 
practical applicative problems, an a priori 
estimate of the number of harmonics required 
to achieve a given error level could be 
obtained by using the theoretical error measure 
Eq.~\eqref{eq:analytical_conv}, if a quick 
estimate of the wake thickness characteristic 
of a given turbomachinery problem is available. 
In the next Section, we show that a reasonable 
estimate can be obtained from a preliminary steady 
computation based on the mixing plane interface condition.
\begin{figure}[htp]
  \centering
  \subfigure[$L = 2 \%$]{\includegraphics[width=.46\textwidth]{RB_MP_0200_error_logY}}\quad
  \subfigure[$L = 5 \%$]{\includegraphics[width=.46\textwidth]{RB_MP_0490_error_logY}}\quad
  \subfigure[$L = 10 \%$]{\includegraphics[width=.46\textwidth]{RB_MP_0965_error_logY}}\quad
  \subfigure[$L = 15 \%$]{\includegraphics[width=.46\textwidth]{RB_MP_1520_error_logY}}\quad
  \caption{Truncation, computed and analytical errors for four wake width.}
  \label{fig:error_comp_curves}
\end{figure}

\subsection{Toward an priori error estimate}
In order to define an \emph{a priori} error measure 
that can be used to estimate the number of 
harmonics required to achieve a reasonable 
convergence of the HB method, we suggest to 
evaluate the wake thickness by using a preliminary 
mixing plane steady computation. Indeed, if 
potential effects due to the downstream row can 
be neglected, the spatial information at the interface 
in the stator block, essentially due to the incoming 
wakes, can be captured without taking into account 
the relative motion between the wheels, \emph{i.e.} 
by means of a mixing plane computation. 
Given the approximated azimuthal distribution at 
the stator interface, we consider the cumulative
energy content of the signal up to a given frequency $f$ 
(or, equivalently, to a given harmonic $N=f/f_1$ where $f_1$ is the
frequency value of the considered unsteadiness). 
The cumulative energy is defined as:
\begin{equation}
    E(f) = \frac{
      \int_0^f | \widehat{g}(\zeta)|^2 \diff \zeta
    }{
      \int_0^\infty | \widehat{g}(\zeta)|^2 \diff \zeta
    },
\end{equation}
where $\widehat{g}$ is the spectrum of the quantity of interest.
By comparison with Eq.~\eqref{eq:def_truncation_error},
the relation between the relative accumulated energy $E$
and the truncation error $\varepsilon_{mxp}$:
\begin{equation}
    E(f) = 1 - \varepsilon_{mxp}^2 (f).
    \label{eq:correspond_E_error}
\end{equation}


Note that this last error measure is based only on 
the amount of unresolved energy that is left 
in a computation if the spatial signal is 
truncated at a given cutoff frequency $f$, 
and does not require any information from the rotor
block, but it depends only on the characteristics 
of the incoming wake.

To check if the new error measure represents an 
accurate estimate of the truncation error of 
an HB simulation, we carry out again a 
parametric study of the error versus different 
wake thicknesses and numbers of harmonics 
(equivalently, cutoff frequencies), and compare 
the results to those of the \emph{a posteriori} error measures 
obtained for the model turbomachinery problem and for the 
theoretical error $\varepsilon_{th}$. 
Results corresponding to $\varepsilon_{mxp}$ are 
superposed to the corresponding curves in Fig.~\ref{fig:error_comp_curves}. 
The \emph{a priori} error measure ($\varepsilon_{mxp}$) matches 
the theoretical estimate ($\varepsilon_{th}$)
and the \emph{a posteriori} measures ($\varepsilon_1$, $\varepsilon_2$)
over a wide range of harmonics. Similarly to the \emph{a posteriori}
errors $\varepsilon_1$ and $\varepsilon_2$, the \emph{a priori} error
exhibits a plateau for high $N$ and high wake thicknesses, 
due to the application of the Fourier transform on a bounded interval. 
We also stress the close agreement between 
$\varepsilon_{mxp}$ and $\varepsilon_{th}$: specifically, 
estimates of the number of harmonics needed to capture 99\% 
of the cumulative energy (equivalently, to get a 
truncation error equal to 10\%) are identical for 
all error measures.

\section{Application to a contra-rotating open rotor configuration}
\label{sec:CROR}
%!TEX root = ../../../adrien_gomar_phd.tex

In contrast to turbomachinery applications, convergence
on CROR configurations
in terms of harmonics has been observed to be
slow on some configurations.

\subsection{Presentation of the cases}

To investigate this issue, two CROR configurations are studied at
different operating conditions:
\begin{enumerate}
\item a Mock-up CROR (noted \mockup) designed by Safran to be
  investigated in a wind tunnel (\emph{i.e.} ground condition:
  $P_i=101,300$~Pa and $T_i=293$~K). Two regimes are considered
  representative of low and high-speed conditions (different rotation
  speeds and blade angles). This configuration is the one
  studied in this thesis and detailed results will be given 
  in Chapters~\ref{cha:dream_ls_isolated} and~\ref{cha:dream_hs_isolated},
\item the Airbus-designed AI-PX7 CROR (noted \aipx) at cruise
  condition: high-speed 
  and flight level (\emph{i.e.}  $P_i=23,842$~Pa and
  $T_i=219.6$~K). This configuration has been studied by
  \citet{ThesisFrancois} in his PhD thesis and is used
  here for comparison.
\end{enumerate}

\subsection{Results of HB computations}

Figures~\ref{fig:mulscv},
\ref{fig:muhscv} and \ref{fig:aipx7cv} show the non-dimensional
entropy at 75\% span computed by the HB method for the three
configurations. The \mockup-LS configuration has the fastest
convergence. There are indeed some spurious entropy waves downstream the blade
row interface for $N=1$ and 2 but none are observed
starting $N=4$.
\begin{figure}[htp]
  \centering
  \subfigure[$N=1$]{\includegraphics[width=.3\textwidth]{dream_LS_N01_entropy_75.jpg}}
  \subfigure[$N=2$]{\includegraphics[width=.3\textwidth]{dream_LS_N02_entropy_75.jpg}}
  \subfigure[$N=3$]{\includegraphics[width=.3\textwidth]{dream_LS_N03_entropy_75.jpg}}
  \subfigure[$N=4$]{\includegraphics[width=.3\textwidth]{dream_LS_N04_entropy_75.jpg}}
  \subfigure[$N=5$]{\includegraphics[width=.3\textwidth]{dream_LS_N05_entropy_75.jpg}}
  \subfigure[$N=6$]{\includegraphics[width=.3\textwidth]{dream_LS_N06_entropy_75.jpg}}
  \caption{\mockup-LS convergence -- Non-dimensional entropy at 75\% span.}
  \label{fig:mulscv}
\end{figure}

For the \mockup-HS configuration, one can
observe in Fig.~\ref{fig:muhscv}(g) that the $N=7$~HB~computation still
presents some spurious waves downstream the interface. It becomes
negligible for a finer sampling. The main difference with the \mockup-LS
configuration is the blade angle. By comparing Fig.~\ref{fig:mulscv}
and Fig.~\ref{fig:muhscv}, one can observe that the blade angle is
higher in the high-speed case. Therefore, even if the wake is of similar
thickness downstream the front rotor, it impacts the axial blade row
interface with a higher angle and therefore looks thinner: Assuming
the flow angle downstream the trailing edge is the same as the
blade incidence angle~$\xi$, the wake thickness observed by the blade row
interface $L_{itf}$ is 
\begin{equation}
  L_{itf}=\frac{L}{\cos(\xi)}.
\end{equation}
When $\xi$ rises from low-speed to high-speed configuration, $L$ will
remain almost constant but $L_{itf}$ will decrease and the spectrum
becomes richer.
\begin{figure}[htp]
  \centering
  \subfigure[$N=1$]{\includegraphics[width=.3\textwidth]{dream_HS_N01_entropy_75.jpg}}
  \subfigure[$N=2$]{\includegraphics[width=.3\textwidth]{dream_HS_N02_entropy_75.jpg}}
  \subfigure[$N=3$]{\includegraphics[width=.3\textwidth]{dream_HS_N03_entropy_75.jpg}}
  \subfigure[$N=4$]{\includegraphics[width=.3\textwidth]{dream_HS_N04_entropy_75.jpg}}
  \subfigure[$N=5$]{\includegraphics[width=.3\textwidth]{dream_HS_N05_entropy_75.jpg}}
  \subfigure[$N=6$]{\includegraphics[width=.3\textwidth]{dream_HS_N06_entropy_75.jpg}}
  \subfigure[$N=7$]{\includegraphics[width=.3\textwidth]{dream_HS_N07_entropy_75.jpg}}
  \subfigure[$N=8$]{\includegraphics[width=.3\textwidth]{dream_HS_N08_entropy_75.jpg}}
  \subfigure[$N=9$]{\includegraphics[width=.3\textwidth]{dream_HS_N09_entropy_75.jpg}}
  \caption{\mockup-HS convergence -- Non-dimensional entropy at 75\% span.}
  \label{fig:muhscv}
\end{figure}

For the \aipx configuration, Fig.~\ref{fig:aipx7cv} shows that the convergence is
not achieved as the finest HB computation ($N=10$) still not capture correctly
the wake through the interface. It is thickened by the low time
resolution. Although the solver is
able to account for an arbitrary number of time samples, the required
memory becomes too demanding and only $N=1$ to 10 were attempted. 
\begin{figure}[htp]
  \centering
  \subfigure[$N=1$]{\includegraphics[width=.3\textwidth]{aipx7_N01_entropy_75.jpg}}
  \subfigure[$N=2$]{\includegraphics[width=.3\textwidth]{aipx7_N02_entropy_75.jpg}}
  \subfigure[$N=3$]{\includegraphics[width=.3\textwidth]{aipx7_N03_entropy_75.jpg}}
  \subfigure[$N=4$]{\includegraphics[width=.3\textwidth]{aipx7_N04_entropy_75.jpg}}
  \subfigure[$N=5$]{\includegraphics[width=.3\textwidth]{aipx7_N05_entropy_75.jpg}}
  \subfigure[$N=6$]{\includegraphics[width=.3\textwidth]{aipx7_N06_entropy_75.jpg}}
  \subfigure[$N=7$]{\includegraphics[width=.3\textwidth]{aipx7_N07_entropy_75.jpg}}
  \subfigure[$N=8$]{\includegraphics[width=.3\textwidth]{aipx7_N08_entropy_75.jpg}}
  \subfigure[$N=9$]{\includegraphics[width=.3\textwidth]{aipx7_N09_entropy_75.jpg}}
  \subfigure[$N=10$]{\includegraphics[width=.3\textwidth]{aipx7_N10_entropy_75.jpg}}
  \caption{\aipx-HS convergence -- Non-dimensional entropy at 75\% span.}
  \label{fig:aipx7cv}
\end{figure}

The are two main differences with the \mockup-HS configuration:
\begin{enumerate}
\item the \aipx configuration is at scale meaning the radial extent is
  several times larger than the \mockup configuration. As the
  pitchwise relative wake width is defined as
  \begin{equation}
    L_{pitch}=L\frac{B}{2\pi R},
  \end{equation}
  the relative wake width will decrease for higher radius~$R$. It also
  explains the difference with classical turbomachinery: as the number
  of blades $B$ can be one order of magnitude higher in the latter
  case than in a CROR configuration and the diameter lower, the
  relative wake thicknesses are higher and the spectrum narrower.
\item the viscosity is also different: applying the Sutherland law for
  air at ground and flight level leads to dynamic viscosities of,
  respectively, $1.807\cdot 10^{-5}$~Pa.s and $1.434\cdot
  10^{-5}$~Pa.s. With lower viscosity, the blade boundary layer is
  thinner and the generated wakes are thinner as well. Furthermore,
  the mixing with the main flow is weaker and the thickening of the
  wakes is also slower leading to a thinner wake reaching the blade
  row interface.
\end{enumerate}

\FloatBarrier

\subsection{Prediction tool based on the wake thickness}
To estimate the wake thickness, a curve fitting algorithm is used to
fit the CFD wakes to the Lakshminarayana and Davino Gaussian wake
law.
Only the relative span between 10\% and 70\% is
considered as elsewhere, the wake interacts with the hub boundary layer and
tip vortex.  This
estimation is plotted in Fig.~\ref{fig:crorwakethick} for the three
configurations.
\begin{figure}[htp]
  \centering
  \includegraphics[width=.45\textwidth]{CROR_THICKNESS.pdf}
  \caption{Estimation of the relative wake thickness for the three contra-rotating
  open rotor configurations.}
  \label{fig:crorwakethick}
\end{figure}
The wake thickness is almost constant along the span for the two HS configurations.
In opposite, the \mockup-LS shows an increase at 50\% of the
relative span. This is due to a large tangential distortion that is
attributed to  flow separation. Thus, the wake width estimation
is not reliable in this region for the \mockup-LS configuration as the
tangential distortion is no longer Gaussian-shaped.
Nevertheless, using Fig.~\ref{fig:crorwakethick}, the wake widths
of the \aipx-HS, the \mockup-HS and the \mockup-LS are approximately
4\%, 9.5\% and 20\%, respectively. 

The level of accumulated energy required 
for a computation to be rigorously converged
is difficult to estimate. 
It seems reasonable, from an engineering standpoint, to consider
that a 99\% accumulation of energy should be a good criterion.
To emphasize that,
the reconstruction of a wake as a function of four levels of cumulative
energy $E$ is depicted in Fig.~\ref{fig:level_of_energy}. 
\begin{figure}[htp]
  \centering
  \includegraphics[width=.5\textwidth]{LEVEL_OF_ENERGY_PAPER.pdf}
  \caption{Reconstructions of a wake depending on
  the energy content kept in the signal.}
  \label{fig:level_of_energy}
\end{figure}
One can see
that a reconstruction using only 50\% of the energy
leads to a signal that has neither
the right wake deficit nor the correct width. Using
90\% and 95\% of the energy improve the resulting shape
but  large secondary
oscillations remain, with a bad capture
of the wake deficit.
In opposite, by using 99\% of the energy to reconstruct
the signal, only minor
oscillations are seen but 
the wake width and deficit are recovered with more than 
95\% accuracy.
Thus, the 99\% energy threshold ensures that the wake
will be correctly transmitted to the opposite row, which is
the prior concern of this paper.
Therefore, based on this value
and the estimation of the wake width
for all the three CROR configurations shown in Fig.~\ref{fig:crorwakethick},
one can evaluate the number of harmonics needed to compute such
applications.
In fact, based on the
analytic formula derived in Sec.~\ref{sec:turbomachine_wake}
and the equivalence of truncation error and accumulated energy given by
Eq.~\eqref{eq:correspond_E_error},
if the wake width is known, one can deduce the
number of harmonics~$N$ needed to capture a target level of accumulated energy
$E$:
\begin{equation}
    N(E) = \frac{\erfc^{-1} \left[1 - E \right]}{
    \sqrt{2 \alpha^\prime}},
    \label{eq:estimation_nb_harms}
\end{equation}
where $\alpha^\prime$ is
the wake parameter as defined in 
Sec.~\ref{sec:turbomachine_wake}:
\begin{equation}
    \alpha^\prime(L) =  \frac{1}{0.693} \left( \frac{\pi L}{2} \right)^2.
\end{equation}
Here, the theoretical estimation of the number of harmonics needed 
to recover 99\% of the energy is then
17, 7 and 3 for, respectively, the \aipx-HS, the \mockup-HS
and the \mockup-LS. These numbers explain why the 
\aipx-HS configuration is still not converged after $N=10$
harmonics. In fact, such a computation leads to recover only
87\% of the signal energy. Figure~\ref{fig:level_of_energy}
supports the argument that with this level of energy, the wake
is not properly captured as a 90\% energy signal
does not accurately estimate the wake deficit and thickness.

With this approach, one can deduce approximately
the number of harmonics needed to compute such CROR
configurations using Fourier-based time methods for a target level
of accumulated energy. The issue is that it is limited to
Gaussian wakes. If the wake shape is very different from a Gaussian
curve or if another tangential
distortion reaches the interface, the present
prediction tool cannot be used. 
However, as demonstrated in Sec.~\ref{sub:comp_w_analytic},
the analytic error and the error based on an
azimuthal Fourier transform of the distortion
seen just upstream the interface for a mixing-plane
configuration are equivalent. 


\subsection{Prediction tool based on an azimuthal Fourier transform}
\label{sub:prediction_tool_azimuthal_fft}
Thus, a more general way to analyze the spectrum in a wake is
to perform an azimuthal Fourier transform at the rows interface
in a mixing-plane computation. It encompasses both the wake analysis done above and also
any tangential disturbances, as for instance
the viscosity effects near the hub or the tip vortex.
Details of the algorithm used to compute the tangential accumulated
energy from a mixing plane computation are given in \ref{app:epsilon_cror_steps}.

To have a global insight of the energy contained in the
tangential distortion across the whole span,
the energy accumulation is plotted using a color map
in Fig.~\ref{fig:crorroxvmapenergy}.
Three contour lines are added to ease the
interpretation: 90\%, 95\%
and 99\% of accumulated energy, corresponding to a truncation
error of respectively 30\%, 20\% and 10\%.
\begin{figure}[htp]
  \centering
  \subfigure[\mockup-LS]{\includegraphics[width=.46\textwidth]{DREAM_LS_RANS_ROE2_SPECTRUM_PPT.pdf}}
  \subfigure[\mockup-HS]{\includegraphics[width=.46\textwidth]{DREAM_HS_RANS_ROE2_SPECTRUM_PPT.pdf}}
  \subfigure[\aipx-HS]{\includegraphics[width=.46\textwidth]{AIPX7_RANS_SPECTRUM_PPT.pdf}}
  \caption{Energy accumulation by harmonics for all spans.}
  \label{fig:crorroxvmapenergy}
\end{figure}
The richer spectrum is observed 
in the wake region between
10\% and 70\% of relative span. This is the region where the wake is
influenced neither by the hub boundary layer nor by the tip
vortex. Therefore the wake drives 
the convergence of HB computations.
Results are in good agreement with the prediction tool
based on the wake thickness. To emphasize that, the number of harmonics
needed to have 99\% of the energy is given in 
Tab.~\ref{tab:predicted_N_CROR} for a relative
span between 10\% and 70\%.
\begin{table}
  \ra{1.3} 
  \centering
  \begin{tabular}{l|ccc}
    \toprule
    configuration & \aipx-HS & \mockup-HS & \mockup-LS \\
    \midrule
    wake thickness & 17 & 7 & 3 \\
    azimuthal Fourier transform & 16 & 7 & 4 \\
    \bottomrule
  \end{tabular}
\caption{Predicted number of harmonics associated to $E = 99\%$ of 
accumulated energy, using two prediction tools,
the first based on the wake thickness and the second based on
an azimuthal Fourier transform.}
\label{tab:predicted_N_CROR}
\end{table}


This prediction tool is more accurate as it handles
wake tangential distortion as well as any other
type of azimuthal distortions. 
Thus, it can be used to predict
the number of harmonics needed to capture a certain
level of energy for any relative span.
The computational time needed to
get the accumulated energy pictures as in
Fig.~\ref{fig:crorroxvmapenergy} is negligible. In fact, it takes less
than a minute.

We verify \emph{a posteriori} that the number of harmonics
provided in Tab.~\ref{tab:predicted_N_CROR} are sufficient
to yield converged HB computations. For the \mockup-LS, the prediction tool
estimate that four harmonics are sufficient. In fact, Fig.~\ref{fig:mulscv}
supports the argument that four harmonics gives a converged simulation as
the difference between $N=4$, $N=5$ and $N=6$ HB computations are barely
visible. For the \mockup-HS, seven harmonics are estimated to be sufficient
while visually, it seems that $N=8$ is converged. 
In fact, one must keep in mind that these criteria just give a lower
bound of the required number of harmonics needed to get the
convergence of the HB method. Indeed, when running a $N$-harmonic HB
computation, the time period is sampled with $2N+1$ time instants
which is, according to the \citet{Nyquist1928}
criteria,
the minimum sampling to get the $N$\textsuperscript{th} of the
fundamental frequency. It does not necessarily mean that the level of
the $N$\textsuperscript{th} harmonic is accurately
predicted. Experience shows that in order to reach this level, one has
to run a $N+1$ or $N+2$ HB computation. 

Figure~\ref{fig:crorroxvmap} shows the non-dimensional
axial momentum extracted at the rotor/rotor interface
from a single-passage mixing-plane computation
for the three considered configurations.
One can observe different wake shapes: the \aipx High-Speed (HS) wake looks much
thinner than the \mockup-HS, which looks
thinner than the Low-Speed (LS) one. Indeed, the latter does not show a
well delimited wake structure all along the span explaining the estimation of the
number of harmonics needed to capture such configurations.
\begin{figure}[htp]
  \centering
  \subfigure[\mockup-LS]{\includegraphics[width=.2\textwidth]{dream_ls.png}}
  \subfigure[\mockup-HS]{\includegraphics[width=.2\textwidth]{dream_hs_roe2.png}}
  \subfigure[\aipx-HS]{\includegraphics[width=.2\textwidth]{aipx7_hs.png}}
  \caption{Non-dimensional axial momentum $(\rho U)/(\rho U)_\infty$ 
  at the rotor/rotor interface (mixing-plane computations).}
  \label{fig:crorroxvmap}
\end{figure}

\chconclu{The accuracy and efficiency of Fourier-based time methods 
used to solve periodic unsteady problems depends on the number of harmonics
chosen to represent the frequency content of the time signal.
In this work we investigate the accuracy and convergence properties 
of Fourier-based time integration methods. The convergence rate 
of these methods, in terms of harmonics required to describe the solution 
with a given level of accuracy, depends on the spectral content of the 
solution itself: Fourier-based time methods are particularly efficient 
for flow problems characterized by a narrow Fourier 
spectrum. Starting from this remark, we try to define a relevant 
indicator of solution regularity in the specific case of turbomachinery 
flows, which represent one of the main applications of Fourier-based 
time methods in Fluid Mechanics.
To this aim, we show that main source of unsteadiness in 
turbomachinery flows is due to the relative motion of wakes 
generated by a given blade row with respect to the downstream row. 
Statistically speaking, the passing wakes are seen by the downstream 
row as an azimuthally advected periodic Gaussian pulse, 
characterized by its relative thickness compared to the pitch 
in between two subsequent blades and by the velocity deficit 
associated to it. We show that the narrower the wake, the larger 
its Fourier spectrum, and the slower the convergence of Fourier-based time methods.
In order to achieve a priori estimates of the number of 
harmonics required to accurately solve a given turbomachinery 
problem, we introduce two error measures based on the relative 
thickness of the passing wakes. It is shown that, for practical 
purposes, these can be preliminarily estimated by running a 
companion steady simulation of the turbomachinery stage. The 
steady simulation is post-processed to extract information about 
the spanwise distribution of wake thickness, and an error criterion 
is used to estimate the number of harmonics required to resolve 99\% 
of the energy content associated to the velocity signal. The 
preliminary step has a negligible cost compared to the overall 
simulation, since the steady computation is used to initialize 
the unsteady run, and extraction of wake characteristics takes 
less than a minute on a single processor. 

The proposed methodology represents an efficient and reliable 
operational tool to guide the choice of the number of harmonics 
for a given turbomachinery problem, and to evaluate beforehand the 
interest of applying or not a Fourier-based time integration scheme 
instead of a classical time-marching scheme.}
