%!TEX root = ../../../adrien_gomar_phd.tex
\subsection{Presentation of the case}
\label{sub:presentation_of_the_case}

A channel configuration is set up to study the properties of the
Fourier-based method used in this thesis, namely the harmonic balance
method. In this 2D channel, a constant left injection is 
supplemented with an unsteady oscillating right pressure condition.
These unsteady pressure waves travel within the flow at the velocity $u + c$ and $u
- c$, where $u$ denotes the local flow velocity and $c$ the sound
velocity. Since the pressure waves are generated at the outlet, only
the $u-c$ waves are seen, resulting in pressure waves propagating
upstream of the channel, which are damped by the effect of
viscosity. Figure~\ref{fig:canal_principle} shows a sketch
of the channel case, illustrating the propagation and attenuation of
the pressure waves.
\begin{figure}[htb]
  \centering
  \includegraphics*[width=0.6\textwidth]{channel_sketch.pdf}
  \caption{Sketch of the channel case.}
  \label{fig:canal_principle}
\end{figure}

\subsection{Numerical setup}

% mesh presentation
It is a 2D channel of length $L_x = 100$~m in the axial
direction and $L_z = 1$~m in the transverse one.
The mesh consists of 1000~points along the axial direction and 10 in the
transverse one, which corresponds to equal spacings in both
directions.

% solver
The \emph{elsA} solver~\cite{Cambier2013} developed by ONERA
is used as well as the time-domain harmonic 
balance method implemented by \citet{JSicot2008}. 
This code solves the RANS equations using a cell-centered
approach on multi-blocks structured meshes.

% boundary conditions
The boundary conditions are: (i)~a constant injection condition for the inlet
where the total pressure $P_{i_0}$ and enthalpy $h_{i_0}$ are set,
(ii)~symmetric conditions for the upper and lower bounds as the flow
is assumed to be symmetric in the transverse direction, and (iii)~a
fluctuating static pressure imposed at the outlet:
\begin{equation}
  P_{outlet}(t) = P_m \left[1 + A_1 \sin(2 \pi f_1 t) +
    A_2 \sin(2 \pi f_2 t) \right],
  \label{eq:outlet_canal}
\end{equation}
where $P_m$ is the temporal average static pressure, $A_n$ the
amplitude of the $n$\textsuperscript{th} mode and $f_n$ its
frequency. The mean outlet static pressure $P_m$ is set to $60\%$ of the
inlet total pressure which is set to $P_{i_0} = 101,325$~Pa.
Through this total-to-static pressure ratio, a mean Mach
number $M_0 = 0.54$ is imposed.

% numerical schemes
This configuration is turbulent as the Reynolds number based on the
inlet flow velocity and the axial length of the channel is about $R_e
\approx 2.0 \times 10^9$.  Turbulence is modeled using the
one-equation model of Spalart and Allmaras~\cite{Spalart1992}, and the
third-order upwind Roe scheme~\cite{Roe1981} is used to compute the
convective fluxes.

\subsection{Validation of the toy problem}
An analytical solution is provided by \citet{Merkle1987} for
incompressible flows. This has been used by \citet{McMullen2001}
to validate the implementation of their NLFD approach.
However, this toy problem is used
to assess the properties of the harmonic balance within
the Navier--Stokes equations framework. Therefore,
this analytical solution will not be used and the validation of
this tool will rely on classical time-marching, namely DTS, solutions.
