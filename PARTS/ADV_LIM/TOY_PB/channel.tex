%!TEX root = ../../../adrien_gomar_phd.tex

A channel configuration is set up to study the properties of the
proposed HB method and the above algorithms for non-uniform time
sampling.  It is a 2D channel of length $L_x = 100$~m in the axial
direction and $L_z = 1$~m in the transverse one.  The boundary
conditions are: (i)~an injection condition for the inlet,
(ii)~symmetric conditions for the upper and lower bounds as the flow
is assumed to be symmetric in the transverse direction, and (iii)~a
fluctuating pressure imposed at the outlet:
\begin{equation}
  P_{outlet}(t) = P_m \cdot \left[1 + A_1 \cdot \sin(2 \pi f_1 t) +
    A_2 \cdot \sin(2 \pi f_2 t) \right],
  \label{eq:outlet_canal}
\end{equation}
where $P_m$ is the temporal average static pressure, $A_n$ the
amplitude of the $n$\textsuperscript{th} mode and $f_n$ its
frequency. The mean outlet pressure $P_m$ is set to $60\%$ of the
inlet total pressure $P_{i0} = 101,325$~Pa.

Pressure waves travel within the flow with the velocity $u + c$ and $u
- c$, where $u$ denotes the local flow velocity and $c$ the sound
velocity. Since the pressure waves are generated at the outlet, only
the $u-c$ waves are visible, resulting in pressure waves propagating
upstream of the channel, which are damped by the effect of
viscosity. Figure~\ref{fig:canal_principle} shows a schematic diagram
of the channel case, illustrating the propagation and attenuation of
the pressure waves.
\begin{figure}[htb]
  \centering
  \includegraphics*[width=0.6\textwidth]{CANAL2_PRINCIPLE.pdf}
  \caption{Schematic diagram of the channel case.}
  \label{fig:canal_principle}
\end{figure}

% mesh presentation
The mesh consists of 997~points along the axial direction and 9 in the
transverse one, which amounts to almost equal spacings in both
directions.

This configuration is turbulent as the Reynolds number based on the
inlet flow velocity and the axial length of the channel is about $R_e
\approx 2.0 \times 10^9$.  Turbulence is modeled using the
one-equation model of Spalart and Allmaras~\cite{Spalart1992}, and the
third-order upwind Roe scheme~\cite{Roe1981} is used to compute the
convective fluxes.