%!TEX root = ../../../adrien_gomar_phd.tex
\subsection{Presentation of the case}
\label{sub:presentation_of_the_case}

The second toy problem considered represents a 2D channel 
with a constant left injection at 
a transonic Mach number ($M=0.7$)
supplemented with a time-varying unsteady back pressure.
As the pressure is oscillating at the outlet, the imposed unsteady pressure
fluctuations travel within the flow at the velocity 
$u + c$ and $u - c$, where $u$ denotes 
the local flow velocity and $c$ the speed of sound.
Since the pressure waves are generated at the outlet, only
the $u-c$ waves are seen, resulting in pressure waves propagating
upstream of the channel. The axial length of the channel is $L_x = 100$~m
and $L_y = 1$~m in the transverse direction.
Figure~\ref{fig:canal_principle} shows a sketch
of the considered channel flow problem.
\begin{figure}[htb]
  \centering
  \includegraphics*[width=0.5\textwidth]{channel_sketch.pdf}
  \caption{Sketch of the channel flow problem.}
  \label{fig:canal_principle}
\end{figure}

\citet{Merkle1987} give an analytical solution
for incompressible flows with small pressure fluctuations, assuming
thus a linear unsteady flow.
However, this toy problem is set up to highlight the properties
of the harmonic balance in a non-linear framework which is not
the hypothesis of \citet{Merkle1987}. Nevertheless, to give confidence
in our forthcoming results for this toy problem,
this last is validated below against a classical time-marching scheme
(see Sec.~\ref{sec:channel_multifreq}).

\subsection{Numerical setup}

% mesh presentation
The mesh consists of 997~points along the axial direction and 9~in the
transverse one, which corresponds to equal spacings in both
directions. 

% boundary conditions
The boundary conditions are: (i)~a constant injection condition for the inlet
where the total pressure $p_{i_0}$ and enthalpy $h_{i_0}$ are set,
(ii)~symmetric conditions for the upper and lower bounds as the flow
is assumed to be symmetric in the transverse direction, and (iii)~a
fluctuating static pressure imposed at the outlet:
\begin{equation}
  p_{s_1}(t) = \overline{p}_{s_1} \left[1 + a_1 \sin(2 \pi f_1 t) +
    a_2 \sin(2 \pi f_2 t) \right],
  \label{eq:outlet_canal}
\end{equation}
where $\overline{p}_{s_1}$ is the temporal average static pressure, $a_n$ the
amplitude of the $n$\textsuperscript{th} mode and $f_n$ its
frequency. Only two modes ($f_1$, $f_2$) are created
but due to the non-linearity of the Navier--Stokes equations,
new term frequencies rise.
The mean velocity of the flow is imposed through a
static pressure condition $\overline{p}_{s_1}$ at the outlet:
\begin{equation}
    \overline{p}_{s_1} = \frac{p_{i_0}}{\left(1 + 
    \frac{\gamma - 1}{2} M_{0}^2 \right) ^ {\frac{\gamma}{ \gamma - 1}}} ,
\end{equation}
the mean velocity is thus set by imposing the
inlet target mean Mach number value $M_{0}$.
We assume here that the flow is isentropic as no
geometrical object disturbes the flow field.

% solver
The \emph{elsA} solver~\cite{Cambier2013} developed by ONERA
is used to solve this toy problem. In fact, 
the aim of this toy problem is 
twofold: i)~highlight the properties of the harmonic balance
approach within a non-linear framework\mytodo{etre plus precis} 
and ii)~use the same
solver as the one used in the application part of this
thesis so that the results shown here can be directly
transposed. 
This code solves the RANS equations using a cell-centered
approach on multi-blocks structured meshes.
Several time-integration schemes
are available, in particular the Dual Time-Stepping~\cite{Jameson1981} (DTS)
as well as the time-domain harmonic 
balance method implemented by \citet{JSicot2008} for the mono-frequential
formulation and extended by \citet{JGuedeney2013} to multi-frequential flows. 


% numerical schemes
The present configuration is turbulent as the Reynolds number based on the
inlet flow velocity and the axial length of the channel is about $R_e
\approx 2.0 \times 10^9$. To this aim, turbulence is modeled using the
one-equation model of \citet{Spalart1992}, and the
Roe's scheme~\cite{Roe1981} along with a third-order MUSCL extrapolation 
is used for the spatial discretization of
the convective fluxes. An implicit backward Euler scheme is used
to march the HB equations in pseudo-time.


\subsection{Validation of the channel flow toy problem}
\label{sec:channel_multifreq}

As no analytical solution is available for this case, we propose
now to validate the channel flow toy problem within a HB framework
in order to have confidence in the forthcoming results on this
toy problem.
To do so, two non-harmonically related
frequencies are chosen as input for the outlet boundary condition:
$f_1 = 3$~Hz and $f_2 = 17$~Hz.

The classical DTS time-marching scheme is taken for comparison.
Convergence in time discretization is obtained after 20~periods using
160~instants per almost-period. Since the frequencies are integers and
coprime, the period is $T=1$~s.  Iterative convergence for the
inner loop is considered achieved when the normalized residuals drop
by $10^{-2}$ within a maximum of 50~sub-iterations.

The results obtained with the DTS scheme are compared to the HB
results for pressure waves amplitudes of $a = a_1 = a_2 = 0.001$
(see Eq.~\eqref{eq:outlet_canal}).  The
transient of the DTS computation is shown in
Fig.~\ref{fig:canal2_transient}, illustrating the wave propagation
with a slight attenuation of the high-frequency waves.
\begin{figure}[htb]
  \centering
  \includegraphics*[width=.5\linewidth]{CANAL2_TRANSIENT_PPT.pdf}
  \caption{DTS computation: transient propagation of the pressure waves.}
  \label{fig:canal2_transient}
\end{figure}

Due to the non-linearity of the Euler equation, the two frequencies
$f_1$ and $f_2$ give rise to linear combinations of them.
Therefore, the results are analyzed for frequencies $1<f< 40~\textrm{Hz}$ and the
dominant frequencies (the one that have the highest amplitudes) are
set for the HB computation.  To do so, pressure signals are probed
upstream, in the middle and downstream of the channel at
$x=[25~\textrm{m}, 50~\textrm{m}, 75~\textrm{m}]$ and $y=0.5$~m
respectively.  The spectrum of the aforementioned unsteady pressure
signals, obtained with a Fourier Transform is plotted in
Fig.~\ref{fig:canal2_dts_fft}.  The labeled frequencies are the
dominant ones, as for each probe, these have a high amplitude. Those
nine frequencies are thus selected as input frequencies for the HB computation.
\begin{figure}[htb]
  \centering
  \includegraphics*[width=.5\linewidth]{channel_dts_fft_plus_sketch.pdf}
  \caption{Spectrum of pressure signals.}
  \label{fig:canal2_dts_fft}
\end{figure}

The HB computation using the previously mentioned frequencies is
run and a discrete Fourier transform is computed at several axis positions
in the middle of the channel ($y=0.5$~m). Figure~\ref{fig:canal2_validation_hbt_gear_amp_vs_axis}
shows the results for the frequencies that have been set for the HB computation.
The overall agreement between the DTS and the HB is fair.  
Some local discrepancies can be
observed upstream for frequencies $f_2 + 3f_1$, $f_2 - f_1$ and $f_2 -
2f_1$. 
\begin{figure}[htb]
  \centering
  \includegraphics*[width=.5\linewidth]{CANAL2_VALIDATION_HBT_GEAR_PPT_AMP_VS_AXIS.pdf}
  \caption{Spatial evolution of the amplitude of the dominant
    frequencies in the channel flow configuration, for $f_1 = 3$~Hz and $f_2 = 17$~Hz.}
  \label{fig:canal2_validation_hbt_gear_amp_vs_axis}
\end{figure}
These are caused by aliasing~\mytodo{haha}
but they are minimal regarding the temporal evolution, as
shown in Fig.~\ref{fig:canal2_validation_hbt_gear_time_ev}, where the
time evolution of pressure signals is extracted at all probes.  The
difference between the HB and the DTS method is negligible proving
that the present toy problem can be used to analyze the properties of 
the HB method proposed in this thesis.
\begin{figure}[htb]
  \centering 
  \subfigure[Probe
  1]{\includegraphics[width=.3\textwidth]{CANAL2_VALIDATION_HBT_GEAR_TIME_EV_PROBE_1_PPT.pdf}}
   \quad\subfigure[Probe
   2]{\includegraphics[width=.3\textwidth]{CANAL2_VALIDATION_HBT_GEAR_TIME_EV_PROBE_2_PPT.pdf}}
   \subfigure[Probe
   3]{\includegraphics[width=.3\textwidth]{CANAL2_VALIDATION_HBT_GEAR_TIME_EV_PROBE_3_PPT.pdf}}
  \caption{Unsteady pressure signals at different axial positions.}
  \label{fig:canal2_validation_hbt_gear_time_ev}
\end{figure}
