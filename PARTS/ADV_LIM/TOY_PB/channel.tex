%!TEX root = ../../../adrien_gomar_phd.tex
\subsection{Presentation of the case}
\label{sub:presentation_of_the_case}

A 2D channel is considered  with a constant left injection
supplemented with a time-varying unsteady back pressure.
As the pressure is oscillating at the outlet, unsteady pressure
fluctuations are created that travel within the flow at the velocity 
$u + c$ and $u - c$, where $u$ denotes 
the local flow velocity and $c$ the sound velocity.
Since the pressure waves are generated at the outlet, only
the $u-c$ waves are seen, resulting in pressure waves propagating
upstream of the channel. Figure~\ref{fig:canal_principle} shows a sketch
of the channel case, illustrating of
the pressure waves.
\begin{figure}[htb]
  \centering
  \includegraphics*[width=0.6\textwidth]{channel_sketch.pdf}
  \caption{Sketch of the channel case.}
  \label{fig:canal_principle}
\end{figure}

\subsection{Numerical setup}

% mesh presentation
It is a 2D channel of length $L_x = 100$~m in the axial
direction and $L_z = 1$~m in the transverse one.
The mesh consists of 1000~points along the axial direction and 10 in the
transverse one, which corresponds to equal spacings in both
directions.

% solver
The \emph{elsA} solver~\cite{Cambier2013} developed by ONERA
is used to solve for the channel flow. Several time-integration schemes
are available: DTS, GEAR, UNS, 
as well as the time-domain harmonic 
balance method implemented by \citet{JSicot2008}. 
This code solves the RANS equations using a cell-centered
approach on multi-blocks structured meshes.

% boundary conditions
The boundary conditions are: (i)~a constant injection condition for the inlet
where the total pressure $p_{i_0}$ and enthalpy $h_{i_0}$ are set,
(ii)~symmetric conditions for the upper and lower bounds as the flow
is assumed to be symmetric in the transverse direction, and (iii)~a
fluctuating static pressure imposed at the outlet:
\begin{equation}
  p_{outlet}(t) = p_m \left[1 + a_1 \sin(2 \pi f_1 t) +
    a_2 \sin(2 \pi f_2 t) \right],
  \label{eq:outlet_canal}
\end{equation}
where $p_m$ is the temporal average static pressure, $a_n$ the
amplitude of the $n$\textsuperscript{th} mode and $f_n$ its
frequency.
The mean velocity of the flow is imposed through a
static pressure condition $p_m$ at the outlet:
\begin{equation}
    p_m = \frac{p_{i_0}}{\left(1 + 
    \frac{\gamma - 1}{2} M_{\inf}^2 \right) ^ {\frac{\gamma}{ \gamma - 1}}} ,
\end{equation}
the mean velocity is thus set by imposing the
target mean Mach number value $M_{\inf}$.
At the azimuthal boundaries, phase-lag conditions~\cite{Erdos1977} 
are used to take into account for the space-time periodicity.

% numerical schemes
This configuration is turbulent as the Reynolds number based on the
inlet flow velocity and the axial length of the channel is about $R_e
\approx 2.0 \times 10^9$.  Turbulence is modeled using the
one-equation model of \citet{Spalart1992}, and the
third-order upwind \citet{Roe1981} scheme is used to compute the
convective fluxes.

\subsection{Validation of the toy problem}
An analytical solution is provided by \citet{Merkle1987} for
incompressible flows. This has been used by \citet{McMullen2001}
to validate the implementation of their NLFD approach.
However, this toy problem is used
to assess the properties of the harmonic balance within
the Navier--Stokes equations framework. Therefore,
this analytical solution will not be used and the validation of
this tool will rely on classical time-marching, namely DTS, solutions.
