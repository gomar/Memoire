%!TEX root = ../../../adrien_gomar_phd.tex

\subsection{Presentation of the case}

The first toy problem concerns the resolution of the
advection equation defined as:
\begin{equation}
  \label{eq:convection}
  \frac{\partial u}{\partial t} + c \frac{\partial u}{\partial x} = 0,
\end{equation}
with the constant advection speed $c$ assumed as positive
and set here to one for simplicity. 
The equation is solved on the domain $x \in [0, L]$. 
Periodic perturbations of different shapes are imposed at the left boundary:
\begin{equation}
   u(0, t) = u_l (t),
\end{equation}
where $u_l$ is a periodic function of period $T=L/c$.
These perturbations are advected across the computational 
domain and leave from the right boundary. After a transient of time length $T_{trans}=L/c$, 
the solution at any point $x$ in the space domain achieves a periodic state. 
The exact solution for this periodic state is a periodic function of the form:
\begin{equation}
    u_{ex}(x,t)=u_l(x/c+t).
\end{equation}
For simplicity, $L$ and $c$ are taken as unity.

\subsection{Numerical setup}

The space derivative is discretized by means of a centered 
fourth-order finite difference scheme on an uniform Cartesian mesh:
\begin{equation}
    \frac{\partial u}{\partial x} (x = x^i, t=t_q) =
    \frac{-u^{i+2}_{q} + 8 u^{i+1}_{q} - 8 u^{i-1}_{q} + u^{i-2}_{q}}{12\Delta x}
    + \mathcal{O} (\Delta x^4),
    \label{eq:convection_center4}
\end{equation}
A very fine space step is used ($\Delta x=5\e{-4}$) in order to rule 
out spatial approximation errors. This corresponds to $2,000$ grid points. 
The solution at the last mesh 
point on the right of the domain is extrapolated 
from the inside. To this aim, a standard second-order 
and a first-order upwind discretization schemes
are used to approximate the space derivative at 
the last two mesh points on the right, respectively:
\begin{align}
    \frac{\partial u}{\partial x} (x = x^{m-1}, t=t_q) &=
    \frac{3 u^{m-1}_{q} - 4 u^{m-2}_{q} + u^{m-3}_{q}}{2\Delta x} + \mathcal{O} (\Delta x^2), \\
    \frac{\partial u}{\partial x} (x = x^m, t=t_q) &= 
    \frac{u^{m}_{q} - u^{m-1}_{q}}{\Delta x} + \mathcal{O} (\Delta x),
\label{eq:upwind_scheme}
\end{align}
where $m$ is the total number of grid points.

Time-discretization is achieved 
through the HB method (described in Sec.~\ref{sec:sm_hb})
with a standard four-step Runge-Kutta method~\cite{Jameson1981}
used to pseudo-time 
march the HB equations to the steady state.
The $k\textsuperscript{th}$ step is evaluated by:
\begin{equation}
    u_k = u_q - \alpha_k \Delta t \left [ 
          c \frac{\partial u_{k-1}}{\partial x} (t=t_q + \alpha_{k-1} \Delta t)
          + D_t(u_k)
          \right],
    \label{eq:convection_rk4}
\end{equation}
where $\alpha_0 = 0$, $\alpha_1 = 1/4$, 
$\alpha_2 = 1/3$, $\alpha_3 = 1/2$, $\alpha_4 = 1$ and 
the HB source term $D_t(u_k)$ is computed using Eq.~\eqref{eq:sm_multi_spectral_operator}.
The CFL number in pseudo-time is set to 1 
to ensure stability of the explicit time-marching scheme.
