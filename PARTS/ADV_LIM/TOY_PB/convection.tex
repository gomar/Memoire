%!TEX root = ../../../adrien_gomar_phd.tex

To properly assess the convergence of harmonic balance computations,
a model problem is set up. The constant convection 
equation is numerically solved within a harmonic balance 
framework. The convection equation is
\begin{equation}
  \label{eq:convection}
  \frac{\partial u}{\partial t} + c \frac{\partial u}{\partial x} = 0,
\end{equation}
where $c$ denotes the constant convection speed and $u$ the velocity. 
It is chosen for two reasons: first, the constant
convection equation is a simple unsteady 
linear partial differential equation
and second, a straightforward analytical solution
exists
which ease the analysis of the numerical solutions. In fact,
it only depends on the boundary conditions:
\begin{equation}
  \label{eq:solconvanalytic}
    u(x, t) = u(x=0, x/c + t) = u_0 (x/c + t),
\end{equation}
where $u_0(t) = u(x=0, t)$ is the temporal evolution of 
the boundary condition ($x=0$). 
The size of the domain is $L_x$ and the number of grid points is
proved to be converged for $2,000$ nodes.

The convection equation is solved within the harmonic
balance framework in time using a finite difference approach
in space.
A $4$\textsuperscript{th} order centered finite
difference scheme is used to evaluate the spatial derivative:
\begin{equation}
    \frac{\partial u}{\partial x} (x = x_i, t=t_q) \approx 
    \frac{-u^{i+2}_{q} + 8 u^{i+1}_{q} - 8 u^{i-1}_{q} + u^{i-2}_{q}}{12\Delta x},
    \label{eq:convection_center4}
\end{equation}
where $u_q^i$ is the concatenation of the velocity for  
all the HB instants,
evaluated at position $i$ within pseudo-iteration $q$.
A four step Runge-Kutta method is then use to time 
march the equation to the steady state with the coefficients $\alpha_0 = 0$,
$\alpha_1 = 1/4$, $\alpha_2 = 1/3$, $\alpha_3 = 1/2$ and $\alpha_4 = 1$.
The k\textsuperscript{th} step is evaluated by:
\begin{equation}
    u_k = u_q - \alpha_k \Delta t \left [ 
          c \frac{\partial u_{k-1}}{\partial x} (t=t_q + \alpha_{k-1} \Delta t)
          + D_t(u_k)
          \right],
    \label{eq:convection_rk4}
\end{equation}
where the HB source term $D_t(u_k)$ is computed using 
Eq.~\eqref{eq:sm_hb_mono_source_term_analytic} if the mono-frequential
formulation is used or Eq.~\eqref{eq:sm_multi_spectral_operator}
for the multi-frequential formulation.
As we use 
an explicit time marching scheme, the CFL number is set to $1$ to ensure stability.

A function is periodically injected at the left boundary
condition~$u_0$.
In this paper, the following functions will be
investigated: a sum of sine functions, a rectangular
function and a Gaussian function. All of these lasts are periodized
using a periodic operator defined as:
\begin{equation}
    \text{period} \colon
    \begin{cases}
        [0: \infty] & \longmapsto[-T/2: T/2]\\
        t & \longmapsto \displaystyle - \frac{T}{2} + t \pmod T,
    \end{cases}
    \label{eq:periodic_operator}
\end{equation}
where $T$ is the period of the unsteady phenomenon. This period 
is chosen so that, when the frequency of the 
phenomenon is set to $1$, only one pattern of the function 
appears at a time in the domain.
This is done by setting $T$ to:
\begin{equation}
   T = \frac{L_x}{c},
   \label{eq:time_spatial_correspondence}
\end{equation}
where $L_x$ is the size of the domain and $c$ the convection speed.

For the right boundary condition, imposing a periodicity condition is
numerically stiff. For this reason, the scheme is degenerated 
to an upwind scheme to avoid wave reflections. It is degenerated to
a 2\textsuperscript{nd} order scheme on the second to last cell
and to a first order scheme on the last cell:
\begin{align}
    \frac{\partial u}{\partial x} (x = x_{m-1}, t=t_q) &\approx 
    \frac{3 u^{m-1}_{q} - 4 u^{m-2}_{q} + u^{m-3}_{q}}{2\Delta x}, \\
    \frac{\partial u}{\partial x} (x = x_m, t=t_q) &\approx 
    \frac{u^{m}_{q} - u^{m-1}_{q}}{\Delta x},
\label{eq:upwind_scheme}
\end{align}
where $m$ is the total number of grid points.

As an analytical solution is known, it is easy to define the error
made by numerically solving the equation with finite difference.
In peculiar, the $\mathcal{L}_2$-norm can be defined as:
\begin{equation}
    \frac{\norm{u-u_{analytic}}_2}{\norm{u_{analytic}}_2} = 
    \sqrt{\frac{\sum_{i=0}^{2N} \left[u(t_i) - u_{analytic}(t_i)  \right]^2}{
    \sum_{i=0}^{2N} \left[u_{analytic}(t_i)\right]^2}},
    \label{eq:l2_norm}
\end{equation}
where $t_i$ are the time instants
in which the harmonic balance solution is sought as 
done in Eq.~\eqref{eq:sm_sampling_hb_var}.
