%!TEX root = ../../../adrien_gomar_phd.tex
\chapter{Presentation of the toy problems}
\label{cha:toy_problems}

\chabstract{In this chapter, two toy problems are
set up to study the advantages and limitations
of the chosen harmonic balance approach. The first
toy problem is the resolution of the linear advection
equation. An
analytic solution is known, allowing to accurately assess
the error introduced by the harmonic balance method.
The second toy problem is a channel flow with an
oscillating back-pressure. It is solved using the 3D Navier--Stokes
equations. The solver used for this last case is the \emph{elsA}
code which is used for the CROR applications.}

\minitoc
\newpage


\section{Periodic linear advection}
\label{sec:toy_convection}
%!TEX root = ../../../adrien_gomar_phd.tex

\subsection{Presentation of the case}

We consider the linear advection equation:
\begin{equation}
  \label{eq:convection}
  \frac{\partial u}{\partial t} + c \frac{\partial u}{\partial x} = 0,
\end{equation}
with the constant advection speed $c$ assumed as positive
and set here to one for simplicity. 
The equation is solved on the domain $x \in [0, 1]$. 
Periodic perturbations of different shapes are imposed at the left boundary:
\begin{equation}
   u(0, t) = u_l (t),
\end{equation}
where $u_l$ is a periodic function of period $T=1/c$.
These perturbations are advected across the computational 
domain and leave from the right boundary. After a transient of time length $T_{trans}=1/c$, 
the solution at any point $x$ in the space domain achieves a periodic state. 
The exact solution for this periodic state is a periodic function of the form:
\begin{equation}
    u_{ex}(x,t)=u_l(x/c+t).
\end{equation}

\subsection{Numerical setup}

The space derivative is discretized by means of a centered 
fourth-order finite difference scheme on an uniform Cartesian mesh:
\begin{equation}
    \frac{\partial u}{\partial x} (x = x^i, t=t_q) \approx 
    \frac{-u^{i+2}_{q} + 8 u^{i+1}_{q} - 8 u^{i-1}_{q} + u^{i-2}_{q}}{12\Delta x},
    \label{eq:convection_center4}
\end{equation}
A very fine space step is used ($\Delta x=0.002$) in order to rule 
out spatial approximation errors. According to the theory 
of characteristics, the solution at the last mesh 
point on the right of the domain is extrapolated 
from the inside. To this aim, a standard second-order 
upwind discretization and a first-order upwind discretization 
are used to approximate the space derivative at 
the last two mesh points on the right, respectively:
\begin{align}
    \frac{\partial u}{\partial x} (x = x^{m-1}, t=t_q) &\approx 
    \frac{3 u^{m-1}_{q} - 4 u^{m-2}_{q} + u^{m-3}_{q}}{2\Delta x}, \\
    \frac{\partial u}{\partial x} (x = x^m, t=t_q) &\approx 
    \frac{u^{m}_{q} - u^{m-1}_{q}}{\Delta x}.
\label{eq:upwind_scheme}
\end{align}
Time-discretization is achieved 
through the HB method described in Sec.~\ref{sec:sm_hb}:
with a standard four-step Runge-Kutta method~\cite{Jameson1981}
used to pseudo-time 
march the HB equations to the steady state.
The $k\textsuperscript{th}$ step is evaluated by:
\begin{equation}
    u_k = u_q - \alpha_k \Delta t \left [ 
          c \frac{\partial u_{k-1}}{\partial x} (t=t_q + \alpha_{k-1} \Delta t)
          + D_t(u_k)
          \right],
    \label{eq:convection_rk4}
\end{equation}
where the HB source term $D_t(u_k)$ is computed using Eq.~\eqref{eq:sm_multi_spectral_operator},
$\alpha_0 = 0$, $\alpha_1 = 1/4$, 
$\alpha_2 = 1/3$, $\alpha_3 = 1/2$ and $\alpha_4 = 1$.
The CFL number in pseudo-time is set to 1 
to ensure stability of the explicit time-marching scheme.

To compare numerical and exact solutions, 
the discrete $\mathcal{L}_2$-norm of the error 
in time is computed over all the time instances
at each grid points over the domain.
Then, the average in 
space is computed.




\section{Channel flow with oscillating back pressure}
\label{sec:toy_channel}
%!TEX root = ../../../adrien_gomar_phd.tex
\subsection{Presentation of the case}
\label{sub:presentation_of_the_case}

A channel configuration is set up to study the properties of the
Fourier-based method used here, namely the harmonic balance
method. In this 2D channel, a constant left injection is 
supplemented with an unsteady harmonic right pressure condition.
Therefore, pressure waves travel within the flow with the velocity $u + c$ and $u
- c$, where $u$ denotes the local flow velocity and $c$ the sound
velocity. Since the pressure waves are generated at the outlet, only
the $u-c$ waves are visible, resulting in pressure waves propagating
upstream of the channel, which are damped by the effect of
viscosity. Figure~\ref{fig:canal_principle} shows a sketch
of the channel case, illustrating the propagation and attenuation of
the pressure waves.
\begin{figure}[htb]
  \centering
  \includegraphics*[width=0.6\textwidth]{channel_sketch.pdf}
  \caption{Sketch of the channel case.}
  \label{fig:canal_principle}
\end{figure}

It is a 2D channel of length $L_x = 100$~m in the axial
direction and $L_z = 1$~m in the transverse one.  The boundary
conditions are: (i)~an injection condition for the inlet,
(ii)~symmetric conditions for the upper and lower bounds as the flow
is assumed to be symmetric in the transverse direction, and (iii)~a
fluctuating pressure imposed at the outlet:
\begin{equation}
  P_{outlet}(t) = P_m \cdot \left[1 + A_1 \cdot \sin(2 \pi f_1 t) +
    A_2 \cdot \sin(2 \pi f_2 t) \right],
  \label{eq:outlet_canal}
\end{equation}
where $P_m$ is the temporal average static pressure, $A_n$ the
amplitude of the $n$\textsuperscript{th} mode and $f_n$ its
frequency. The mean outlet pressure $P_m$ is set to $60\%$ of the
inlet total pressure $P_{i0} = 101,325$~Pa.

\subsection{Numerical setup}

An analytical solution is provided by \citet{Merkle1987} for
incompressible flows. However, since this toy problem is used
to assess the properties of the harmonic balance within
the Navier--Stokes equations framework, a 


% mesh presentation
The mesh consists of 1000~points along the axial direction and 10 in the
transverse one, which corresponds to equal spacings in both
directions.

This configuration is turbulent as the Reynolds number based on the
inlet flow velocity and the axial length of the channel is about $R_e
\approx 2.0 \times 10^9$.  Turbulence is modeled using the
one-equation model of Spalart and Allmaras~\cite{Spalart1992}, and the
third-order upwind Roe scheme~\cite{Roe1981} is used to compute the
convective fluxes.

\chconclu{}
