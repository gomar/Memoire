%!TEX root = ../../../adrien_gomar_phd.tex
\chapter{Presentation of the toy problems}
\label{cha:toy_problems}

\chabstract{In this chapter, three toy problems are
set up to thoroughly study the properties
of the chosen harmonic balance approach. The first
toy problem is the resolution of the periodic linear advection
equation. An
analytic solution is known, allowing to accurately assess
the harmonic balance method from a theoretical point of view.
The second toy problem is a channel flow with an
oscillating back-pressure. It is solved using the 2D Navier--Stokes
equations. This allows to highlight the properties of the harmonic
balance method in a non-linear equation framework. 
Finally, the third toy problem is a model problem of a turbomachinery configuration.
A Gaussian function representative of a wake is injected and convected through
the domain using the 3D Euler equations. This test case
will be used to assess the specific problems arising when using
the harmonic balance approach in turbomachinery computations.
Furthermore, 
the solver used for these last two cases is the \emph{elsA}
code which will be used for the CROR applications.}

\minitoc
\newpage


\section{Periodic linear advection}
\label{sec:toy_convection}
%!TEX root = ../../../adrien_gomar_phd.tex

\subsection{Presentation of the case}

We consider the linear advection equation:
\begin{equation}
  \label{eq:convection}
  \frac{\partial u}{\partial t} + c \frac{\partial u}{\partial x} = 0,
\end{equation}
with the constant advection speed $c$ assumed as positive
and set here to one for simplicity. 
The equation is solved on the domain $x \in [0, 1]$. 
Periodic perturbations of different shapes are imposed at the left boundary:
\begin{equation}
   u(0, t) = u_l (t),
\end{equation}
where $u_l$ is a periodic function of period $T=1/c$.
These perturbations are advected across the computational 
domain and leave from the right boundary. After a transient of time length $T_{trans}=1/c$, 
the solution at any point $x$ in the space domain achieves a periodic state. 
The exact solution for this periodic state is a periodic function of the form:
\begin{equation}
    u_{ex}(x,t)=u_l(x/c+t).
\end{equation}

\subsection{Numerical setup}

The space derivative is discretized by means of a centered 
fourth-order finite difference scheme on an uniform Cartesian mesh:
\begin{equation}
    \frac{\partial u}{\partial x} (x = x^i, t=t_q) \approx 
    \frac{-u^{i+2}_{q} + 8 u^{i+1}_{q} - 8 u^{i-1}_{q} + u^{i-2}_{q}}{12\Delta x},
    \label{eq:convection_center4}
\end{equation}
A very fine space step is used ($\Delta x=0.002$) in order to rule 
out spatial approximation errors. According to the theory 
of characteristics, the solution at the last mesh 
point on the right of the domain is extrapolated 
from the inside. To this aim, a standard second-order 
upwind discretization and a first-order upwind discretization 
are used to approximate the space derivative at 
the last two mesh points on the right, respectively:
\begin{align}
    \frac{\partial u}{\partial x} (x = x^{m-1}, t=t_q) &\approx 
    \frac{3 u^{m-1}_{q} - 4 u^{m-2}_{q} + u^{m-3}_{q}}{2\Delta x}, \\
    \frac{\partial u}{\partial x} (x = x^m, t=t_q) &\approx 
    \frac{u^{m}_{q} - u^{m-1}_{q}}{\Delta x}.
\label{eq:upwind_scheme}
\end{align}
Time-discretization is achieved 
through the HB method described in Sec.~\ref{sec:sm_hb}:
with a standard four-step Runge-Kutta method~\cite{Jameson1981}
used to pseudo-time 
march the HB equations to the steady state.
The $k\textsuperscript{th}$ step is evaluated by:
\begin{equation}
    u_k = u_q - \alpha_k \Delta t \left [ 
          c \frac{\partial u_{k-1}}{\partial x} (t=t_q + \alpha_{k-1} \Delta t)
          + D_t(u_k)
          \right],
    \label{eq:convection_rk4}
\end{equation}
where the HB source term $D_t(u_k)$ is computed using Eq.~\eqref{eq:sm_multi_spectral_operator},
$\alpha_0 = 0$, $\alpha_1 = 1/4$, 
$\alpha_2 = 1/3$, $\alpha_3 = 1/2$ and $\alpha_4 = 1$.
The CFL number in pseudo-time is set to 1 
to ensure stability of the explicit time-marching scheme.

To compare numerical and exact solutions, 
the discrete $\mathcal{L}_2$-norm of the error 
in time is computed over all the time instances
at each grid points over the domain.
Then, the average in 
space is computed.




\section{Channel flow with oscillating back pressure}
\label{sec:toy_channel}
%!TEX root = ../../../adrien_gomar_phd.tex
\subsection{Presentation of the case}
\label{sub:presentation_of_the_case}

A channel configuration is set up to study the properties of the
Fourier-based method used here, namely the harmonic balance
method. In this 2D channel, a constant left injection is 
supplemented with an unsteady harmonic right pressure condition.
Therefore, pressure waves travel within the flow with the velocity $u + c$ and $u
- c$, where $u$ denotes the local flow velocity and $c$ the sound
velocity. Since the pressure waves are generated at the outlet, only
the $u-c$ waves are visible, resulting in pressure waves propagating
upstream of the channel, which are damped by the effect of
viscosity. Figure~\ref{fig:canal_principle} shows a sketch
of the channel case, illustrating the propagation and attenuation of
the pressure waves.
\begin{figure}[htb]
  \centering
  \includegraphics*[width=0.6\textwidth]{channel_sketch.pdf}
  \caption{Sketch of the channel case.}
  \label{fig:canal_principle}
\end{figure}

It is a 2D channel of length $L_x = 100$~m in the axial
direction and $L_z = 1$~m in the transverse one.  The boundary
conditions are: (i)~an injection condition for the inlet,
(ii)~symmetric conditions for the upper and lower bounds as the flow
is assumed to be symmetric in the transverse direction, and (iii)~a
fluctuating pressure imposed at the outlet:
\begin{equation}
  P_{outlet}(t) = P_m \cdot \left[1 + A_1 \cdot \sin(2 \pi f_1 t) +
    A_2 \cdot \sin(2 \pi f_2 t) \right],
  \label{eq:outlet_canal}
\end{equation}
where $P_m$ is the temporal average static pressure, $A_n$ the
amplitude of the $n$\textsuperscript{th} mode and $f_n$ its
frequency. The mean outlet pressure $P_m$ is set to $60\%$ of the
inlet total pressure $P_{i0} = 101,325$~Pa.

\subsection{Numerical setup}

An analytical solution is provided by \citet{Merkle1987} for
incompressible flows. However, since this toy problem is used
to assess the properties of the harmonic balance within
the Navier--Stokes equations framework, a 


% mesh presentation
The mesh consists of 1000~points along the axial direction and 10 in the
transverse one, which corresponds to equal spacings in both
directions.

This configuration is turbulent as the Reynolds number based on the
inlet flow velocity and the axial length of the channel is about $R_e
\approx 2.0 \times 10^9$.  Turbulence is modeled using the
one-equation model of Spalart and Allmaras~\cite{Spalart1992}, and the
third-order upwind Roe scheme~\cite{Roe1981} is used to compute the
convective fluxes.

\section{Model turbomachinery configuration}
\label{sec:model_tbm}
%!TEX root = ../../../adrien_gomar_phd.tex
\subsection{Numerical Setup}
We consider a simplified configuration modeling a turbomachinery 
stage. The configuration consists of a spatially 
periodic azimuthal perturbation advected downstream 
of the inlet boundary of the computational domain. 
The domain is made of two grid blocks in relative 
motion, so that the perturbation, which is steady 
in the upstream block, is seen as unsteady by the 
downstream one, and is thus representative of 
turbomachinery wakes advected across an inter-wheel interface.
The blocks are generated in cylindrical
coordinates such that the presented configuration
can be assimilated to a slice of 
a turbomachinery stage.
Without loss of generality, 
we set the rotation velocity of the upstream block to zero (stator). 
The stator is composed of $B_{stator} = 10$
"virtual" blades and the rotor by $B_{rotor} = 12$ "virtual" blades.
These are termed virtual blades as no blade is actually meshed.
This is a typical pitch ratio encountered 
in contra-rotating open rotor applications in which 
the first row contains more blades 
than the second (see Sec.~\ref{sec:CROR}). Indeed, in
these applications, the number
of blades is typically smaller than classical
turbomachinery configurations. %This means that a wake
% with a given width will be seen thinner because of the bigger
% relative pitch. 

A wake is axially injected at the inlet of the
stator block following the Lakshminarayana and
Davino
similarity law defined in Eq.~\eqref{eq:similarity}.
It is schematically represented in 
Fig.~\ref{fig:rotating_blocks}.
This is thus a representative model problem of the wake shed
by an upstream row that crosses the rows interface, 
here the stator-rotor interface.

The flow is modeled through the 
Euler equations in order to avoid wake thickening
associated with viscous effects. 
The velocity is not imposed at the inlet directly
but rather through the total pressure and total enthalpy distributions:
\begin{equation}
  \label{eq:rotatingblocks_ptot}
    p_{i0} (\theta) = p_{i_{ref}} \left[1 - 
        \Delta p_i \cdot e^{
          -0.693 \left( 2 \frac{\theta}{L} \right) ^ 2}\right],
\end{equation}
\begin{equation}
  \label{eq:rotatingblocks_htot}
    h_{i0} (\theta) = h_{i_{ref}} \left[1 - 
        \Delta h_i \cdot e^{
          -0.693 \left( 2 \frac{\theta}{L} \right) ^ 2}\right],
\end{equation}
where $p_{i0}$ is the inlet total pressure, $\Delta p_i$ the total pressure
deficit in the wake,
$h_{i0}$ the inlet total enthalpy, $\Delta h_i$ the total enthalpy
deficit in the wake and $L$ the wake width.
% As the total enthalpy and total pressure deficits are not
% easy to define, they are estimated from a turbomachinery
% simulation. The idea here is to impose a distortion that
% is physical. This does not remove any generality
% to the current approach.
To impose a realistic distortion, the total pressure and
enthalpy deficits are estimated from a separate turbomachinery simulation.
This leads to $\Delta p_i = 0.025$ and 
$\Delta h_i = - 0.007$.
The negative sign is due to overturning in the wake, which
is due to velocity composition, and therefore specific to rotors.
The static pressure $p_{s_1}$ is imposed at the outlet:
\begin{equation}
    p_{s_1} = \frac{p_{ref}}{\left(1 + 
    \frac{\gamma - 1}{2} M_{ref}^2 \right) ^ {\frac{\gamma}{ \gamma - 1}}} ,
\end{equation}
the mean velocity is thus set by imposing the
target mean Mach number value $M_{ref}$.
At the azimuthal boundaries, phase-lag conditions~\cite{Erdos1977} 
are used to take into account for the space-time periodicity. %, 
% allowing to compute only a single blade passage, 
% as explained in Sec.~\ref{sec:turbomachinery_adaptation}.

Roe's scheme~\cite{Roe1981} with second-order MUSCL extrapolation 
is used for the spatial discretization of
the convective fluxes, and an implicit backward Euler scheme is used
to march the HB equations in pseudo-time.

A parametric study is carried out over the two parameters 
that influence the truncation error as defined in 
Eq.~\eqref{eq:def_truncation_error}: the number of harmonics and the wake width.
The number of harmonics used for the computations ranges from 1 to $25$.
%in order to ease 
%the post processing but 
%Note that a harmonic balance simulation represents $2N+1$ steady
%computations coupled together by a source term. This means that the
%equivalent mesh for $N=25$ computation is $(2 * 25 + 1) * 170,000 = 8,670,000$
%grid point mesh, which is large for a reduce order model problem.
The wake width $L$, that drives
Eq.~\eqref{eq:rotatingblocks_ptot} and~\eqref{eq:rotatingblocks_htot} varies
between $1\%$ and $30\%$ according to a logarithmic scale to ease 
the visualization of the results. $375$ computations are performed in total. 

Each grid block has a radial extent of five grid points \emph{i.e.} four cells. 
The azimuthal grid density in the stator and rotor blocks is kept similar
to guarantee a consistent capture of the wake on each side of the interface.
To do so, if $\Delta \theta_{cell}$ denotes the azimuthal length of a cell
at the interface, then:
\begin{equation}
   \Delta \theta_{cell} = \frac{2\pi}{B_{stator}}~\frac{1}{N_{stator}}
   = \frac{2\pi}{B_{rotor}}~\frac{1}{N_{rotor}},
   \label{eq:az_spatial_discretization_1}
\end{equation}
where $N_{stator}$ and $N_{rotor}$ are the number of cells
in the stator and the rotor, respectively. 

% This equation gives finally the relation
% between $N_{stator}$ and $N_{rotor}$ that ensures a uniform azimuthal
% grid density at the interface:
% \begin{equation}
%    12~N_{stator} = 10~N_{rotor}.
%    \label{eq:az_spatial_discretization_2}
% \end{equation}
Mesh convergence for the thinnest wake (1\% of the pitch)
is obtained with 500~cells in the azimuthal direction of
the stator which gives
600~cells for the rotor block.
30~grid points are put in the axial direction. Moreover, a constant
aspect ratio of 5 with respect to the azimuthal length of the cells is
kept, which sets the axial length of the case.
This leads to a total number of grid points of approximately 
170,000. 
Note that the memory requirement of an HB 
simulation is $2N+1$ times that of the equivalent steady case. 
An equivalent steady computation to $N=25$ would 
thus require 
$(2 \times 25 + 1) \times 170,000 = 8,670,000$ grid point mesh.
The grid used for the computations is shown in Fig.~\ref{fig:rotating_blocks}.
\begin{figure}[htb]
  \centering
  \begin{tabular}{cc}
    \includegraphics[height=.3\textheight]{ROTATING_BLOCKS.pdf}
    &
    \includegraphics[height=.3\textheight]{RB_mesh_3D.png}\\
    Principle diagram 
    &
    Mesh, one out every five points
  \end{tabular}
\caption{Model turbomachinery configuration.}
\label{fig:rotating_blocks}
\end{figure}
Convergence of the iterative procedure used to solve the HB equations is achieved 
after 3,000 iterations for 
all the simulations. 

\chconclu{}
