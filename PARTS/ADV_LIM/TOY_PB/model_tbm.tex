%!TEX root = ../../../adrien_gomar_phd.tex

\subsection{Presentation of the case}

The last toy problem considered is a simplified 
configuration modeling a turbomachinery 
stage. The configuration consists of a spatially 
periodic azimuthal perturbation propagated downstream 
of the inlet boundary of the computational domain. 
The domain is made of two grid blocks in relative 
motion, so that the perturbation, which is steady 
in the upstream block, is seen as unsteady by the 
downstream one, and is thus representative of 
turbomachinery wakes propagated across an inter-wheel interface.
The blocks are generated in cylindrical
coordinates such that the presented configuration
can be assimilated to a slice of 
a turbomachinery stage.
Without loss of generality, 
we set the rotation velocity of the upstream block to zero (stator). 
The stator is composed of $B_{stator} = 10$
"virtual" blades and the rotor by $B_{rotor} = 12$ "virtual" blades.
The stator's pitch is therefore larger than the rotor's.
These are termed virtual blades as no blade is actually meshed.
The chosen number of blades represents 
a typical pitch ratio encountered 
in contra-rotating open rotor applications in which 
the first row contains more blades 
than the second (see Chap.~\ref{cha:cror}). Indeed, in
these applications, the number
of blades is typically smaller than classical
turbomachinery configurations.

\citet{Lakshminarayana1980} showed that the wake
behind turbomachinery blade follows a similarity law for the velocity. 
It can be empirically approximated by a Gaussian function:
\begin{equation}
    u (\theta) = u_m - 
        \Delta u \cdot e^{
          -0.693 \left( 2 \frac{\theta}{L} \right) ^ 2},
    \label{eq:similarity}
\end{equation}
where $u_m$ denotes the free-stream velocity, $\Delta u$ the axial wake velocity deficit,
$\theta$ the azimuthal coordinate and $L$ the wake width,
defined as the full width at half maximum.
A wake is axially injected at the inlet of the
turbomachinery model problem following the \citet{Lakshminarayana1980}
similarity law.

Figure~\ref{fig:model_tbm_sketch} shows a sketch
of the considered channel flow problem.
\begin{figure}[htb]
  \centering
  \includegraphics*[width=0.2\textwidth]{model_tbm_sketch.pdf}
  \caption{Sketch of the channel flow problem.}
  \label{fig:model_tbm_sketch}
\end{figure}
It is thus a representative model problem of the wake shed
by an upstream row that crosses the rows interface, 
here the stator-rotor interface.

\subsection{Numerical Setup}

The flow is modeled through the 
Euler equations in order to avoid wake thickening and vanishing
associated with viscous effects. 
The velocity is not imposed at the inlet directly
but rather through the total pressure and total enthalpy distributions:
\begin{equation}
  \label{eq:rotatingblocks_ptot}
    p_{i0} (\theta) = p_{i_{ref}} - 
        \Delta p_i \cdot e^{
          -0.693 \left( 2 \frac{\theta}{L} \right) ^ 2},
\end{equation}
\begin{equation}
  \label{eq:rotatingblocks_htot}
    h_{i0} (\theta) = h_{i_{ref}} - 
        \Delta h_i \cdot e^{
          -0.693 \left( 2 \frac{\theta}{L} \right) ^ 2},
\end{equation}
where $p_{i_{ref}}$ is the inlet reference total pressure, $\Delta p_i$ the total pressure
deficit in the wake,
$h_{i0}$ the inlet total enthalpy, $\Delta h_i$ the total enthalpy
deficit in the wake and $L$ the wake width.
% As the total enthalpy and total pressure deficits are not
% easy to define, they are estimated from a turbomachinery
% simulation. The idea here is to impose a distortion that
% is physical. This does not remove any generality
% to the current approach.
To impose a realistic distortion, the total pressure and
enthalpy deficits are estimated from a separate turbomachinery simulation.
This leads to $\Delta p_i = 0.025$ and 
$\Delta h_i = - 0.007$.
The negative sign is due to overturning in the wake, which
is due to velocity composition, and therefore specific to rotors.
The static pressure $p_{s_1}$ is imposed at the outlet:
\begin{equation}
    p_{s_1} = \frac{p_{i_{ref}}}{\left(1 + 
    \frac{\gamma - 1}{2} M_{0}^2 \right) ^ {\frac{\gamma}{ \gamma - 1}}} ,
\end{equation}
the mean velocity is thus set by imposing the
target mean Mach number value $M_{0}$.
At the azimuthal boundaries, phase-lag conditions~\cite{Erdos1977} 
are used to take into account for the space-time periodicity.

Roe's scheme~\cite{Roe1981} with second-order MUSCL extrapolation 
is used for the spatial discretization of
the convective fluxes, and an implicit backward Euler scheme is used
to march the HB equations in pseudo-time.

Each grid block has a radial extent of five grid points \emph{i.e.} four cells. 
The azimuthal grid density in the stator and rotor blocks is kept similar
to guarantee a consistent capture of the wake on each side of the interface.
To do so, if $\Delta \theta_{cell}$ denotes the azimuthal length of a cell
at the interface, then:
\begin{equation}
   \Delta \theta_{cell} = \frac{2\pi}{B_{stator}}~\frac{1}{N_{stator}}
   = \frac{2\pi}{B_{rotor}}~\frac{1}{N_{rotor}},
   \label{eq:az_spatial_discretization_1}
\end{equation}
where $N_{stator}$ and $N_{rotor}$ are the number of cells
in the stator and the rotor, respectively. 

Several wake thickness will be tested in Sec~\ref{sec:rotating_blocks}
from 30\% down to 1\% of the pitch. For this thinnest wake, 
mesh convergence
is obtained with 500~cells in the azimuthal direction of
the stator which gives
600~cells for the rotor block.
30~grid points are put in the axial direction. Moreover, a constant
aspect ratio of 5 with respect to the azimuthal length of the cells is
kept, which sets the axial length of the case.
This leads to a total number of grid points of approximately 
170,000. 

Convergence of the iterative procedure used to solve the HB equations is achieved 
after 3,000 iterations for 
all the simulations. 