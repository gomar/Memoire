%!TEX root = ../../../adrien_gomar_phd.tex
\chapter{Isolated high-speed CROR configuration}
\label{cha:dream_hs_isolated}

\chabstract{To further assess the proposed
multi-frequential harmonic balance method along
with a decoupled approach, a more demanding
case is studied, namely a high-speed CROR configuration.
The number of harmonics required to compute such a
configuration is shown to be higher than the 
low-speed configuration. The unsteady rigid-motion
computations are analyzed and aeroelastic
simulations are then carried out, showing that 
the proposed approach is robust enough to 
assess such configurations.}

\newpage

\section{Presentation of the case}
\label{sec:dream_hs_presentation}
%!TEX root = ../../../adrien_gomar_phd.tex

\begin{figure}[htp]
  \centering
  \includegraphics[width=.3\textwidth]{DREAM_HS_wall.png}
  \caption{High-speed isolated contra-rotating open rotor geometry.}
  \label{fig:dream_hs_wall}
\end{figure}

The configuration studied in this Chapter is the same as the
previous one but with a different angle of attack of
the blades. The geometry is shown in Figure~\ref{fig:dream_hs_wall}.
This geometry is the High-Speed (HS) version of the previous one, 
representative of the cruise flight condition. The rotation speed being kept
almost constant between the two configurations, the only way to ensure
a proper adaptation of the velocity field is to change the angle of 
attack the blades.

The main input parameters of the case are recalled in
Tab.~\ref{tab:dream_hs_flight_condition}.
\begin{table}[htp]
  \ra{1.3} \centering
  \begin{tabular}{cccc}
    \toprule
    $M_0$ & $|\Omega|$ & $J$ & $M_{tip}$ \\
    \midrule
    $0.73$ & $6057$ RPM & 3.7 & 0.96  \\
    \bottomrule
  \end{tabular}
  \caption{High-speed isolated contra-rotating open rotor flight condition parameters.}
  \label{tab:dream_hs_flight_condition}
\end{table} 
The inflow Mach number is within the transonic range. Its high value
can suggest the appearance of shocks in the flow field. 
This is emphasized by the $M_{tip}$ which is
near from being supersonic.


\section{Numerical setup}
\label{sec:dream_hs_numerical}
%!TEX root = ../../../adrien_gomar_phd.tex

The same topology and number of grid points is used to
mesh this high-speed configuration. 
If nothing particular is said, the same numerical parameters
are kept for the study of this
high-speed configuration.
The reader is referred to 
Sec.~\ref{sec:dream_ls_numerical} for detailed information.

The same numerical approach as the low-speed configuration
is chosen for the aeroelastic computations that will
be presented in
Sec.~\ref{sec:dream_hs_ael_results}.
For practical reasons, the harmonic balance computations are run with
the same number of frequencies as the low-speed configuration,
namely five frequencies in total. This might be not sufficient
as suggested by the prediction tool shown bellow in 
Sec.~\ref{sec:dream_hs_spectral_convergence}. A partial 
convergence study is therefore conducted afterwards 
in Sec.~\ref{sub:dream_hs_convergence_ael}.
In the rear rotor,
the harmonics of the front rotor blade passing frequency
are chosen. In the front rotor, the first frequency is the
frequency associated to the vibration of the blade and the
remaining ones are the harmonics of the rear rotor blade 
passing frequency. 
The time instances are automatically chosen using the OPT
algorithm which leads to 
a condition number always lower than $1.1$, ensuring thus
the stability of the computations.

\paragraph{Influence of the spatial discretization}
\label{sub:dream_hs_spatial_discretization}

The same exercise as done in 
Sec.~\ref{sub:dream_ls_spatial_discretization} is made below
Four schemes are evaluated based 
on the convergence of the computations
and of the integrated 
results (similarity coefficients).
For the \citet{Jameson1981} scheme, the artificial viscosities
are chosen as follow: $\kappa_2 = 1.0$
and $\kappa_4 = 0.016$. We will see that the computation converges
with this low $\kappa_4$ coefficient. Therefore, only this coefficient
will be tested.

The convergence of the computations using the four spatial schemes
is reported in Figure~\ref{fig:DREAM_HS_RESIDUALS_PPT}. The convergence is good
for all the spatial schemes. In fact, more than five orders of magnitude
are lost on the residuals.
\begin{figure}[htp]
  \centering
  \includegraphics*[width=0.50\textwidth]{SPACE_SCHEME_DIFF_HS_RESIDUALS.pdf}
  \caption{High-speed isolated configuration: convergence 
  of steady computations using different spatial schemes.}
  \label{fig:DREAM_HS_RESIDUALS_PPT}
\end{figure}

The values of the similarity coefficients obtained with
all the spatial schemes is reported in 
Figure~\ref{fig:dream_hs_space_scheme_coeff}. Arbitrarily, the values are
given as a ratio over the Roe~2 values. The first-order
upwind scheme (Roe~1) give similarity coefficients that are
several percent lower than the Roe~2 value. The other schemes
give results that are less than 1\% close, therefore and for
consistence with the approach retained for the low-speed configuration
the Roe~2 scheme is chosen for the following computations.
\begin{figure}[htp]
  \centering
  \includegraphics[width=.5\textwidth]{SPACE_SCHEME_DIFF_HS_COEFF.pdf}
  \caption{High-speed isolated configuration: convergence of 
  similarity coefficients using different spatial schemes.}
  \label{fig:dream_hs_space_scheme_coeff}
\end{figure}



\section{Steady results}
\label{sec:dream_hs_steady_results}
%!TEX root = ../../../adrien_gomar_phd.tex

\subsection{Analysis of the convergence}
\label{sub:dream_hs_steady_conv}

The convergence of the simulation is obtained 
after 500 iterations for both
the residuals and the similarity coefficients 
(Figure~\ref{fig:dream_HS_convergence_roe2}). More than
five orders of magnitude are obtained for the residuals and
the similarity coefficients are stabilized starting below
$1,000$ iterations. According to \citet{Casey2000},
this means that the steady simulation is converged.
\begin{figure}[htp]
  \centering
  \subfigure[residuals]{\includegraphics[width=.35\textwidth]{DREAM_HS_RESIDUALS_PPT.pdf}}
  \subfigure[$C_T$]{\includegraphics[width=.35\textwidth]{DREAM_HS_FORCES_CT_PPT.pdf}}
  \subfigure[$C_P$]{\includegraphics[width=.35\textwidth]{DREAM_HS_FORCES_CP_PPT.pdf}}
  \subfigure[$\eta$]{\includegraphics[width=.35\textwidth]{DREAM_HS_FORCES_ETA_PPT.pdf}}
  \caption{High-speed isolated configuration: convergence of the steady
  computation.}
  \label{fig:dream_HS_convergence_roe2}
\end{figure}

\subsection{Similarity coefficients}
\label{sub:dream_hs_sim_coeff}

The similarity coefficients are reported in 
Tab.~\ref{tab:dream_HS_sim_coeff}. 
They are representative of a cruise
propeller (see Eq.~\eqref{eq:estimation_sim_coeff}). 
Firstly, the thrust is higher
on the rear rotor than on the front rotor, even though it is
relatively well distributed. Bare in mind that the rear rotor
similarity coefficient is normalized by the front rotor diameter. Therefore, 
the thrust produced by the rear rotor is larger than the one of the front rotor, 
relatively to its diameter. 
Secondly, the power coefficient is similar for both the front and the
rear rotor. As the power coefficient represents the mechanical input
given to flow, this means that the mechanical distribution is 
well repartitioned, which is a design wish for the integrity of
the drive shaft. In fact a non-uniform power coefficient
might give an additional momentum on the drive shaft which can
deteriorates its mechanical properties.
\begin{table}[htp]
  \ra{1.3} \centering
  \begin{tabular}{ccc||ccc|ccc}
    \toprule
    \multicolumn{3}{c||}{global} & \multicolumn{3}{c|}{front} & \multicolumn{3}{c}{rear} \\
    $C_T$ & $C_P$ & $\eta$ & $C_{T_f}$ & $C_{P_f}$ & $\eta_f$ & $C_{T_r}$ & $C_{P_r}$ & $\eta_r$ \\
    \midrule
    0.902 & 3.849 & 0.858 & 0.411 & 1.926 & 0.780 & 0.491 & 1.922 & 0.935 \\
    \bottomrule
  \end{tabular}
  \caption{High-speed isolated configuration: similarity coefficients.}
  \label{tab:dream_HS_sim_coeff}
\end{table}

\subsection{Radial profiles}
\label{sub:dream_hs_radial_profiles}

Radial profiles are computed on the steady results and 
reported in Figure~\ref{fig:dream_HS_radial_profiles}.
\begin{figure}[htp]
  \centering
  \subfigure[absolute Mach number]{
    \label{fig:dream_HS_radial_profiles_ma}
    \includegraphics[width=.72\textwidth]{DREAM_HS_RANS_AZI_MEAN_PPT_macha.pdf}}
  \subfigure[absolute flow angle]{
    \label{fig:dream_HS_radial_profiles_alpha}
    \includegraphics[width=.72\textwidth]{DREAM_HS_RANS_AZI_MEAN_PPT_alpha.pdf}}
  \caption{High-speed isolated configuration: radial profiles.}
\end{figure}
\setcounter{figure}{\value{figure}-1}
\begin{figure}[htp]
  \centering
  \setcounter{subfigure}{3}
  \subfigure[static pressure]{
    \label{fig:dream_HS_radial_profiles_ps}
    \includegraphics[width=.72\textwidth]{DREAM_HS_RANS_AZI_MEAN_PPT_ps.pdf}}
  \subfigure[stagnation temperature]{
    \label{fig:dream_HS_radial_profiles_ti}
    \includegraphics[width=.72\textwidth]{DREAM_HS_RANS_AZI_MEAN_PPT_ti.pdf}}
  \subfigure[stagnation pressure]{
    \label{fig:dream_HS_radial_profiles_pi}
    \includegraphics[width=.72\textwidth]{DREAM_HS_RANS_AZI_MEAN_PPT_pi.pdf}}
  \caption{High-speed isolated configuration: radial profiles (contd.).}
  \label{fig:dream_HS_radial_profiles}
\end{figure}

The absolute Mach number (Figure~\ref{fig:dream_HS_radial_profiles_ma}) 
increases from its inflow
value ($M_a = 0.73$) up to around $M_a=0.76$. This represents
a 4\% increase that has to be compared to the 100\% increase
for the low-speed configuration Nevertheless, this represents
an absolute $\Delta M_a = 0.03$ increase for this high-speed
configuration where it was $\Delta M_a = 0.2$
for the low-speed configuration. The stream tube contraction
is smaller than the low-speed configuration 
as the increase in velocity remains bounded.
It seems that the front rotor tip vortices do not interact
with the rear rotor blades as the small increase near 90\%
of the span, which is attributed to the core of the vortices,
is not contracted by the stream tube.

The absolute pitch angle (Figure~\ref{fig:dream_HS_radial_profiles_alpha}) 
of the velocity highlights, again, the advantage
of using a CROR compared to a single row propeller system. In fact,
the flow is deviated by the front rotor from its axial direction
to a mean $5^\circ$ velocity vector. The rear rotor then deviates
back the flow field to make it almost purely axial with exceptions
near the hub and near the front tip vortex region ($0.75 \leq R/R_f \leq 0.95$).
This explains the efficiency of a CROR propeller system also for high-speed
inflow conditions, which is the design priority.

The static pressure (Figure~\ref{fig:dream_HS_radial_profiles_ps}) increases
by at-most 10\% which is larger than it was for the low-speed
inflow condition. These are clearly compressibility effects due
to the high inflow Mach number. As the goal of a CROR is
to create thrust through a large mass-flow, this
static pressure rise helps increasing the mass-flow.
In fact, using roughly the perfect gas state-equation,
a static pressure increase is seen as a density increase that
participates to a high mass-flow, hence the contribution to thrust.
In fact, a larger part of the energy is converted to static
pressure and not to velocity. The potential effects can be seen in the
$P2$ and $P4$ planes. In fact, the pressure decreases before
crossing a rotor blades. This is due to the acceleration of the
fluid that is done at each rotor crossing. This increase of velocity
creates a decrease in static pressure that is seen upstream the rotors.

With the stagnation temperature rise shown
in Figure~\ref{fig:dream_HS_radial_profiles_ti}, one can say that
the rotors provide energy to the fluid to create the thrust.
The stagnation pressure is shown in Figure~\ref{fig:dream_HS_radial_profiles_pi}.
At each rotor crossing it increases since both the absolute velocity and
the static pressure increase. 
Moreover, the stream tube contraction seems to be lessened
compared to the low-speed flight condition. It will have to be
confirmed by the forthcoming unsteady results.

\subsection{Flow field around the blades}
\label{sub:dream_hs_blades}

Relative Mach number contours and the pressure coefficient
$k_p$ are shown in Figure~\ref{fig:dream_HS_mach_kp} for both the
front and the rear rotors.
\begin{figure}[htp]
 \centering
 \begin{tabular}{rccc}
   & $k_p$ front rotor
   & $k_p$ rear rotor
   & relative Mach number\\
   \rotatebox{90}{\qquad\qquad 25~\%} 
   & \includegraphics[width=0.28\textwidth]{DREAM_HS_KP_25_FRONT_PPT.pdf}
   & \includegraphics[width=0.28\textwidth]{DREAM_HS_KP_25_REAR_PPT.pdf}
   & \includegraphics[width=0.28\textwidth]{DREAM_HS_RANS_roe2_sa_slice_r_25_mach_rel.png}\\
   \rotatebox{90}{\qquad\qquad 50~\%} 
   & \includegraphics[width=0.28\textwidth]{DREAM_HS_KP_50_FRONT_PPT.pdf}
   & \includegraphics[width=0.28\textwidth]{DREAM_HS_KP_50_REAR_PPT.pdf}
   & \includegraphics[width=0.28\textwidth]{DREAM_HS_RANS_roe2_sa_slice_r_50_mach_rel.png}\\
   \rotatebox{90}{\qquad\qquad 75~\%} 
   & \includegraphics[width=0.28\textwidth]{DREAM_HS_KP_75_FRONT_PPT.pdf}
   & \includegraphics[width=0.28\textwidth]{DREAM_HS_KP_75_REAR_PPT.pdf}
   & \includegraphics[width=0.28\textwidth]{DREAM_HS_RANS_roe2_sa_slice_r_75_mach_rel.png}\\
   \rotatebox{90}{\qquad\qquad 90~\%} 
   & \includegraphics[width=0.28\textwidth]{DREAM_HS_KP_90_FRONT_PPT.pdf}
   & \includegraphics[width=0.28\textwidth]{DREAM_HS_KP_90_REAR_PPT.pdf}
   & \includegraphics[width=0.28\textwidth]{DREAM_HS_RANS_roe2_sa_slice_r_90_mach_rel.png}\\
   \rotatebox{90}{\qquad\qquad 95~\%} 
   & \includegraphics[width=0.28\textwidth]{DREAM_HS_KP_95_FRONT_PPT.pdf}
   & \includegraphics[width=0.28\textwidth]{DREAM_HS_KP_95_REAR_PPT.pdf}
   & \includegraphics[width=0.28\textwidth]{DREAM_HS_RANS_roe2_sa_slice_r_95_mach_rel.png}  
 \end{tabular}
 \caption{High-speed isolated configuration: pressure coefficient and relative Mach
 number contours at different radial position.}
 \label{fig:dream_HS_mach_kp}
\end{figure}

On the front rotor, for relative span $R / R_f = 25 \%$,
a compression is observed near the leading edge ($x/c \leq 0.2$)
of the suction side. The pressure coefficient is then
almost constant up to the trailing edge. For all
relative spans,
a small negative incidence is indicated by a crossing
in $k_p$ values for all spans of the front rotor,
near the leading edge. This might indicate
that either the incidence of the blades or the rotation speed is
not well adapted for this inflow condition.
The shock that was seen near the leading edge seems to change to a
weak shock wave for higher relative spans ($R / R_f \geq 50 \%$).
On the suction side of the blade, the pressure coefficient $k_p$
increases such that for $R / R_f \geq 75 \%$, a shock is observed
on the suction side of the blade at $x/c \approx 0.7$.

On the rear rotor, the flow field seems to be better adapted to
the inflow conditions. In fact, no negative incidence is seen.
A shock is observed at $x/c \approx 0.5$ for all relative spans
that moves toward the trailing edge as
for $R / R_f = 90 \%$, it is located at $x/c \approx 0.6$.
In fact, as the relative span increases, the relative Mach
number grows which explains the movement of the shock
toward the trailing edge.
For $R / R_f = 95 \%$, no shock is seen but rather a smooth
compression of the flow. This is due to the 
leaving of rear rotor tip vortices that splits the
pressure gradient responsible for the shock.
In fact a low velocity trace seems to indicate
a tip vortex. Globally, the pressure coefficients on the 
rear rotor have a higher integral, explaining
the higher thrust coefficients observed in Sec.~\ref{sub:dream_hs_sim_coeff}.

To investigate the structure of the tip vortices, axial
cuts of entropy made at planes $P3$, $P4$, $P5$ and $P6$
are shown in Figure~\ref{fig:dream_HS_steady_entropy}. One can notice
that the vortices look much thinner compared to the
low-speed inflow condition ones. This is due to the
staggering angle of the blades. In fact, as the inflow 
Mach number is greater, the staggering angle should be
decreased so that an almost constant rotation speed can be kept.
Therefore, projecting the tip vortices on axial cuts will make
the high-speed one thinner. Contrary to the low-speed
inflow condition, the trace of the front rotor tip vortices
seen in plane $P4$ seems to pass above the rear rotor tip
vortices. This was also observed when looking at the radial profiles.
This will be deeply investigated with unsteady simulations
in Sec.~\ref{sec:dream_hs_rigid_results}.
\begin{figure}[htp]
  \centering
  \subfigure[$P3$]{\includegraphics[width=.35\textwidth]{DREAM_HS_RANS_roe2_sa_slice_x_front_1_entropy.png}}
  \subfigure[$P4$]{\includegraphics[width=.35\textwidth]{DREAM_HS_RANS_roe2_sa_slice_x_rear_0_entropy.png}}
  \subfigure[$P5$]{\includegraphics[width=.35\textwidth]{DREAM_HS_RANS_roe2_sa_slice_x_rear_1_entropy.png}}
  \subfigure[$P6$]{\includegraphics[width=.35\textwidth]{DREAM_HS_RANS_roe2_sa_slice_x_rear_2_entropy.png}}
  \caption{High-speed isolated configuration: axial cut of entropy.}
   \label{fig:dream_HS_steady_entropy}
\end{figure}

\section{\emph{A priori} estimate of the required number of harmonics}
\label{sec:dream_hs_spectral_convergence}
%!TEX root = ../../../adrien_gomar_phd.tex

\subsection{Using the prediction tool}
\label{sub:dream_hs_conv_hb_prediction_tool}

\begin{figure}[htp]
  \centering
  \includegraphics*[width=0.5\textwidth]{DREAM_HS_RANS_ROE2_SPECTRUM_PPT.pdf}
  \caption{High-speed isolated configuration: prediction of the number
  of harmonics needed to simulate the configuration.}
  \label{fig:DREAM_HS_RANS_ROE2_SPECTRUM_PPT}
\end{figure}

\subsection{Analyzing the similarity coefficients}
\label{sub:dream_hs_conv_hb_sim_coeff}
\begin{table}[htp]
  \ra{1.3} \centering
  \begin{tabular}{rccccc}
    \toprule
    & steady & HB $N=1$ & HB $N=2$ & HB $N=3$ & HB $N=4$ \\
    \midrule
    $C_T$  &  & & \\
    $C_P$  &  & & \\
    $\eta$ &  & & \\
    \bottomrule
  \end{tabular}
  \caption{High-speed isolated configuration: analysis of the number of harmonics
  required to capture the mean similarity coefficients.}
  \label{tab:dream_hs_hb_conv_sim}
\end{table}

\subsection{Analyzing the blade response}
\label{sub:dream_hs_conv_hb_blade_response}



\subsection{Analyzing the radial cuts}
\label{sub:dream_hs_conv_hb_slice_r}


\section{Unsteady rigid-motion results}
\label{sec:dream_hs_rigid_results}
%!TEX root = ../../../adrien_gomar_phd.tex

\subsection{Similarity coefficients}
\label{sub:dream_hs_hb_sim_coeff}

\begin{figure}[htp]
  \centering
  \subfigure[front rotor]{\includegraphics[width=.35\textwidth]{DREAM_HS_TSM_FORCES_INST_FRONT_PPT.pdf}}
  \subfigure[rear rotor]{\includegraphics[width=.35\textwidth]{DREAM_HS_TSM_FORCES_INST_REAR_PPT.pdf}}
  \caption{High-speed isolated configuration: unsteadiness seen by the rotors.}
  \label{fig:dream_hs_hb_unst_coeff}
\end{figure}

\subsection{Two-dimensional results: harmonic blade response}
\label{sub:dream_hs_hb_blade_response}

\begin{figure}[htp]
 \ra{1.3} \centering
 \begin{tabular}{cccc}
    \multicolumn{2}{c}{\includegraphics[width=0.3\textwidth]{dream_LS_blade_resp_scale_H01_front.pdf}} &
    \multicolumn{2}{c}{\includegraphics[width=0.3\textwidth]{dream_LS_blade_resp_scale_H01_rear.pdf}} \\
    \includegraphics[width=0.15\textwidth]{DREAM_LS_TSM_N4_roe2_sa_blade_response_front_H01_SS.png}
    & \includegraphics[width=0.15\textwidth]{DREAM_LS_TSM_N4_roe2_sa_blade_response_front_H01_PS.png}
    & \includegraphics[width=0.15\textwidth]{DREAM_LS_TSM_N4_roe2_sa_blade_response_rear_H01_PS.png}
    & \includegraphics[width=0.15\textwidth]{DREAM_LS_TSM_N4_roe2_sa_blade_response_rear_H01_SS.png} \\
    \multicolumn{2}{c}{\emph{Front rotor blade}}
    & \multicolumn{2}{c}{\emph{Rear rotor blade}} \\
    suction side & pressure side & pressure side & suction side
 \end{tabular}
 \caption{High-speed isolated configuration: harmonic response of the front
 rotor blades.}
 \label{fig:dream_hs_hb_blade_response}
\end{figure}

\subsection{Two-dimensional results: axial cuts}
\label{sub:dream_hs_hb_axial_cuts}

\begin{figure}[htp]
  \centering
  \subfigure[$P3$]{\includegraphics[width=.35\textwidth]{DREAM_LS_TSM_N4_roe2_sa_slice_x_front_1_entropy.png}}
  \subfigure[$P4$]{\includegraphics[width=.35\textwidth]{DREAM_LS_TSM_N4_roe2_sa_slice_x_rear_0_entropy.png}}
  \subfigure[$P5$]{\includegraphics[width=.35\textwidth]{DREAM_LS_TSM_N4_roe2_sa_slice_x_rear_1_entropy.png}}
  \subfigure[$P6$]{\includegraphics[width=.35\textwidth]{DREAM_LS_TSM_N4_roe2_sa_slice_x_rear_2_entropy.png}}
  \caption{High-speed isolated configuration: axial cuts of entropy.}
   \label{fig:dream_hs_hb_axial_cut_entropy}
\end{figure}

\subsection{Two-dimensional results: radial cut of harmonic pressure}
\label{sub:dream_hs_hb_radial_cuts}

\begin{figure}[htp]
  \centering
  \includegraphics*[width=0.40\textwidth]{DREAM_LS_TSM_N4_roe2_sa_slice_r_70_ps.png}
  \caption{High-speed isolated configuration: radial cut of the first harmonic of the
  static pressure normalized by the inflow static pressure.}
  \label{fig:dream_hs_hb_radial_cuts}
\end{figure}

\begin{figure}[htp]
  \centering
  \includegraphics*[width=0.40\textwidth]{DREAM_LS_TSM_N4_roe2_sa_slice_r_70_machrel.png}
  \caption{High-speed isolated configuration: radial cut of the first harmonic of the
  relative Mach number.}
  \label{fig:dream_hs_hb_radial_cuts_machrel}
\end{figure}

\section{Aeroelastic results}
\label{sec:dream_hs_ael_results}
%!TEX root = ../../../adrien_gomar_phd.tex

\subsection{Stability curve}
\label{sub:dream_hs_ael_curve}

The damping curves for the two modes of this high-speed
configuration are shown in Fig.~\ref{fig:dream_hs_ael_damping}.
The damping is positive for all the inter-blade phase angles
and modes, which clears this configuration for flutter.
In fact, the damping is around $2.1$ and $0.35$
for the torsional mode and the flection mode, respectively.
The torsional is much more damped than the flection one.
In opposite to the low-speed configuration, the 
variation range is similar for both modes. 
The minimum damping is obtained for IBPA=$30^\circ$
for the 2F mode and IBPA=$-30^\circ$ for the 1T mode.
To further analyze the aeroelastic behavior of the front rotor 
blades, the local damping is computed.
\begin{figure}[htp]
  \centering
  \subfigure[mode 2F]{\includegraphics[width=.35\textwidth]{DREAM_HS_DAMPING_MODE_2F.pdf}}
  \subfigure[mode 1T]{\includegraphics[width=.35\textwidth]{DREAM_HS_DAMPING_MODE_1T.pdf}}
  \caption{High-speed isolated configuration: integrated damping for modes 2F and 1T.}
  \label{fig:dream_hs_ael_damping}
\end{figure}

\subsection{Local excitation}
\label{sub:dream_hs_ael_local_damping}

The local excitation is shown on the pressure side and
the suction side of the front rotor blades in 
Figure~\ref{fig:dream_hs_ael_local_damping}. It is the
local damping given in each cell divided by the 
surface of the cell. It is therefore expressed in
m\textsuperscript{-2}.
\begin{figure}[htp]
 \ra{1.3} \centering
 \begin{tabular}{r|cccc}
   \toprule
   & \multicolumn{4}{c}{
        \includegraphics[width=0.22\textwidth]{dream_hs_damping_scale.pdf}} \\
   & \multicolumn{2}{c}{mode 2F} & \multicolumn{2}{c}{mode 1T} \\
   \midrule
   \rotatebox{90}{\quad\quad\quad IBPA $= -60^\circ$} 
   & \includegraphics[width=0.12\textwidth]{DREAM_HS_HBT_N5_AEL_H1M2FD-3_roe3_sa_local_damping_SS.png}
   & \includegraphics[width=0.12\textwidth]{DREAM_HS_HBT_N5_AEL_H1M2FD-3_roe3_sa_local_damping_PS.png}
   & \includegraphics[width=0.12\textwidth]{DREAM_HS_HBT_N5_AEL_H1M1TD-3_roe3_sa_local_damping_SS.png}
   & \includegraphics[width=0.12\textwidth]{DREAM_HS_HBT_N5_AEL_H1M1TD-3_roe3_sa_local_damping_PS.png} \\
   \rotatebox{90}{\quad\quad\quad IBPA $= -30^\circ$} 
   & \includegraphics[width=0.12\textwidth]{DREAM_HS_HBT_N5_AEL_H1M2FD-1_roe3_sa_local_damping_SS.png}
   & \includegraphics[width=0.12\textwidth]{DREAM_HS_HBT_N5_AEL_H1M2FD-1_roe3_sa_local_damping_PS.png}
   & \includegraphics[width=0.12\textwidth]{DREAM_HS_HBT_N5_AEL_H1M1TD-1_roe3_sa_local_damping_SS.png}
   & \includegraphics[width=0.12\textwidth]{DREAM_HS_HBT_N5_AEL_H1M1TD-1_roe3_sa_local_damping_PS.png} \\
   \rotatebox{90}{\quad\quad\quad IBPA $= 30^\circ$} 
   & \includegraphics[width=0.12\textwidth]{DREAM_HS_HBT_N5_AEL_H1M2FD1_roe3_sa_local_damping_SS.png}
   & \includegraphics[width=0.12\textwidth]{DREAM_HS_HBT_N5_AEL_H1M2FD1_roe3_sa_local_damping_PS.png}
   & \includegraphics[width=0.12\textwidth]{DREAM_HS_HBT_N5_AEL_H1M1TD1_roe3_sa_local_damping_SS.png}
   & \includegraphics[width=0.12\textwidth]{DREAM_HS_HBT_N5_AEL_H1M1TD1_roe3_sa_local_damping_PS.png} \\
   \rotatebox{90}{\quad\quad\quad IBPA $= 60^\circ$} 
   & \includegraphics[width=0.12\textwidth]{DREAM_HS_HBT_N5_AEL_H1M2FD3_roe3_sa_local_damping_SS.png}
   & \includegraphics[width=0.12\textwidth]{DREAM_HS_HBT_N5_AEL_H1M2FD3_roe3_sa_local_damping_PS.png}
   & \includegraphics[width=0.12\textwidth]{DREAM_HS_HBT_N5_AEL_H1M1TD3_roe3_sa_local_damping_SS.png}
   & \includegraphics[width=0.12\textwidth]{DREAM_HS_HBT_N5_AEL_H1M1TD3_roe3_sa_local_damping_PS.png} \\
   & suction side & pressure side & suction side & pressure side \\
   \bottomrule
 \end{tabular}
 \caption{High-speed isolated configuration: local excitation for modes 2F and 1T.}
 \label{fig:dream_hs_ael_local_damping}
\end{figure}

Firstly, the level of local excitation is larger for the
1T mode than it is for the 2F mode. This is one explication
for the difference in damping amplitude observed above.
This can be attributed again to the displacement related to the
1T mode. In fact, this mode has the tendency to change the
local angle of attack of the blade, yielding an unadapted
inflow velocity. This angle of attack drives, for the most part,
the aerodynamic behavior around the blade. Therefore, a small
change in angle of attack can have a strong impact on the loading and
the local excitation might be emphasized.
Compared to the low-speed configuration, the local excitation
is always positive on the leading edge of the blade, meaning
that the change in angle of attack and in dihedral angle
produces for the torsional and the flection mode, respectively,
is a positive feature for the damping.

Secondly, the influence of the IBPA remains limited for the
two modes. The global phenomenology is
kept unchanged even though the amplitude varies.
For the torsional mode, the local excitation of
the tip of the blade seems to be sensitive 
to the IBPA. This is the tip vortex region, and
advanced post-processing procedures might be required
to fully understand the behavior of local excitation
near the tip of the blades.

Globally the shape of the local excitation contours
is much more complicated on the high-speed configuration
compared to the low-speed one. In fact, 
not only the modes inflection lines become inflection lines
for the local excitation, but also the shock and 
the flow that develops in the tip region are important.

\subsection{Influence of the number of harmonics on the aeroelastic results}
\label{sub:dream_hs_convergence_ael}

To assess the convergence of the capture of the damping by the multi-frequential
HB approach, four computations are run on the second flection mode
of the high-speed CROR configuration. For each computation, the 
vibration frequency is considered with one to several harmonics
of the blade passing frequency of the rear rotor. As indicated in
Sec.~\ref{par:choice_of_frequencies}, several approaches exist to
truncate the frequency sets. Here, only the "cross grid" truncation 
pattern is used with only one aeroelastic frequency.
\begin{figure}[htp]
 \ra{1.3} \centering
 \begin{tabular}{r|cccc}
   \toprule
   & \multicolumn{4}{c}{
        \includegraphics[width=0.22\textwidth]{dream_hs_damping_scale.pdf}} \\
   & $N=2$ & $N=3$ & $N=4$ & $N=5$ \\
   \midrule
   \rotatebox{90}{\quad\quad\quad suction side} 
   & \includegraphics[width=0.12\textwidth]{DREAM_HS_HBT_N2_AEL_H1M2FD-1_roe3_sa_local_damping_SS.png}
   & \includegraphics[width=0.12\textwidth]{DREAM_HS_HBT_N3_AEL_H1M2FD-1_roe3_sa_local_damping_SS.png}
   & \includegraphics[width=0.12\textwidth]{DREAM_HS_HBT_N4_AEL_H1M2FD-1_roe3_sa_local_damping_SS.png}
   & \includegraphics[width=0.12\textwidth]{DREAM_HS_HBT_N5_AEL_H1M2FD-1_roe3_sa_local_damping_SS.png} \\
   \rotatebox{90}{\quad\quad\quad pressure side} 
   & \includegraphics[width=0.12\textwidth]{DREAM_HS_HBT_N2_AEL_H1M2FD-1_roe3_sa_local_damping_PS.png}
   & \includegraphics[width=0.12\textwidth]{DREAM_HS_HBT_N3_AEL_H1M2FD-1_roe3_sa_local_damping_PS.png}
   & \includegraphics[width=0.12\textwidth]{DREAM_HS_HBT_N4_AEL_H1M2FD-1_roe3_sa_local_damping_PS.png}
   & \includegraphics[width=0.12\textwidth]{DREAM_HS_HBT_N5_AEL_H1M2FD-1_roe3_sa_local_damping_PS.png} \\
   \bottomrule
   \multicolumn{1}{c}{}& \\
   \multicolumn{1}{c}{} & \multicolumn{4}{c}{
        \includegraphics[width=0.45\textwidth]{DREAM_HS_COMVERGENCE_DAMPING.pdf}} \\
 \end{tabular}
 \caption{High-speed isolated configuration: convergence of local excitation and damping.}
 \label{fig:dream_hs_ael_convergence_damping}
\end{figure}
The local excitation and damping results are 
shown in Fig.~\ref{fig:dream_hs_ael_convergence_damping}.
Raising the number of harmonics of the rear rotor blade passing frequency
changes the damping by at-most 10\% but the contours
of the local excitation are kept nearly unchanged. Only small differences
are observed, that integrated produces the 10\%
observed on the damping value. Further investigation of the
convergence for the choice of frequencies are needed, but
this has not been done on this work.


\chconclu{The multi-frequential harmonic balance
method along with a decoupled approach has been
assessed on a high-speed CROR configuration.
The steady computations reveal highly non-linear 
features, namely shocks. The prediction tool
defined in Chap.~\ref{cha:limitations_convergence}
is then used to estimate the number 
of harmonics required to simulate the configuration.
Seven harmonics are needed for the high-speed configuration
whereas only four were required for the low-speed one
for the same accuracy.
With no need for trial and error simulations 
to select the number of harmonics, results
are shown to be consistent, namely tip vortices and wake interactions
are well captured. 
Aeroelastic simulations are then undertaken.
The results are scrutinized with focus on the integrated damping
and the local excitation of the blades.
This configuration is shown to be flutter free, 
similarly to the low-speed one.
The proposed methodology is thus demonstrated to be able to
tackle demanding industrial configurations.}
