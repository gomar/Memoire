%!TEX root = ../../../adrien_gomar_phd.tex

The same topology and number of grid points is used to
mesh this high-speed configuration. The reader is referred to 
Sec.~\ref{sec:dream_ls_numerical} for detailed information.

\paragraph{Influence of the spatial discretization}
\label{sub:dream_hs_spatial_discretization}

To assess the different spatial scheme for this high-speed
CROR configuration, the same exercise as done in 
Sec.~\ref{sub:dream_ls_spatial_discretization} is made below.
The same four schemes are evaluated based 
on the convergence of the computation
and the convergence of the integrated 
results (similarity coefficients).
For the \citet{Jameson1981} scheme, the artificial viscosities
are chosen as follow: $\kappa_2 = 1.0$ as the configuration is likely to 
see shocks and $\kappa_4 = 0.016$. We will see that the computation converges
with this low $\kappa_4$ coefficient. Therefore, only this coefficient
will be tested.

The convergence of the computations using the four spatial schemes
is shown in Fig.~\ref{fig:DREAM_HS_RESIDUALS_PPT}. The convergence is good
for all the spatial schemes. In fact, more than five order of magnitude
is lost on the residuals.
\begin{figure}[htb]
  \centering
  \includegraphics*[width=0.40\textwidth]{SPACE_SCHEME_DIFF_HS_RESIDUALS.pdf}
  \caption{High-speed isolated configuration: convergence 
  of steady computations using different spatial schemes.}
  \label{fig:DREAM_HS_RESIDUALS_PPT}
\end{figure}
The convergence 

