%!TEX root = ../../../adrien_gomar_phd.tex

\subsection{Analysis of the convergence}
\label{sub:dream_hs_steady_conv}

The convergence of the simulation is obtained 
after one thousand iterations for both
the residuals and the similarity coefficients 
(Figure~\ref{fig:dream_HS_convergence_roe2}). More than
five order of magnitude is obtained for the residuals and
the similarity coefficients are stabilized starting below
$1,000$ iterations. According to \citet{Casey2000},
this means that the steady simulation is converged.
\begin{figure}[htp]
  \centering
  \subfigure[residuals]{\includegraphics[width=.35\textwidth]{DREAM_HS_RESIDUALS_PPT.pdf}}
  \subfigure[$C_T$]{\includegraphics[width=.35\textwidth]{DREAM_HS_FORCES_CT_PPT.pdf}}
  \subfigure[$C_P$]{\includegraphics[width=.35\textwidth]{DREAM_HS_FORCES_CP_PPT.pdf}}
  \subfigure[$\eta$]{\includegraphics[width=.35\textwidth]{DREAM_HS_FORCES_ETA_PPT.pdf}}
  \caption{High-speed isolated configuration: convergence of the steady
  computation.}
  \label{fig:dream_HS_convergence_roe2}
\end{figure}

\subsection{Similarity coefficients}
\label{sub:dream_hs_sim_coeff}

The similarity coefficients are representative of a cruise
propeller (see Eq.~\eqref{eq:estimation_sim_coeff}). 
Firstly, the thrust is higher
on the rear rotor than on the front rotor, even though it is
relatively well distributed. Bare in mind that the rear rotor
similarity coefficient is normalized by the front rotor diameter. Therefore, 
the thrust produced by the rear rotor is larger than the one of the front rotor, 
relatively to its diameter. 
Secondly, the power coefficient is similar for both the front and the
rear rotor. As the power coefficient represents the mechanical input
given to flow, this means that the mechanical distribution is 
well repartitioned, which is a design wish.
\begin{table}[htp]
  \ra{1.3} \centering
  \begin{tabular}{ccc|cccccc}
    \toprule
    $C_T$ & $C_P$ & $\eta$ & $C_{T_f}$ & $C_{P_f}$ & $\eta_f$ & $C_{T_r}$ & $C_{P_r}$ & $\eta_r$ \\
    \midrule
    0.9017 & 3.8485 & 0.8577 & 0.4105 & 1.9262 & 0.7801 & 0.4912 & 1.9223 & 0.9354 \\
    \bottomrule
  \end{tabular}
  \caption{High-speed isolated configuration: similarity coefficients.}
  \label{tab:dream_HS_sim_coeff}
\end{table}

\subsection{Radial profiles}
\label{sub:dream_hs_radial_profiles}

Radial profiles are computed on the steady results and 
reported in Fig.~\ref{fig:dream_HS_radial_profiles}.

The absolute Mach number (Fig.~\ref{fig:dream_HS_radial_profiles_ma}) 
increases from its inflow
value ($M_a = 0.73$) up to around $M_a=0.76$. This represents
a 4\% increase that has to be compared to the 100\% increase
for the low-speed configuration. The stream tube contraction
is barely seen as the increase in velocity remains bounded.
It seems that the front rotor tip vortices do not interact
with the rear rotor blades as the small increase near 90\%
of the span, which is attributed to the core of the vortices,
is not contracted by the stream tube.

The absolute angle (Fig.~\ref{fig:dream_HS_radial_profiles_alpha}) 
of the velocity highlights, again, the advantage
of using a CROR compared to a single row propeller system. In fact,
the flow is deviated by the front rotor from its axial direction
to a mean $5^\circ$ velocity vector. The rear rotor then deviates
back the flow field to make it almost purely axial with exceptions
near the hub and near the front tip vortex region ($0.75 \leq R/R_f \leq 0.95$).
This explains the efficiency of a CROR propeller system also for high-speed
inflow conditions.

The static pressure (Fig.~\ref{fig:dream_HS_radial_profiles_ps}) increases
by at-most 10\% which is larger than it was for the low-speed
inflow condition. In fact, a larger part of the energy is converted to static
pressure and not to velocity. The potential effects can be seen in the
$P2$ and $P4$ planes. In fact, the pressure decreases before
crossing a rotor blades. This is due to the acceleration of the
fluid that is done at each rotor crossing. This increase of velocity
creates a decrease in static pressure that is seen upstream the rotors.

The stagnation pressure is shown in Fig.~\ref{fig:dream_HS_radial_profiles_pi}.
At each rotor crossing it increases since both the absolute velocity and
the static pressure increase. 
With the stagnation temperature rise shown
in Fig.~\ref{fig:dream_HS_radial_profiles_ti}, one can say that
the rotors provide energy to the fluid to create the thrust.
Moreover, the stream tube contraction seems to be lessened
compared to the low-speed flight condition. It will have to be
confirmed by the forthcoming unsteady results.

\begin{figure}[htp]
  \centering
  \subfigure[absolute Mach number]{
    \label{fig:dream_HS_radial_profiles_ma}
    \includegraphics[width=.72\textwidth]{DREAM_HS_RANS_AZI_MEAN_PPT_macha.pdf}}
  \subfigure[absolute flow angle]{
    \label{fig:dream_HS_radial_profiles_alpha}
    \includegraphics[width=.72\textwidth]{DREAM_HS_RANS_AZI_MEAN_PPT_alpha.pdf}}
  \caption{High-speed isolated configuration: radial profiles.}
\end{figure}
\setcounter{figure}{\value{figure}-1}
\begin{figure}[htp]
  \centering
  \setcounter{subfigure}{3}
  \subfigure[static pressure]{
    \label{fig:dream_HS_radial_profiles_ps}
    \includegraphics[width=.72\textwidth]{DREAM_HS_RANS_AZI_MEAN_PPT_ps.pdf}}
  \subfigure[stagnation pressure]{
    \label{fig:dream_HS_radial_profiles_pi}
    \includegraphics[width=.72\textwidth]{DREAM_HS_RANS_AZI_MEAN_PPT_pi.pdf}}
  \subfigure[stagnation temperature]{
    \label{fig:dream_HS_radial_profiles_ti}
    \includegraphics[width=.72\textwidth]{DREAM_HS_RANS_AZI_MEAN_PPT_ti.pdf}}
  \caption{High-speed isolated configuration: radial profiles (contd.).}
  \label{fig:dream_HS_radial_profiles}
\end{figure}

\subsection{Flow field around the blades}
\label{sub:dream_hs_blades}

Relative Mach number contours and the pressure coefficient
$-k_p$ are shown in Fig.~\ref{fig:dream_HS_mach_kp} for both the
front and the rear rotors. The flow field is much more complicated than
the low-speed configuration. Looking at the $-k_p$ profiles first, on can
see that on the front rotor, no shock is seen by rather a compression

\begin{figure}[htp]
 \centering
 \begin{tabular}{rccc}
   & $-k_p$ front rotor
   & $-k_p$ rear rotor
   & relative Mach number\\
   \rotatebox{90}{\qquad\qquad 25~\%} 
   & \includegraphics[width=0.28\textwidth]{DREAM_HS_KP_25_FRONT_PPT.pdf}
   & \includegraphics[width=0.28\textwidth]{DREAM_HS_KP_25_REAR_PPT.pdf}
   & \includegraphics[width=0.28\textwidth]{DREAM_HS_RANS_roe2_sa_slice_r_25_mach_rel.png}\\
   \rotatebox{90}{\qquad\qquad 50~\%} 
   & \includegraphics[width=0.28\textwidth]{DREAM_HS_KP_50_FRONT_PPT.pdf}
   & \includegraphics[width=0.28\textwidth]{DREAM_HS_KP_50_REAR_PPT.pdf}
   & \includegraphics[width=0.28\textwidth]{DREAM_HS_RANS_roe2_sa_slice_r_50_mach_rel.png}\\
   \rotatebox{90}{\qquad\qquad 75~\%} 
   & \includegraphics[width=0.28\textwidth]{DREAM_HS_KP_75_FRONT_PPT.pdf}
   & \includegraphics[width=0.28\textwidth]{DREAM_HS_KP_75_REAR_PPT.pdf}
   & \includegraphics[width=0.28\textwidth]{DREAM_HS_RANS_roe2_sa_slice_r_75_mach_rel.png}\\
   \rotatebox{90}{\qquad\qquad 90~\%} 
   & \includegraphics[width=0.28\textwidth]{DREAM_HS_KP_90_FRONT_PPT.pdf}
   & \includegraphics[width=0.28\textwidth]{DREAM_HS_KP_90_REAR_PPT.pdf}
   & \includegraphics[width=0.28\textwidth]{DREAM_HS_RANS_roe2_sa_slice_r_90_mach_rel.png}\\
   \rotatebox{90}{\qquad\qquad 95~\%} 
   & \includegraphics[width=0.28\textwidth]{DREAM_HS_KP_95_FRONT_PPT.pdf}
   & \includegraphics[width=0.28\textwidth]{DREAM_HS_KP_95_REAR_PPT.pdf}
   & \includegraphics[width=0.28\textwidth]{DREAM_HS_RANS_roe2_sa_slice_r_95_mach_rel.png}  
 \end{tabular}
 \caption{High-speed isolated configuration: pressure coefficient and relative Mach
 number contours at different radial position.}
 \label{fig:dream_HS_mach_kp}
\end{figure}

\begin{figure}[htp]
  \centering
  \subfigure[$P3$]{\includegraphics[width=.35\textwidth]{DREAM_HS_RANS_roe2_sa_slice_x_front_1_entropy.png}}
  \subfigure[$P4$]{\includegraphics[width=.35\textwidth]{DREAM_HS_RANS_roe2_sa_slice_x_rear_0_entropy.png}}
  \subfigure[$P5$]{\includegraphics[width=.35\textwidth]{DREAM_HS_RANS_roe2_sa_slice_x_rear_1_entropy.png}}
  \subfigure[$P6$]{\includegraphics[width=.35\textwidth]{DREAM_HS_RANS_roe2_sa_slice_x_rear_2_entropy.png}}
  \caption{High-speed isolated configuration: axial cut of entropy.}
   \label{fig:dream_HS_steady_entropy}
\end{figure}