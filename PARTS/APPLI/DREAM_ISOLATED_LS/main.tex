%!TEX root = ../../../adrien_gomar_phd.tex
\chapter{Isolated low-speed CROR configuration}
\label{cha:dream_ls_isolated}

\chabstract{The studies performed in the previous chapters 
are finally used together to simulate the aeroelasticity
of a low-speed CROR configuration. First,
the steady results are analyzed to provide insight into the flow
physics and give confidence in the results. The prediction tool
defined in Chap.~\ref{cha:limitations_convergence}
is then used to estimate the number of harmonics required to
simulate the unsteady rigid-motion response 
of the CROR using the harmonic balance approach.
Aeroelastic simulations are then carried-out using the decoupled
approach that has been validated in the previous chapter. The OPT
algorithm develop in Chap.~\ref{cha:limitations_condition_number}
is used to ensure the stability of the computations.
Local excitation contours and the integrated damping are finally
analyzed.}


\newpage

\section{Presentation of the case}
\label{sec:dream_presentation}
%!TEX root = ../../../adrien_gomar_phd.tex


\begin{figure}[htp]
  \centering
  \includegraphics[width=.3\textwidth]{DREAM_LS_wall.png}
  \caption{Low-speed isolated contra-rotating open rotor geometry.}
  \label{fig:dream_ls_wall}
\end{figure}

The studied configuration is a pusher contra-rotating open rotor
that comes from the know-how of SAFRAN-Snecma 
shown in Fig.~\ref{fig:dream_ls_wall} for the
Low-speed (LS) flight condition, representative of the take-off.
The simulated configuration does not include the spinner as the
experimental setup does not take into account this part of the geometry.
Sadly, the experimental results were not available for comparison
at the time this PhD thesis was written.

\begin{table}[htp]
  \ra{1.3} \centering
  \begin{tabular}{cccc}
    \toprule
    $M_0$ & $|\Omega|$ & $J$ & $M_{tip}$ \\
    \midrule
    $0.2$ & $5739$ tr.min\textsuperscript{-1} & 1.06 & 0.63 \\
    \bottomrule
  \end{tabular}
  \caption{Low-speed isolated contra-rotating open rotor flight condition parameters.}
  \label{tab:dream_ls_flight_condition}
\end{table} 
Table~\ref{tab:dream_ls_flight_condition} recalls the main
parameters of the case: the inflow Mach number $M_0$,
the absolute value of the rotation speed of both rotors $|\Omega|$,
the advance ratio $J$ (whose definition is given in Chap.~\ref{cha:cror})
and the Mach number at the tip of
the front rotor blades based on the inflow velocity and the advance ratio:
\begin{equation}
	M_{tip} = M_0 \sqrt{1 + \left(\frac{\pi}{J} \right)^2}
  \label{eq:m_tip}
\end{equation}
Equation~\ref{eq:m_tip} is actually a simple transcription of the velocity triangle
applied to the infinite velocity and to the rotation speed perceived
at the front blade tip:
\begin{equation}
    M_{tip} = M_0 \sqrt{1 + \left(\frac{\pi}{J}\right)^2} = 
    \frac{V_0}{\sqrt{\gamma R t_0}} \sqrt{1 + \left(\frac{
    	\cancel{\pi} \cdot \Omega D}{
    	2 \cancel{\pi} \cdot V_0}\right)^2} =
    \sqrt{\frac{V_0^2 + (\Omega R)^2}{\gamma R t_0}}
    \label{eq:m_tip_explained}
\end{equation}

The inflow Mach number $M_0$ is within the incompressible range
($M_0 < 0.3$). As the CFD flow solver used here is \emph{elsA}
(see Appendix~\ref{app:elsa}) which is a compressible code, 
a preconditionner might be needed for the computations to converge. 
Hopefully, the fluid is accelerated by the two rotors
and the tip Mach number is high enough to not use any preconditionning.
However, let us bare in mind that this range of Mach number might
be tedious for a compressible flow solver.
The advance ratio $J$ is around~1 which is a classical value for
low-speed propellers~\cite{Bousquet2012}. Note that the rotation speed is almost
one order of magnitude larger than expected. 
In fact, a full-scale CROR diameter is around 5~meters. With this rotation speed,
this would lead to a tip Mach number greater than~4. This high rotation speed is 
actually here to compensated for the small radius of the blades, which is~33~cm
for experimental purposes, while maintaining the same similarity coefficients.


\section{Numerical setup}
\label{sec:dream_ls_numerical}
%!TEX root = ../../../adrien_gomar_phd.tex

The mesh considered to compute this
CROR configuration is a single-blade passage meshed
with an O4H topology show in Fig.~\ref{fig:dream_mesh}. This is a classical
topology for turbomachinery computations that is here applied to 
a CROR.
\begin{figure}[htb]
  \centering
  \subfigure[Topology]{
    \label{fig:dream_mesh}
    \includegraphics[height=.4\textwidth]{dream_mesh.png}}
  \subfigure[Detailed topology with number of grid points]{
    \label{fig:dream_ls_mesh}
    \includegraphics[height=.4\textwidth]{dream_ls_mesh.pdf}}
  \caption{Low-speed isolated configuration mesh topology.}
\end{figure}
The number of points is reported in 
Fig.~\ref{fig:dream_ls_mesh} for a blade-to-blade section. 
129~points discretize the blade, 45~the pitch and 181~the radial
extent. The same number of points is used for the front
and the rear rotors. These number of grid points are
classical inputs for steady RANS computations.

As a CROR is not shrouded, a sufficiently large
far-field domain is taken to ensure a minimum influence
of the far-field boundary conditions on the results.
The computational domain is shown in Fig.~\ref{fig:dream_farfield}.
The radial extent is $3D$ while the axial one is $3.5D$.
\citet{Peters2012} consider an axial extent of $7.5D$
with a radial extent of $4D$ while \citet{Zachariadis2011}
consider $2.5D$ and $3.6D$, respectively. We are thus in 
the mid-range of the values taken in the literature.
\begin{figure}[htb]
  \centering
  \includegraphics[width=.4\textwidth]{dream_farfield.pdf}
  \caption{Low-speed isolated configuration far-field domain and boundary conditions.}
  \label{fig:dream_farfield}
\end{figure}
As highlighted by underlined text in Fig.~\ref{fig:dream_farfield},
the boundary conditions used are: (i)~adiabatic walls
for the blades and the shroud (or spinner) and (ii)~constant
stagnation values used at the far-field.
In opposite to the study made previously on the STCF11
configuration (see Sec.~\ref{sec:stcf11_numerical}),
the mesh stems from literature and industrial best
practices and thus will not be assessed.

Turbulence is modeled using the one-equation model of
\citet{Spalart1992}.  Roe's scheme~\cite{Roe1981} along with a 
second-order MUSL extrapolation 
is used to compute the convective fluxes.
The maximum CFL number is set to~10 for the steady 
computations and the HB simulations.

\subsection{Influence of the spatial discretization} % (fold)
\label{sub:dream_ls_spatial_discretization}

To assess the influence of spatial discretization, four 
space schemes are used to simulate the low-speed CROR configuration.
These four schemes are the \citet{Jameson1981} scheme (noted JST) with artificial
viscosities $\kappa_4 = 0.016$, $\kappa_4 = 0.032$, $\kappa_4 = 0.064$
and $\kappa_2$ equal to $0.5$ as the operating point should not 
lead to shocks. In addition to this scheme, three upwind
Roe's scheme~\cite{Roe1981} along with no extrapolation (noted Roe~1),
a second order (noted Roe~2) or a third-order (noted Roe~3) 
MUSCL extrapolations are used.

The convergence of the different computations is show 
in Fig.~\ref{fig:dream_ls_space_scheme_residual}
for the four schemes. The convergence is not 
very good. Only the Roe~1 and Roe~2 spatial schemes give 
a convergence that has an acceptable slope. In contrary,
the JST~$\kappa_4 = 0.016$ diverges and the three
remaining schemes hardly converge. The higher the
viscosity parameter $\kappa_4$ of the JST scheme, the better
the convergence. Exceeding $\kappa_4 = 0.064$ should
warn us that something might be wrong with the computation.
As stated previously, this is actually due to the range of Mach
number in which this low-speed configuration operates. In fact,
a part of the computation can be said to be within the incompressible
range ($M \leq 0.3$) and therefore not adapted to a compressible
flow solver which is the case of the \emph{elsA} code 
(see Appendix~\ref{app:elsa}).
\begin{figure}[htb]
  \centering
  \includegraphics[width=.5\textwidth]{DREAM_LS_RESIDUALS_PPT.pdf}
  \caption{Convergence of the steady computations.}
  \label{fig:dream_ls_space_scheme_residual}
\end{figure}

To differentiate the spatial schemes, 
the steady results for the similarity coefficients are reported
in Fig.~\ref{fig:dream_ls_space_scheme_coeff} for all spatial scheme, 
except the diverging JST~$\kappa_4 = 0.016$ computation.
The results are normalized to the Roe~2 values.
The Roe~2, Roe~3, and the two JST schemes give equivalent
similarity coefficients as the difference is smaller than 1~\%.
In opposite, the first order upwind scheme Roe~1 give a 5~\%
difference for both the traction coefficient $C_T$ and the efficiency $\eta$.
Cross-crossing these results with the convergence of the computations
given by Fig.~\ref{fig:dream_ls_space_scheme_residual}, the Roe~2
scheme is kept for the following as it give both a good convergence
of the residuals and the similarity coefficients.
\begin{figure}[htb]
  \centering
  \includegraphics[width=.5\textwidth]{space_scheme_diff.pdf}
  \caption{Convergence of the steady computations, comparison of similarity coefficients.}
  \label{fig:dream_ls_space_scheme_coeff}
\end{figure}


\section{Steady results}
\label{sec:dream_ls_steady_results}
%!TEX root = ../../../adrien_gomar_phd.tex

\subsection{Analysis of the convergence}
\label{sub:dream_ls_conv_coeff}

The convergence of the steady computation using the Roe~2 space scheme
is reported in Fig.~\ref{fig:dream_ls_convergence_roe2}. The residual
show a four order of magnitude decrease and the similarity
coefficients are convergence starting at 500~iterations.
Therefore, according to \citet{Casey2000}, the
solution is considered to be converged.
\begin{figure}[htb]
  \centering
  \subfigure[residuals]{\includegraphics[width=.35\textwidth]{DREAM_LS_RESIDUALS_PPT.pdf}}
  \subfigure[$C_T$]{\includegraphics[width=.35\textwidth]{DREAM_LS_FORCES_CT_PPT.pdf}}
  \subfigure[$C_P$]{\includegraphics[width=.35\textwidth]{DREAM_LS_FORCES_CP_PPT.pdf}}
  \subfigure[$\eta$]{\includegraphics[width=.35\textwidth]{DREAM_LS_FORCES_ETA_PPT.pdf}}
  \caption{Low-speed isolated configuration: convergence of the steady
  computation.}
  \label{fig:dream_ls_convergence_roe2}
\end{figure}

\subsection{Radial profiles}
\label{sub:dream_ls_radial_profiles}

Six radial profiles are extracted,
positioned as shown in Fig.~\ref{fig:dream_ls_position_radial}
using Antares (see Appendix~\ref{app:antares}). The absolute
Mach number, absolute flow angle, static pressure, static
temperature, stagnation pressure and stagnation temperature
are shown in Fig.~\ref{fig:dream_ls_radial_profiles}
against the radial position expressed
relative to the radius of the front rotor blade.

The absolute Mach number, show in 
Fig.~\ref{fig:dream_ls_radial_profiles_ma}, is constantly increased through
the two rotors as it goes from the inflow condition value $M=0.2$
up to $M=0.4$. Note that above $R/h=1$, the Mach number
almost recovers the inflow condition. Moreover, it can be infer from the
$M_a$ evolution, that the stream tube is contracting.

The motivation for adding a second rotor to a propeller
was to recover the energy lost by the swirling flow
(recall Sec.~\ref{sub:cror_velocity_triangle}).
The absolute angle of the flow is shown in 
Fig.~\ref{fig:dream_ls_radial_profiles_alpha}. The front rotor
deviates the absolute velocity of almost $20^\circ$, justifying the need
for a second rotor. Between the fourth and the fifth extraction plane, namely
passing through the rear rotor, straighten the flow up. In fact,
the deflection angle is now close to $0^\circ$ for $0.3 \leq R/h \leq 0.7$.
Below that, the deflection angle remains negative. In the tip vortex region
of the rear rotor, one can the effect of the two tip vortices: between 
$0.8 \leq R/h \leq 0.9$, the front rotor tip vortex is seen as the 
deflection angle is positive, which is consistent with the positive
peak observe near the blade tip region in plane $P3$ and $P4$.

The goal of a CROR is to create thrust through an acceleration of 
the flow, not to produce static pressure as in a compressor stage.
This is highlighted in Fig.~\ref{fig:dream_ls_radial_profiles_ps}
where the evolution of the static pressure is shown.
In fact, the evolution of the static pressure lies with 2~\%
of its inflow value. A small increase is observed at each
rotor crossing. Upstream the rotors, the potential effects can
be seen. Actually, the flow is accelerated by the rotors, this acceleration
yielding a decrease of the static pressure 
(roughly through the Bernouilli theorem) and this pressure deficit is observed in
planes $P1$, $P2$ and $P4$.

The stagnation pressure evolution is shown in 
Fig.~\ref{fig:dream_ls_radial_profiles_pi}. As the static pressure
and the absolute Mach number is grows along with the crossing of the rotors,
it is logical to have an increase in the stagnation pressure.
It is actually a marker of the energy exchanged between the rotor
and the fluid.

Figure~\ref{fig:dream_ls_radial_profiles_ti}
shows the stagnation temperature.
An increase is observed at each 
rotor crossing. This is consistent with the
Euler theorem that states that the variation of total enthalpy (roughly
the stagnation temperature) is equal to the variation of the product of
the azimuthal deflection of the fluid and the rotation velocity. As a rotor
deviates the fluid, the stagnation temperature should increase, 
hence the consistence.
\begin{figure}[htb]
  \centering
  \subfigure[position of the extraction planes]{
    \label{fig:dream_ls_position_radial}
    \includegraphics[width=.55\textwidth]{dream_position_azi_mean.pdf}}
  \subfigure[absolute Mach number]{
    \label{fig:dream_ls_radial_profiles_ma}
    \includegraphics[width=.72\textwidth]{DREAM_LS_RANS_AZI_MEAN_PPT_macha.pdf}}
  \subfigure[absolution flow angle]{
    \label{fig:dream_ls_radial_profiles_alpha}
    \includegraphics[width=.72\textwidth]{DREAM_LS_RANS_AZI_MEAN_PPT_alpha.pdf}}
  \caption{Low-speed isolated configuration: radial profiles.}
\end{figure}
\setcounter{figure}{\value{figure}-1}
\begin{figure}[htb]
  \centering
  \setcounter{subfigure}{3}
  \subfigure[static pressure]{
    \label{fig:dream_ls_radial_profiles_ps}
    \includegraphics[width=.72\textwidth]{DREAM_LS_RANS_AZI_MEAN_PPT_ps.pdf}}
  \subfigure[stagnation pressure]{
    \label{fig:dream_ls_radial_profiles_pi}
    \includegraphics[width=.72\textwidth]{DREAM_LS_RANS_AZI_MEAN_PPT_pi.pdf}}
  \subfigure[stagnation temperature]{
    \label{fig:dream_ls_radial_profiles_ti}
    \includegraphics[width=.72\textwidth]{DREAM_LS_RANS_AZI_MEAN_PPT_ti.pdf}}
  \caption{Low-speed isolated configuration: radial profiles (contd.).}
  \label{fig:dream_ls_radial_profiles}
\end{figure}

These 1D results provide us confidence. In fact, the flow physic that
was expected is observed in the results. To further analyze the simulation,
2D results are presented in the following section.

\subsection{Flow field around the blades}
\label{sub:dream_ls_flow_field}

Contours of the relative Mach number are shown in 
Fig.~\ref{fig:dream_ls_mach_kp} along with the pressure coefficient
for both the front and the rear rotor
$k_p$ defined as:
\begin{equation}
   -k_p = - \frac{p_s - p_{s_0}}{\rho n^2 D^2}.
\end{equation}
The $- k_p$ should be interpreted as follow, an increasing $- k_p$
means that the pressure gradient is negative, namely the flow
is accelerating. Therefore, the positive  $-k_p$ range is attributed
to the suction side and the negative to the pressure side. The shape of
the pressure coefficients is schematically represented 
in Fig.~\ref{fig:subsonic_airfoil}. The stagnation point is highlighted
by the minimum of the pressure coefficient. On the pressure side 
($-k_p < 0$) the flow then accelerates
toward the trailing edge with a favorable pressure gradient. 
On the suction side, a rapid acceleration of the fluid
is observed near the leading edge 
($\partial (-k_p) / \partial x \gg 0$) followed by
a deceleration of the fluid along with an
adverse pressure gradient.
Nevertheless, the velocity of the fluid is always superior
on the suction side compared to the pressure side.



As inferred by the tip Mach number value $M_{tip}$ of the blade,
the relative Mach number does not 
cross the supersonic boundary $M_{rel} = 1$.



\begin{figure}[htb]
 \centering
 \begin{tabular}{rccc}
   & $-k_p$ front rotor
   & $-k_p$ rear rotor
   & relative Mach number\\
   \rotatebox{90}{\qquad\qquad 25~\%} 
   & \includegraphics[width=0.28\textwidth]{DREAM_LS_KP_25_FRONT_PPT.pdf}
   & \includegraphics[width=0.28\textwidth]{DREAM_LS_KP_25_REAR_PPT.pdf}
   & \includegraphics[width=0.28\textwidth]{DREAM_LS_RANS_roe2_sa_slice_r_25_mach_rel.png}\\
   \rotatebox{90}{\qquad\qquad 50~\%} 
   & \includegraphics[width=0.28\textwidth]{DREAM_LS_KP_50_FRONT_PPT.pdf}
   & \includegraphics[width=0.28\textwidth]{DREAM_LS_KP_50_REAR_PPT.pdf}
   & \includegraphics[width=0.28\textwidth]{DREAM_LS_RANS_roe2_sa_slice_r_50_mach_rel.png}\\
   \rotatebox{90}{\qquad\qquad 75~\%} 
   & \includegraphics[width=0.28\textwidth]{DREAM_LS_KP_75_FRONT_PPT.pdf}
   & \includegraphics[width=0.28\textwidth]{DREAM_LS_KP_75_REAR_PPT.pdf}
   & \includegraphics[width=0.28\textwidth]{DREAM_LS_RANS_roe2_sa_slice_r_75_mach_rel.png}\\
   \rotatebox{90}{\qquad\qquad 90~\%} 
   & \includegraphics[width=0.28\textwidth]{DREAM_LS_KP_90_FRONT_PPT.pdf}
   & \includegraphics[width=0.28\textwidth]{DREAM_LS_KP_90_REAR_PPT.pdf}
   & \includegraphics[width=0.28\textwidth]{DREAM_LS_RANS_roe2_sa_slice_r_90_mach_rel.png}\\
   \rotatebox{90}{\qquad\qquad 95~\%} 
   & \includegraphics[width=0.28\textwidth]{DREAM_LS_KP_95_FRONT_PPT.pdf}
   & \includegraphics[width=0.28\textwidth]{DREAM_LS_KP_95_REAR_PPT.pdf}
   & \includegraphics[width=0.28\textwidth]{DREAM_LS_RANS_roe2_sa_slice_r_95_mach_rel.png}  
 \end{tabular}
 \caption{Low-speed isolated configuration: pressure coefficient and relative Mach
 number contours at different radial position.}
 \label{fig:dream_ls_mach_kp}
\end{figure}


\begin{figure}[htb]
  \centering
  \includegraphics*[width=0.40\textwidth]{subsonic_airfoil.pdf}
  \caption{}
  \label{fig:subsonic_airfoil}
\end{figure}

\begin{figure}[htb]
  \centering
  \subfigure[$P3$]{\includegraphics[width=.35\textwidth]{DREAM_LS_RANS_roe2_sa_slice_x_front_1_entropy.png}}
  \subfigure[$P4$]{\includegraphics[width=.35\textwidth]{DREAM_LS_RANS_roe2_sa_slice_x_rear_-1_entropy.png}}
  \subfigure[$P5$]{\includegraphics[width=.35\textwidth]{DREAM_LS_RANS_roe2_sa_slice_x_rear_1_entropy.png}}
  \subfigure[$P6$]{\includegraphics[width=.35\textwidth]{DREAM_LS_RANS_roe2_sa_slice_x_rear_2_entropy.png}}
  \caption{}
\end{figure}

\section{Spectral convergence of the harmonic balance computations}
\label{sec:dream_ls_spectral_convergence}
%!TEX root = ../../../adrien_gomar_phd.tex

\subsection{Using the prediction tool}
\label{sub:dream_ls_conv_hb_prediction_tool}

The prediction tool developed in 
Sec.~\ref{sub:prediction_tool_azimuthal_fft} is applied
to the studied configuration to evaluate the
required number of harmonics needed for the
harmonic balance approach to be converged.
The result is shown in Fig.~\ref{fig:DREAM_LS_RANS_ROE2_SPECTRUM_PPT}.
To capture 99\% of the energy, four harmonics are needed, even though
at a relative span superior to 60\%, only three harmonics would be needed.
However, as the implementation of the harmonic balance used
in this work does not allow a varying number of harmonics through the
configuration, four harmonics is supposed to be sufficient to efficiently 
represent the unsteady flow field.
\begin{figure}[htp]
  \centering
  \includegraphics*[width=0.5\textwidth]{DREAM_LS_RANS_ROE2_SPECTRUM_PPT.pdf}
  \caption{Low-speed isolated configuration: prediction of the number
  of harmonics needed to simulate the configuration.}
  \label{fig:DREAM_LS_RANS_ROE2_SPECTRUM_PPT}
\end{figure}

\subsection{Analyzing the similarity coefficients}
\label{sub:dream_ls_conv_hb_sim_coeff}
To confirm the number of harmonics needed to ensure the convergence
of harmonic balance computations, simulations are run with up to four
harmonics. The strategy used to launch the computation is as follow:
the steady computation is used as an initial guess for the $N=1$ HB computation.
Then each new HB computation is launched with the previous one as initial
solution.

Two harmonics are actually necessary to converge the temporal mean 
of the similarity coefficients as show 
in Tab.~\ref{tab:dream_ls_hb_conv_sim}. After that, a slight evolution of the
coefficients is still seen but represents a change lower than 0.01\%
of the $N=4$ results, hence the convergence. 
\begin{table}[htp]
  \ra{1.3} \centering
  \begin{tabular}{rccccc}
    \toprule
    & steady & HB $N=1$ & HB $N=2$ & HB $N=3$ & HB $N=4$ \\
    \midrule
    $C_T$  & $1.1319$ & $1.1334$ & $1.1330$ & $1.1330$ & $1.1329$ \\
    $C_P$  & $2.0927$ & $2.0951$ & $2.0944$ & $2.0945$ & $2.0946$ \\
    $\eta$ & $0.5726$ & $0.5727$ & $0.5726$ & $0.5726$ & $0.5725$ \\
    \bottomrule
  \end{tabular}
  \caption{Low-speed isolated configuration: analysis of the number of harmonics
  required to capture the mean similarity coefficients.}
  \label{tab:dream_ls_hb_conv_sim}
\end{table}
Note that a steady computation is sufficient to retrieve
the temporal mean value of the similarity coefficients.
In fact, a maximum of 0.1\% difference is observed by
comparing the mixing plane and harmonic balance results.

\subsection{Analyzing the blade response}
\label{sub:dream_ls_conv_hb_blade_response}
Of course, analyzing integrated results to assess the converge of
a computation is a primarily step that should be complemented with
a local analysis. In this way, a discrete Fourier transform is
performed to analyze the first harmonic of the static pressure on the 
rear rotor blades. Due to the passing
of the front rotor wakes, these blades will experience a
high level of unsteadinesses. It is therefore considered as a
bottleneck in the convergence of the HB computations, hence its analysis.
The results are shown in 
Fig.~\ref{fig:dream_ls_hb_blade_response_conv}. 
\begin{figure}[htp]
 \ra{1.3} \centering
 \begin{tabular}{r|cccc}
   \toprule
   & \multicolumn{2}{c}{mean} & \multicolumn{2}{c}{1\textsuperscript{st} harmonic} \\
   & \multicolumn{2}{c}{
        \includegraphics[width=0.22\textwidth]{dream_ls_blade_resp_scale_mean.pdf}} 
   & \multicolumn{2}{c}{
        \includegraphics[width=0.22\textwidth]{dream_ls_blade_resp_scale_H01_rear.pdf}} \\
   \midrule
   \rotatebox{90}{\quad\quad\quad steady} 
   & \includegraphics[width=0.10\textwidth]{DREAM_LS_RANS_roe2_sa_blade_response_rear_PS.png}
   & \includegraphics[width=0.10\textwidth]{DREAM_LS_RANS_roe2_sa_blade_response_rear_SS.png}
   &   &\\
   \rotatebox{90}{\quad\quad HB $N=1$} 
   & \includegraphics[width=0.10\textwidth]{DREAM_LS_TSM_N1_roe2_sa_blade_response_rear_mean_PS.png}
   & \includegraphics[width=0.10\textwidth]{DREAM_LS_TSM_N1_roe2_sa_blade_response_rear_mean_SS.png}
   & \includegraphics[width=0.10\textwidth]{DREAM_LS_TSM_N1_roe2_sa_blade_response_rear_H01_PS.png}
   & \includegraphics[width=0.10\textwidth]{DREAM_LS_TSM_N1_roe2_sa_blade_response_rear_H01_SS.png} \\
   \rotatebox{90}{\quad\quad HB $N=2$} 
   & \includegraphics[width=0.10\textwidth]{DREAM_LS_TSM_N2_roe2_sa_blade_response_rear_mean_PS.png}
   & \includegraphics[width=0.10\textwidth]{DREAM_LS_TSM_N2_roe2_sa_blade_response_rear_mean_SS.png}
   & \includegraphics[width=0.10\textwidth]{DREAM_LS_TSM_N2_roe2_sa_blade_response_rear_H01_PS.png}
   & \includegraphics[width=0.10\textwidth]{DREAM_LS_TSM_N2_roe2_sa_blade_response_rear_H01_SS.png} \\
   \rotatebox{90}{\quad\quad HB $N=3$} 
   & \includegraphics[width=0.10\textwidth]{DREAM_LS_TSM_N3_roe2_sa_blade_response_rear_mean_PS.png}
   & \includegraphics[width=0.10\textwidth]{DREAM_LS_TSM_N3_roe2_sa_blade_response_rear_mean_SS.png}
   & \includegraphics[width=0.10\textwidth]{DREAM_LS_TSM_N3_roe2_sa_blade_response_rear_H01_PS.png}
   & \includegraphics[width=0.10\textwidth]{DREAM_LS_TSM_N3_roe2_sa_blade_response_rear_H01_SS.png} \\
   \rotatebox{90}{\quad\quad HB $N=4$} 
   & \includegraphics[width=0.10\textwidth]{DREAM_LS_TSM_N4_roe2_sa_blade_response_rear_mean_PS.png}
   & \includegraphics[width=0.10\textwidth]{DREAM_LS_TSM_N4_roe2_sa_blade_response_rear_mean_SS.png}
   & \includegraphics[width=0.10\textwidth]{DREAM_LS_TSM_N4_roe2_sa_blade_response_rear_H01_PS.png}
   & \includegraphics[width=0.10\textwidth]{DREAM_LS_TSM_N4_roe2_sa_blade_response_rear_H01_SS.png} \\
   \bottomrule
 \end{tabular}
 \caption{Low-speed isolated configuration: analysis of the number of harmonics
  required to capture the harmonic response of the rear rotor blades.}
 \label{fig:dream_ls_hb_blade_response_conv}
\end{figure}

Only one harmonic is needed to convergence the time-average value on the
rear rotor blades. Actually, the steady computation gave already a good prediction
of this time averaged value. This is due to the range on which this
low-speed CROR configuration operates. The Mach number is almost within the
incompressible range. As such, the non-linearities of the Navier--Stokes equations
remains small and steady approaches give good results.

Two to three harmonics are needed to converge the first harmonic 
pressure response on the rear rotor blade. This is a rough estimation
as a small convergence of the
harmonic pressure rise on the suction side of the blade between HB $N=2$, 
$N=3$ and $N=4$ computations can be seen. 

\subsection{Analyzing the radial cuts}
\label{sub:dream_ls_conv_hb_slice_r}
The final assessment of the convergence is done on radial cuts
of entropy made at 75\% of the height of the rear rotor blade. 
Remember that a propeller is supposed to be the most efficient at 75\% of the
blade (see Chapter~\ref{cha:cror}) which justifies the analysis done at this height.
The entropy spurious waves vanishes when computing the HB $N=4$ even though
the HB $N=3$ computation give a relatively smooth entropy field.
\begin{figure}[htp]
  \centering
  \subfigure[HB $N=1$]{
  \includegraphics[width=.3\textwidth]{DREAM_LS_TSM_N1_roe2_sa_slice_r_75_entropy.png}}
  \subfigure[HB $N=2$]{
  \includegraphics[width=.3\textwidth]{DREAM_LS_TSM_N2_roe2_sa_slice_r_75_entropy.png}}
  \subfigure[HB $N=3$]{
  \includegraphics[width=.3\textwidth]{DREAM_LS_TSM_N3_roe2_sa_slice_r_75_entropy.png}}
  \subfigure[HB $N=4$]{
  \includegraphics[width=.3\textwidth]{DREAM_LS_TSM_N4_roe2_sa_slice_r_75_entropy.png}}
  \caption{Low-speed isolated configuration: analysis of the number of harmonics
  required to capture the wake at a 75\% height radial cut.}
  \label{fig:dream_ls_hb_slice_r_conv}
\end{figure}
Nevertheless, the prediction tool, the
similarity coefficients, the harmonic blade response and the radial cuts give us confidence in
the HB $N=4$ computation. It is therefore chosen to further analyze the unsteady 
results on the HB $N=4$ computation.


\section{Unsteady rigid-motion results}
\label{sec:dream_ls_rigid_results}
%!TEX root = ../../../adrien_gomar_phd.tex

\subsection{Similarity coefficients}
\label{sub:dream_ls_hb_sim_coeff}

Figure~\ref{fig:dream_ls_hb_unst_coeff} depicts the
unsteady variation of the thrust coefficient $C_T$ on 
both the front and the rear rotor.
The time is 
normalized by the reference period of the current rotor 
(the rotation frequency~$n$) and the thrust coefficient is normalized
by its temporal mean value. This allows to assess the unsteady variations
over one reference period. 

The level of unsteadiness is rather
the same on both rotors. It represents an envelope of approximately
$\pm 3\permil$ of the temporal mean value. This level is not negligible and
justifies the use of unsteady methods on CROR configurations. 
Moreover, even though wakes are shed behind the front rotor
that impinge the rear rotor blades, the level of unsteadiness
perceived by the rear rotor is close to the front rotor ones.
Actually, the rear rotor sees more unsteady flow
phenomena but on a smaller area. This can be one of the reasons
explaining the equal level of
unsteadiness observed on the front and rear rotor blades.
\begin{figure}[htp]
  \centering
  \subfigure[front rotor]{\includegraphics[width=0.45\textwidth]{DREAM_LS_TSM_FORCES_INST_FRONT_PPT.pdf}}
  \subfigure[rear rotor]{\includegraphics[width=0.45\textwidth]{DREAM_LS_TSM_FORCES_INST_REAR_PPT.pdf}}
  \caption{Low-speed isolated configuration: unsteadiness seen by the rotors.}
  \label{fig:dream_ls_hb_unst_coeff}
\end{figure}

To assess in more detail the unsteady flow
field seen by the blade, a harmonic analysis on the
blades is performed in the following section.

\subsection{Two-dimensional results: harmonic blade response}
\label{sub:dream_ls_hb_blade_response}

A discrete Fourier transform is computed on the blades
for the unsteady static pressure variable. This gives an
idea of the level of unsteadiness seen locally by the blades.
The amplitude of the first harmonic of the blade
passing frequency of the opposite rotor is shown in 
Figure~\ref{fig:dream_ls_hb_blade_response} for both blades.
The legend is in logarithmic scale and it is different
for the front and rear rotor blades. In fact, even though the
integrated level of unsteadiness is relatively the same, this
fails when looking at local results. Roughly, the harmonic
amplitude of the static pressure on the rear rotor blade is
ten times larger.
\begin{figure}[htp]
 \ra{1.3} \centering
 \begin{tabular}{cccc}
    \multicolumn{2}{c}{\includegraphics[width=0.3\textwidth]{dream_ls_blade_resp_scale_H01_front.pdf}} &
    \multicolumn{2}{c}{\includegraphics[width=0.3\textwidth]{dream_ls_blade_resp_scale_H01_rear.pdf}} \\
    \includegraphics[width=0.15\textwidth]{DREAM_LS_TSM_N4_roe2_sa_blade_response_front_H01_SS.png}
    & \includegraphics[width=0.15\textwidth]{DREAM_LS_TSM_N4_roe2_sa_blade_response_front_H01_PS.png}
    & \includegraphics[width=0.15\textwidth]{DREAM_LS_TSM_N4_roe2_sa_blade_response_rear_H01_PS.png}
    & \includegraphics[width=0.15\textwidth]{DREAM_LS_TSM_N4_roe2_sa_blade_response_rear_H01_SS.png} \\
    \multicolumn{2}{c}{\emph{Front rotor blade}}
    & \multicolumn{2}{c}{\emph{Rear rotor blade}} \\
    suction side & pressure side & pressure side & suction side
 \end{tabular}
 \caption{Low-speed isolated configuration: harmonic response of the front
 rotor blades.}
 \label{fig:dream_ls_hb_blade_response}
\end{figure}

On the front rotor blade, the level is large as it is
close to $0.1\%$ of the inflow static pressure.
Moreover, the pressure side
exhibits a larger level of unsteadiness compared to the
suction side. This is due to the relative position of the
blades which makes the pressure side more vulnerable to
potential effects. In fact, as can be seen on radial cuts
(as for instance in Figure~\ref{fig:dream_ls_hb_slice_r_conv})
the suction side is shield from the flow
field coming from downstream. The intensity is not
uniform along span with a relatively higher amplitude of
unsteadiness at the tip of the blade and near the hub
on the pressure side. On the suction side, the largest level
of unsteadiness is observed near the hub.

On the rear rotor blade, the level
of unsteadiness is much larger than the one observed on
the front rotor blade. 
Here, the level of unsteadiness
goes up to 1\% of the inflow static pressure.
This is mostly due to the wake passing
shed by the front rotor blades. In fact, on the leading
edge of the suction side of the rear rotor blade, 
a strong harmonic response is observed, while it is 
much smaller on the pressure side. Bearing in mind that 
the suction side sees the wake passing, one can deduce
that this strong harmonic response is attributed to wake passing.
In addition to that, in the tip region of the rear rotor blade, 
a stronger level of unsteadiness is observed. As mentioned
previously, tip vortices are shed by the front rotor blades.
Even though the rear rotor blades are clipped, as 
the stream tube contracts, there is a chance that
the front rotor blades tip vortices interact with the 
rear rotor blades. This is investigated by analyzing
axial cuts of entropy.

\subsection{Two-dimensional results: axial cuts}
\label{sub:dream_ls_hb_axial_cuts}

Axial cuts of entropy at four planes ($P3$, $P4$, $P5$ and $P6$)
are shown in Figure~\ref{fig:dream_ls_hb_axial_cut_entropy}.
The steady results are also reported for comparison.
Compared to a steady computation, the harmonic balance
approach allows to capture the impact of the front rotor
tip vortices on the rear rotor. Between plane $P3$
and $P4$, they have been diffused thanks to the viscosity effects.
The interaction of the front and the rear rotor tip vortices
is highlighted in the $P5$ plane. At the end, in plane $P6$
the vortices have almost merged and a large entropy
trace remains. This confirms the impact
of the front rotor tip vortices on the
rear rotor blades, which explains the large static pressure
fluctuations observed in the tip of the rear 
rotor blades. Nevertheless, for this
configuration, the steady mixing plane approach 
provides good results when comparing the axial
cuts downstream the rear rotor ($P6$).

\begin{figure}[htp]
 \ra{1.3} \centering
 \begin{tabular}{rcc}
   & steady
   & HB $N=4$ \\
   \rotatebox{90}{\qquad\qquad\qquad $P3$} & \includegraphics[width=.35\textwidth]{DREAM_LS_RANS_roe2_sa_slice_x_front_1_entropy.png}
   & \includegraphics[width=.35\textwidth]{DREAM_LS_TSM_N4_roe2_sa_slice_x_front_1_entropy.png} \\
   \rotatebox{90}{\qquad\qquad\qquad $P4$} & \includegraphics[width=.35\textwidth]{DREAM_LS_RANS_roe2_sa_slice_x_rear_-1_entropy.png}
   & \includegraphics[width=.35\textwidth]{DREAM_LS_TSM_N4_roe2_sa_slice_x_rear_0_entropy.png} \\
   \rotatebox{90}{\qquad\qquad\qquad $P5$} & \includegraphics[width=.35\textwidth]{DREAM_LS_RANS_roe2_sa_slice_x_rear_1_entropy.png}
   & \includegraphics[width=.35\textwidth]{DREAM_LS_TSM_N4_roe2_sa_slice_x_rear_1_entropy.png} \\
   \rotatebox{90}{\qquad\qquad\qquad $P6$} & \includegraphics[width=.35\textwidth]{DREAM_LS_RANS_roe2_sa_slice_x_rear_2_entropy.png}
   & \includegraphics[width=.35\textwidth]{DREAM_LS_TSM_N4_roe2_sa_slice_x_rear_2_entropy.png} \\
 \end{tabular}
 \caption{Low-speed isolated configuration: axial cuts of entropy.}
 \label{fig:dream_ls_hb_axial_cut_entropy}
\end{figure}

\subsection{Two-dimensional results: radial cut of harmonic pressure}
\label{sub:dream_ls_hb_radial_cuts}

To further analyze the unsteadinesses that are seen
in a CROR configuration, a radial cut at 75\% of the
rear rotor height of the first harmonic
of the static pressure is show in 
Figure~\ref{fig:dream_ls_hb_radial_cuts}.
The two rotors rotating in opposite direction, a steady
field for the former is seen as unsteady by the latter and \emph{vice-versa}.
Therefore the flow field is, by nature, discontinuous at the rows interface.
\begin{figure}[htp]
  \centering
  \includegraphics*[width=0.45\textwidth]{DREAM_LS_TSM_N4_roe2_sa_slice_r_70_ps.png}
  \caption{Low-speed isolated configuration: radial cut of the first harmonic of the
  static pressure normalized by the inflow static pressure.}
  \label{fig:dream_ls_hb_radial_cuts}
\end{figure}

On the front rotor side of the interface, a high amplitude azimuthal 
pattern of static pressure is seen. It is representative
of the potential effects: the blades deviates the stream lines and
as the two rotors rotate in opposite directions, theses deviations
are finally seen as unsteady flow features by the front rotor.

On the rear rotor side of the interface, no azimuthal pattern 
of static pressure is observed near the interface.
Conversely, the pressure unsteadinesses are mostly observed near the rear rotor blades.
These pressure unsteadinesses come actually from the unsteady wake passing.
In fact, the absolute velocity deficit in the wakes shed by the front rotor
is seen as unsteady by the rear rotor. However, at the blade walls,
the velocity is necessary null. These 
velocity fluctuations are actually seen as pressure fluctuations since the presence
of the blades transforms velocity into pressure at blade walls.
Figure~\ref{fig:dream_ls_hb_radial_cuts_machrel}
supports this argument as the amplitude of the Mach number fluctuations
vanishes in regions where the pressure fluctuations grows near the
rear rotor blades.
\begin{figure}[htp]
  \centering
  \includegraphics*[width=0.45\textwidth]{DREAM_LS_TSM_N4_roe2_sa_slice_r_70_machrel.png}
  \caption{Low-speed isolated configuration: radial cut of the first harmonic of the
  relative Mach number.}
  \label{fig:dream_ls_hb_radial_cuts_machrel}
\end{figure}



\section{Aeroelastic results}
\label{sec:dream_ls_ael_results}
%!TEX root = ../../../adrien_gomar_phd.tex

\subsection{Numerical setup}
\label{sub:dream_ls_ael_numerical}

Two structural modes are considered for the aeroelastic study of this 
configuration: the second bending/flexion mode and the first torsion mode
of the front rotor.
The shape of the modes is shown in Fig.~\ref{fig:dream_ls_ael_modes}
with an arbitrary amplitude that allows the visualization.
Two inflection lines are seen for the 2F mode, while only
one is seen for the 1T, hence its designation.
\begin{figure}[htp]
  \centering
  \subfigure[2F]{\includegraphics[height=.35\textwidth]{mode_2F.png}}
  \subfigure[1T]{\includegraphics[height=.35\textwidth]{mode_1T.png}}
  \caption{Low-speed isolated configuration: structural modes considered.}
  \label{fig:dream_ls_ael_modes}
\end{figure}

Four nodal diameters are considered, corresponding to IBPA
values: $[-60^\circ, -30^\circ, 30^\circ, 60^\circ]$. The frequency
considered for theses modes are not correlated with the blade
passing frequency of the rear rotor. Therefore, to simulate such
a configuration using the harmonic balance approach, the
multi-frequential framework should be adopted. The ratio of the
blade passing frequency $f_{BPF}$ of rear rotor (the one that is
seen by the front rotor, see Sec.~\ref{sec:cror_unsteady})
over the frequency of the mode considered is reported in
Tab.~\ref{tab:dream_ls_ael_freq_bpf}. 

\begin{table}[htp]
  \ra{1.3} \centering
  \begin{tabular}{r|cc}
    \toprule
    &  mode 2F (IBPA $30^\circ$/$60^\circ$) & mode 1T (IBPA $30^\circ$/$60^\circ$) \\
    \midrule
    $f_{BPF} / f_{AEL}$ & 3.87/3.85 & 3.20/3.17 \\ 
    $\kappa (E)$, EVE~2N+1 & 31.3/29.2 & 1.68/1.84 \\
    $\kappa (E)$, OPT & 1.01/1.07 & 1.02/1.08 \\
    \bottomrule
  \end{tabular}
  \caption{Low-speed isolated configuration:}
  \label{tab:dream_ls_ael_freq_bpf}
\end{table} 

not correlated to the blade
passing frequency of the configuration. The multi-frequential
formulation of the harmonic balance approach is used 
(see Appendix~\ref{app:elsa}). 

\chconclu{The multi-frequential harmonic balance
approach along with the decoupled approach has
been used in this chapter to simulate the flutter
behavior of a low-speed CROR computation. It is shown
that the local excitation varies in correlation
with the inflection lines of the modes and with 
a change in aerodynamic behavior. This configuration is
shown to be cleared from flutter as the damping
is positive for all modes and all inter-blade phase angles.
To further assess the approach, a more demanding 
configuration is studied, namely a high-speed
CROR.}
