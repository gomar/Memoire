%!TEX root = ../../../adrien_gomar_phd.tex


\begin{figure}[htp]
  \centering
  \includegraphics[width=.3\textwidth]{DREAM_LS_wall.png}
  \caption{Low-speed isolated contra-rotating open rotor geometry.}
  \label{fig:dream_ls_wall}
\end{figure}

The studied configuration is a pusher contra-rotating open rotor
that comes from the know-how of SAFRAN-Snecma 
shown in Fig.~\ref{fig:dream_ls_wall} for the
Low-speed (LS) flight condition, representative of the take-off.
The simulated configuration does not include the spinner as the
experimental setup does not take into account this part of the geometry.
Sadly, the experimental results were not available for comparison
at the time this PhD thesis was written.

\begin{table}[htp]
  \ra{1.3} \centering
  \begin{tabular}{cccc}
    \toprule
    $M_0$ & $|\Omega|$ & $J$ & $M_{tip}$ \\
    \midrule
    $0.2$ & $5739$ tr.min\textsuperscript{-1} & 1.06 & 0.63 \\
    \bottomrule
  \end{tabular}
  \caption{Low-speed isolated contra-rotating open rotor flight condition parameters.}
  \label{tab:dream_ls_flight_condition}
\end{table} 
Table~\ref{tab:dream_ls_flight_condition} recalls the main
parameters of the case: the inflow Mach number $M_0$,
the absolute value of the rotation speed of both rotors $|\Omega|$,
the advance ratio $J$ (whose definition is given in Chap.~\ref{cha:cror})
and the Mach number at the tip of
the front rotor blades based on the inflow velocity and the advance ratio:
\begin{equation}
	M_{tip} = M_0 \sqrt{1 + \left(\frac{\pi}{J} \right)^2}
  \label{eq:m_tip}
\end{equation}
Equation~\ref{eq:m_tip} is actually a simple transcription of the velocity triangle
applied to the infinite velocity and to the rotation speed perceived
at the front blade tip:
\begin{equation}
    M_{tip} = M_0 \sqrt{1 + \left(\frac{\pi}{J}\right)^2} = 
    \frac{V_0}{\sqrt{\gamma R t_0}} \sqrt{1 + \left(\frac{
    	\cancel{\pi} \cdot \Omega D}{
    	2 \cancel{\pi} \cdot V_0}\right)^2} =
    \sqrt{\frac{V_0^2 + (\Omega R)^2}{\gamma R t_0}}
    \label{eq:m_tip_explained}
\end{equation}

The inflow Mach number $M_0$ is within the incompressible range
($M_0 < 0.3$). As the CFD flow solver used here is \emph{elsA}
(see Appendix~\ref{app:elsa}) which is a compressible code, 
a preconditionner might be needed for the computations to converge. 
Hopefully, the fluid is accelerated by the two rotors
and the tip Mach number is high enough to not use any preconditionning.
However, let us bare in mind that this range of Mach number might
be tedious for a compressible flow solver.
The advance ratio $J$ is around~1 which is a classical value for
low-speed propellers~\cite{Bousquet2012}. Note that the rotation speed is almost
one order of magnitude larger than expected. 
In fact, a full-scale CROR diameter is around 5~meters. With this rotation speed,
this would lead to a tip Mach number greater than~4. This high rotation speed is 
actually here to compensated for the small radius of the blades, which is~33~cm
for experimental purposes, while maintaining the same similarity coefficients.
