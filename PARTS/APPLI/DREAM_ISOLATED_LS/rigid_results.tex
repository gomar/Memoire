%!TEX root = ../../../adrien_gomar_phd.tex

\subsection{Spectral convergence of the harmonic balance computations} 
\label{sub:dream_ls_hb_convergence}

\paragraph{Using the prediction tool}
\label{par:dream_ls_conv_hb_prediction_tool}

The prediction tool developed in 
Sec.~\ref{sub:prediction_tool_azimuthal_fft} is applied
to the studied configuration to evaluate the
required number of harmonics needed for the
harmonic balance approach to be converged.
The result is shown in Fig.~\ref{fig:DREAM_LS_RANS_ROE2_SPECTRUM_PPT}.
To capture 99\% of the energy, four harmonics are needed, even though
at a relative span superior to 60\%, only three harmonics would be needed.
However, as the implementation of the harmonic balance used
in this work does not allow a varying number of harmonics through the
configuration, four harmonics is supposed to be sufficient to efficiently 
represent the unsteady flow field.
\begin{figure}[htp]
  \centering
  \includegraphics*[width=0.5\textwidth]{DREAM_LS_RANS_ROE2_SPECTRUM_PPT.pdf}
  \caption{Low-speed isolated configuration: prediction of the number
  of harmonics needed to simulate the configuration.}
  \label{fig:DREAM_LS_RANS_ROE2_SPECTRUM_PPT}
\end{figure}

\paragraph{Using the similarity coefficients}
\label{par:dream_ls_conv_hb_sim_coeff}

Two harmonics are actually necessary to converge the temporal mean 
of the similarity coefficients as show 
in Tab.~\ref{tab:dream_ls_hb_conv_sim}. A slight evolution of the
coefficients is still seen but represents a change inferior to 0.01\%
of the $N=4$ results, hence the convergence. 
\begin{table}
  \ra{1.3} \centering
  \begin{tabular}{rccccc}
    \toprule
    & steady & HB $N=1$ & HB $N=2$ & HB $N=3$ & HB $N=4$ \\
    \midrule
    $C_T$  & $1.1319$ & $1.1334$ & $1.1330$ & $1.1330$ & $1.1329$ \\
    $C_P$  & $2.0927$ & $2.0951$ & $2.0944$ & $2.0945$ & $2.0946$ \\
    $\eta$ & $0.5726$ & $0.5727$ & $0.5726$ & $0.5726$ & $0.5725$ \\
    \bottomrule
  \end{tabular}
  \caption{Low-speed isolated configuration: analysis of the number of harmonics
  required to capture the mean similarity coefficients.}
  \label{tab:dream_ls_hb_conv_sim}
\end{table}
Note that a steady computation is sufficient to retrieve
the temporal mean value of the similarity coefficients.
In fact, a maximum of 0.1\% difference is observed by
comparing the results obtained with the harmonic balance and the
mixing plane approach.

\paragraph{Using the blade response}
\label{par:dream_ls_conv_hb_blade_response}

\subsection{Integrated results: similarity coefficients}
\label{sub:dream_ls_hb_sim_coeff}

\subsection{One-dimensional results: radial profiles}
\label{sub:dream_ls_hb_radial_profiles}

\subsection{Two-dimensional results: radial and axial cuts}
\label{sub:dream_ls_hb_axial_radial_cuts}
