%!TEX root = ../../../adrien_gomar_phd.tex

\subsection{Spectral convergence of the harmonic balance computations} 
\label{sub:dream_ls_hb_convergence}

The prediction tool developed in 
Sec.~\ref{sub:prediction_tool_azimuthal_fft} is applied
to the studied configuration to evaluate the
required number of harmonics needed for the
harmonic balance approach to be converged.
The result is shown in Fig.~\ref{fig:DREAM_LS_RANS_ROE2_SPECTRUM_PPT}.
To capture 99\% of the energy, four harmonics are needed, even though
at a relative span superior to 60\%, only three harmonics would be needed.
However, as the implementation of the harmonic balance used
in this work does not allow a varying number of harmonics through the
configuration, four harmonics is the supposed converged value.
\begin{figure}[htbp]
  \centering
  \includegraphics*[width=0.5\textwidth]{DREAM_LS_RANS_ROE2_SPECTRUM_PPT.pdf}
  \caption{caption}
  \label{fig:DREAM_LS_RANS_ROE2_SPECTRUM_PPT}
\end{figure}

The computations have been run in different manners. First the steady computation
is run
\begin{table}[htb]
  \ra{1.3} \centering
  \begin{tabular}{rcccccccc}
    \toprule
    & steady & HB $N=1$ & HB $N=2$ & HB $N=3$ & HB $N=4$ & HB $N=5$ & HB $N=6$ & HB $N=7$\\
    \midrule
    $C_T$  & $1.4301$ &  \\
    $C_P$  & $2.8554$ &  \\
    $\eta$ & $0.5659$ &  \\
    \bottomrule
  \end{tabular}
  \caption{}
  \label{tab:}
\end{table}



