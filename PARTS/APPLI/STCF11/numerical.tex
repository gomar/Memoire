%!TEX root = ../../../adrien_gomar_phd.tex

% mesh presentation
The blade passage is meshed using an O4H topology (Fig.~\ref{fig:stcf11_mesh}).  
The number of grid points along the blade
chord axis is~160 and the computed $y^+$ at the walls is $\mathcal{O}(1)$. % proving
The blade has the same profile along the spanwise direction and no
twist. Therefore, a 2.5D mesh is used with five points in the radial direction, with a spanwise
extent representing $1\%$ of the chord. 

% boundary conditions
The boundary conditions used for this case are: (i)~an
injection condition  for the inlet (with a relative flow angle
set to the  experimental value), (ii)~a constant static pressure
condition for the outlet,  (iii)~an adiabatic no-slip condition on
blade walls, and (iv)~periodic or phase-lagged conditions for azimuthal boundaries depending on the  
prescribed IBPA.

% numerical parameters
Turbulence is modeled using the one-equation model of
Spalart-Allmaras~\cite{Spalart1992}.  The third-order upwind Roe
scheme~\cite{Roe1981} is used to compute the convective fluxes.
The maximum
CFL number is set to~20 for the steady computations,  the inner loop
of the DTS scheme and the HB simulations.  For the DTS scheme,  
convergence in time discretization is obtained
after 20~periods using 128~instants per period.  Iterative convergence 
for the inner loop is considered achieved when the normalized
residuals drop by $5\cdot 10^{-2}$ (within a maximum of
50~sub-iterations).