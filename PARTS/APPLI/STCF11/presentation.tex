%!TEX root = ../../../adrien_gomar_phd.tex

For external-flow aeroelasticity, the HB approach has 
been thoroughly validated by \citet{Gopinath2005, Woodgate2009} and \citet{JDufour2009}, 
mostly for the AGARD test cases of \citet{Davis1982}.
Experimental data for turbomachinery aeroelasticity are more scarce: 
the STandard aeroelastic ConFigurations (STCF) experiments 
of \citet{Fransson1999} are the 
reference in this respect, and have been widely used 
to validate different numerical approaches by \citet{Sbardella2001,
Duta2002,Campobasso2003,Cinnella2004} and \citet{Huang2013a} whose
uses a similar harmonic balance approach as the one proposed in
this work.
The experiments
are composed of 11~turbomachinery configurations that have been
thoroughly investigated experimentally in an 
annular test rig at \'Ecole 
Polytechnique F\'ed\'erale de Lausanne.

In particular, the 11\textsuperscript{th} standard configuration is a
turbine stator composed of 20~blades, and tested
in the late 1990's by \citet{Fransson1999}.
The experimental results have been found to be highly reproducible and
therefore suitable for code validation.  Moreover,
two flow regimes are considered, one subsonic and one transonic.
In this respect, the transonic case allows to distinguish
solvers able to capture non-linear unsteady effects. This is why this particular
case is used here, since HB methods are meant to capture non-linear unsteady
features. However, it must be pointed out that LUR approaches have
been validated using the transonic case and show fair agreement with experimental
data~\cite{Sbardella2001, Duta2002,Campobasso2003}.

% geometry presentation
The geometry profile and the results are available over the
internet~\cite{stcf11web}. 
To characterize the two flow regimes, measurements of static and total pressures 
as well as flow angles are done in two planes $e_0$ located $0.3$ axial chord upstream 
of the turbine
blade and $e_1$ located $0.6$ axial chord downstream
as shown in Figure~\ref{fig:stcf11_measurements}.
\begin{figure}[htp]
  \centering
  \includegraphics*[width=0.40\textwidth]{stcf11_measurements.pdf}
  \caption{Position of the measurement planes in the STCF~11 configuration.}
  \label{fig:stcf11_measurements}
\end{figure}
The results are given in terms of
inlet Mach number $M_0$, inlet total pressure $p_{i_0}$, 
inlet flow angle $\beta_0$, outlet isentropic
Mach number $M_{1_{is}}$ and outlet static pressure $p_{s_1}$. 
The isentropic Mach number is the Mach number 
computed if the stagnation pressure was taken constant (without loss)
\begin{equation}
    M_{is} = \sqrt{\frac{2}{\gamma -1}
        \left[\left( \frac{p_{i_0}}{p_s} \right)^{\frac{\gamma - 1}{\gamma}}  
        - 1 \right]},
\end{equation}
where $p_{i_0}$ is the constant total pressure and $p_s$
the local static pressure.
It is actually
one way to interpret the static pressure as a velocity.
The experimental results measured at plane $e_0$
and $e_1$ are given in Tab.~\ref{tab:stcf11_steady_results}. 
These will be used later on to set the boundary conditions 
of the CFD computations. To allow local validation of the steady
flow, the experimental results of 
isentropic Mach number are given at blade wall.
\begin{table}[htp]
  \ra{1.3} \centering
  \begin{tabular}{lccccc}
    \toprule
    \phantom{abdefghijk}& $M_0~[-]$ & $p_{i_0}~[\text{Pa}]$ & $\beta_0~[\circ]$ & $M_{1_{is}}~[-]$ & $p_{s_1}~[\text{Pa}]$ \\
    \midrule
    Subsonic & $0.31$ & $124,600$ & $15.2$ & $0.69$  & $90,700$ \\
    Transonic & $0.4$ & $229,800$ & $34$    & $0.99$ & $122,400$ \\
    \bottomrule
  \end{tabular}
  \caption{Steady experimental results for the STCF~11 configuration.}
  \label{tab:stcf11_steady_results}
\end{table} 

For aeroelastic investigations, the blades oscillate harmonically in the first bending mode
at a reduced frequency of $f_{c} =\pi c f/U_{outlet, exp} = 0.2134$ 
for the subsonic case and $0.1549$ for the
transonic case. Aeroelastic
results are available such as the first harmonic of the unsteady pressure
coefficient at blade walls (amplitude and phase), for several nodal
diameters. These are measured using piezo-resistive pressure transducers.

The damping is evaluated at blade walls through through the
expression given by~\citet{Fransson1999}
\begin{equation}
    \delta [-] = - \sum^{\#~pts}_{k=0} \frac{c}{h} 
      \frac{|\widehat{p}_k|}{(p_{i_0} - p_{s_0})} S_k \arg (\widehat{p}_k),
\end{equation}
where $c$ is the chord length,
$h$ the bending amplitude, $| \widehat{p} |$ 
and $\arg (\widehat{p})$ are the modulus and the phase of the
complex first harmonic of static pressure, respectively, $S$ the surface
and $k$ denotes the k\textsuperscript{th}
grid point at blade walls.
The damping strongly varies under small changes in the
local distribution. It is therefore recommended to look at the local
distributions. No experimental damping curves are given. In fact,
there is too few measurement points to integrate the results with
confidence. However, \citet{Fransson1999}
provide numerical results of the damping curve obtained with a potential code.

