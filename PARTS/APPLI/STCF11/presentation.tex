%!TEX root = ../../../adrien_gomar_phd.tex

Experimental data for turbomachinery aeroelasticity are more scarce: 
the Standard Aeroelastic Configurations experiments 
of \citet{Fransson1999} are the 
reference in this respect, and have been widely used 
to validate different numerical approaches by \citet{Sbardella2001,
Duta2002,Campobasso2003} and \citet{Cinnella2004}. 

The $11$\textsuperscript{th} standard configuration is a
turbine stator composed of $20$~blades, and tested at EPF-Lausanne
in the late $1990$'s by \citet{Fransson1999}.
The test cascade is placed in a 
annular test rig as depicted in Fig.\ref{fig:annular_channel}.
\begin{figure}[htbp]
  \centering
  \includegraphics*[width=0.40\textwidth]{./ANNULAR_CHANNEL.pdf}
  \caption{Annular test rig for the standard configurations}
  \label{fig:annular_channel}
\end{figure}
The experimental results have been found to be highly reproducible and
therefore suitable for code validation~\cite{Fransson1999}.  

% geometry presentation
The geometry profile and the results are available over the
internet~\cite{stcf11web}.  To allow local validation of the steady
flow, the isentropic Mach number is given at blade wall.

The blades oscillate harmonically in the first bending mode
at a reduced frequency of $f_{c} =\pi \cdot c \cdot
f/U_{outlet, exp} = 0.2134$ for the subsonic case and $0.1549$ for the
transonic case. Aeroelastic
results are available such as the first harmonic of the unsteady pressure
coefficient at blade walls (amplitude and phase), for several nodal
diameters. The integrated
results, such as the damping, strongly vary under small changes in the
local distribution. It is therefore recommended to look at the local
distributions.