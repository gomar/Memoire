%!TEX root = ../../../adrien_gomar_phd.tex

For external-flow aeroelasticity, the HB approach has 
been thoroughly validated~\cite{Gopinath2005, Woodgate2009, JDufour2009}, 
mostly for the AGARD test cases of Davis~\cite{Davis1982}.
Experimental data for turbomachinery aeroelasticity are more scarce: 
the STandard aeroelastic ConFigurations (STCF) experiments 
of \citet{Fransson1999} are the 
reference in this respect, and have been widely used 
to validate different numerical approaches by \citet{Sbardella2001,
Duta2002,Campobasso2003} and \citet{Cinnella2004}. The experiments
are composed of 11~turbomachinery configurations that have been
thoroughly  investigated experimentally at EPF-Lausanne shown in an 
annular test rig in Fig.\ref{fig:annular_channel} .
\begin{figure}[htbp]
  \centering
  \includegraphics*[width=0.40\textwidth]{annular_channel.pdf}
  \caption{Annular test rig for the standard configurations.}
  \label{fig:annular_channel}
\end{figure}

In particular, the 11\textsuperscript{th} standard configuration is a
turbine stator composed of 20~blades, and tested
in the late 1990's by \citet{Fransson1999}.
The experimental results have been found to be highly reproducible and
therefore suitable for code validation~\cite{Fransson1999}.  Moreover,
two inflow conditions are considered, one subsonic and one transonic.
In this respect, the transonic case allows to distinguish
solvers able to capture non-linear unsteady effects. This is why this particular
case is used here, since HB methods are meant to capture non-linear unsteady
features.

% geometry presentation
The geometry profile and the results are available over the
internet~\cite{stcf11web}.
As said above, two inflow conditions are available, one subsonic and
one transonic. 
To characterize these two operating points, measurements of static and total pressures 
as well as flow angles are done in two plane $e_1$ located upstream 
($e_1 = 0.018~\text{m}$) of the turbine
blade and $e_2$ located downstream ($e_2 = 0.035~\text{m}$) as shown in Fig.~\ref{fig:stcf11_measurements}.
\begin{figure}[htbp]
  \centering
  \includegraphics*[width=0.40\textwidth]{stcf11_measurements.pdf}
  \caption{Position of the measurement planes in the STCF~11 configuration}
  \label{fig:stcf11_measurements}
\end{figure}
The results are given in terms of
inlet Mach number $M_1$, inlet total pressure $P_{i_1}$, 
inlet flow angle $\beta_1$, outlet isentropic
Mach number $M_{2_{is}}$ and outlet static pressure $P_{s_2}$. 
The isentropic Mach number is the Mach number 
computed if the stagnation pressure would stay constant. It is actually
one way to interpret the static pressure as a velocity.
The results are given in Tab.~\ref{tab:stcf11_steady_results}.
\begin{table}[htbp]
  \ra{1.3} \centering
  \begin{tabular}{lccccc}
    \toprule
    \phantom{abdefghijk}& $M_1~[-]$ & $P_{i_1}~[\text{Pa}]$ & $\beta_1~[\circ]$ & $M_{2_{is}}~[-]$ & $P_{s_2}~[\text{Pa}]$ \\
    \midrule
    Subsonic & $0.31$ & $124,600$ & $15.2$ & $0.69$  & $90,700$ \\
    Transonic & $0.4$ & $229,800$ & $34$    & $0.99$ & $122,400$ \\
    \bottomrule
  \end{tabular}
  \caption{Steady experimental results for the STCF~11 configuration.}
  \label{tab:stcf11_steady_results}
\end{table} 

To allow local validation of the steady
flow, the isentropic Mach number is given at blade wall. These
are measured using pressure taps.

For aeroelastic investigations, the blades oscillate harmonically in the first bending mode
at a reduced frequency of $f_{c} =\pi \cdot c \cdot
f/U_{outlet, exp} = 0.2134$ for the subsonic case and $0.1549$ for the
transonic case. Aeroelastic
results are available such as the first harmonic of the unsteady pressure
coefficient at blade walls (amplitude and phase), for several nodal
diameters. These are measured using piezo-resistive pressure transducers.
The integrated
results, such as the damping, strongly vary under small changes in the
local distribution. It is therefore recommended to look at the local
distributions.
