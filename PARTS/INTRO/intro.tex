%!TEX root = ../../adrien_gomar_phd.tex

\chapter{Introduction}

Global warming could be one of the biggest challenge human kind
may have to face in the forthcoming decades if not years.
In fact, according to the last scientific
report of the Intergovernmental Panel on Climate Change 
(IPCC)~\cite{IPCC2013}
"Warming of the climate system is unequivocal, 
and since the 1950s, many of the observed 
changes are unprecedented over decades to millennia".
"It is extremely likely that human influence has 
been the dominant cause of the 
observed warming since the mid-20\textsuperscript{th} 
century".
"The largest contribution to total radiative 
forcing is caused by the increase in the atmospheric 
concentration of $CO_2$ since 1750".

A part of these emissions comes from the
transport industry and in particular the
aeronautical industry. 
For that reason, the European commission has set
demanding objectives for 2050 on the pollutant emissions
through the
Advisory Council for 
Aeronautics Research in Europe (ACARE):
the noise, $CO_2$ and $NO_x$ emissions should be reduced by 
$65\%$, $75\%$ and $80\%$ respectively
(Figure~\ref{fig:flightpath_2050}).
\begin{figure}[htp]
  \centering
  \includegraphics*[width=0.40\textwidth]{flightpath_2050.pdf}
  \caption{European Commission goals for the aeronautical industry.}
  \label{fig:flightpath_2050}
\end{figure}

Hopefully for the earth,
the rarefaction of crude oil helps the decision making.
In particular, in the seventies, the two oil crisis showed the aeronautical 
industries its dependence toward energy resources. 
To face this issue, the U.S. Senate directed NASA in 1975
to look for every potential fuel-saving concept for aircraft
engines. The Advanced Turboprop
project was born~\cite{Hager1988} and led to the
concept of contra-rotating open rotor. This new
engine concept showed large fuel savings
along with higher noise emissions due to the absence of
a duct. Combined with the decrease of the price of the
barrel in the late eighties, the contra-rotating open rotor 
never reached the commercial aviation.

Today, the cost of a barrel is almost at its maximum as shown
in Fig.~\ref{fig:crude_oil_price}.
\begin{figure}[htp]
  \centering
  \includegraphics*[width=0.40\textwidth]{crude_oil_price.pdf}
  \caption{Evolution of the cost of a barel from $1861$ to $2012$, from BP~\cite{bpreview2013}}
  \label{fig:crude_oil_price}
\end{figure}
In parallel, Airbus forecast a doubled number of passengers in
$2031$. To allow a sustainable air transportation, new
concepts are needed for both the engines and the 
aircraft in general.
Several concepts have emerged such as: lightweight construction
with advanced composite structure, airport collaborative decision
making with continuous climb departure and less waiting in taxi,
aerodynamically optimized wing geometries as laminar wings for instance
and finally, full efficient engines to name but a few.
Three new type of engines are currently studied: the
Ultra-High ByPass Ratio (HBPR) engine that is based on a
large diameter engine, improving thus the
propulsive efficiency and the Contra-Rotating Open Rotor (CROR)
engine that relies on two rows of contra-rotating propellers
that proves its efficiency during experiments in the
eighties.

The CROR is the prior interest of this PhD thesis.
Several challenges, such as aerodynamic,
aeroacoustic and aeroelasticity are still open for CROR
to become a viable engine for the next generation aircraft.
As mentioned previously, noise emissions on such an architecture
have justified the abandon of this concept in the late eighties.
In the last five years, several research teams have tackle
the issue of noise on CROR. In this work, we propose to 
investigate a less studied challenge: the aeroelasticity.

 








% fil rouge: "there is always a need for developing efficent methods 
%   for applications to blade designs. In a design cycle, 
%   a large number of flow solutions are sought to interact 
%   iterativeley or concurrently with various options 
%   opportunities and constraints from other disciplines"
  
% mettre en place les outils numerique pour faire AEL des CROR
%   notions de base pour comprendre: 1) perfo des crors, 2) AEL 
%   3) méthodes numeriques









\section*{Outline of this work}
\label{sec:outline_of_this_work}

\subsection*{Toward harmonic balance computations}
\label{sub:toward_harmonic_balance_computations}

Several studies have been performed at CERFACS on the use
of the harmonic balance approach for CROR configurations 
before the current work.
\citet{Yabili2010} first evaluate the harmonic balance approach
on a radial slice of a CROR configuration and showed that an
important number of harmonics (more than~10) was required,
which is consistent with the results provided in 
Chap.~\ref{cha:limitations_convergence}.
\citet{ThesisFrancois} made his PhD on the assessment of
harmonic balance approach a one blade-passage 3D CROR configuration.
The aim of the present PhD thesis is to analyze ...



