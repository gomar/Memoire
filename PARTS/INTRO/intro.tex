%!TEX root = ../../adrien_gomar_phd.tex

\chapter*{Introduction}

Global warming is one of the biggest issue that human being
has to face in the forthcoming centuries. 










In the seventies, the two oil crisis showed the aeronautical 
industries its dependence toward energy resources. In fact,
airplanes are driven by engines that rely on  


To face this issue, the U.S. Senate directed NASA in $1975$
to look for every potential fuel-saving concept. The Advanced Turboprop
project was born~\cite{Hager1988}. The reflection led to the
concept of contra-rotating open rotor.

In an airplane engine, 
the thrust $F_x$ and the propulsive efficiency $\eta_{pr}$ are given by:
\begin{equation}
	\begin{split}
		F_x &= Q \Delta V, \\
		\eta_{pr} &= \displaystyle \frac{1}{1 + \displaystyle \frac{\Delta V}{2 V_0}},
	\end{split}
\end{equation}
where $Q$ is the mass-flow, $\Delta V$ is the inlet/outlet velocity difference and $V_0$
the inlet velocity. These equations means that the 
smaller the velocity difference $\Delta V$, the higher the propulsive efficiency.
Then, to increase the thrust $F_x$ which is the reason being of an engine,
the only way is to work on the mass-flow $Q$.
The current

In this way, a propeller is more efficient than a jet. In fact, a jet 
moves small mass of gas at high velocity while propellers moves 
large mass of air at low velocity, increasing the propulsive efficiency.

Only ten years later, the cost of 
the barrel decreased for twenty years to almost retrieve its original
value. 

In the beginning of the eighties, the cost of 
the barrel decreased for almost twenty years and then increased again.
Today, the cost of a barrel is almost at its maximum as shown
in Fig.~\ref{fig:crude_oil_price}.
\begin{figure}[htbp]
  \centering
  \includegraphics*[width=0.40\textwidth]{crude_oil_price.pdf}
  \caption{Evolution of the cost of a barel from $1861$ to $2012$ \cite{bpreview2013}}
  \label{fig:crude_oil_price}
\end{figure}

Global air traffic is expected to increase significantly in the next 20 years, 
by approximately 5\% per year for passenger flights and 6\% 
per year for freight (Airbus [4], Boeing [33]) : (thèse vincent blandeau)


Nevertheless, Airbus forecast a doubled number of passengers in
$2031$. To allow a sustainable air transportation, the aeronautical
industry should reduce its environmental footprint. For instance,
the European Commission recently published the \emph{Flightpath $2050$}.
In this document, the aeronautical industries are set goals for $2050$
shown in Fig.~\ref{fig:flightpath_2050}.
\begin{figure}[htbp]
  \centering
  \includegraphics*[width=0.40\textwidth]{flightpath_2050.pdf}
  \caption{European Commission goals for the aeronautical industry. }
  \label{fig:flightpath_2050}
\end{figure}
The noise, CO2 and NOx emissions should be reduced of 
respectively $65\%$, $75\%$ and $80\%$.
These are ambitious goals, needing technological breakthrough.

The main source of pollutant emission is the engine. Two ways of reducing
these are through a better control of the combustion and better
efficiency of the device in global.

How to increase the mass-flow ??
HBR or propeller