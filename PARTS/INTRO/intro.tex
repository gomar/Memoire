%!TEX root = ../../adrien_gomar_phd.tex

\chapter{Introduction}

Global warming could be one of the biggest challenge human kind
may have to face in the forthcoming decades if not years.
In fact, according to the last scientific
report of the Intergovernmental Panel on Climate Change 
(IPCC)~\cite{IPCC2013}
"Warming of the climate system is unequivocal, 
and since the 1950s, many of the observed 
changes are unprecedented over decades to millennia".
"It is extremely likely that human influence has 
been the dominant cause of the 
observed warming since the mid-20\textsuperscript{th} 
century".
"The largest contribution to total radiative 
forcing is caused by the increase in the atmospheric 
concentration of $CO_2$ since 1750".

A part of these pollutant emissions comes from the
transport industry and in particular the
aeronautical industry. 
For that reason, the European commission has set
demanding objectives for 2050 on the pollutant emissions
through the
Advisory Council for 
Aeronautics Research in Europe (ACARE):
the noise, $CO_2$ and $NO_x$ emissions should be reduced by 
$65\%$, $75\%$ and $80\%$ respectively
(Figure~\ref{fig:flightpath_2050}).
\begin{figure}[htp]
  \centering
  \includegraphics*[width=0.40\textwidth]{flightpath_2050.pdf}
  \caption{European Commission goals for the aeronautical industry.}
  \label{fig:flightpath_2050}
\end{figure}

Hopefully for the earth,
the rarefaction of crude oil accelerates the decision making.
In particular, in the seventies, the two oil crisis showed the aeronautical 
industries its dependence toward energy resources. 
To face this issue, the U.S. Senate directed NASA in 1975
to look for every potential fuel-saving concept for aircraft
engines. The Advanced Turboprop
project was born~\cite{Hager1988} and led to the
concept of contra-rotating open rotor. This new
engine concept showed large fuel savings
along with higher noise emissions due to the absence of
a duct. Combined with the decrease of the price of the
barrel in the late eighties, the contra-rotating open rotor 
never reached the commercial aviation.

Today, the cost of a barrel is almost at its maximum as shown
in Fig.~\ref{fig:crude_oil_price}.
\begin{figure}[htp]
  \centering
  \includegraphics*[width=0.40\textwidth]{crude_oil_price.pdf}
  \caption{Evolution of the cost of a barel from $1861$ to $2012$, from BP~\cite{bpreview2013}.}
  \label{fig:crude_oil_price}
\end{figure}
In parallel, Airbus forecast a doubled number of passengers in
$2031$. Therefore, to allow a sustainable air transportation, new
concepts are needed for both the engines and the 
aircraft in general.
Several have emerged among which: lightweight construction
with advanced composite structure, airport collaborative decision
making with continuous climb departure and less waiting in taxi,
aerodynamically optimized wing geometries as laminar wings for instance
and finally, full efficient engines to name but a few.
In this last field, two main type of engines are currently studied: the
High ByPass Ratio (HBPR) engine that is based on a
large diameter engine, improving thus the
propulsive efficiency and the Contra-Rotating Open Rotor (CROR)
engine that relies on two rows of contra-rotating propellers
that proves its efficiency during experiments in the
eighties.


The industrial design of turbomachinery, and by extension contra-rotating
open rotors, is usually based on steady flow analysis, 
for which the reference simulation tool are the three-dimensio\-nal Reynolds-Averaged 
Navier--Stokes (RANS) steady computations. However, this approach finds its limits 
when unsteady phenomena become dominant. This is the case of 
contra-rotating open rotors where the interaction between the
two rotors is of prior importance.
In such a context, engineers now 
need to account for unsteady-flow effects as early as possible in the design 
cycle, which makes efficiency of unsteady computations a key issue. 
A specificity of turbomachinery flows is their periodicity, 
at least as far as the mean field properties are considered. 

Several unsteady approaches to simulate CROR configurations
have been investigated at CERFACS during the
last four years: \citet{Burnazzi2010} evaluated the phase-lag approach
applied to 3D contra-rotating open rotor configuration and showed
that unsteadinesses can be captured using a mesh that only spans one
blade-passage per row. 
Some years ago, \citet{ThesisSicot} implemented a 
Fourier-based time method, namely the Harmonic Balance (HB),
into the \emph{elsA} CFD code that is used at CERFACS. Applied to turbomachinery
configurations, this method showed a computational gain
of one to two orders of magnitude 
compared to classical time-marching approaches.
This is why, in the continuity of the work
of \citet{Burnazzi2010}, \citet{Yabili2010} assessed 
the harmonic balance approach applied to 
a $2.5$D radial slice of a CROR configuration. 
The capability
of the HB approach to efficiently retrieve the unsteadinesses
at a lower cost, compared to classical time-marching approach,
 was not conclusive.
Therefore, \citet{ThesisFrancois}
investigated classical unsteady approaches for
turbomachinery computations applied to CROR simulations
among which the HB approach in his PhD thesis. 
He showed that
the harmonic balance method can retrieve unsteady
flow features for a reduced cost but at a gain that
is relatively smaller compared to what was
obtained on former turbomachinery applications. 
In parallel, \citet{ThesisGuedeney} extended the harmonic
balance approach to a multi-frequential framework. 
This method allows then to compute unsteadinesses whose frequencies
are not harmonically related.

Several challenges, such as aerodynamic,
aeroacoustic and aeroelasticity are still open 
for contra-rotating open rotor
to become a viable engine for the next generation aircraft.
In this PhD thesis, we propose to assess the aeroelasticity of 
contra-rotating open rotor by using the multi-frequential
harmonic balance developed by \citet{ThesisGuedeney}.
In fact, the main unsteadinesses of the flow field
are known to be correlated with the so-called
blade passing frequency. This frequency depends on the
rotation speed of the rotor and the number of blades
of the opposite rotor. In contrast to that, the 
frequency that drives the aeroelasticity of CROR
blades depends on their structural properties.
As such, the frequencies of both the aerodynamic
field and the aeroelasticity are not harmonically
related and justifies the use of the multi-frequential
formulation of the harmonic balance approach.


In this framework of the studies done at CERFACS,
the aim of this PhD thesis is to assess the
multi-frequential harmonic balance
to estimate the flutter properties of contra-rotating open rotor
configurations. In this way, the memoir is divided in three parts:
\begin{itemize}
	\item the first part presents general information on 
	contra-rotating open rotors (Chapter~\ref{cha:cror}),
	the basic equations that govern the aeroelasticity of
	turbomachinery/CROR and the approach retained to simulate it
	(Chapter~\ref{cha:ael}),
	finally the mathematical mechanisms that allow the derivation
	of Fourier-based time methods and their underlying principles
	(Chapter~\ref{cha:spectral_methods}).
	\item the second part presents the advantages and limitations
	of Fourier-based time methods. Three toys problems representative
	of the main phenomena encountered when simulating CROR
	aeroelasticity are first presented (Chapter~\ref{cha:toy_problems}).
	Then, the advantage of using the harmonic balance 
	approach to estimate the temporal derivative is assessed
	(Chapter~\ref{cha:advantages}). It is emphasized
	that a large CPU gain can be expected in the
	case of contra-rotating open rotor aeroelasticity. 
	In opposite, when using the multi-frequential harmonic
	balance, mathematical properties can lead to divergence
	of the computation through a high condition number
	(Chapter~\ref{cha:limitations_condition_number}). This is 
	first highlighted on the toy problems and then solved using
	an original optimization algorithm.
	Finally, the convergence of the harmonic balance 
	that was shown to be case dependent is
	assessed (Chapter~\ref{cha:limitations_convergence}). 
	It is demonstrated that the difference in computational
	gain is linked to the thickness of the wakes observed behind
	contra-rotating open rotor blades. Based on this observation,
	a prediction tool is developed to estimate the
	number of harmonics needed to compute a given turbomachinery (and CROR)
	configuration. The relative CPU gain to be expected can thus be estimated
	and help the decision making in choosing an unsteady approach
	over another one.
	\item based on the work done in the second part,
	the proposed approach is applied on different configurations
	in the third part. Firstly, the approach retained in this thesis, 
	namely the harmonic balance method along with a weak 
	aeroelastic coupling approach, is validated against experimental 
	results and other numerical approaches found in the
	literature on a reference configuration 
	(Chapter~\ref{cha:stcf11}). This give us confidence
	to apply the approach on an industrial isolated contra-rotating
	open rotor application at low-speed (Chapter~\ref{cha:dream_ls_isolated})
	and high-speed (Chapter~\ref{cha:dream_hs_isolated})
	flight conditions. The aeroelastic results and discussed based
	on the computed unsteady flow field. Finally, the applicability
	of the proposed approach is demonstrated to be out of scope
	of the method when dealing with an installed configuration 
	(Chapter~\ref{cha:hera3_installed}).
\end{itemize}
