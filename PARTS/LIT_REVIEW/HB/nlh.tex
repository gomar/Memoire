%!TEX root = ../../../adrien_gomar_phd.tex

Originally developed by \citet{He1998} and \citet{Ning1998},
the NLH method
relies on a decomposition of the conservative variables into a
time-averaged part plus an unsteady perturbation
\begin{equation}
	u = \overline{u} + u^\prime,
	\label{eq:sm_nlh_decomposition}
\end{equation}
where $\overline{\vphantom{u}.}$ denotes the time-averaging operator and
$.^\prime$ its unsteady perturbation counterpart.
By injecting Eq.~\eqref{eq:sm_nlh_decomposition} into
Eq.~\eqref{eq:sm_nonlinear_convection_conservative}, one gets
\begin{equation}
	\frac{\partial u^\prime}{\partial t} + 
	\frac{1}{2}\frac{\partial}{\partial x} \left[
	\overline{u}^2 + 2 \overline{u} u^\prime + u^\prime u^\prime \right] = 
	0.
	\label{eq:sm_nlh_step_1}
\end{equation}
The equation for the time-averaged part can be obtained by time-averaging
Equation~\eqref{eq:sm_nlh_step_1}
\begin{equation}
	\frac{\partial}{\partial x}
	\left[\overline{u}^2 + 
	\overline{u^\prime u^\prime}\right] =
	0,
	\label{eq:sm_nlh_step_2}
\end{equation}
The term $\overline{u^\prime u^\prime}$
accounts for the non-linearities of the considered equations. 
This term reflects the influence of the unsteady contribution to
the time-average, which was neglected in the LUR approach. It
is called the non-linear 
(or the deterministic) stress terms, by analogy with
the Reynolds stress terms. 
The equation for the unsteady perturbation is then obtained by keeping
the first-order terms of the unsteady Eq.~\eqref{eq:sm_nlh_step_1}.
This means that the term $u^\prime u^\prime$ is neglected, yielding
\begin{equation}
	\frac{\partial u^\prime}{\partial t} + 
	\frac{\partial}{\partial x} \left[\overline{u} u^\prime \right] = 
	0.
	\label{eq:sm_nlh_step_3}
\end{equation}
Note that neglecting the high-order terms 
(namely $u^\prime u^\prime$ for the Burger's equation) 
is almost similar to
linearizing the equation. However, in the NLH approach,
the time-averaged $\overline{u^\prime u^\prime}$ 
of $u^\prime u^\prime$ is kept in
Equation~\eqref{eq:sm_nlh_step_2} which accounts for a part of the
non-linearities. Thus, the method is not linear. Equations~\eqref{eq:sm_nlh_step_2} 
and~\eqref{eq:sm_nlh_step_3} 
are simultaneously solved, leading to a two-way coupling.

\subsection{Mono-frequential formulation}
Up to now, no assumption has been made neither on the velocity $u$,
nor on its time-averaged part or unsteady perturbation part.
Assuming now that the velocity perturbation 
is periodic in time with period
$T=2 \pi / \omega$,
the unsteady perturbation can be decomposed into 
a Fourier series
\begin{equation}
	u^\prime = \sum_{\genfrac{}{}{0pt}{}{k=-\infty}{k \neq 0}}^{\infty} 
	\widehat{u}_k e^{i \omega k t},
	\label{eq:sm_nlh_decomposition_pert}
\end{equation}
where the $k=0$ term is omitted as it is accounted for in
the $\overline{u}$ part.
The complex exponentials family forming
an orthogonal basis, we retrieve for all harmonics 
$-\infty \leq k \leq \infty, \; k \neq 0$
\begin{equation}
	i \omega k \widehat{u}_k + 
	\frac{\partial}{\partial x} \left[ \overline{u} \widehat{u}_k\right] =
	0,~\forall k.
	\label{eq:sm_nlh_decomposition_pert_part1}
\end{equation}
Each one of harmonic Eq.~\eqref{eq:sm_nlh_decomposition_pert_part1}
represents now a steady-flow-like equation as no temporal
derivative is present anymore.

The term $\overline{u^\prime u^\prime}$ remains in the time-averaged
Equation~\eqref{eq:sm_nlh_step_2}
and needs to be computed. It can be 
directly worked out when the harmonics are known 
from Eq.~\eqref{eq:sm_nlh_decomposition_pert_part1}
\begin{equation}
	\begin{split}
		u^\prime u^\prime &= 
		\left[
			\sum_{\genfrac{}{}{0pt}{}{k=-\infty}{k \neq 0}}^{\infty} \widehat{u}_k e^{i \omega k t} 
		\right]
		\left[
			\sum_{\genfrac{}{}{0pt}{}{k=-\infty}{k \neq 0}}^{\infty} \widehat{u}_k e^{i \omega k t} 
		\right] \\
		&= \sum_{\genfrac{}{}{0pt}{}{k=-\infty}{k \neq 0}}^{\infty} (\widehat{u}_k)^2
		   e^{i 2 \omega k t} +
		   2 \sum_{\genfrac{}{}{0pt}{}{k,j=-\infty}{k \neq j \neq 0}}^{\infty} 
		   \widehat{u}_k \widehat{u}_j e^{i \omega (k + j) t}.
	\end{split}
\end{equation}
Thus, the time-average becomes
\begin{equation}
	\begin{split}
		\overline{u^\prime u^\prime} &= 
		\frac{1}{T} \int_{t=0}^{T} \left[ 
			\sum_{\genfrac{}{}{0pt}{}{k=-\infty}{k \neq 0}}^{\infty} (\widehat{u}_k)^2
		   	e^{i 2 \omega k t} +
		   	2 \sum_{\genfrac{}{}{0pt}{}{k,j=-\infty}{k \neq j \neq 0}}^{\infty} 
		   	\widehat{u}_k \widehat{u}_j e^{i \omega (k + j) t} 
		\right] \diff t\\
		&= \frac{2}{T} \int_{t=0}^{T} \sum_{\genfrac{}{}{0pt}{}{k,j=-\infty}{k \neq j \neq 0}}^{\infty} 
		   	\widehat{u}_k \widehat{u}_j 
		   	e^{i \omega (k + j) t} \diff t,
	\end{split}
	\label{eq:sm_nlh_temp}
\end{equation}
as the first term is periodic with a zero mean value. Moreover,
this olds true for the terms $j \neq -k$. Therefore,
Equation~\eqref{eq:sm_nlh_temp} becomes
\begin{equation}
		\overline{u^\prime u^\prime} = \frac{2}{T} \int_{t=0}^{T} 
			\sum_{\genfrac{}{}{0pt}{}{k=-\infty}{k \neq 0}}^{\infty} 
			\widehat{u}_k \widehat{u}_{-k}  \diff t.
\end{equation}
As $\widehat{u}_k$ and $\widehat{u}_{-k}$ are complex conjugates,
$\overline{u^\prime u^\prime}$ is finally equal to
\begin{equation}
	\overline{u^\prime u^\prime} = 
	2 \sum_{\genfrac{}{}{0pt}{}{k=-\infty}{k \neq 0}}^{\infty} |\widehat{u}_k|^2.
	\label{eq:sm_nlh_deterministic_stress_terms}
\end{equation}
This last equation depends only on the computed harmonics, meaning
that no term is modeled. Moreover, this term couples the
time-average solution with the unsteady perturbation
and takes into account a part of the 
non-linearities of the considered equation, which makes a
great difference with the LUR approach presented in Sec.~\ref{sub:sm_lur}.

Finally, as computing an infinite number of harmonics is 
numerically not feasible,
it is truncated at order $N$. 
This is a fair assumption as, for most
of the physical flows, the energy
is concentrated on a finite number of frequencies.
Moreover, the goal of Fourier-based time
methods is to have a compact representation of the unsteady time
signals.
As for a mesh grid convergence, the number of harmonics $N$
will directly impact the accuracy of the unsteady representation
of the signal.
The discussion on the
convergence of Fourier-based time methods will be introduced mathematically
in Sec.~\ref{sec:spectral_accuracy} and discussed later on in this 
work in Chap.~\ref{cha:limitations_convergence}.

To summarize, the NLH
method applied to Eq.~\eqref{eq:sm_nonlinear_convection_conservative}
gives $2N$ \added{complex} perturbation equations and one time-averaged 
equation. \added{As $\widehat{u}_k$ and $\widehat{u}_{-k}$
are complex conjugates, this yields $N$ complex equations and
one time-averaged (real) equation,} making $2N+1$ \added{real} equations in total. 
A pseudo-time ($\tau$) derivative is
added to march the equations in pseudo-time to the steady-state 
solution of all the harmonics
\begin{equation}
	\fbox{$
	\begin{dcases}
		\frac{\partial \overline{u}}{\partial \tau} + 
		\frac{\partial}{\partial x}
			\left[\overline{u}^2 + 
			\overline{u^\prime u^\prime}\right] &=
			0, \\
		\frac{\partial \widehat{u}_k}{\partial \tau} + 
		i \omega k \widehat{u}_k + 
			\frac{\partial}{\partial x} 
			\left[ \overline{u} \widehat{u}_k\right] &= 
			0, \: k \in [1, N]
	\end{dcases}
	$}
	\label{eq:sm_nlh_subset_eq}
\end{equation}
The equations are coupled by the deterministic 
stress term $\overline{u^\prime u^\prime}$
defined in Eq.~\eqref{eq:sm_nlh_deterministic_stress_terms}.
The term $u^\prime u^\prime$ is neglected in this formulation.

\subsection{Multi-frequential formulation}

\citet{He2002} extended the method to a multi-frequential
formulation. Instead of writing the perturbation
using a Fourier series as defined in Eq.~\eqref{eq:sm_nlh_decomposition_pert},
it is written using a sum of harmonics each of which
having an angular frequency $\omega_k$
\begin{equation}
	u^\prime = \sum_{\genfrac{}{}{0pt}{}{k=-N}{k \neq 0}}^{N} 
	\widehat{u}_k e^{i \omega_k t}.
	\label{eq:sm_nlh_decomposition_pert_multi}
\end{equation}
Note that the term $k \omega$ in Eq.~\eqref{eq:sm_nlh_decomposition_pert}
is now replaced by $\omega_k$ meaning that the frequencies can be chosen
arbitrarily.
The derivation of the equations is kept the same and the following
$2N+1$ subset of equations is finally obtained
\begin{equation}
	\fbox{$
	\begin{dcases}
		\frac{\partial \overline{u}}{\partial \tau} +
		\frac{\partial}{\partial x}
			\left[\overline{u}^2 + 
			\overline{u^\prime u^\prime}\right] &=
			0, \\
		\frac{\partial \widehat{u}_k}{\partial \tau} + 
		i \omega_k \widehat{u}_k + 
			\frac{\partial}{\partial x} 
			\left[ \overline{u} \widehat{u}_k\right] &= 
			0, \: k \in [1, N]
	\end{dcases}
	$}
	\label{eq:sm_nlh_subset_eq_multi}
\end{equation}
However, as the complex exponentials 
($e^{i \omega_k t}$) do not form
an orthogonal basis, writing Eq.~\eqref{eq:sm_nlh_subset_eq_multi}
for each harmonic $k \in [1, N]$ is mathematically
not true. \citet{He2002} argued that the terms
are collected for each harmonic. 
The same development is made by \citet{Vilmin2006}.

The coupling deterministic stress term is evaluated using the
same equation as for the mono-frequential formulation
\begin{equation}
	\overline{u^\prime u^\prime} = 
	2 \sum_{\genfrac{}{}{0pt}{}{k=-\infty}{k \neq 0}}^{\infty} |\widehat{u}_k|^2.
	\label{eq:sm_nlh_deterministic_stress_terms2}
\end{equation}
To give a mathematical framework to prove this assertion, 
let us consider the specific example of $u^\prime$ taken as
\begin{equation}
	u^\prime = (\widehat{u}_{-1} e^{-i t} + \widehat{u}_{1} e^{i t}) +
		(\widehat{u}_{-2} e^{-i \pi t} + \widehat{u}_{2} e^{i \pi t}),
\end{equation}
namely, we consider Eq.~\eqref{eq:sm_nlh_decomposition_pert_multi}
with the specific angular frequencies: $\omega_1 = 1$ and $\omega_2 = \pi$.
The cross-term $u^\prime u^\prime$ is then equal to
\begin{equation}
	\begin{split}
		u^\prime u^\prime = 
			&(\widehat{u}_{-1})^2 e^{-i 2 t}
			+ (\widehat{u}_{1})^2 e^{i 2 t}
			+ (\widehat{u}_{-2})^2 e^{- i 2 \pi t}
			+ (\widehat{u}_{2})^2 e^{i 2 \pi t} \\
		&+ 2 \left[
				\widehat{u}_{-1} \widehat{u}_{-2} e^{i (-1 -\pi) t} 
				+ \widehat{u}_{-1} \widehat{u}_{2} e^{i (-1 + \pi) t}
				+ \widehat{u}_{1} \widehat{u}_{-2} e^{i (1 - \pi) t} 
				+ \widehat{u}_{1} \widehat{u}_{2} e^{i (1 + \pi) t} 
			 \right] \\
		&+ 2 \widehat{u}_{-1}\widehat{u}_{1}
			 	+ 2 \widehat{u}_{-2}\widehat{u}_{2}.
	\end{split}
	\label{eq:sm_nlh_temp2}
\end{equation}
In the mono-frequential framework, the frequencies are harmonically related
and a common period $T$ exists. This logically leads to the definition
of the mean value $\overline{f}$ of such a time-varying function $f(t)$ as
\begin{equation}
	\overline{f} = \frac{1}{T} \int_0^T f(t) \diff t.
\end{equation}
In the current multi-frequential example, 
$\pi$ and $1$ are not integer multiples of a common fundamental
frequency. For this specific example, a common period $T$ does
not exist. \citet{Besicovitch1932}
defines a mathematical framework for such functions, called
almost-periodic functions. In this framework
the temporal mean value is shown to exist. It is defined as
\begin{equation}
	\overline{f} = \lim_{X \to \infty} \frac{1}{X} \int_0^X f(t) \diff t.
\end{equation}
Applying the time-averaged operator to Eq.~\eqref{eq:sm_nlh_temp2} yields
\begin{equation}
	\overline{u^\prime u^\prime} = 2 (\widehat{u}_{-1}\widehat{u}_{1} + 
		\widehat{u}_{-2}\widehat{u}_{2}) = 2 (|\widehat{u}_1|^2 + |\widehat{u}_2|^2),
\end{equation}
as the mean value of purely harmonic functions is zero.
Extending this demonstration to the arbitrary case of a
multi-frequential perturbation 
(Eq.~\eqref{eq:sm_nlh_decomposition_pert_multi}) leads to
the general expression given in Eq.~\eqref{eq:sm_nlh_deterministic_stress_terms2}.


\subsection{Extensions}

\paragraph{Navier--Stokes equations}
As shown above, since the development of the NLH
method is made in the frequency domain, applying the method to
complex equations can be difficult. For the Navier--Stokes equations,
this step is tedious due to the number of equations to treat. Nevertheless, 
\citet{He1998, Chen2001, He2002} and \citet{Vilmin2006} have
done this and the reader is referred to these papers
for a detailed description.
Note that in all those publications, turbulence is modeled
using only the time-averaged quantities.
This is another assumption as the turbulent field, in a wake
for instance, is seen unsteady in the opposite row frame
of reference~\cite{Lakshminarayana1980}. Thus, this
unsteadiness is not taken into account by the NLH method.

\paragraph{Turbomachinery computations}
Originally, the NLH method has been developed for 
turbomachinery applications. \citet{He1998} and
\citet{Ning1998} computed isolated turbomachinery
configurations. To reduce the domain to a single 
blade-to-blade passage, they consider a periodic
boundary condition for the time-averaged part and a
phase-lagged boundary condition for the perturbation part on the
azimuthal boundaries
\begin{equation}
    \begin{split}
    	\overline{u}_U &= \overline{u}_L, \\
    	u^\prime_U &= u^\prime_L e^{i \sigma},
    \end{split}
\end{equation}
where subscripts $U$ and $L$ denote 
the upper and the lower boundaries, respectively, and $\sigma$ is the
inter-blade phase angle. This allows to compute
isolated vibrating configurations thanks to 
the phase theorem of \citet{Lane1956} (see Sec.~\ref{sub:lane_theorem}).

\citet{Chen2001} added a rotor/stator treatment
to allow the computation of stage configurations. 
To do so, the perturbation 
$u^\prime$ is exchanged at the interface using
an azimuthal Fourier transform, whereas
the time-average field $\overline{u}$ 
and the deterministic stresses 
$\overline{u^\prime u^\prime}$
are azimuthally flux-averaged.
To this aim, the azimuthal variations of 
the perturbation $\widehat{u} (\theta_i)$
are spatially Fourier transformed ($\widetilde{u}_i$)
and exchanged at the interface 
($\widetilde{u}_i = \widetilde{u}_j$). Subscript $i$ and $j$ denotes,
respectively, the upstream and the downstream rows.
This is schematically represented in 
Figure~\ref{fig:bnd_sliding_chen2001}.
\begin{figure}[htp]
  \centering
  \includegraphics*[scale=0.25]{bnd_sliding_chen2001.pdf}
  \caption{Exchange of the variables at rows interface as described by \citet{Chen2001}.}
  \label{fig:bnd_sliding_chen2001}
\end{figure}
Still the method is restricted to mono-frequential problems, since only
one stage is considered, each row seeing the
opposite blade passing frequency. Considering
the time-averaged field to be constant in the azimuthal 
direction at the interface seems fair (in case 
without clocking effects), 
but there is
no reason for $\overline{u^\prime u^\prime}$ to be so.

\citet{He2002} extended the method to take into
account multi-stage configurations through the
development of a multi-frequential formulation.
In fact, in such applications, 
a sandwiched row will see unsteadinesses coming
from the upstream row (mainly wake effects) and
potential effects from the downstream rows. In the
general case where the surrounding rows do not have the
same blade passing frequencies, multiple frequencies
can be present in the current row.
The same treatment is used at the rows interfaces meaning
that the time-averaged quantities are flux-averaged and the
fluctuations are exchanged through their azimuthal
Fourier transform.

\citet{Vilmin2006} extended the rotor/stator
interface to a non-matching join sliding mesh interface which
leads to the continuity of the unsteady flow field at the interface.
The main difference with the previous treatment is that
$\overline{u}$ and $\overline{u^\prime u^\prime}$ 
are not flux-averaged but rather spatially Fourier transformed,
which leads to the continuity of $u$ at the interface, when
the number of harmonics is sufficient.

\paragraph{Clocking effect}
\begin{figure}[htp]
  \centering 
  \includegraphics[width=.5\textwidth]{clocking_effect.pdf}
  \caption{Different clocking positions for a stator/rotor/stator
  configuration.}
  \label{fig:sm_nlh_clocking_effect}
\end{figure}
Figure~\ref{fig:sm_nlh_clocking_effect} shows three
different clocking positions (sometimes also referred 
to as the indexing positions)
of the first stator
in a stator/rotor/stator configuration. In this figure,
the first clocking position is aligned with the second stator.
The second and the third clocking positions are not aligned with
the second stator, which may give different level of perceived
unsteadiness by the second stator. At a design phase, the engineer can choose any
relative position of the rows and thus any clocking position.
The relative position of both stator is of
prior interest to choose the best clocking position. 
In fact, the wakes that are shed behind the first stator
are cut by the rotor blades and transmitted to 
the second stator row. The stators being fixed, the wake of
the first stator is seen as a stationary wave in the second stator.
Hence, the importance of their relative position. For instance,
\citet{Huber1996} showed that
on their 1.5 stage turbine, the variation of efficiency due to clocking
position was equal to $0.8\%$ of efficiency points, showing the
importance of the clocking effect.

The brute force approach to compute the clocking effect on a
configuration is to consider all the relative positions. This means
that the geometry of the stator should be rotated for each new 
clocking position and hence a new unsteady computation should be 
run. The innovative procedure proposed by 
\citet{He2002} is to consider the clocking effect as a steady wave.
In fact, as both stator are fixed, a steady perturbation
shed behind the first stator is still steady in the second stator.
In terms of frequencies, a steady perturbation
can be assimilated to a zero frequency mode. 
In \citet{He2002} and \citet{Vilmin2009}, 
a perturbation with a zero frequency
is thus additionally computed. The clocking effect can then be evaluated by
post-processing the Fourier coefficient of the zeroth frequency mode.
Recently, the computation of clocking effects on
arbitrary configurations has been made possible
by \citet{Vilmin2013a}. This allows its use for 
pylon/rotor/rotor applications for instance, which
is the configuration encountered in an installed CROR.

\subsection{Numerical cost}
Compared to the LUR method, the number of equations to solve is 
not constant here. In fact, if $N$ denotes the number of harmonics
computed in total (sum of the number of computed harmonics per perturbations)
and if $\mathdollar_{\text{RANS}}$ 
denotes the CPU and memory cost associated to
one steady computation, \replaced{$N$ complex perturbation}{$2N$ harmonic} 
equations \added{(\emph{i.e.} $2N$ real equations)} and 
one time-average equation
are solved, thus
\begin{equation}
	\mathdollar_{\text{NLH}} = (2N+1) \times \mathdollar_{\text{RANS}}.
\end{equation}
However, \citet{Vilmin2006} do not apply the NLH formulation
to the turbulent equation (in their case, 
the one equation of \citet{Spalart1992}). Therefore,
as only five equations are solved using the NLH approach, the 
turbulent equation being solved as a steady one,
the cost becomes
\begin{equation}
	\mathdollar_{\text{NLH}} = \frac{5 \times (2N+1) + 1}{6} \times \mathdollar_{\text{RANS}}.
\end{equation}

