%!TEX root = ../../adrien_gomar_phd.tex

\chapter*{Acknowledgments}
\thispagestyle{empty}

What a pleasure to finally write those lines !! This means that
the end is close ;)
I will try to make this quick, even though everyone knows that this
will be the most read part in this manuscript, starting by you ;)
I am not fooling myself.

First, I want to thank the members of the jury. I have been
delighted by the meticulous review made by Professor Li He and Professor 
Christophe Corre on this manuscript. Your remarks, observations
and questions
have participated to make this manuscript more complete and accurate, and
for that I thank you.
Then I want to thank the other members of the Jury: Pascal Ferrand, 
Jean-Camille Chassaing and Clément Dejeu for being here on D-Day,
for the chats that we had
and for the feedback that you gave me. This PhD defense day will remain as one of the
best days of my life, yet stressful.
As for the PhD defense, I will now switch in French as it will
be easier for me to make proper acknowledgments.

Je te tiens à remercier chaleureusement Paola Cinnella, ma
directrice de thèse. Pendant ces trois années, tu as été un pilier
de cette thèse et ce malgré la distance. 
Je te remercie d'avoir pris le temps à chaque fois
de m'aiguiller dans les choix "stratégiques", de remettre en cause
certaines de mes certitudes, et surtout pour ton regard neuf sur les méthodes
spectrales. J'ai beaucoup apprécié travailler avec toi.
Je tiens aussi à remercier Fredéric Sicot, mon encadrant au CERFACS.
Merci pour le temps que tu as pris pour m'expliquer les fondements de 
la TSM (même s'il faudrait mieux que je l'appelle HB, right ;)
et aussi pour le temps que tu as passé avec moi sur elsA. 
Merci surtout de m'avoir laissé la liberté et le temps
d'explorer des pistes parfois farfelues, j'ai beaucoup appris 
des réussites, mais surtout des échecs. Enfin, merci à Jean-françois Boussuge
de m'avoir accepté dans l'équipe AAM.
Je me rappellerais du son
de tes pas se dirigeant vers mon bureau ;) avec
en prime une remise en cause "douce" de certains de mes choix.
Cela m'a permis de m'affirmer là où j'étais encore trop
peu sur de moi, donc merci ! A contrario, je me rappellerais aussi 
de cette boite de chocolat pour nous remercier pour Antares :P
Un peu de douceur dans ce monde de brute right ?

Je souhaiterais remercier chaleureusement
l'ensemble de l'équipe AAM. Cela va être long et je vais
essayer d'oublier personne: je me lance. 
Merci Nico d'avoir cru en moi sur un simple coup
de téléphone. Merci pour ta bonne humeur quelque soit les
circonstances et ta gentillesse lorsque je suis venu te solliciter
pour la relecture de mon mémoire. Au passage désolé de ne pas
t'avoir payé cette dernière bière. Je n'ai pas trouvé le temps de t'inviter
en tête à tête au pub ... Merci Guillaume pour m'avoir encadré
sur mon stage. J'aurais aimé que cette thèse se passe avec toi comme
encadrant car j'ai toujours aimé la pédagogie que
tu prends pour expliquer les choses. 
Néanmoins, merci d'être resté dans les parages et d'avoir
continué à bosser avec nous !
Merci Marco, Lokmann
et JC, mes acolytes de stage. Je me suis marré pendant 6 mois.
Je me rappellerais de cette clim improvisée et des craquages de Marco !!
JC, encore toi, bah oui car comme moi tu es resté en thèse ;)
Merci pour ta gentillesse à toute épreuve ! Merci Sophie de
la team acoustique. Je suis désolé, tu as été ma target préférée
de fin de thèse, mais bon je t'ai toujours charriée en toute amitié,
j'espère que tu ne m'en voudras pas ;) 
Merci à l'autre belge de notre équipe, hein Nadège!! 
Sacré maman, un vrai petit caractère comme il faut !! Merci Benjamin
pour tous nos échanges sur le CROR. Avant nos discussions, je n'y
comprenais rien ;) merci pour ta vision physique et tous les échanges
que l'on a pu avoir sur nos thèses respectives. Cela fait du bien
d'échanger right ? Et merci d'avoir créé le terme "script jetable",
cela restera dans mon vocable. Merci à Thomas. J'ai adoré bosser avec
toi !! Je pense que dans un autre contexte, j'aurais cherché à
monter une boite avec toi !! Merci aussi pour cette visite personnalisée
de Bordeaux avec son fameux plan d'eau ... No comment. Pour moi tu es
devenu un véritable ami. Merci Gaëlle
pour ta fraicheur et pour m'avoir pris un nombre de jetons incalculables.
Je ne doute pas que ta thèse va bien se passer.
Merci à Remy, le plus AAM des combus, pour ton opiniâtreté sur Paraview.
Je te l'accorde, Paraview sait faire des smileys, 
mais non je ne l'utiliserais pas ;)
Merci à Bill bocquet et Flore. L'ex bureau improbable mais pourtant
bien là. Vous m'avez bien fait marrer. Je n'oublie pas Majd, Carlos et Julien.
La relève. Bon courage pour la thèse et faites ça dans la bonne humeur, vous
verrez la fin est vraiment agréable !
Merci à Marc, marcounet pour les intimes, merci d'avoir accepté de me prêter à taux
zéro des
jetons lorsque je n'en avais plus. Quand tu veux en Angleterre pour un
potage tomate !! Merci à Vieux gui pour les midi baby foot. J'aurais aimé
te battre un peu plus, car ton jeux m'a toujours déconcerté !! 
Merci à Guillaume pour ces bons chocolats et NON, Antares n'est pas buggé !!
C'est juste le mec qui est entre la chaise et le clavier ... 
énigme quand tu nous tiens ;) 
Merci Pierre d'avoir repris mon appart ! J'espère que tu t'y plairas
(même si je n'en doute pas un instants !).
Last but not least, merci à François pour ces
trois années de travail / déconnade / craquage le vendredi soir / 
petits gateaux / python / et de Pink Floyd qui claque. J'ai vraiment
eu le bureau qu'il fallait pour bosser et lâcher la pression de temps
en temps (souvent ?). Et non, je ne te remercierais pas
pour tes questions sur Git. J'ai jamais compris comment 
un mec aussi brillant pouvait avoir du mal
avec un soft aussi banal. Faut croire qu'il n'y a pas de règles ;)
Je n'oublie pas les CERFACSiens qui ont quitté le navire entre temps:
Antoine, Grace et Lulu.
Je n'oublie pas non plus l'équipe CSG. Merci Gérard,
tu as supporté mes coups de fil répétés, Fabrice pour ton
support sur les macs, Isabelle pour les problèmes de
compilation. C'est votre équipe qui fait la force du CERFACS.
Enfin, merci à L'administration du CERFACS, Michèle pour tes
conseils éclairés sur les aspects légaux, Marie pour
ton aide au jour le jour et pour faire le relais des 
congés !! Nicole, notamment pour ta relecture et les
corrections que tu as apportées sur notre premier
article. Lydia et Brigitte, pour votre accueil
et votre aide au quotidien.
Enfin merci à Chantal pour ta bonne humeur, pour 
cette bonne galette et pour t'être occupé de
Nahel pendant la soutenance, même si je ne pense
pas que c'était un calvaire pour toi !

Je souhaiterais aussi remercier Nicolas Binder et
Xavier Carbonneau. 
Merci de m'avoir initié au monde des turbomachines ;) Pour moi,
le débit c'est dieux et personne ne croit en l'expérience
sauf celui qu'il l'a faite et tout le monde croit en la
simulation sauf celui qu'il l'a fait. 
C'est ça que je devais retenir, right ?
Merci d'avoir pris le temps de 
discuter avec moi de mes travaux de thèse même si
vous n'étiez pas directement impliqués.
Merci à Yann Colin pour les échanges que l'on a pu avoir
sur les CRORs, mais surtout sur mon avenir. Aujourd'hui,
je suis convaincu que j'ai fait le bon choix et je te remercie
de m'avoir aiguillé !
Merci à Thieu pour m'avoir mis sur la voie de la CFD ;)
Aujourd'hui, on peut le dire, je ne porte plus de moufles, enfin
j'espère ...

\newpage

\`A la fin, il reste les amis et la famille ...
Merci à mon poto Remy pour les afterworks nécessaires
à la réussite de cette thèse. Merci aussi pour avoir été
là quand ça n'allait vraiment pas et aussi quand ça
allait pour fêter ça. Merci d'être venu à la soutenance.
Merci aussi à May d'avoir fait le déplacement, même
si le sujet, well, n'était pas trop ta tasse de thé.
Merci à mes frères Aurélien et Victor, à ma soeur Salomé et à mes
Parents (I mean lolosse of course, l'autre ne t'arrive pas
à la cheville) pour votre soutien au cours de ces trois années.
On se rend pas compte à quel point votre présence m'a permis de
me dépasser. Merci aussi à JC, Cécile et la famille Hugon, passez 
nous voir en Angleterre quand vous voulez. Merci enfin à Géraldine, celle qui 
m'accompagne depuis tant d'années. Merci pour les coups
de pieds que tu m'as mis quand il fallait et pour ton
amour au quotidien. Cette thèse a été réussie grace à ton aide,
ton soutien et ta compréhension lors de ces longues nuits 
de travail et ces moments de doutes. Merci m’avoir soutenu. Je t'aime.
Merci enfin au petit nouveau de ma famille, mon fils Nahel.
Même si tu m'as donné du fil à retordre avant ma soutenance,
cela se voit déjà que tu es un gentil. Je t'aime.

\begin{flushright}
\emph{\`A Paris, le 6 mai 2014}
\end{flushright}
