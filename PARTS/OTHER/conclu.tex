%!TEX root = ../../adrien_gomar_phd.tex

\chapter*{Conclusion}

\section*{Summary of the results}

\subsection*{On the multi-frequential harmonic balance approach}

When unsteadiness is related to a single frequency and its
harmonics, Fourier analysis leads to a natural choice for time instances:
they are evenly spaced over the period. In this case, the mathematical
problem is numerically well-posed, meaning that the conditioning of
the operators ensures the convergence of the approach.
In opposite, when several arbitrary frequencies are 
considered, as for instance contra-rotating open rotor
aeroelasticity, the multi-frequential harmonic balance approach
is required.

In \hyperref[cha:limitations_condition_number]{\emph{Chapter~5}},
we demonstrated that the time sampling has a major effect on the
stability of the multi-frequential harmonic balance 
method, due to the condition number of the Fourier
transform matrix. One way to tackle this issue, 
is to consider a non-uniform time sampling
along with an algorithm to properly choose them
as proposed by \citet{ThesisGuedeney}.
The Almost-Periodic Fourier Transform algorithm (APFT) 
algorithm, originally developed by \citet{Kundert1988} and implemented by 
\citet{ThesisGuedeney}, is shown to improve the
Fourier matrix orthogonality and thus the condition number.
However, for segregated frequencies, the condition number
remains too large to be used within an industrial context.
As the aeroelasticity of contra-rotating open rotors is by essence
composed of segregated frequencies, new algorithms are needed.
This is why, a gradient-based OPTimization algorithm (OPT) 
has been developed in the current work.
It directly minimizes the condition number thanks to a
gradient-based optimization method. This last has proved to
give a condition number that is almost unity for any input frequencies,
thus alleviating the stability issues attributed to 
the condition number.
Therefore, the non-uniform time sampling proposed by \citet{ThesisGuedeney}
used together with the OPT algorithm 
developed in the present contribution
enables to tackle problems with large frequency 
separation or large unsteadinesses, namely CROR aeroelasticity
can be considered.

\subsection*{On the convergence of Fourier-based time methods}

Efficiency of Fourier-based time methods results 
from a trade-off between accuracy and 
costs requirements.
On one hand, the accuracy of Fourier-based
time methods depends on the number of harmonics
used to represent the frequency content of the time 
signal; on the other hand, computational costs and 
memory consumption of the computations also scale
with the number of harmonics. 
Nevertheless, this number of harmonics is configuration-dependent 
and hardly predictable. Studies on the convergence of 
Fourier-based time methods for turbomachinery simulations 
have been previously reported in the literature, 
but with scattered results. 

In \hyperref[cha:limitations_convergence]{\emph{Chapter~6}}
we investigated the accuracy and convergence properties 
of Fourier-based time integration methods. The convergence rate 
of these methods, in terms of harmonics required to describe the solution 
with a given level of accuracy, depends on the spectral content of the 
solution itself: Fourier-based time methods are particularly efficient 
for flow problems characterized by a narrow Fourier 
spectrum. Starting from this remark, we try to define a relevant 
indicator of solution regularity in the specific case of turbomachinery 
flows, which represent one of the main applications of Fourier-based 
time methods in Fluid Mechanics.
To this aim, we show that the main source of unsteadiness in 
turbomachinery flows is due to the relative motion of wakes 
generated by a given blade row with respect to the downstream row. 
Statistically speaking, the passing wakes are seen by the downstream 
row as an azimuthally advected periodic Gaussian pulse, 
characterized by its relative thickness
and by the velocity deficit 
associated to it. We show that the narrower the wake, the larger 
its Fourier spectrum, and the slower the convergence of Fourier-based time methods.
In order to achieve a priori estimates of the number of 
harmonics required to accurately solve a given turbomachinery 
problem, we introduce two error measures based on the relative 
thickness of the passing wakes. It is shown that, for practical 
purposes, these can be preliminarily estimated by running a 
companion steady simulation of the turbomachinery stage. The 
steady simulation is post-processed to extract information about 
the spanwise distribution of wake thickness, and an error criterion 
is used to estimate the number of harmonics required to resolve 99\% 
of the energy content associated to the velocity signal. The 
preliminary step has a negligible cost compared to the overall 
simulation, since the steady computation is used to initialize 
the unsteady run, and extraction of wake characteristics takes 
less than a minute on a single processor. 

\subsection*{On the aeroelasticity of contra-rotating open rotors}

The proposed weak coupling approach along with
an harmonic balance approach has been
validated on the $11^{th}$ standard configuration. 
The results show that the harmonic balance approach provides local
and global results close to the reference time-marching scheme 
with only $N=1$ harmonic in the time period. At the cost of a memory
increase (roughly equal to the number of instances used in the harmonic balance
simulations), the computational saving is seven for this
particular case compared to a phase-lag approach combined
with a time-marching scheme. 
Moreover, the results are
in good agreement with the experimental data and with the results
found in the literature.





\section*{Future work}

\begin{figure}[htp]
  \centering
  \subfigure[2F]{\includegraphics[width=.4\textwidth]{HERA3_INSTALLED_wall.png}}
  \subfigure[1T]{\includegraphics[width=0.40\textwidth]{HERA3_INSTALLED_RANS_SPECTRUM_PPT.pdf}}
  \caption{Low-speed isolated configuration: structural modes considered.}
  \label{fig:hera3_perspectives}
\end{figure}
