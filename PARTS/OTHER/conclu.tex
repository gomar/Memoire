%!TEX root = ../../adrien_gomar_phd.tex

\chapter*{Conclusion}

The present PhD thesis aims at applying the HB approach to the 
aeroelasticity of a new type of aircraft engine: 
the contra-rotating open rotor. The method is 
first validated on analytical and linear and non-linear 
numerical test problems in \hyperref[cha:validation_hb]{\emph{Chapter~4}}. 
Two issues are raised, which prevent the use of such an approach 
on industrial configurations: problem conditioning and the
convergence of the method. Original methodologies are developed 
to improve the condition number of the simulations 
(\hyperref[cha:limitations_condition_number]{\emph{Chapter~5}})
and to provide a priori estimates of the number of harmonics 
required to achieve a given convergence level
(\hyperref[cha:limitations_convergence]{\emph{Chapter~6}}). 
The HB method is then validated for a standard configuration 
for turbomachinery aeroelasticity in \hyperref[cha:stcf11]{\emph{Chapter~7}}. 
The results are shown to be in fair agreement 
with the experimental data. The applicability of the method 
is finally demonstrated for aeroelastic 
simulations of contra-rotating open rotors
in \hyperref[cha:dream_ls_isolated]{\emph{Chapter~8}}
and \hyperref[cha:dream_hs_isolated]{\emph{Chapter~9}}

\section*{Summary of the results}

\subsection*{On the conditioning of multi-frequential harmonic balance methods}

When unsteadiness is related to a single frequency and its
harmonics, Fourier analysis leads to a natural choice for the time instances
needed to compute the source term:
they are evenly spaced over the period. In this case, the mathematical
problem is numerically well-defined, meaning that the conditioning of
the operators ensures the convergence of the approach.
In opposite, when several arbitrary frequencies are 
considered, as for instance CROR
aeroelasticity, the multi-frequential harmonic balance approach
is required and its source term can be ill-conditioned.

In \hyperref[cha:limitations_condition_number]{\emph{Chapter~5}},
we demonstrated that the time sampling has a major effect on the
stability of the multi-frequential harmonic balance 
method, due to the condition number of the discrete Fourier
transform matrix. One way to tackle this issue, 
is to consider a non-uniform time sampling
along with an algorithm to properly choose the time instances
as proposed by \citet{ThesisGuedeney}.
The Almost-Periodic Fourier Transform algorithm (APFT) 
algorithm, originally developed by \citet{Kundert1988} and implemented by 
\citet{ThesisGuedeney}, is shown to improve the discrete
Fourier transform matrix condition number.
However, for segregated frequencies, this condition number
is shown to remain too large to be used within an industrial context.

As the aeroelasticity of CROR is by essence
composed of segregated frequencies, new algorithms are needed.
This is why, a gradient-based OPTimization algorithm (OPT) 
has been developed in the current work.
It directly minimizes the condition number thanks to a
gradient-based optimization method. This last has proved to
give a condition number that is almost unity for any input frequencies,
thus alleviating the stability issues encountered for arbitrary
multi-frequential HB computations.
This is a pre-processing procedure
implemented into Antares (see Appendix~\ref{app:antares})
that takes less than a minute.
Therefore, the non-uniform time sampling proposed by \citet{ThesisGuedeney}
used together with the OPT algorithm 
developed in the present contribution
enables to tackle problems with large frequency 
separation or large unsteadinesses, namely CROR aeroelasticity
can be considered.
This work has been published in:
\begin{quote}
	{\small T. Gu\'edeney, \emph{A. Gomar}, F. Gallard, F. Sicot, G. Dufour, and G. Puigt. 
	Non-Uniform Time Sampling for Multiple-Frequency Harmonic Balance Computations. 
	\emph{Journal of Computational Physics}, 236:317--345, March 2013}
\end{quote}


\subsection*{On the convergence of Fourier-based time methods}

Efficiency of Fourier-based time methods results 
from a trade-off between accuracy and 
costs requirements.
On one hand, the accuracy depends on the number of harmonics
used to represent the frequency content of the time 
signal; on the other hand, computational costs and 
memory consumption of the computations also scale
with the number of harmonics. 
The problem is that this number is 
configuration-dependent and hardly predictable. 
Moreover, a high number of harmonics
($\gg 10$) can prevent the use of such an approach,
as it might be more expansive than the classical time-marching approaches.
This is particularly true on CROR configurations were the number
of harmonics needed to reach convergence
have been shown to be greater than ten
on some configurations~\cite{ThesisFrancois}.

In \hyperref[cha:limitations_convergence]{\emph{Chapter~6}}
we investigated the accuracy and convergence properties 
of Fourier-based time methods. It is highlighted that the convergence rate 
of these methods, in terms of harmonics required to describe the solution 
with a given level of accuracy, depends on the spectral content of the 
solution itself: Fourier-based time methods are particularly efficient 
for flow problems characterized by a narrow Fourier 
spectrum. 

We showed that the main source of unsteadiness in 
turbomachinery flows is due to the relative motion of wakes 
generated by a given blade row with respect to the downstream row.
Statistically speaking, the passing wakes are seen by the downstream 
row as an azimuthally advected periodic Gaussian pulse, 
characterized by its relative thickness
and by the velocity deficit associated to it.
The Fourier transform of a Gaussian function being analytical,
a truncation error has been defined, which showed that the narrower the wake, 
the larger the Fourier spectrum resulting in a slower convergence 
of Fourier-based time methods.

Based on this knowledge of the phenomenon,
we showed on a model turbomachinery computation, that
the analytical truncation error can be \emph{a priori} 
estimated using a mixing-plane steady computation
using the azimuthal accumulated energy.
Applying the \emph{a priori} error estimate to 
the steady computation of any turbomachinery configuration
provides the number of harmonics required 
to achieve a given convergence level.
It encompasses both the wake distortions and also
any tangential disturbances, as for instance
the viscosity effects near the hub or the tip vortices.
We finally stress that a 10\% error (or equivalently 
a 99\% accumulation of energy) is a good threshold
that ensures the continuity of the wakes at the rows
interfaces. Finally, this allows to \emph{a priori}
estimate the number of harmonics required to simulate
a given turbomachinery configuration.
This work has been submitted in:
\begin{quote}
	{\small \emph{A. Gomar}, Q. Bouvy, F. Sicot, G. Dufour, P. Cinnella, and B. Fran\c cois. Convergence of Fourier-based time methods for turbomachinery wake passing problems. 
	\emph{Journal of Computational Physics}, submitted in December 2013}
\end{quote}

This preliminary step has a negligible cost compared to the overall harmonic balance
simulation, since the steady computation is classically used to initialize 
the unsteady run, and extraction of energy accumulation across span takes 
less than a minute on a single processor. The capability of the
tool to estimate the number of harmonics needed
to converge an HB computation is verified on the industrial low-speed CROR configuration
studied in \hyperref[cha:dream_ls_isolated]{\emph{Chapter~8}}.

\subsection*{On the aeroelasticity of contra-rotating open rotors}

In \hyperref[cha:stcf11]{\emph{Chapter~7}}, 
the proposed weak coupling approach along with
an harmonic balance approach has been
validated on the $11^{th}$ standard aeroelastic turbomachinery
configuration.
The results show that the harmonic balance approach provides local
and global results close to the reference time-marching scheme 
with only $N=1$ harmonic in the time period. At the cost of a memory
increase (roughly equal to the number of instances used in the harmonic balance
simulations), the computational saving is seven for this
particular case compared to a phase-lag approach combined
with a time-marching scheme. Moreover, the results are
in good agreement with the experimental data and with the results
found in the literature, validating the current approach.






\section*{Future work}

\begin{figure}[htp]
  \centering
  \subfigure[2F]{\includegraphics[width=.4\textwidth]{HERA3_INSTALLED_wall.png}}
  \subfigure[1T]{\includegraphics[width=0.40\textwidth]{HERA3_INSTALLED_RANS_SPECTRUM_PPT.pdf}}
  \caption{Low-speed isolated configuration: structural modes considered.}
  \label{fig:hera3_perspectives}
\end{figure}
