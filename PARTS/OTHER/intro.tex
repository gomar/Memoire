%!TEX root = ../../adrien_gomar_phd.tex

\chapter{Introduction}

Global warming may be one of the biggest challenges that human kind
will have to face in the forthcoming decades if not years.
According to the last scientific
report of the Intergovernmental Panel on Climate Change 
(IPCC)~\cite{IPCC2013}
"The largest contribution to total radiative 
forcing\footnote{namely global warming} 
is caused by the increase in the atmospheric 
concentration of $CO_2$ since 1750".

A part of these $CO_2$ emissions and in general pollutants stems from the
transport industry and in particular the
aeronautical industry. 
Hopefully for the earth,
the rarefaction of crude oil accelerates the decision making.
In particular, in the seventies, the two oil crisis showed the aeronautical 
industries their dependence toward energy resources. 
To face this issue, the U.S. Senate directed NASA in 1975
to look for every potential fuel-saving concept for aircraft
engines. The Advanced Turboprop
project was born~\cite{Hager1988} and led to the
concept of Contra-Rotating Open Rotor (CROR). It differentiates
from the Contra-Rotating Propeller (CRP) engine, that has been
studied by the United-Kingdom (at the beginning of the twenties century) 
and Russia (in the forties), by the Mach number for which it is designed:
the CROR is meant to work on transonic inflow conditions,
enabling its use for commercial aviation.
This CROR concept showed a potential for large fuel savings
but led to higher noise emissions due to the absence of
a duct. The high noise emissions 
combined with the decrease of the price of the
barrel in the late eighties, the contra-rotating open rotor 
never reached the commercial aviation.

Today, the cost of the barrel is almost at its maximum as shown
in Figure~\ref{fig:crude_oil_price}.
\begin{figure}[htp]
  \centering
  \includegraphics*[width=0.5\textwidth]{crude_oil_price.pdf}
  \caption{Evolution of the cost of a barel from $1861$ to $2012$, from BP~\cite{bpreview2013}.}
  \label{fig:crude_oil_price}
\end{figure}
In parallel, Airbus forecasts a doubled number of passengers in
$2031$~\cite{AirbusForecast2013}. For that reason, the European commission has set, through the
Advisory Council for Aeronautics Research in Europe (ACARE),
demanding objectives to reduce these emissions by 2050:
the noise, $CO_2$ and $NO_x$ emissions should be reduced by 
$65\%$, $75\%$ and $80\%$, respectively, with respect to
the value of 2000.
Therefore, to allow a sustainable air transportation, new
concepts are needed for both the engines and the 
aircraft in general.
Several have emerged, among which lightweight construction
with advanced composite structure, airport collaborative decision
making with continuous climb departure and less waiting in taxi,
aerodynamically optimized wing geometries, \emph{e.g.} laminar wings,
and fuel efficient engines, to name but a few.
For the latter, two main types of engine are currently studied: the
Ultra-High ByPass Ratio (UHBPR) engine that is based on a
larger fan exhaust engine improving thus the
propulsive efficiency, and the CROR
engine that relies on two rows of contra-rotating rotors,
that proved its viability during experiments within the framework of
the Advanced Turboprop project of NASA~\cite{Hager1988}.
In this work, the focus will be on the CROR engine.
\newline 

The industrial design of turbomachinery, and by extension of contra-rotating
open rotors, is usually based on steady flow analysis, 
for which the reference simulation tool are the three-dimensio\-nal Reynolds-Averaged 
Navier--Stokes (RANS) steady computations. However, this approach finds its limits 
when unsteady phenomena become dominant. This is the case of 
contra-rotating open rotors, where the interaction between the
two rotors is of prior importance. 
In such a
context, engineers need now tools to account for these effects as
early as possible in the design cycle. With the growth of
computational power, unsteady computations are entering industrial
practice, but the associated restitution time remains an obstacle for
daily basis applications.  For this reason, efficient
unsteady approaches are receiving a lot of attention. 

At CERFACS, several unsteady approaches have been investigated 
to reduce the computational time associated with the unsteady simulation of 
CROR configurations. 
These are seldom carried on the whole
circumference of the annulus due to the high computational
cost. A first approach is therefore to assume cyclic periodicity,
which allows to solve for only one blade passage and thus drastically
reduce the computational domain. 
In the turbomachinery community, the phase-lag approach has shown to be
a very efficient method to reduce the computational domain while
maintaining a good capture of the unsteady flow physics. 
In this way, \citet{Burnazzi2010} evaluated the phase-lag approach
by applying it
to a 3D contra-rotating open rotor configuration. He showed
that the interactions between the two rotors can be retrieved, allowing
thus a large computational time reduction.
An additional gain can be expected by working on
the temporal scheme used to solve the equations. 
To achieve
this, Fourier-based methods for periodic flows have undergone major
developments in the last decade (see \citet{He2010} for a recent
review).  The basic idea is to decompose
time-dependent flow variables into Fourier series, which are then
injected into the equations of the problem. The time-domain problem is
thus made equivalent to a frequency-domain problem, where the complex
Fourier coefficients are the new unknowns. At this point, two
strategies coexist to obtain the solution. The first one is to solve
directly the Fourier coefficients, using a dedicated
frequency-domain solver, as proposed by \citet{He1998}. The second strategy is to cast the
problem back to the time domain using the inverse Fourier transform, as
proposed by \citet{Hall2002} with the Harmonic
Balance (HB) method. The unsteady time-marching problem is thus
transformed into a set of steady equations coupled by a source term.
This term corresponds to a spectral approximation of the 
time-derivative in the exact equations. The main advantage of solving in the time domain is
that it can be implemented in an existing classical RANS solver,
taking advantage of all classical convergence-accelerating techniques
for steady state problems.
\citet{ThesisSicot} implemented
the HB method into the \textit{elsA}~\cite{Cambier2013} CFD code
that is used at CERFACS. Applied to turbomachinery
configurations, this method showed a computational gain
of one to two orders of magnitude 
compared to classical time-marching approaches.
Applied to CROR configurations, \citet{Yabili2010}
showed that the computational time reduction
was not conclusive. In fact, a large number of 
harmonics compared to turbomachinery configurations
was needed to properly capture the unsteadinesses, lowering
the computational gain.
Therefore, \citet{ThesisFrancois} deeply
investigated the different unsteady approaches available 
for turbomachinery computations, among which the HB approach, 
and evaluated it for CROR simulations. 
The author confirmed that
the harmonic balance method can retrieve unsteady
flow features for a reduced cost at a gain that
is relatively smaller compared to what was
obtained on former turbomachinery applications.
In parallel, \citet{ThesisGuedeney} extended the harmonic
balance approach to a multi-frequential framework. 
This method allows then to compute unsteadinesses whose frequencies
are not harmonically related, which opens new perspectives.
\newline

Several challenges, such as aerodynamic,
aeroacoustic and aeroelasticity are still open 
for contra-rotating open rotor
to become a viable engine for the next generation aircraft.
In this work, we assess the aeroelasticity of 
contra-rotating open rotor by using the multi-frequential
harmonic balance approach developed and implemented 
by \citet{ThesisGuedeney}.
Actually, the main unsteadinesses of the flow field
are known to be correlated with the so-called
blade passing frequency. This frequency depends on the
rotation speed of the current rotor and the opposite rotor
number of blades. In contrast to that, the 
frequency that drives the aeroelasticity of CROR
blades depends on their structural properties.
As such, the frequencies of both the aerodynamic
field and the aeroelasticity are not harmonically
related, which justifies the use of the multi-frequential
formulation of the harmonic balance approach.


The aim of this study is to assess the
multi-frequential harmonic balance
to estimate the flutter properties of contra-rotating open rotor
configurations. In this way, the dissertation is divided in three parts:
\begin{itemize}
	\item \hyperref[part1]{\emph{Part I}} first presents general information on 
	contra-rotating open rotors (\hyperref[cha:cror]{\emph{Chapter~1}}).
	The basic equations governing the aeroelasticity of
	turbomachinery are presented and the chosen numerical approach 
	is detailed (\hyperref[cha:ael]{\emph{Chapter~2}}).
	Then, the mathematical framework that allows the derivation
	of Fourier-based time methods and their underlying properties
	are presented (\hyperref[cha:spectral_methods]{\emph{Chapter~3}}),
	with focus on the multi-frequential harmonic balance approach.
	\item \hyperref[part2]{\emph{Part II}} 
	presents the advantages and limitations
	of Fourier-based time methods. The chosen approach, namely
	the harmonic balance, is validated for linear and non-linear
	equations in \hyperref[cha:validation_hb]{\emph{Chapter~4}}. Both the
	mono-frequential and the multi-frequential formulations
	are shown to give spectral accuracy, which is a convergence
	property specific to Fourier-based time methods.
	It is emphasized
	that a large CPU gain can be expected in the
	case of contra-rotating open rotor aeroelasticity. 
	However, high condition number can lead to divergence
	of the computations when using the multi-frequential harmonic
	balance approach
	(\hyperref[cha:limitations_condition_number]{\emph{Chapter~5}}). This is 
	first highlighted on two model problems and then cured using
	an original optimization algorithm.
	Finally, the convergence of the harmonic balance 
	that was shown to be case dependent is
	assessed (\hyperref[cha:limitations_convergence]{\emph{Chapter~6}}). 
	It is demonstrated that the difference in computational
	gain is linked to the thickness of the wakes observed behind
	turbomachinery blades, which extends to CROR blades. Based on this observation,
	a prediction tool is developed to estimate the
	number of harmonics needed to compute a given turbomachinery (and CROR)
	configuration using a mixing plane computation 
	The relative CPU gain to be expected can thus be estimated
	and help the decision making in choosing an unsteady approach
	over another one.
	\item based on the work done in the second part,
	the proposed approach retained in this thesis, 
	namely the multi-frequential harmonic balance method along with a weak 
	aeroelastic coupling approach, is applied on different configurations
	in \hyperref[part3]{\emph{Part III}}. First, it is validated 
	on a reference configuration against experimental 
	results and other numerical approaches found in the literature
	(\hyperref[cha:stcf11]{\emph{Chapter~7}}). This gives us confidence
	to apply the approach on an industrial isolated contra-rotating
	open rotor application at low-speed (\hyperref[cha:dream_ls_isolated]{\emph{Chapter~8}})
	and high-speed (\hyperref[cha:dream_hs_isolated]{\emph{Chapter~9}})
	flight conditions. The aeroelastic results are discussed based
	on the computed unsteady flow field.
\end{itemize}
