%!TEX root = ../../adrien_gomar_phd.tex

\chapter{Introduction}

Global warming may be one of the biggest challenge that human kind
will have to face in the forthcoming decades if not years.
According to the last scientific
report of the Intergovernmental Panel on Climate Change 
(IPCC)~\cite{IPCC2013}
"The largest contribution to total radiative 
forcing\footnote{namely global warming} 
is caused by the increase in the atmospheric 
concentration of $CO_2$ since 1750".

A part of $CO_2$ emissions and in general pollutant comes from the
transport industry and in particular the
aeronautical industry. 
Hopefully for the earth,
the rarefaction of crude oil accelerates the decision making.
In particular, in the seventies, the two oil crisis showed the aeronautical 
industries its dependence toward energy resources. 
To face this issue, the U.S. Senate directed NASA in 1975
to look for every potential fuel-saving concept for aircraft
engines. The Advanced Turboprop
project was born~\cite{Hager1988} and led to the
concept of contra-rotating open rotor. This new
engine concept showed large fuel savings
along with higher noise emissions due to the absence of
a duct. Combined with the decrease of the price of the
barrel in the late eighties, the contra-rotating open rotor 
never reached the commercial aviation.

Today, the cost of a barrel is almost at its maximum as shown
in Fig.~\ref{fig:crude_oil_price}.
\begin{figure}[htp]
  \centering
  \includegraphics*[width=0.40\textwidth]{crude_oil_price.pdf}
  \caption{Evolution of the cost of a barel from $1861$ to $2012$, from BP~\cite{bpreview2013}.}
  \label{fig:crude_oil_price}
\end{figure}
In parallel, Airbus forecast a doubled number of passengers in
$2031$. For that reason, the European commission has set
demanding objectives for 2050 for these emissions
through the
Advisory Council for 
Aeronautics Research in Europe (ACARE):
the noise, $CO_2$ and $NO_x$ emissions should be reduced by 
$65\%$, $75\%$ and $80\%$ respectively
(Figure~\ref{fig:flightpath_2050}).
\begin{figure}[htp]
  \centering
  \includegraphics*[width=0.40\textwidth]{flightpath_2050.pdf}
  \caption{European Commission goals for the aeronautical industry.}
  \label{fig:flightpath_2050}
\end{figure}
Therefore, to allow a sustainable air transportation, new
concepts are needed for both the engines and the 
aircraft in general.
Several have emerged, among which lightweight construction
with advanced composite structure, airport collaborative decision
making with continuous climb departure and less waiting in taxi,
aerodynamically optimized wing geometries as laminar wings for instance
and finally fuel efficient engines to name but a few.
For the latter, two main types of engine are currently studied: the
High ByPass Ratio (HBPR) engine that is based on a
large diameter engine improving thus the
propulsive efficiency and the Contra-Rotating Open Rotor (CROR)
engine that relies on two rows of contra-rotating propellers
that proves its viability during experiments within the framework of
the previously mentioned Advanced Turboprop project of NASA~\cite{Hager1988}.

The industrial design of turbomachinery, and by extension contra-rotating
open rotors, is usually based on steady flow analysis, 
for which the reference simulation tool are the three-dimensio\-nal Reynolds-Averaged 
Navier--Stokes (RANS) steady computations. However, this approach finds its limits 
when unsteady phenomena become dominant. This is the case of 
contra-rotating open rotors where the interaction between the
two rotors is of prior importance. 
In such a
context, engineers now need tools to account for these effects as
early as possible in the design cycle. With the growth of
computational power, unsteady computations are entering industrial
practice, but the associated restitution time remains an obstacle for
daily basis applications.  For this reason, efficient
unsteady approaches are receiving a lot of attention. 

At CERFACS, several unsteady approaches have been investigated 
to reduce the computational time associated with the unsteady simulation of 
CROR configurations. 
These are seldom carried on the whole
circumference of the annulus due to the high computational
cost. A first approach is therefore to assume cyclic periodicity,
which allows to solve for only one blade passage and thus drastically
reduce the computational domain. 
In the turbomachinery community, the phase-lag approach has shown to be
a very efficient method to reduce the computational domain while
maintaining a good capture of the unsteady flow physics. 
In this way, \citet{Burnazzi2010} evaluated the phase-lag approach, largely
used in the turbomachinery community, and
applied to a 3D contra-rotating open rotor configuration. He showed
that the interactions between the two rotors can be retrieve, allowing
thus a large computational time reduction.
A second approach 
is to work on the time-integration algorithm to reduce
the computational cost as compared to standard time-marching techniques. To achieve
this, Fourier-based methods for periodic flows have undergone major
developments in the last decade (see \citet{He2010} for a recent
review).  The basic idea is to decompose
time-dependent flow variables into Fourier series, which are then
injected into the equations of the problem. The time-domain problem is
thus made equivalent to a frequency-domain problem, where the complex
Fourier coefficients are the new unknowns. At this point, two
strategies coexist to obtain the solution. The first one is to solve
directly the Fourier coefficients, using a dedicated
frequency-domain solver, as proposed by \citet{He1998}. The second strategy is to cast the
problem back to the time domain using the inverse Fourier transform, as
proposed by \citet{Hall2002} with the Harmonic
Balance (HB) method. The unsteady time-marching problem is thus
transformed into a set of steady equations coupled by a source term
that is a high-order spectral evaluation of the time-derivative of the
initial equations. The main advantage of solving in the time domain is
that it can be implemented in an existing classical RANS solver,
taking advantage of all classical convergence-accelerating techniques
for steady state problems.
\citet{ThesisSicot} implemented
the HB method into the \emph{elsA} CFD code 
that is used at CERFACS. Applied to turbomachinery
configurations, this method showed a computational gain
of one to two orders of magnitude 
compared to classical time-marching approaches.
Applied to CROR configurations, \citet{Yabili2010}
showed that the computational time reduction
was not conclusive. In fact, a large number of 
harmonics compared to turbomachinery configurations
was needed to properly capture the unsteadinesses, lowering
the computational gain.
Therefore, \citet{ThesisFrancois} deeply
investigated the different unsteady approaches available 
for turbomachinery computations and applied it to CROR simulations
among which the HB approach. 
He confirmed that
the harmonic balance method can retrieve unsteady
flow features for a reduced cost at a gain that
is relatively smaller compared to what was
obtained on former turbomachinery applications.
In parallel, \citet{ThesisGuedeney} extended the harmonic
balance approach to a multi-frequential framework. 
This method allows then to compute unsteadinesses whose frequencies
are not harmonically related.
\newline

Several challenges, such as aerodynamic,
aeroacoustic and aeroelasticity are still open 
for contra-rotating open rotor
to become a viable engine for the next generation aircraft.
In this PhD thesis, we propose to assess the aeroelasticity of 
contra-rotating open rotor by using the multi-frequential
harmonic balance developed by \citet{ThesisGuedeney}.
In fact, the main unsteadinesses of the flow field
are known to be correlated with the so-called
blade passing frequency. This frequency depends on the
rotation speed of the rotor and the number of blades
of the opposite rotor. In contrast to that, the 
frequency that drives the aeroelasticity of CROR
blades depends on their structural properties.
As such, the frequencies of both the aerodynamic
field and the aeroelasticity are not harmonically
related and justifies the use of the multi-frequential
formulation of the harmonic balance approach.


The aim of this PhD thesis is to assess the
multi-frequential harmonic balance
to estimate the flutter properties of contra-rotating open rotor
configurations. In this way, the memoir is divided in three parts:
\begin{itemize}
	\item \hyperref[part1]{\emph{Part I}} presents general information on 
	contra-rotating open rotors (\hyperref[cha:cror]{\emph{Chapter~1}}),
	the basic equations that govern the aeroelasticity of
	turbomachinery/CROR are then presented and the approach retained 
	to simulate it is detailed (\hyperref[cha:ael]{\emph{Chapter~2}}).
	Finally the mathematical mechanisms that allow the derivation
	of Fourier-based time methods and their underlying properties
	(\hyperref[cha:spectral_methods]{\emph{Chapter~3}}) are presented and the 
	multi-frequential harmonic balance approach is chosen.
	\item \hyperref[part2]{\emph{Part II}} 
	presents the advantages and limitations
	of Fourier-based time methods. The chosen approach, namely
	the harmonic balance, is validated for linear and non-linear
	equations in \hyperref[cha:validation_hb]{\emph{Chapter~4}}. Both the
	mono-frequential and the multi-frequential formulations
	are shown to give spectral accuracy.
	It is emphasized
	that a large CPU gain can be expected in the
	case of contra-rotating open rotor aeroelasticity. 
	Sadly, when using the multi-frequential harmonic
	balance, mathematical properties can lead to divergence
	of the computation through a high condition number
	(\hyperref[cha:limitations_condition_number]{\emph{Chapter~5}}). This is 
	first highlighted on the toy problems and then solved using
	an original optimization algorithm.
	Finally, the convergence of the harmonic balance 
	that was shown to be case dependent is
	assessed (\hyperref[cha:limitations_convergence]{\emph{Chapter~6}}). 
	It is demonstrated that the difference in computational
	gain is linked to the thickness of the wakes observed behind
	contra-rotating open rotor blades. Based on this observation,
	a prediction tool is developed to estimate the
	number of harmonics needed to compute a given turbomachinery (and CROR)
	configuration from a mixing plane computation in 
	\hyperref[cha:model_tbm]{\emph{Chapter~7}}. 
	The relative CPU gain to be expected can thus be estimated
	and help the decision making in choosing an unsteady approach
	over another one.
	\item based on the work done in the second part,
	the proposed approach retained in this thesis, 
	namely the multi-frequential harmonic balance method along with a weak 
	aeroelastic coupling approach, is applied on different configurations
	in \hyperref[part3]{\emph{Part III}}. Firstly, it is validated against experimental 
	results and other numerical approaches found in the
	literature on a reference configuration
	(\hyperref[cha:stcf11]{\emph{Chapter~8}}). This give us confidence
	to apply the approach on an industrial isolated contra-rotating
	open rotor application at low-speed (\hyperref[cha:dream_ls_isolated]{\emph{Chapter~9}})
	and high-speed (\hyperref[cha:dream_hs_isolated]{\emph{Chapter~10}})
	flight conditions. The aeroelastic results are discussed based
	on the computed unsteady flow field.
\end{itemize}
