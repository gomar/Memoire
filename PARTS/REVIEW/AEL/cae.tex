%!TEX root = ../../../adrien_gomar_phd.tex

Solving equation~\ref{eq:ael_motion_eq} analytically is generally 
not feasible. In fact, in turbomachinery, 
the flow exhibits non-linear features such as turbulence, shock and
boundary-layer interaction, to name but a few, that are out of reach for
analytical methods.

Two main strategies exist then for solving equation~\ref{eq:ael_motion_eq}:
the strong-coupling and the weak-coupling. The strong-coupling 
approach solves either the equation directly or two solvers are coupled and 
compute the aerodynamic and structural response of the system, respectively.
The strong coupling remains computationally expensive~\cite{Bartels2007}
and numerically stiff~\cite{Datta2008}.
It is therefore not used in this thesis.

In opposite, the weak-coupling approach has been widely used
in the turbomachinery aeroelasticity community~\cite{Marshall1996}.
This method uses a modal approach to identify the structural modes.
These modes are then prescribed with an harmonic motion in the aerodynamic
flow solver. The aerodynamic force is then post-processed to 
analyze if it amplifies the motion of the blade or damps it.

\subsection{Modal analysis}
\label{sub:modal_analysis}

The aerodynamic force $F(t)$ and
the structural damping matrix $D$ are considered to be zero
and Eq.~\eqref{eq:ael_motion_eq} becomes:
\begin{equation}
	M \ddot{x}(t) + K x(t) = 0.
	\label{eq:ael_motion_eq_free_response}
\end{equation}
Considering now that the displacement vector $x(t)$ is harmonic
yields the eigen-value problem:
\begin{equation}
	\mdet \left(K - \omega^2 M  \right) = 0.
	\label{eq:ael_motion_eq_eigen_value}
\end{equation}
The solution of this equation are the modes $\psi_r$
and their frequency $\omega_r$, verifying:
\begin{equation}
	\left(K - \omega_r^2 M  \right) \psi_r = 0.
\end{equation}
The modes define a modal basis 
$\Psi = [\psi_0, \psi_1, \dots \psi_n]$.
Once the modal basis
is identified, either by mean of a Finite
Element model or an experimental identification, 
equation~\ref{eq:ael_motion_eq} becomes:
\begin{equation}
  \label{eq:2}
  M_m \ddot{q}(t) + D_m \dot{q}(t) + K_m q (t) - \Psi^\top F(t)=0, \quad x(t) = \Psi q(t).
\end{equation}
$M_m$, $D_m$ and $K_m$ are the modal mass, 
damping and stiffness, respectively.
The weak coupling approach assumes the linearity of the response of
the fluid with respect to the displacement of the structure. Therefore
small displacements are assumed and the so-called Generalized
Aerodynamic Forces (GAF) are linearized, which adds aerodynamic
stiffness~$K_A$ and damping~$D_A$:
\begin{equation}
  \label{eq:4}
  \Psi^\top F(t) = D_A\dot{q}(t) + K_A q(t).
\end{equation}
In order to estimate the unsteady aerodynamic forces $F(t)$, 
a fluid simulation is run with a prescribed complex harmonic motion of the
structure:
\begin{equation}
  \label{eq:6}
  q(t) = \widehat{q} e^{(p t + \sigma)}
\end{equation}
A stability analysis is then performed in the frequency domain
which leads to:
\begin{equation}
  \label{eq:5}
  q=\widehat{q}e^{(p t + \sigma)}\Rightarrow\left(
    p^2M + p(D-D_A) + (K-K_A)
  \right)\widehat{q}=0,
\end{equation}
where the Laplace variable $p$ is of the form
$p=i\omega(1+i\alpha)$. Finally, considering only weakly damped or
amplified modes (i.e. $|\alpha| \ll 1$), the damping of the
fluid/structure coupled system reads $\alpha=-\Re e(p)/\Im m(p)$.


\subsection{Structural dynamics of turbomachinery blade}
\label{sub:structural_dynamics_of_turbomachinery_blade}

The modes are classified by their global shape: 
bending/flexion (noted F) and torsion (noted T) 
modes are the main ones. Then they are classified
depending on the number of deflection lines that they
have. If one deflection line is present in a flexion 
mode, it is called 1F and 2F if two deflection lines are
seen, as shown in Fig.~\ref{fig:blade_mode_shape}.
\begin{figure}[htp]
  \centering
  \includegraphics*[width=0.40\textwidth]{blade_mode_shape.pdf}
  \caption{Blade mode shape nomenclature.}
  \label{fig:blade_mode_shape}
\end{figure}

\subsection{Phase theorem}
\label{sub:lane_theorem}

The interblade phase angle has been demonstrated
to be very important for the flutter response of a blade~\cite{


\subsection{S curve}
\label{sub:s_curve}
