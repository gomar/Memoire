%!TEX root = ../../../adrien_gomar_phd.tex

Solving equation~\ref{eq:ael_motion_eq} analytically is generally 
not feasible. In fact, in turbomachinery, 
the flow exhibits non-linear features such as turbulence, shock and
boundary-layer interaction, to name but a few, that are out of reach for
analytical methods.

Two main strategies exist then for solving equation~\ref{eq:ael_motion_eq}:
the strong-coupling and the weak-coupling strategies. The strong-coupling 
approach solves either the equation directly or two solvers are coupled and 
compute the aerodynamic and structural response of the system, respectively.
The strong coupling remains computationally expensive~\cite{Bartels2007}
and numerically stiff~\cite{Datta2008}.
It is therefore not used in this thesis.

In opposite, the weak-coupling approach has been widely used
in the turbomachinery aeroelasticity community~\cite{Marshall1996}.
This method uses a modal approach to identify the structural modes.
These modes are then prescribed with an harmonic motion in the aerodynamic
flow solver. The aerodynamic force is then post-processed to 
analyze if it amplifies the motion of the blade or damps it.

\subsection{Modal analysis}
\label{sub:modal_analysis}

To identify the structural modes, the aerodynamic force $f(t)$ and
the structural damping matrix $D$ are considered to be zero
and Eq.~\eqref{eq:ael_motion_eq} becomes:
\begin{equation}
	M \ddot{x}(t) + K x(t) = 0.
	\label{eq:ael_motion_eq_free_response}
\end{equation}
Considering now that the displacement vector $x(t)$ is harmonic
yields the eigen-value problem:
\begin{equation}
	\mdet \left(K - \omega^2 M  \right) = 0.
	\label{eq:ael_motion_eq_eigen_value}
\end{equation}
The solution of this equation are the modes $\psi_r$
and their associated pulsations $\omega_r$, verifying:
\begin{equation}
	\left(K - \omega_r^2 M  \right) \psi_r = 0.
\end{equation}
The modes define a modal basis 
$\Psi = [\psi_0, \psi_1, \dots, \psi_n]$.
Once it
is identified, either by mean of a Finite
Element model or an experimental identification, 
equation~\ref{eq:ael_motion_eq} becomes:
\begin{equation}
  \label{eq:2}
  M_m \ddot{q}(t) + D_m \dot{q}(t) + K_m q (t) - \Psi^\top f(t)=0, \quad x(t) = \Psi q(t).
\end{equation}
$M_m$, $D_m$ and $K_m$ are the modal mass, 
damping and stiffness, respectively expressed as:
\begin{equation}
    M_m = \Psi ^{-1} M, \quad D_m = \Psi ^{-1} D, \quad K_m = \Psi ^{-1} K, \Psi ^{-1}  = \Psi ^{T}.
\end{equation}
As the modes are, by definition, orthogonal,
$M_m$, $D_m$ and $K_m$ are diagonal matrices and
equation~\eqref{eq:2} is a system of completely decoupled equations.

\subsection{Structural dynamics of turbomachinery blade}
\label{sub:structural_dynamics_of_turbomachinery_blade}

The modes are classified by their global shape, among which 
bending/flexion (noted~F) and torsion (noted~T) 
modes are the main ones. Then they are classified
depending on the number of deflection lines that they
have. If one deflection line is present in a flexion 
mode, it is called 1F and 2F if two deflection lines are
seen, as shown in Fig.~\ref{fig:blade_mode_shape}.
\begin{figure}[htp]
  \centering
  \includegraphics*[width=0.40\textwidth]{blade_mode_shape.pdf}
  \caption{Blade mode shape nomenclature.}
  \label{fig:blade_mode_shape}
\end{figure}

\subsection{Phase theorem}
\label{sub:lane_theorem}

In 1956, \citet{Lane1956} 
demonstrated for the case of small vibration amplitude,
that each blade in a turbomachinery vibrates with
identical modal amplitudes with a constant Inter-Blade
Phase Angle (IBPA) sometimes noted $\sigma$. For a rotor with $B$ blades,
the possible values are:
\begin{equation}
    \fbox{$\textrm{IBPA} [^\circ] = \displaystyle \frac{360 \times n_d}{B}$} \quad n_d \in [0, B-1],
\end{equation}
where $n_d$ is the nodal diameter.
A zero degree value IBPA means that the blades are vibration in phase, a $180^\circ$ or
$-180^\circ$ IBPA means that the blades vibrates in phase opposition.

\subsection{Weak-coupling approach}
\label{sub:weak_coupling_approach}

The modes being identified, these are prescribed
with a small vibration amplitude and a harmonic motion.
Due to the phase theorem, the easiest way to express
the mode is ti use a complex notation.
The displacement vector projected on the modal basis becomes then:
\begin{equation}
   \widehat{x}(t) = (h_r + i h_i) e^{i \omega t},
   \label{eq:harm_vib_displ_vector}
\end{equation}
where $h_r$ and $h_i$ are the real and imaginary  displacement
mode, respectively.
As the motion is harmonic, the fluid response is
supposed to be harmonic too.
In particular, the aerodynamic force $f (t)$ exerted by the fluid is due to the
static pressure and can be expressed as:
\begin{equation}
    \widehat{f}(t) = (p_r + i p_i) S e^{i \omega t}.
\end{equation}
The damping can then be computed by considering the 
work per cycle $W_c$ defined as:
\begin{equation}
    W_c = \int_0^T \dot{x} (t) \times f(t) \diff t, \quad T = \frac{2 \pi}{\omega}.
\end{equation}
Using the complex approach:
\begin{equation}
    W_c = \int_0^T \Re (\dot{\widehat{x}} (t)) \times \Re (\widehat{f}(t)) \diff t.
\end{equation}
The development of this equation leads to:
\begin{equation}
    W_c = \pi S \left[h_r p_i - h_i p_r \right]
\end{equation}
According to \citet{Carta1967}, the aerodynamic 
damping $\delta$ can be expressed using the
work per cycle $W_c$, which gives:
\begin{equation}
    \fbox{$
    \delta [-] = - \displaystyle \frac{\pi S \left[h_r p_i - h_i p_r \right]}{2 M_m \omega^2}.
    $}
\end{equation}
The mechanical damping $D_m$ is difficult to estimate
but is negligible compared to the aerodynamic damping~\cite{Mikolajczak1975}.
Therefore, estimating only the aerodynamic damping is the discriminant test case.

\subsection{Stability curve}
\label{sub:s_curve}

The damping as a function of the IBPA, sometimes
referred as the stability or S curve, is used to
display the aeroelastic results. It is shown in
Fig.~\ref{fig:s-curve}. The shape of this curve is
known to display an S for most of the
turbomachinery configurations. 
In our simulations, we will check that this empirical
statement is observed.
The negatively damped modes are said to
be unstable and can be subject to flutter. 
The least stable modes are usually at low IBPA.
\begin{figure}[htp]
  \centering
  \includegraphics*[width=0.40\textwidth]{s-curve.pdf}
  \caption{Stability curve.}
  \label{fig:s-curve}
\end{figure}
