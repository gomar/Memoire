%!TEX root = ../../../adrien_gomar_phd.tex

\subsection{What is aeroelasticity ?}
\label{sub:ael_what_is_aeroelasticity}

The aeroelasticity, also called dynamic aeroelasticity,
is the interaction of three forces:
the aerodynamic ($\mathcal{A}$), the elastic ($\mathcal{E}$) and
the inertial forces ($\mathcal{I}$) as 
shown in Fig.~\ref{fig:ael_collar_triangle}. 
This is thus a 
multi-physic problem involving fluid dynamics and
solid mechanics.
\begin{figure}[htb]
  \centering
  \includegraphics*[width=0.40\textwidth]{collar_triangle.pdf}
  \caption{Collar diagram.}
  \label{fig:ael_collar_triangle}
\end{figure} 

\subsection{Equation of motion}
\label{sub:ael_equation}

The governing equation of the dynamic aeroelasticity is
the combination of the fluid dynamics and the solid mechanics
equations:
\begin{equation}
	m \ddot{x} + c \dot{x} + k x = F(t)
	\label{eq:ael_motion_eq}
\end{equation}
where $m$ is the mass, $c$ the damping, $k$ the stiffness, $x$ the deformation
coordinate and $F(t)$ the aerodynamic force, which is governed
by the Navier--Stokes equations. In this equation, $m$
and $k$ are characteristics of the material. 

