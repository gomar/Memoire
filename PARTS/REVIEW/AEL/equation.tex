%!TEX root = ../../../adrien_gomar_phd.tex

From a structural point of view, the dynamic aeroelasticity
of turbomachineries is governed by:
\begin{equation}
	M \vec{\ddot{x}} + d \vec{\dot{x}} + k \vec{x} = \vec{F}(t)
	\label{eq:ael_motion_eq}
\end{equation}
where $m$, $d$ and $k$ are the mass, the damping 
and the stiffness matrices, respectively.
$\vec{x}$ and $\vec{F}(t)$ denote the displacement 
and aerodynamic force vectors.

Solving these equations analytically is not feasible
within a turbomachinery. In fact, in this system, 
the flow exhibits non-linear features like turbulence, shock,
boundary-layer interaction that are out of reach for
analytical methods. Moreover, on the structural viewpoint,
the shape of the blades can also encounter non-linearities
due to large deformations for instance.

In the literature, 