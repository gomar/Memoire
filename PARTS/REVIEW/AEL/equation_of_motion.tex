%!TEX root = ../../../adrien_gomar_phd.tex

\subsection{Structural equation}
\label{sub:structural_equation}

From a structural point of view, 
the dynamic aeroelasticity is governed by:
\begin{equation}
	M \ddot{x}(t) + D \dot{x}(t) + K x(t) = F(t)
	\label{eq:ael_motion_eq}
\end{equation}
where $M$, $D$ and $K$ are the structural mass, damping 
and stiffness matrices, respectively.
$x(t)$ and $F(t)$ denote the displacement 
and aerodynamic force vectors, respectively. The displacement
vector is defined relatively to the 
steady state position of the system. In turbomachinery
and by extension in CRORs, it is the steady state position
in rotation.

\subsection{Solving the equation of motion}
\label{sub:solving_eq_ael}

Solving this equations analytically is generally 
not feasible. In fact, in turbomachinery, 
the flow exhibits non-linear features such as turbulence, shock,
boundary-layer interaction that are out of reach for
analytical methods.

One way to solve for Eq.~\eqref{eq:ael_motion_eq}
is to first perform a free-response analysis.
In this analysis, the aerodynamic force $F(t)$ and
the structural damping matrix $D$ are considered to be zero.
Eq.~\eqref{eq:ael_motion_eq} becomes then:
\begin{equation}
	M \ddot{x}(t) + K x(t) = 0.
	\label{eq:ael_motion_eq_free_response}
\end{equation}
Considering now that the displacement vector $x(t)$ is harmonic
yields the eigen-value problem:
\begin{equation}
	\mdet \left(K - \omega^2 M  \right) = 0.
	\label{eq:ael_motion_eq_eigen_value}
\end{equation}
The solution of this equation are the modes $\psi_r$
and their frequency $\omega_r$, verifying:
\begin{equation}
	\left(K - \omega_r^2 M  \right) \psi_r = 0.
\end{equation}
The modes define a modal basis 
$\Psi = [\psi_0, \psi_1, \dots \psi_n]$.

\subsection{Modal analysis}
\label{sub:modal_analysis}

Once the modal basis
is identified, either by mean of a Finite
Element model or an experimental identification, 
equation~\ref{eq:ael_motion_eq} becomes:
\begin{equation}
  \label{eq:2}
  M_m \ddot{q}(t) + D_m \dot{q}(t) + K_m q (t) - \Psi^\top F(t)=0, \quad x(t) = \Psi q(t).
\end{equation}
$M_m$, $D_m$ and $K_m$ are the modal mass, 
damping and stiffness, respectively.
The weak coupling approach assumes the linearity of the response of
the fluid with respect to the displacement of the structure. Therefore
small displacements are assumed and the so-called Generalized
Aerodynamic Forces (GAF) are linearized, which adds aerodynamic
stiffness~$K_A$ and damping~$D_A$:
\begin{equation}
  \label{eq:4}
  \Psi^\top F(t) = D_A\dot{q}(t) + K_A q(t).
\end{equation}
In order to estimate the unsteady aerodynamic forces $F(t)$, 
a fluid simulation is run with a prescribed harmonic motion of the
structure:
\begin{equation}
  \label{eq:6}
  q(t)=\cos(\omega t).
\end{equation}
A stability analysis is then performed in the frequency domain:
\begin{equation}
  \label{eq:5}
  q=\hat{q}e^{p t}\Rightarrow\left(
    p^2M + p(D-D_A) + (K-K_A)
  \right)\hat{q}=0,
\end{equation}
where the Laplace variable $p$ is of the form
$p=i\omega(1+i\alpha)$. Finally, considering only weakly damped or
amplified modes (i.e. $|\alpha| \ll 1$), the damping of the
fluid/structure coupled system reads $\alpha=-\Re e(p)/\Im m(p)$.

\subsection{Structural dynamics of turbomachinery blade}
\label{sub:structural_dynamics_of_turbomachinery_blade}

The modes are classified by their global shape: 
bending/flexion (noted F) and torsion (noted T) 
modes are the main ones. Then they are classified
depending on the number of deflection lines that they
have. If one deflection line is present in a flexion 
mode, it is called 1F and 2F if two deflection lines are
seen, as shown in Fig.~\ref{fig:blade_mode_shape}.
\begin{figure}[htp]
  \centering
  \includegraphics*[width=0.40\textwidth]{blade_mode_shape.pdf}
  \caption{Blade mode shape nomenclature.}
  \label{fig:blade_mode_shape}
\end{figure}


