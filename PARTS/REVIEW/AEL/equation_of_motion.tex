%!TEX root = ../../../adrien_gomar_phd.tex

\subsection{Structural equation}
\label{sub:structural_equation}

From a structural point of view, 
the dynamic aeroelasticity is governed by:
\begin{equation}
	m \vec{\ddot{x}} + d \vec{\dot{x}} + k \vec{x} = \vec{F}(t)
	\label{eq:ael_motion_eq}
\end{equation}
where $m$, $d$ and $k$ are the structural mass, damping 
and stiffness matrices, respectively.
$\vec{x}$ and $\vec{F}(t)$ denote the displacement 
and aerodynamic force vectors, respectively. The displacement
vector is defined relatively to the 
steady state position in rotation.
Solving this equations analytically is generally 
not feasible. In fact, in turbomachinery, 
the flow exhibits non-linear features such as turbulence, shock,
boundary-layer interaction that are out of reach for
analytical methods.

One way to solve for Eq.~\eqref{eq:ael_motion_eq}
is to first perform a free-response analysis.
In this analysis, the aerodynamic force $F(t)$ and
the structural damping matrix $d$ are considered to be zero.
Eq.~\eqref{eq:ael_motion_eq} becomes then:
\begin{equation}
	m \vec{\ddot{x}} + k \vec{x} = 0.
	\label{eq:ael_motion_eq_free_response}
\end{equation}
Considering now that the displacement vector $x(t)$ is harmonic
yields the eigen-value problem:
\begin{equation}
	\mdet \left(k - \omega^2 m  \right) = 0.
	\label{eq:ael_motion_eq_eigen_value}
\end{equation}
The solution of this equation are the modes $\psi_r$
and their frequency $\omega_r$, verifying:
\begin{equation}
	\left(k - \omega_r^2 m  \right) \psi_r = 0.
\end{equation}

\subsection{Modal analysis}
\label{sub:modal_analysis}

Ones the modes of the blade $\psi_r$ are identified,
a modal analysis can be performed.


\subsection{Structural dynamics of turbomachinery blade}
\label{sub:structural_dynamics_of_turbomachinery_blade}

The modes are classified by their global shape: 
bending/flexion (noted F) and torsion (noted T) 
modes are the main ones. Then they are classified
depending on the number of deflection lines that they
have. If one deflection line is present in a flexion 
mode, it is called 1F and 2F if two deflection lines are
seen, as shown in Fig.~\ref{fig:blade_mode_shape}.
\begin{figure}[htbp]
  \centering
  \includegraphics*[width=0.40\textwidth]{blade_mode_shape.pdf}
  \caption{Blade mode shape nomenclature.}
  \label{fig:blade_mode_shape}
\end{figure}


