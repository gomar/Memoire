%!TEX root = ../../../adrien_gomar_phd.tex
\chapter{Introduction to aeroelasticity}
\label{cha:ael}

\chabstract{In this chapter, the basic elements
to understand aeroelasticity in turbomachinery and by extension
in CROR are detailed. Firstly, the
definition and the basic equations governing dynamic
aeroelasticity are presented. The two main aeroelastic
phenomena that develop in turbomachinery,
forced response and flutter, are then presented.
The latter is investigated in this thesis and the computational
approach retained to compute it, namely the weak-coupling approach,
is presented. The variables that are used to quantify the 
flutter boundary are finally presented.}


\newpage

\section{What is aeroelasticity}
\label{sec:what_is_ael}
%!TEX root = ../../../adrien_gomar_phd.tex

The aeroelasticity, also called dynamic aeroelasticity,
is the interaction of three forces:
the aerodynamic ($\mathcal{A}$), the elastic ($\mathcal{E}$) and
the inertial forces ($\mathcal{I}$) as 
shown in Fig.~\ref{fig:ael_collar_triangle}. 
\begin{figure}[htb]
  \centering
  \includegraphics*[width=0.40\textwidth]{collar_triangle.pdf}
  \caption{Collar diagram.}
  \label{fig:ael_collar_triangle}
\end{figure} 
This is thus a 
multi-physical problem involving the field of fluid dynamics and
solid mechanics. 

\section{Main aeroelastic phenomena in turbomachinery}
\label{sec:ael_phenomena}
%!TEX root = ../../../adrien_gomar_phd.tex

% \subsection{Static aeroelasticity}
% \label{sub:static_aeroelasticity}

% Consider a row of a turbomachinery that is in
% rotation at a given speed. Due to the centrifugal forces
% and the structural and elastic properties of the blade, this
% last assumes a deformed position. This is static aeroelasticity.
% As the shape of the blade changes, it is not
% optimal anymore for the inflow conditions 
% and a loss of efficiency is to be expect.
% From an engineering standpoint, the problem is thus inverse:
% what should be the rigid design of a blade so that under 
% loads, the shape is optimum?

% When the forces acting on the blade
% are too high to achieve a static equilibrium, divergence can occur,
% which is a destructive event.
% However, according to \citet{Marshall1996}, the stiffness of the
% materials used in turbomachinery is large enough to
% prevent from static divergence in turbomachines.
% This phenomenon will not be studied in this work.

\subsection{Forced response}
\label{sub:forced_response}

As shown previously in Sec.~\ref{sec:cror_unsteady}, wakes and
potentials effects give rise to unsteady fluctuations in 
CROR configurations. These fluctuations 
can generate large vibration levels on the blades.
When the assembly modes are excited by the rotation speed
or its multiples, resonance can occur,
hence the term forced response. 
The frequency associated to the rotation speed or its multiples
is called Engine Order (EO).
At design, one step to minimize forced response is
to use the Campbell diagram show in Fig.~\ref{fig:campbell}
which schematically represents such resonance.
Blue points shows the crossing of engine order and 
the blade eigenfrequencies within the operating range. 
The Campbell diagram does not give any information of
the absolution level of vibration. Therefore, it is mostly
used to rank the potential designs~\cite{Marshall1996}.
\begin{figure}[htbp]
  \centering
  \includegraphics*[width=0.40\textwidth]{campbell.pdf}
  \caption{Campbell diagram with forced response (blue circles)
  and flutter behavior (red stars).}
  \label{fig:campbell}
\end{figure}



\subsection{Flutter}
\label{sub:flutter}

Flutter is defined as a self-excited, unstable 
self-sustained vibration of a blade in turbomachinery. 
One of the most impressive
manifestation of flutter occurred November 7\textsuperscript{th}, 1940.
Four month after being build, the bridge experienced 
torsional flutter excited by a $64$ \mbox{km/h} wind.
The 1T and 2T modes were observed.
A few hours latter, the bridge felt down as seen in 
Fig.~\ref{fig:tacoma_bridge}. Hopefully, no human
was injured, but this event showed the importance
of taking into account the flutter phenomenon as
it is very energetic and can be a destructive event.
\begin{figure}[htb]
  \centering
  \subfigure[Torsion mode]{
      \includegraphics[height=.3\textwidth]{tac06.png}}
  \subfigure[Failure of the bridge]{
      \includegraphics[height=.3\textwidth]{tac09.png}}
  \caption{Tacoma Narrows bridge flutter, from \citet{Smith1974}.}
  \label{fig:tacoma_bridge}
\end{figure}

Three vibration scenarios can appear for flutter.
The first scenario is the damped (or positively damped) 
vibration meaning
that the vibration amplitude decreases with respect to time, 
as shown in Fig.~\ref{fig:flutter_damped}.
This is the most wanted behavior as the system tends to
as stable point. In this case, the blade is said to
be flutter-free.
The second scenario is the amplified (or negatively damped)
vibration as shown in Fig.~\ref{fig:flutter_amplified}. 
This was the scenario that most likely occurred for the
previous example of the Tacoma bridge. This scenario ultimately
leads to failure which is not acceptable. Furthermore, as
detailed in Sec.~\ref{sec:cror_challenges}, the blades of 
a CROR shall not fail otherwise the aircraft might
be destroyed.
The last scenario is the Limit Cycle Oscillation (LCO) vibration.
In this scenario, the deformation increases until a certain 
amplitude and then stays constant. This scenario is not
destructive by essence compared to the amplified scenario. However,
if the blade is repetitively excited by LCO, the blade
can fail as a consequence of the structure fatigue.
\begin{figure}[htb]
  \centering
  \subfigure[damped]{
      \label{fig:flutter_damped}
      \includegraphics[width=.3\textwidth]{flutter_damped.pdf}}
  \subfigure[amplified]{
      \label{fig:flutter_amplified}
      \includegraphics[width=.3\textwidth]{flutter_amplified.pdf}}
  \subfigure[Limit cycle oscillation]{
      \label{fig:LCO}
      \includegraphics[width=.3\textwidth]{LCO.pdf}}
  \caption{Different vibration scenario for the flutter phenomenon.}
\end{figure}

The development of one scenario over another one is linked to
the fluid response to the vibration of the blade. In fact,
if the aerodynamic loads projected on the direction of the vibration
is positive, this means that the displacement will be amplified. 
In opposite, if the force is in opposed direction, the vibration will be damped.
The out-of-phase component of the aerodynamic force compared to
the displacement vector will give finally the sign of the aerodynamic damping.
The amplitude will give its strength. 




\section{\texorpdfstring{\underline{C}}{C}omputational 
\texorpdfstring{\underline{A}}{A}ero\texorpdfstring{\underline{E}}{E}lasticity (CAE)}
\label{sec:cae}
%!TEX root = ../../../adrien_gomar_phd.tex

Solving equation~\ref{eq:ael_motion_eq} analytically is generally 
not feasible. In fact, in turbomachinery, 
the flow exhibits non-linear features such as turbulence, shock and
boundary-layer interaction, to name but a few, that are out of reach for
analytical methods.

Two main strategies exist then for solving equation~\ref{eq:ael_motion_eq}:
the strong-coupling and the weak-coupling. The strong-coupling 
approach solves either the equation directly or two solvers are coupled and 
compute the aerodynamic and structural response of the system, respectively.
The strong coupling remains computationally expensive~\cite{Bartels2007}
and numerically stiff~\cite{Datta2008}.
It is therefore not used in this thesis.

In opposite, the weak-coupling approach has been widely used
in the turbomachinery aeroelasticity community~\cite{Marshall1996}.
This method uses a modal approach to identify the structural modes.
These modes are then prescribed with an harmonic motion in the aerodynamic
flow solver. The aerodynamic force is then post-processed to 
analyze if it amplifies the motion of the blade or damps it.

\subsection{Modal analysis}
\label{sub:modal_analysis}

The aerodynamic force $f(t)$ and
the structural damping matrix $D$ are considered to be zero
and Eq.~\eqref{eq:ael_motion_eq} becomes:
\begin{equation}
	M \ddot{x}(t) + K x(t) = 0.
	\label{eq:ael_motion_eq_free_response}
\end{equation}
Considering now that the displacement vector $x(t)$ is harmonic
yields the eigen-value problem:
\begin{equation}
	\mdet \left(K - \omega^2 M  \right) = 0.
	\label{eq:ael_motion_eq_eigen_value}
\end{equation}
The solution of this equation are the modes $\psi_r$
and their frequency $\omega_r$, verifying:
\begin{equation}
	\left(K - \omega_r^2 M  \right) \psi_r = 0.
\end{equation}
The modes define a modal basis 
$\Psi = [\psi_0, \psi_1, \dots, \psi_n]$.
Once the modal basis
is identified, either by mean of a Finite
Element model or an experimental identification, 
equation~\ref{eq:ael_motion_eq} becomes:
\begin{equation}
  \label{eq:2}
  M_m \ddot{q}(t) + D_m \dot{q}(t) + K_m q (t) - \Psi^\top f(t)=0, \quad x(t) = \Psi q(t).
\end{equation}
$M_m$, $D_m$ and $K_m$ are the modal mass, 
damping and stiffness, respectively expressed as:
\begin{equation}
    M_m = \Psi ^ T M, \quad D_m = \Psi ^ T D, \quad K_m = \Psi ^ T K.
\end{equation}
As the modes are, by definition, orthogonal,
$M_m$, $D_m$ and $K_m$ are diagonal matrices and
equation~\eqref{eq:2} is a system of completely decoupled equations.

\subsection{Structural dynamics of turbomachinery blade}
\label{sub:structural_dynamics_of_turbomachinery_blade}

The modes are classified by their global shape: 
bending/flexion (noted~F) and torsion (noted~T) 
modes are the main ones. Then they are classified
depending on the number of deflection lines that they
have. If one deflection line is present in a flexion 
mode, it is called 1F and 2F if two deflection lines are
seen, as shown in Fig.~\ref{fig:blade_mode_shape}.
\begin{figure}[htp]
  \centering
  \includegraphics*[width=0.40\textwidth]{blade_mode_shape.pdf}
  \caption{Blade mode shape nomenclature.}
  \label{fig:blade_mode_shape}
\end{figure}

\subsection{Phase theorem}
\label{sub:lane_theorem}

In 1956, \citet{Lane1956} 
demonstrated analytically that each blade in a turbomachinery vibrates with
identical modal amplitudes with a constant InterBlade
Phase Angle (IBPA) sometimes noted $\sigma$. For a rotor with $B$ blades,
the possible values are:
\begin{equation}
    \fbox{$\textrm{IBPA} [^\circ] = \displaystyle \frac{360 \times n_d}{B}$} \quad n_d \in [0, B-1],
\end{equation}
where $n_d$ is the nodal diameter.
A zero degree value IBPA means that the blades are vibration in phase, a $180^\circ$ or
$-180^\circ$ IBPA means that the blades vibrates in phase opposition.

\subsection{Weak-coupling approach}
\label{sub:weak_coupling_approach}

The modes being identified, these are prescribed
with a small vibration amplitude and a harmonic motion.
Due to the phase theorem, the easiest way to express
the mode is ti use a complex notation.
The displacement vector projected on the modal basis becomes then:
\begin{equation}
   \widehat{x}(t) = (h_r + i h_i) e^{i \omega t},
   \label{eq:harm_vib_displ_vector}
\end{equation}
where $h_r$ and $h_i$ are the real and complex displacement
mode, respectively.
Note that we use a complex approach here.
As the motion is harmonic, the fluid response is
supposed to be harmonic too.
In particular, the aerodynamic force $f (t)$ exerted by the fluid is due to the
static pressure and can be expressed as:
\begin{equation}
    \widehat{f}(t) = (p_r + i p_i) S e^{i \omega t}.
\end{equation}
The damping can then be computed by considering the 
work per cycle $W_c$ defined as:
\begin{equation}
    W_c = \int_0^T \dot{x} (t) \times f(t) \diff t, \quad T = \frac{2 \pi}{\omega}.
\end{equation}
Using the complex approach:
\begin{equation}
    W_c = \int_0^T \Re (\dot{\widehat{x}} (t)) \times \Re (\widehat{f}(t)) \diff t.
\end{equation}
The development of this equation leads to:
\begin{equation}
    W_c = \pi S \left[h_r p_i - h_i p_r \right]
\end{equation}
According to \citet{Carta1967}, the aerodynamic 
damping $\delta$ can be expressed using the
work per cycle $W_c$, which gives to:
\begin{equation}
    \fbox{$
    \delta [-] = - \displaystyle \frac{\pi S \left[h_r p_i - h_i p_r \right]}{2 M_m \omega^2}.
    $}
\end{equation}
The mechanical damping $D_m$ is difficult to estimate
but is negligible compared to the aerodynamic damping~\cite{Mikolajczak1975}.
Therefore, estimating only the aerodynamic damping makes sense.

\subsection{Stability curve}
\label{sub:s_curve}

The damping as a function of the IBPA, sometimes
referred as the stability or S curve, is used to
display the aeroelastic results. It is shown in
Fig.~\ref{fig:s-curve}. The shape of this curve is
known to display an S for most of the
turbomachinery configuration. 
In our simulations, we will check that this empirical
statement is observed.
The negatively damped modes are said to
be unstable and can be subject to flutter. 
The least stable modes are usually at low IBPA.
\begin{figure}[htp]
  \centering
  \includegraphics*[width=0.40\textwidth]{s-curve.pdf}
  \caption{Stability curve shape for turbomachinery}
  \label{fig:s-curve}
\end{figure}


