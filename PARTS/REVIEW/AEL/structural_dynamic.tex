%!TEX root = ../../../adrien_gomar_phd.tex

One way to solve for Eq.~\eqref{eq:ael_motion_eq}
is to first perform a free-response analysis.
In this analysis, the aerodynamic force $F(t)$ and
the structural damping matrix $d$ are considered to be zero.
Eq.~\eqref{eq:ael_motion_eq} becomes then:
\begin{equation}
	m \vec{\ddot{x}} + k \vec{x} = 0.
	\label{eq:ael_motion_eq_free_response}
\end{equation}
Considering now that the displacement vector $x(t)$ is harmonic
yields the eigen-value problem:
\begin{equation}
	\mdet \left(k - \omega^2 m  \right) = 0.
	\label{eq:ael_motion_eq_eigen_value}
\end{equation}
The solution of this equation are the modes $\psi_r$
and their frequency $\omega_r$, verifying:
\begin{equation}
	\left(k - \omega_r^2 m  \right) \psi_r = 0.
\end{equation}

The modes are classified by their general shape: 
bending/flexion (noted F) and torsion (noted T) 
modes are the main ones. Then they are classified
depending on the number of deflection lines that they
have. If one deflection line is present in a flexion 
mode, it is called 1F and 2F if two deflection lines are
seen, as shown in Fig.~\ref{fig:blade_mode_shape}.
\begin{figure}[htbp]
  \centering
  \includegraphics*[width=0.40\textwidth]{blade_mode_shape.pdf}
  \caption{Blade mode shape nomenclature.}
  \label{fig:blade_mode_shape}
\end{figure}
