%!TEX root = ../../../adrien_gomar_phd.tex

The study of aeroelasticity in turbomachines takes its origin
in the first engines failure during the sixties~\cite{Dugundji2003}.
Also called dynamic aeroelasticity,
it is the interaction between three forces:
the aerodynamic ($\mathcal{A}$), the elastic ($\mathcal{E}$) and
the inertial forces ($\mathcal{I}$) as 
shown by the \citet{Collar1946} triangle represented in 
Fig.~\ref{fig:ael_collar_triangle}. 
\begin{figure}[htp]
  \centering
  \includegraphics*[width=0.40\textwidth]{collar_triangle.pdf}
  \caption{Collar triangle.}
  \label{fig:ael_collar_triangle}
\end{figure}

From a structural point of view, 
the dynamic aeroelasticity is governed by:
\begin{equation}
	M \ddot{x}(t) + D \dot{x}(t) + K x(t) = f(t)
	\label{eq:ael_motion_eq}
\end{equation}
where $M$, $D$ and $K$ are the structural mass, damping 
and stiffness matrices, respectively.
$x(t)$ and $f(t)$ denote the displacement 
and aerodynamic force vectors, respectively. The displacement
vector is defined relatively to the 
steady state position of the system. In turbomachinery
and by extension in CRORs, it is the steady state position
in rotation.
