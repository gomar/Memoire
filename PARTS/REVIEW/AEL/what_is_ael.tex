%!TEX root = ../../../adrien_gomar_phd.tex

Aeroelasticity, also called dynamic aeroelasticity,
is the interaction between three forces:
the aerodynamic ($\mathcal{A}$), the elastic ($\mathcal{E}$) and
the inertial forces ($\mathcal{I}$) as 
shown in Fig.~\ref{fig:ael_collar_triangle}. 
\begin{figure}[htp]
  \centering
  \includegraphics*[width=0.40\textwidth]{collar_triangle.pdf}
  \caption{Collar diagram.}
  \label{fig:ael_collar_triangle}
\end{figure}

The improvement of engine consumption goes through a 
decrease of the mass of the engine and a better aerodynamic.
The latter is obtained using 3D shaped blades while the former
comes with lightweight blades. In the framework of
aeroelasticity, improving the engines means pushing the boundaries
of aeroelasticity. In fact lightweight blades subject
to strong aerodynamic effects are more prone to deflect as 
$\mathcal{A}$ and $\mathcal{I}$ forces become bigger.
