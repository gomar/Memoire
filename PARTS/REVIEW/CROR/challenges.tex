%!TEX root = ../../../adrien_gomar_phd.tex

\paragraph{Classification}
Figure~\ref{fig:cror_challenges} depicts the current challenges associated
with CROR configurations. Three main fields are involved: the aerodynamics, the
aeroelastics and the aeroacoustics fields.
\begin{figure}[htbp]
  \centering
  \includegraphics*[width=0.60\textwidth]{challenges.pdf}
  \caption{Challenges raised by contra-rotating open rotor configurations.}
  \label{fig:cror_challenges}
\end{figure}
The first challenge is on the aerodynamic of the CROR. In fact, theoretically, 
the CROR is meant to have a better propulsive efficiency than a turbofan and a
propeller. However, as it is a new architecture, studies need to be conducted
to understand the flow physic that develops within it.
\citet{Hendricks2011} developed an open-rotor cycle model based
on experimental performance characteristics made at NASA. This is 
an empiric approach that suffers of the absence of new designs. To
overcome this, \citet{Bechet2011} used a lifting-line code to
initialize a gradient optimization procedure based on mixing-plane
computations. This last step allows to gain almost a half point of
the CROR efficiency. This is more general than an empiric strategy
if the mixing-plane computations are reliable to assess the performance
parameters of CROR. This study has been performed by \citet{Zachariadis2011}.
They compared the performance prediction of mixing plane computations
to experimental data made on a open-rotor test case.
The agreement is fair for the thrust and power coefficients.
As small discrepancies are present, these are amplified when computing
the efficient which basically the ratio of the two coefficients.
\citet{Vion2011} and \citet{Stuermer2008} used unsteady
CFD computations to assess the unsteady performance and flow features.
\citet{Stuermer2008} \citet{Francois2013} demonstrated through a code to code comparison
that CFD was mature enough to estimate the 1P~loads.

Lot of research effort is put on the third challenge which
is aeroacoustic. In fact, in the late eighties at NASA, \citet{Hager1988}
conducted a large project on innovative propulsion systems for the
next generation aircrafts. The potential of the CROR configuration
was identified but the noise emitted was so high that the solution
offered was to put noise limiter in the fuselage. It results in 
an increase of the weight. Together with the decrease of the
barrel in the late eighties, the CROR never reached the commercial
aviation.
