%!TEX root = ../../../adrien_gomar_phd.tex

Several challenges are still open for CROR
to become a viable engine for the next generation aircraft.
In this way, we classify and describe each of them in the following sections.

\paragraph{Classification}
Figure~\ref{fig:cror_challenges} depicts current challenges associated
with CROR configurations. Three main fields are involved: aerodynamics,
aeroacoustics and aeroelasticity.
\begin{figure}[htp]
  \centering
  \includegraphics*[scale=0.8]{challenges.pdf}
  \caption{Challenges raised by contra-rotating open rotor configurations.}
  \label{fig:cror_challenges}
\end{figure}

\paragraph{Aerodynamics}
Theoretically, 
the CROR is meant to have a better propulsive efficiency than a turbofan or a
propeller. However, as it is a new architecture, studies need to be conducted
to understand its flow physics. In particular,
aerodynamic interactions between the two rotors need to be better understood.

The research on the aerodynamic of the CRORs is divided in two main
axis: the first axis deals with the design of CRORs while the second
analyzes the unsteady flow physics that develop on given design.

Toward the first axis, 
\citet{Hendricks2011} developed an open-rotor cycle model based
on experimental performance characteristics made at NASA. This is 
an empiric approach that suffers from the impossibility to build new designs.
\citet{Peters2012} developed a similar code to design their CROR. The aeroacoustic
characteristics of the final design is assessed by a 
full annulus unsteady simulation even though the design is 
based on experimental correlations.
To improve the approach to design new CRORs, 
\citet{Bechet2011} used a lifting-line code to
initialize a gradient optimization procedure based on mixing-plane
computations. This led to a gain of almost a half point
in CROR efficiency. This is more general than an empiric strategy
if the mixing-plane computations are reliable to assess the performance
parameters of CROR. 

Toward the second axis, \citet{Zachariadis2011}
compared the performance prediction of mixing plane computations
to experimental data made on an open-rotor test case.
They found a fair agreement for the thrust and power coefficients, however
small discrepancies on the coefficients led to significant errors on their ratio,
\emph{i.e.} the efficiency.
\citet{Vion2011} and \citet{Stuermer2008} used unsteady
CFD computations to assess the unsteady performance and flow features.
\citet{Stuermer2008} and \citet{Francois2013} demonstrated through a code to code comparison
that CFD was mature enough to estimate in-plane forces.

\paragraph{Aeroacoustics}
Lot of research efforts are put on the second challenge which
is aeroacoustic since the absence of a duct allows noise generated
by CROR to propagate far away.
In the late eighties, \citet{Hager1988}
conducted at NASA a large project on innovative propulsion systems for the
next generation aircrafts. The potential of the CROR configuration
was identified but the noise emitted was so high that the only way
thought to use such an engine was to put noise liners in the fuselage. This resulted in 
increased weight. This is why, today, a lot of research effort is put on the
understanding and mastering of noise sources in CRORs.
Two main types of noise have been identified: tonal noise which comes from
the interaction of both rotors and is mainly present at low-speed flight conditions 
(namely take-off and landing)
and broadband noise which comes from turbulence and is predominant
at high-speed flight conditions (namely cruise).
Several CFD studies have been performed in the literature.
\citet{Peters2012} showed that unsteady CFD simulation is able
to reproduce the aeroacoustic footprint of a CROR. They then optimized
their CROR and showed that this optimized CROR design may be mature enough
for noise certification. \citet{Hoffer2012} and \citet{Ferrante2013}
developed an efficient CFD approach to simulate the aeroacoustics of CRORs.
It is based on a Fourier-based time method. The approach is able to
account for incidence effects which is particularly interesting
considering that the noise of installed configuration is drastically
different from the isolated one (see \citet{Hager1988}).

\paragraph{Aeroelasticity}
The third challenge is the less studied in the numerical literature.
Two main aeroelastic phenomena have been identified during preliminary studies
during the eighties by \citet{Hager1988}: whirl flutter, \emph{i.e.} the self-excited
movement of the whole nacelle, and blade flutter, \emph{i.e.} the vibration
of the blades.
For a turbofan engine to achieve certification, it must be 
demonstrated that one released fan blade can be safely contained 
within the engine’s fan case as written in the 
Certification Specifications for Engines (CSE) of the EASA:
\begin{quote}
	"It must be demonstrated that any single compressor or turbine blade will be contained after Failure and 
that no Hazardous Engine Effect can arise as a result of other Engine damage likely to occur before 
Engine shut down following a blade Failure"
\end{quote}
In the case of propellers and contra-rotating open rotors, due to the absence of a nacelle,
this can not be done. To achieve certification, it must be demonstrated that the probability of a blade
failure (or any failure) should not exceed $1e^{-8}$ per propeller flight hour as written in 
the Certification Specifications for Propellers (CSP) of the EASA:
\begin{quote}
	"It must be shown that Hazardous Propeller Effects will not occur at a rate in excess of that defined 
as Extremely Remote. The estimated probability for individual failures may be insufficiently precise 
to enable the total rate for Hazardous Propeller Effects to be assessed. For Propeller certification, it 
is acceptable to consider that the intent of this paragraph is achieved if the probability of a 
Hazardous Propeller Effect arising from an individual failure can be predicted to be not greater than 
$1e^{-8}$ per Propeller flight hour. It will also be accepted that, in dealing with probabilities of this low 
order of magnitude, absolute proof is not possible and reliance must be placed on engineering 
judgment and previous experience combined with sound design and test philosophies" 
\end{quote}
This explains why aeroelasticity of contra-rotating open rotors should be assessed.
To the author's knowledge, only whirl flutter has been investigated
in the CROR literature by \citet{CISicot2011a} and 
\citet{Verley2013} and these studies mainly
discuss the simulation tools needed to compute such a phenomenon as
no experimental data are available.
