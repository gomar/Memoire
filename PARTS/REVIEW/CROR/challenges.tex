%!TEX root = ../../../adrien_gomar_phd.tex

\paragraph{Classification}
Figure~\ref{fig:cror_challenges} depicts the current challenges associated
with CROR configurations. Three main fields are involved: the aerodynamics, the
aeroelastics and the aeroacoustics fields.
\begin{figure}[htbp]
  \centering
  \includegraphics*[width=0.60\textwidth]{challenges.pdf}
  \caption{Challenges raised by contra-rotating open rotor configurations.}
  \label{fig:cror_challenges}
\end{figure}

\paragraph{Aerodynamic}
Theoretically, 
the CROR is meant to have a better propulsive efficiency than a turbofan and a
propeller. However, as it is a new architecture, studies need to be conducted
to understand the flow physic that develops within it. In particular,
the aerodynamic interaction between the two rotors needs to be understand.
\citet{Hendricks2011} developed an open-rotor cycle model based
on experimental performance characteristics made at NASA. This is 
an empiric approach that suffers from the impossibility to build new designs.
\citet{Peters2012} developed an equivalent code to design their CROR but the
final design is assessed for aeroacoustic purposes by full annulus unsteady simulations.
To improve this, \citet{Bechet2011} used a lifting-line code to
initialize a gradient optimization procedure based on mixing-plane
computations. This last step allows to gain almost a half point of
the CROR efficiency. This is more general than an empiric strategy
if the mixing-plane computations are reliable to assess the performance
parameters of CROR. This study has been performed by \citet{Zachariadis2011}.
They compared the performance prediction of mixing plane computations
to experimental data made on a open-rotor test case.
The agreement is fair for the thrust and power coefficients.
As small discrepancies are present, these are amplified when computing
the efficient which basically the ratio of the two coefficients.
\citet{Vion2011} and \citet{Stuermer2008} used unsteady
CFD computations to assess the unsteady performance and flow features.
\citet{Stuermer2008} \citet{Francois2013} demonstrated through a code to code comparison
that CFD was mature enough to estimate the in-plane forces.

\paragraph{Aeroacoustic}
Lot of research effort is put on the third challenge which
is aeroacoustic due to the absence of a duct. 
In fact, in the late eighties at NASA, \citet{Hager1988}
conducted a large project on innovative propulsion systems for the
next generation aircrafts. The potential of the CROR configuration
was identified but the noise emitted was so high that the solution
offered was to put noise limiter in the fuselage. It results in 
an increase of the weight. Together with the decrease of the price of the
barrel in the late eighties, the CROR never reached the commercial
aviation. This is why, today, a lot of research effort is put on the
understanding and mastering of noise sources in CRORs.
Several CFD studies have been performed in the literature.
\citet{Peters2012} showed that unsteady CFD simulation is able
to reproduce the aeroacoustic footprint of a CROR. They then optimized
their CROR and showed that optimized CROR design may be mature enough
for noise certification. \citet{Hoffer2012} and \citet{Ferrante2013}
developed an efficient CFD approach to simulation the aeroacoutic of CRORs.
It is based on time Fourier-based method. The method is able to
take into account for incidence effects.

\paragraph{Aeroelastic}
The aeroelastic challenge is less studied in the literature.
To the author knowledge, only whirl flutter has been investigated
in the CROR literature by \citet{Verley2013} and this study mainly
discusses the simulation tools needed to compute such a phenomena as
no experimental data are available.
