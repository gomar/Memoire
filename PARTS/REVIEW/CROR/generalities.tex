%!TEX root = ../../../adrien_gomar_phd.tex

% Consider the general form of the momentum equation
% \begin{equation}
% 	\frac{\partial}{\partial t} \iiint_\mathcal{V} \rho \vec{V} \diff \mathcal{V}
% 	+ \iint_S \left(\rho \vec{V} \cdot \diff \vec{s} \right) \vec{V} 
% 	= - \iint_S P \diff \vec{s}
% 	- \iint_S \tau_p \diff \vec{s}
% 	+ \iiint_\mathcal{V} \rho \vec{F} \diff \mathcal{V},
% \end{equation}
% where $\rho$ is the density, $\vec{V}$ the velocity vector, $P$ the pressure force, 
% $\tau_p$ the viscosity force and $\vec{F}$ the volumetric forces (gravity for instance).

% Consider now a control volume representative of a turbojet engine.
% Moreover, we suppose that:
% \begin{itemize}
% 	\item the pressure $P_\infty$ outside this control volume is constant,
% 	\item the viscosity effects and volumetric forces are negligible 
% compared to the pressure forces,
% 	\item the flow is steady.
% \end{itemize}
% Under these assumption, the momentum equation becomes:
% \begin{equation}
% 	\iint_S \left(\rho \vec{V} \cdot \diff \vec{s} \right) \vec{V} 
% 	= - \iint_S P \diff \vec{s}
% \end{equation}
% The computation of the surface integral can be decomposed as:
% \begin{equation}
% 	\iint_S = \iint_{AB} + \iint_{BC} + \iint_{CD} + \iint_{DA}.
% \end{equation}
% Between $AD$ and $BC$, $\vec{V}$ is orthogonal to $\diff \vec{s}$
% and projecting the equation on the $x$ axis gives:
% \begin{equation}
% 	\begin{split}
% 		\iint_S \left(\rho \vec{V} \cdot \diff \vec{s} \right) \vec{V} \cdot \vec{x}
% 		&=
% 		\iint_{AB} \left(\rho \vec{V} \cdot \diff \vec{s} \right) \vec{V} \cdot \vec{x} +
% 		\iint_{CD} \left(\rho \vec{V} \cdot \diff \vec{s} \right) \vec{V} \cdot \vec{x} \\
% 		&= - \rho_{\infty} V_{\infty} S_{1} (- V_{\infty}) +
% 		\rho_{2} V_{2} S_{2} (- V_{2})
% 	\end{split}
% \end{equation}

Consider the conservative equation of momentum:
\begin{equation}
	\frac{\partial \rho \vec{V}}{\partial t} 
	+ \nabla \cdot (\rho \vec{V} \otimes \vec{V} + p \mathbb{I} - \vec{\vec{\Sigma}}_v) = 0
\end{equation}
where $\rho$ is the density, $\vec{V}$ the velocity vector, $p$ the pressure and
$\vec{\vec{\Sigma}}_v$ the remaining forces (viscosity, gravity and so on).
Applying this equation on two closed domains $\Sigma_i$ and $\Sigma_o$ as
shown in Fig.~\ref{fig:cror_control_volume}.
\begin{figure}[htb]
  \centering
  \includegraphics*[width=0.30\textwidth]{control_volume.pdf}
  \caption{Domains used for the application of the momentum equation.}
  \label{fig:cror_control_volume}
\end{figure}
The domain $\Sigma_o$ represents a fluid domain outside from the
engine encompass by the solid domain $\Sigma_i$.
Taking a steady state hypothesis, one can write:
\begin{equation}
	\oint_{\Sigma_i} \left(\rho \vec{V} \otimes \vec{V} + 
	                       p \mathbb{I} - 
	                       \vec{\vec{\Sigma}}_v \right) \cdot \vec{n} \diff S
    =
   	\oint_{\Sigma_o} \left(\rho \vec{V} \otimes \vec{V} + 
	                       p \mathbb{I} - 
	                       \vec{\vec{\Sigma}}_v \right) \cdot \vec{n} \diff S.
\end{equation} 
As $\Sigma_o$ is an arbitrary domain, we can take it sufficiently
away from the engine so that $\vec{\vec{\Sigma}}_v$ becomes zero.
Moreover, 
\begin{equation}
	\oint_{\Sigma_i} \left(\rho \vec{V} \otimes \vec{V} \right) \cdot \vec{n} \diff S = 0
\end{equation}
since the surface is solid. If $\vec{F}$ denotes
\begin{equation}
	\vec{F} = \oint_{\Sigma_i} \left(p \mathbb{I} - 
	\vec{\vec{\Sigma}}_v \right) \cdot \vec{n} \diff S,
\end{equation}
then
\begin{equation}
	\vec{F} = \oint_{\Sigma_o} \left(\rho \vec{V} \otimes \vec{V} +
	p \mathbb{I} \right) \cdot \vec{n} \diff S,
\end{equation}
This last equation is vectorial, meaning that all efforts on 
$\oint_{\Sigma_i}$ are taken into account.

The thrust $F_x$ of an engine can be written as:
\begin{equation}
	F_x = \dot{m} \Delta V,
\end{equation}
where $m$ is the mass-flow rate going through the
propeller and $\Delta V$ is
the increment of axial velocity.


