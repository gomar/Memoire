%!TEX root = ../../../adrien_gomar_phd.tex

\subsection{The thrust formula}
\label{sub:cror_thrust}

Consider the conservative equation of momentum:
\begin{equation}
	\frac{\partial \rho \vec{V}}{\partial t} 
	+ \nabla \cdot (\rho \vec{V} \otimes \vec{V} + p \mathbb{I} - \vec{\vec{\Sigma}}_v) = 0
\end{equation}
where $\rho$ is the density, $\vec{V}$ the velocity vector, $p$ the pressure and
$\vec{\vec{\Sigma}}_v$ the remaining forces (viscosity, gravity and so on).
Applying this equation on two closed domains $\Sigma$ and $\Sigma^\prime$ as
shown in Fig.~\ref{fig:cror_control_volume}.
\begin{figure}[htb]
  \centering
  \includegraphics*[width=0.30\textwidth]{control_volume.pdf}
  \caption{Domains used for the application of the momentum equation.}
  \label{fig:cror_control_volume}
\end{figure}
The domain $\Sigma^\prime$ represents a fluid domain outside from the
engine encompass by the solid domain $\Sigma$.
Taking a steady state hypothesis, one can write:
\begin{equation}
	\oint_{\Sigma} \left(\rho \vec{V} \otimes \vec{V} + 
	                       p \mathbb{I} - 
	                       \vec{\vec{\Sigma}}_v \right) \cdot \vec{n} \diff S
    =
   	\oint_{\Sigma^\prime} \left(\rho \vec{V} \otimes \vec{V} + 
	                       p \mathbb{I} - 
	                       \vec{\vec{\Sigma}}_v \right) \cdot \vec{n} \diff S.
\end{equation} 
As $\Sigma^\prime$ is an arbitrary domain, we can take it sufficiently
away from the engine so that $\vec{\vec{\Sigma}}_v$ becomes zero.
Moreover, 
\begin{equation}
	\oint_{\Sigma} \left(\rho \vec{V} \otimes \vec{V} \right) \cdot \vec{n} \diff S = 0
\end{equation}
since the surface is solid. If $\vec{F}$ denotes
\begin{equation}
	\vec{F} = \oint_{\Sigma} \left(p \mathbb{I} - 
	\vec{\vec{\Sigma}}_v \right) \cdot \vec{n} \diff S,
\end{equation}
then
\begin{equation}
	\vec{F} = \oint_{\Sigma^\prime} \left(\rho \vec{V} \otimes \vec{V} +
	p \mathbb{I} \right) \cdot \vec{n} \diff S,
\end{equation}
This last equation is vectorial, meaning that all efforts on 
$\oint_{\Sigma}$ are taken into account.
If $\Sigma^\prime$ is a stream tube, and projecting the equation
onto the $x$-axis gives the formula for the thrust $F_x$:
\begin{equation}
	F_x = Q V_{out} + p_{out} S_{out}
	- Q V_{in} - p_{in} S_{in},
\end{equation}
using the notation of Fig.~\ref{fig:cror_control_volume}.

Far downstream of the engine, 
the thrust $F_x$ can be written as:
\begin{equation}
	F_x = \dot{m} \Delta V_x,
\end{equation}
where $m$ is the mass-flow rate going through the
propeller and $\Delta V_x$ is
the increment of axial velocity.

\subsection{The propulsive efficiency}
\label{sub:cror_efficiency}

The global propulsive efficiency $\eta$ measures the 
success in converting a mechanical power into a
propulsive power. It results from the combination
of the cycle efficiency $\eta_{C}$ and the propulsive efficiency
$\eta_{PR}$ as shown in Fig.~\ref{fig:cror_efficiency}
\begin{figure}[htb]
  \centering
  \includegraphics*[width=0.40\textwidth]{efficiency.pdf}
  \caption{Efficiency relations from mechanical power to propulsive power.}
  \label{fig:cror_efficiency}
\end{figure}

\paragraph{Cycle efficiency}
The cycle efficiency measures the success in converting the mechanical
input into a kinetic energy. The mechanical power delivered as input
can be computed through the first principle of thermodynamic. In absence
of heat exchange, 
\begin{equation}
	P_m = \dot{m} \left(hi_{in} - hi_{out} \right)
\end{equation}
