%!TEX root = ../../../adrien_gomar_phd.tex

\subsection{Thrust formula}
\label{sub:cror_thrust}

Consider the conservative equation of momentum:
\begin{equation}
	\frac{\partial \rho \vec{V}}{\partial t} 
	+ \nabla \cdot (\rho \vec{V} \otimes \vec{V} + p \mathbb{I} - \vec{\vec{\Sigma}}_v) = 0
\end{equation}
where $\rho$ is the density, $\vec{V}$ the velocity vector, $p$ the pressure and
$\vec{\vec{\Sigma}}_v$ the remaining forces (viscosity, gravity and so on).
Applying this equation on two closed domains $\Sigma$ and $\Sigma^\prime$ as
shown in Fig.~\ref{fig:cror_control_volume}.
\begin{figure}[htb]
  \centering
  \includegraphics*[width=0.30\textwidth]{control_volume.pdf}
  \caption{Domains used for the application of the momentum equation.}
  \label{fig:cror_control_volume}
\end{figure}
The domain $\Sigma^\prime$ represents a fluid domain outside from the
engine encompass by the solid domain $\Sigma$.
Taking a steady state hypothesis, one can write:
\begin{equation}
	\oint_{\Sigma} \left(\rho \vec{V} \otimes \vec{V} + 
	                       p \mathbb{I} - 
	                       \vec{\vec{\Sigma}}_v \right) \cdot \vec{n} \diff S
    =
   	\oint_{\Sigma^\prime} \left(\rho \vec{V} \otimes \vec{V} + 
	                       p \mathbb{I} - 
	                       \vec{\vec{\Sigma}}_v \right) \cdot \vec{n} \diff S.
\end{equation} 
As $\Sigma^\prime$ is an arbitrary domain, we can take it sufficiently
away from the engine so that $\vec{\vec{\Sigma}}_v$ becomes zero.
Moreover, 
\begin{equation}
	\oint_{\Sigma} \left(\rho \vec{V} \otimes \vec{V} \right) \cdot \vec{n} \diff S = 0
\end{equation}
since the surface is solid. If $\vec{F}$ denotes
\begin{equation}
	\vec{F} = \oint_{\Sigma} \left(p \mathbb{I} - 
	\vec{\vec{\Sigma}}_v \right) \cdot \vec{n} \diff S,
\end{equation}
then
\begin{equation}
	\vec{F} = \oint_{\Sigma^\prime} \left(\rho \vec{V} \otimes \vec{V} +
	p \mathbb{I} \right) \cdot \vec{n} \diff S,
\end{equation}
This last equation is vectorial, meaning that all efforts on 
$\oint_{\Sigma}$ are taken into account.
If $\Sigma^\prime$ is a stream tube, and projecting the equation
onto the $x$-axis gives the formula for the thrust $F_x$:
\begin{equation}
	F_x = Q V_{out} + p_{out} S_{out}
	- Q V_{in} - p_{in} S_{in},
\end{equation}
using the notation of Fig.~\ref{fig:cror_control_volume}.

Far downstream of the engine, 
the thrust $F_x$ can be written as:
\begin{equation}
	F_x = \dot{m} \Delta V_x,
	\label{eq:cror_thrust}
\end{equation}
where $m$ is the mass-flow rate going through the
propeller and $\Delta V_x$ is
the increment of axial velocity.

\subsection{Global propulsive efficiency}
\label{sub:cror_efficiency}

The global propulsive efficiency $\eta$ measures the 
success in converting a mechanical power into a
propulsive power. It results from the combination
of the cycle efficiency $\eta_{C}$ and the propulsive efficiency
$\eta_{PR}$:
\begin{equation}
	\eta = \eta_{C} \times \eta_{PR}.
\end{equation}
This is schematically represented in Fig.~\ref{fig:cror_efficiency}.
\begin{figure}[htb]
  \centering
  \includegraphics*[width=0.40\textwidth]{efficiency.pdf}
  \caption{Efficiency relations from mechanical power to propulsive power.}
  \label{fig:cror_efficiency}
\end{figure}


\paragraph{Cycle efficiency}
The cycle efficiency measures the success in converting the mechanical
power $P_m$ into a kinetic power $P_k$. The mechanical power delivered as input
can be computed through the first principle of thermodynamic. In fact, in absence
of heat exchange, the mechanical power can be estimated as:
\begin{equation}
	P_m = \dot{m} (h_{i_{out}} - h_{i_{in}}),
\end{equation}
where $h_i$ is the total enthalpy.
The kinetic power is given by:
\begin{equation}
	P_k = \dot{m} \left(\frac{1}{2} V^2_{out} -
	\frac{1}{2} V^2_{in} \right).
\end{equation}
This leads to a cycle efficiency that can be expressed as:
\begin{equation}
	\eta_{C} = \frac{V_{out} - V_{in}}{2 (h_{i_{out}} - h_{i_{in}})}
\end{equation}

\paragraph{Propulsive efficiency}
The propulsive efficiency $\eta_{PR}$ measures the success
in creating a propulsive power $P_{pr}$ from a
kinetic power $P_k$:
\begin{equation}
	\eta_{PR} = \frac{P_{pr}}{P_k}
\end{equation}
The propulsive power is computed using the thrust $F_x$:
\begin{equation}
	P_{pr} = F_x \cdot V_{\infty},
\end{equation}
where $V_{\infty}$ is the free-stream velocity.
Finally, if the free-stream velocity is the inlet velocity $V_{in}$
and the inlet and outlet velocities are purely axial:
\begin{equation}
	\eta_{PR} = \frac{1}{1 + \frac{V_{out} - V_{in}}{2 V_{in}}}
	\label{eq:cror_propulsive_efficiency}
\end{equation}

\subsection{Toward propeller engines}
\label{sub:cror_toward_propeller}

The combination of Eq.~\eqref{eq:cror_thrust} and 
Eq.~\eqref{eq:cror_propulsive_efficiency} shows that
increasing the propulsive efficiency while keeping the same amount
of thrust can be done through a higher mass-flow rate. This is 
the basic justification of the use of propeller for low consumption engines.
In fact, as the blades are not with a nacelle, the mass-flow rate is not limited
by the architecture