%!TEX root = ../../../adrien_gomar_phd.tex

For an aircraft in steady flight conditions, 
lift balances weight and 
thrust balances drag. This explains why engineers try
indefinitely to reduce weight while increasing
thrust. A trade-off between those two is to work
on the propulsive efficiency of the engine. In this
section, general information on propulsion
that leads to the concepts of propeller and
contra-rotating open rotor are given.

\subsection{Thrust equation}
\label{sub:cror_thrust}
Consider the conservative equation of momentum
\begin{equation}
	\frac{\partial \rho \vec{V}}{\partial t} 
	+ \nabla \cdot (\rho \vec{V} \otimes \vec{V} + p \mathbb{I} - \vec{\vec{\Sigma}}_v) = 0,
\end{equation}
where $\rho$ is the density, $\vec{V}$ the velocity vector, $p$ the pressure and
$\vec{\vec{\Sigma}}_v$ the viscous stress terms.
Consider two closed domains $\Sigma$ and $\Sigma^\prime$ as
shown in Fig.~\ref{fig:cror_control_volume}.
\begin{figure}[htp]
  \centering
  \includegraphics*[width=0.30\textwidth]{control_volume.pdf}
  \caption{Domains used for the application of the momentum equation.}
  \label{fig:cror_control_volume}
\end{figure}
The domain $\Sigma^\prime$ represents a fluid domain outside from the
engine encompassed by the solid domain $\Sigma$.
Taking a steady state hypothesis, one can write
\begin{equation}
	\oint_{\Sigma} \left(\rho \vec{V} \otimes \vec{V} + 
	                       p \mathbb{I} - 
	                       \vec{\vec{\Sigma}}_v \right) \cdot \vec{n} \diff S
    =
   	\oint_{\Sigma^\prime} \left(\rho \vec{V} \otimes \vec{V} + 
	                       p \mathbb{I} - 
	                       \vec{\vec{\Sigma}}_v \right) \cdot \vec{n} \diff S,
\end{equation} 
where $\vec{n}$ is the normal vector.
As $\Sigma^\prime$ is an arbitrary domain, we can take it sufficiently
away from the engine so that $\vec{\vec{\Sigma}}_v$ becomes zero (\emph{i.e.}
viscosity stress terms are null).
Moreover, 
\begin{equation}
	\oint_{\Sigma} \left(\rho \vec{V} \otimes \vec{V} \right) \cdot \vec{n} \diff S = 0,
\end{equation}
since the surface is solid ($\vec{V} = \vec{0}$ on wall). 
If $\vec{F}$ denotes the resultant forces acting on $\Sigma$
\begin{equation}
	\vec{F} = \oint_{\Sigma} \left(p \mathbb{I} - 
	\vec{\vec{\Sigma}}_v \right) \cdot \vec{n} \diff S,
\end{equation}
then
\begin{equation}
	\vec{F} = \oint_{\Sigma^\prime} \left(\rho \vec{V} \otimes \vec{V} +
	p \mathbb{I} \right) \cdot \vec{n} \diff S.
\end{equation}
Assuming that $\Sigma^\prime$ is a stream tube, and projecting the equation
onto the $x$-axis gives the formula for the thrust $F_x$
\begin{equation}
	F_x = \dot{m} V_{out} + p_{out} S_{out}
	- \dot{m} V_{in} - p_{in} S_{in},
\end{equation}
using the notation of Fig.~\ref{fig:cror_control_volume}.

Far downstream of the engine $S_{in} = S_{out}$ and
considering that we have an adapted nozzle ($p_{in} = p_{out}$),
the thrust $F_x$ can be written as
\begin{equation}
	\fbox{$
	F_x = \dot{m} (V_{out} - V_{in}) = \dot{m} \Delta V_x
	$}
	\label{eq:cror_thrust}
\end{equation}
where $\dot{m}$ is the mass-flow rate going through the
propeller and $\Delta V_x$ is
the increment of axial velocity. From this simple equation,
one can see that to increase the thrust $F_x$, there are two parameters:
the mass-flow and the axial velocity increment.

\subsection{Global propulsive efficiency}
\label{sub:cror_efficiency}

The global propulsive efficiency $\eta$ measures the 
success in converting a mechanical power into a
propulsive power. It results from the combination
of the kinetic efficiency $\eta_{K}$ and the propulsive efficiency
$\eta_{PR}$
\begin{equation}
	\eta = \eta_{K} \times \eta_{PR}.
\end{equation}
This is schematically represented in Fig.~\ref{fig:cror_efficiency}.
\begin{figure}[htp]
  \centering
  \includegraphics*[width=0.40\textwidth]{efficiency.pdf}
  \caption{Efficiency relations from mechanical power to propulsive power.}
  \label{fig:cror_efficiency}
\end{figure}

\paragraph{Kinetic efficiency}
The kinetic efficiency measures the success in converting the mechanical
power $P_m$ into a kinetic power $P_k$
\begin{equation}
	\eta_K = \frac{P_k}{P_m}.
\end{equation}

The mechanical power delivered as input
can be computed through the first thermodynamic principle. In fact, in absence
of heat exchange, the mechanical power $P_m$ can be estimated as
\begin{equation}
	P_m = \dot{m} (h_{i_{out}} - h_{i_{in}}),
\end{equation}
where $h_i$ is the total enthalpy and subscript $in$ and $out$ are
the input and output, respectively, of the propulsion system as represented
in Fig.~\ref{fig:cror_control_volume}.
The kinetic power $P_k$ is given by
\begin{equation}
	P_k = \dot{m} \left(\frac{1}{2} V^2_{out} -
	\frac{1}{2} V^2_{in} \right).
\end{equation}
This leads to a kinetic efficiency that can be expressed as
\begin{equation}
	\eta_{K} = \frac{V^2_{out} - V^2_{in}}{2 (h_{i_{out}} - h_{i_{in}})}
\end{equation}

\paragraph{Propulsive efficiency}
The propulsive efficiency $\eta_{PR}$ measures the success
in creating a propulsive power $P_{pr}$ from a
kinetic power $P_k$
\begin{equation}
	\eta_{PR} = \frac{P_{pr}}{P_k}.
\end{equation}
The propulsive power is computed using the thrust $F_x$
\begin{equation}
	P_{pr} = F_x \times V_{\infty},
\end{equation}
where $V_{\infty}$ is the free-stream velocity.
Finally, if the free-stream velocity is the inlet velocity $V_{in}$
and the inlet and outlet velocities are purely axial
\begin{equation}
	\fbox{$
	\eta_{PR} = \displaystyle \frac{1}{1 + \frac{V_{out} - V_{in}}{2 V_{in}}}
	$}
	\label{eq:cror_propulsive_efficiency}
\end{equation}
This formula means that the most efficient engine produces
a very small velocity increment.

\subsection{Toward propeller engines}
\label{sub:cror_toward_propeller}

One way to improve the environmental footprint of
airplanes engines is to increase the propulsive efficiency
by reducing the kinetic power needed to drive the engine.
Doing so while maintaining the thrust can be achieved through
a higher mass-flow rate. Two new concepts are thus derived from
this simple statement: the High ByPass-Ratio (HBPR) which
is basically a turbofan with a larger fan exhaust, and the
propeller whose mass-flow rate is not limited
by the architecture, as the blades are not within a nacelle.
In the following section, the propeller engine will be detailed
and the drawbacks of such an architecture will be highlighted to
motivate the use
of a second propeller row, yielding the contra-rotating open rotor
architecture.


