%!TEX root = ../../../adrien_gomar_phd.tex

In aviation, lift balances the weight and 
thrust balances the drag. Thus engineer try
indefinitely to reduce weight and increase the
thrust. A trade-off between those two is to work
on the propulsive efficiency of the engine. In this
section, general information on propulsion are given
that leads to the concepts of propeller and
contra-rotating open rotor.

\subsection{Thrust equation}
\label{sub:cror_thrust}
\todo{a reecrire en utilisant annexe de Sasha}
Consider the conservative equation of momentum:
\begin{equation}
	\frac{\partial \rho \vec{V}}{\partial t} 
	+ \nabla \cdot (\rho \vec{V} \otimes \vec{V} + p \mathbb{I} - \vec{\vec{\Sigma}}_v) = 0
\end{equation}
where $\rho$ is the density, $\vec{V}$ the velocity vector, $p$ the pressure and
$\vec{\vec{\Sigma}}_v$ the remaining forces as for instance the viscosity and
the gravity.
Consider two closed domains $\Sigma$ and $\Sigma^\prime$ as
shown in Fig.~\ref{fig:cror_control_volume}.
\begin{figure}[htb]
  \centering
  \includegraphics*[width=0.30\textwidth]{control_volume.pdf}
  \caption{Domains used for the application of the momentum equation.}
  \label{fig:cror_control_volume}
\end{figure}
The domain $\Sigma^\prime$ represents a fluid domain outside from the
engine encompassed by the solid domain $\Sigma$.
Taking a steady state hypothesis, one can write:
\begin{equation}
	\oint_{\Sigma} \left(\rho \vec{V} \otimes \vec{V} + 
	                       p \mathbb{I} - 
	                       \vec{\vec{\Sigma}}_v \right) \cdot \vec{n} \diff S
    =
   	\oint_{\Sigma^\prime} \left(\rho \vec{V} \otimes \vec{V} + 
	                       p \mathbb{I} - 
	                       \vec{\vec{\Sigma}}_v \right) \cdot \vec{n} \diff S.
\end{equation} 
As $\Sigma^\prime$ is an arbitrary domain, we can take it sufficiently
away from the engine so that $\vec{\vec{\Sigma}}_v$ becomes zero (e.g.
gravity and viscosity forces are null).
Moreover, 
\begin{equation}
	\oint_{\Sigma} \left(\rho \vec{V} \otimes \vec{V} \right) \cdot \vec{n} \diff S = 0
\end{equation}
since the surface is solid. If $\vec{F}$ denotes
\begin{equation}
	\vec{F} = \oint_{\Sigma} \left(p \mathbb{I} - 
	\vec{\vec{\Sigma}}_v \right) \cdot \vec{n} \diff S,
\end{equation}
then
\begin{equation}
	\vec{F} = \oint_{\Sigma^\prime} \left(\rho \vec{V} \otimes \vec{V} +
	p \mathbb{I} \right) \cdot \vec{n} \diff S,
\end{equation}
This last equation is vectorial, meaning that all efforts on 
$\oint_{\Sigma}$ are taken into account by $\vec{F}$.
If $\Sigma^\prime$ is a stream tube, and projecting the equation
onto the $x$-axis gives the formula for the thrust $F_x$:
\begin{equation}
	F_x = \dot{m} V_{out} + p_{out} S_{out}
	- \dot{m} V_{in} - p_{in} S_{in},
\end{equation}
using the notation of Fig.~\ref{fig:cror_control_volume}.

Far downstream of the engine, $p_{in} S_{in} = p_{out} S_{out}$, meaning
that the thrust $F_x$ can be written as:
\begin{equation}
	F_x = \dot{m} (V_{out} - V_{in}) = \dot{m} \Delta V_x,
	\label{eq:cror_thrust}
\end{equation}
where $\dot{m}$ is the mass-flow rate going through the
propeller and $\Delta V_x$ is
the increment of axial velocity. From this simple equation of the thrust,
one can see that to increase the thrust $F_x$, there are two parameters:
the mass-flow and the axial velocity increment.

\subsection{Global propulsive efficiency}
\label{sub:cror_efficiency}

The global propulsive efficiency $\eta$ measures the 
success in converting a mechanical power into a
propulsive power. It results from the combination
of the cycle efficiency $\eta_{C}$ and the propulsive efficiency
$\eta_{PR}$:
\begin{equation}
	\eta = \eta_{C} \times \eta_{PR}.
\end{equation}
This is schematically represented in Fig.~\ref{fig:cror_efficiency}.
\begin{figure}[htb]
  \centering
  \includegraphics*[width=0.40\textwidth]{efficiency.pdf}
  \caption{Efficiency relations from mechanical power to propulsive power.}
  \label{fig:cror_efficiency}
\end{figure}


\paragraph{Cycle efficiency}
The cycle efficiency measures the success in converting the mechanical
power $P_m$ into a kinetic power $P_k$:
\begin{equation}
	\eta_C = \frac{P_k}{P_m}.
\end{equation}

The mechanical power delivered as input
can be computed through the first principle of thermodynamic. In fact, in absence
of heat exchange, the mechanical power $P_m$ can be estimated as:
\begin{equation}
	P_m = \dot{m} (h_{i_{out}} - h_{i_{in}}),
\end{equation}
where $h_i$ is the total enthalpy.
The kinetic power $P_k$ is given by:
\begin{equation}
	P_k = \dot{m} \left(\frac{1}{2} V^2_{out} -
	\frac{1}{2} V^2_{in} \right).
\end{equation}
This leads to a cycle efficiency that can be expressed as:
\begin{equation}
	\eta_{C} = \frac{V^2_{out} - V^2_{in}}{2 (h_{i_{out}} - h_{i_{in}})}
\end{equation}

\paragraph{Propulsive efficiency}
The propulsive efficiency $\eta_{PR}$ measures the success
in creating a propulsive power $P_{pr}$ from a
kinetic power $P_k$:
\begin{equation}
	\eta_{PR} = \frac{P_{pr}}{P_k}
\end{equation}
The propulsive power is computed using the thrust $F_x$:
\begin{equation}
	P_{pr} = F_x \cdot V_{\infty},
\end{equation}
where $V_{\infty}$ is the free-stream velocity.
Finally, if the free-stream velocity is the inlet velocity $V_{in}$
and the inlet and outlet velocities are purely axial:
\begin{equation}
	\eta_{PR} = \frac{1}{1 + \frac{V_{out} - V_{in}}{2 V_{in}}}
	\label{eq:cror_propulsive_efficiency}
\end{equation}

\subsection{Toward propeller engines}
\label{sub:cror_toward_propeller}

The combination of Eq.~\eqref{eq:cror_thrust} and 
Eq.~\eqref{eq:cror_propulsive_efficiency} shows 
that to increase the propulsive efficiency, the outlet
velocity $V_{out}$ should decrease. Keeping the same amount
of thrust can thus be done through a higher mass-flow rate. This is 
the basic justification of the use of propeller for low consumption engines.
In fact, as the blades are not with a nacelle, the mass-flow rate is not limited
by the architecture. In opposite, the mass-flow of 
turbofan engines is limited by the size of the nacelle. The two architectures,
namely the CROR and the High ByPass-Ratio (HBPR) are currently studied by
the motorist and the aircraft manufacturers. Here, we will focus on
propeller and CROR configurations.

