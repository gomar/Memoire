%!TEX root = ../../../adrien_gomar_phd.tex
\chapter{Contra-rotating open rotors}
\label{cha:cror}

\chabstract{The advantage of propeller based engine
is demonstrated through the use of the thrust and propulsive efficiency
equations. The geometry of propellers is first presented, 
following by the main physical phenomena that develop 
in this system through, finally,
the estimation of the performance parameters. Contra-rotating
open rotors are then presented as an extension of propellers.
A specific emphasis is put on the unsteady phenomena
that develops in a CROR and the challenges associated to these
to become a viable future aircraft engine.}

\minitoc
\newpage

\section{Generalities of propulsion}
\label{sec:cror_intro}
%!TEX root = ../../../adrien_gomar_phd.tex

For an aircraft in steady flight conditions, 
lift balances weight and 
thrust balances drag. This explains why engineers try
indefinitely to reduce weight while increasing
thrust. A trade-off between those two is to work
on the propulsive efficiency of the engine. In this
section, general information on propulsion
that leads to the concepts of propeller and
contra-rotating open rotor are given.

\subsection{Thrust equation}
\label{sub:cror_thrust}
Consider the conservative equation of momentum
\begin{equation}
	\frac{\partial \rho \vec{V}}{\partial t} 
	+ \nabla \cdot (\rho \vec{V} \otimes \vec{V} + p \mathbb{I} - \vec{\vec{\Sigma}}_v) = 0,
\end{equation}
where $\rho$ is the density, $\vec{V}$ the velocity vector, $p$ the pressure and
$\vec{\vec{\Sigma}}_v$ the viscous stress terms.
Consider two closed domains $\Sigma$ and $\Sigma^\prime$ as
shown in Fig.~\ref{fig:cror_control_volume}.
\begin{figure}[htp]
  \centering
  \includegraphics*[width=0.30\textwidth]{control_volume.pdf}
  \caption{Domains used for the application of the momentum equation.}
  \label{fig:cror_control_volume}
\end{figure}
The domain $\Sigma^\prime$ represents a fluid domain outside from the
engine encompassed by the solid domain $\Sigma$.
Taking a steady state hypothesis, one can write
\begin{equation}
	\oint_{\Sigma} \left(\rho \vec{V} \otimes \vec{V} + 
	                       p \mathbb{I} - 
	                       \vec{\vec{\Sigma}}_v \right) \cdot \vec{n} \diff S
    =
   	\oint_{\Sigma^\prime} \left(\rho \vec{V} \otimes \vec{V} + 
	                       p \mathbb{I} - 
	                       \vec{\vec{\Sigma}}_v \right) \cdot \vec{n} \diff S,
\end{equation} 
where $\vec{n}$ is the normal vector.
As $\Sigma^\prime$ is an arbitrary domain, we can take it sufficiently
away from the engine so that $\vec{\vec{\Sigma}}_v$ becomes zero (\emph{i.e.}
viscosity stress terms are null).
Moreover, 
\begin{equation}
	\oint_{\Sigma} \left(\rho \vec{V} \otimes \vec{V} \right) \cdot \vec{n} \diff S = 0,
\end{equation}
since the surface is solid ($\vec{V} = \vec{0}$ on wall). 
If $\vec{F}$ denotes the resultant forces acting on $\Sigma$
\begin{equation}
	\vec{F} = \oint_{\Sigma} \left(p \mathbb{I} - 
	\vec{\vec{\Sigma}}_v \right) \cdot \vec{n} \diff S,
\end{equation}
then
\begin{equation}
	\vec{F} = \oint_{\Sigma^\prime} \left(\rho \vec{V} \otimes \vec{V} +
	p \mathbb{I} \right) \cdot \vec{n} \diff S.
\end{equation}
Assuming that $\Sigma^\prime$ is a stream tube, and projecting the equation
onto the $x$-axis gives the formula for the thrust $F_x$
\begin{equation}
	F_x = \dot{m} V_{out} + p_{out} S_{out}
	- \dot{m} V_{in} - p_{in} S_{in},
\end{equation}
using the notation of Fig.~\ref{fig:cror_control_volume}.

Far downstream of the engine $S_{in} = S_{out}$ and
considering that we have an adapted nozzle ($p_{in} = p_{out}$),
the thrust $F_x$ can be written as
\begin{equation}
	\fbox{$
	F_x = \dot{m} (V_{out} - V_{in}) = \dot{m} \Delta V_x
	$}
	\label{eq:cror_thrust}
\end{equation}
where $\dot{m}$ is the mass-flow rate going through the
propeller and $\Delta V_x$ is
the increment of axial velocity. From this simple equation,
one can see that to increase the thrust $F_x$, there are two parameters:
the mass-flow and the axial velocity increment.

\subsection{Global propulsive efficiency}
\label{sub:cror_efficiency}

The global propulsive efficiency $\eta$ measures the 
success in converting a mechanical power into a
propulsive power. It results from the combination
of the kinetic efficiency $\eta_{K}$ and the propulsive efficiency
$\eta_{PR}$
\begin{equation}
	\eta = \eta_{K} \times \eta_{PR}.
\end{equation}
This is schematically represented in Fig.~\ref{fig:cror_efficiency}.
\begin{figure}[htp]
  \centering
  \includegraphics*[width=0.40\textwidth]{efficiency.pdf}
  \caption{Efficiency relations from mechanical power to propulsive power.}
  \label{fig:cror_efficiency}
\end{figure}

\paragraph{Kinetic efficiency}
The kinetic efficiency measures the success in converting the mechanical
power $P_m$ into a kinetic power $P_k$
\begin{equation}
	\eta_K = \frac{P_k}{P_m}.
\end{equation}

The mechanical power delivered as input
can be computed through the first thermodynamic principle. In fact, in absence
of heat exchange, the mechanical power $P_m$ can be estimated as
\begin{equation}
	P_m = \dot{m} (h_{i_{out}} - h_{i_{in}}),
\end{equation}
where $h_i$ is the total enthalpy and subscript $in$ and $out$ are
the input and output, respectively, of the propulsion system as represented
in Fig.~\ref{fig:cror_control_volume}.
The kinetic power $P_k$ is given by
\begin{equation}
	P_k = \dot{m} \left(\frac{1}{2} V^2_{out} -
	\frac{1}{2} V^2_{in} \right).
\end{equation}
This leads to a kinetic efficiency that can be expressed as
\begin{equation}
	\eta_{K} = \frac{V^2_{out} - V^2_{in}}{2 (h_{i_{out}} - h_{i_{in}})}
\end{equation}

\paragraph{Propulsive efficiency}
The propulsive efficiency $\eta_{PR}$ measures the success
in creating a propulsive power $P_{pr}$ from a
kinetic power $P_k$
\begin{equation}
	\eta_{PR} = \frac{P_{pr}}{P_k}.
\end{equation}
The propulsive power is computed using the thrust $F_x$
\begin{equation}
	P_{pr} = F_x \times V_{\infty},
\end{equation}
where $V_{\infty}$ is the free-stream velocity.
Finally, if the free-stream velocity is the inlet velocity $V_{in}$
and the inlet and outlet velocities are purely axial
\begin{equation}
	\fbox{$
	\eta_{PR} = \displaystyle \frac{1}{1 + \frac{V_{out} - V_{in}}{2 V_{in}}}
	$}
	\label{eq:cror_propulsive_efficiency}
\end{equation}
This formula means that the most efficient engine produces
a very small velocity increment.

\subsection{Toward propeller engines}
\label{sub:cror_toward_propeller}

One way to improve the environmental footprint of
airplanes engines is to increase the propulsive efficiency
by reducing the kinetic power needed to drive the engine.
Doing so while maintaining the thrust can be achieved through
a higher mass-flow rate. Two new concepts are thus derived from
this simple statement: the High ByPass-Ratio (HBPR) which
is basically a turbofan with a larger fan exhaust, and the
propeller whose mass-flow rate is not limited
by the architecture, as the blades are not within a nacelle.
In the following section, the propeller engine will be detailed
and the drawbacks of such an architecture will be highlighted to
motivate the use
of a second propeller row, yielding the contra-rotating open rotor
architecture.




\section{Propeller}
\label{sec:cror_propeller}
%!TEX root = ../../../adrien_gomar_phd.tex

\subsection{Geometry}
\label{sub:cror_propeller_geometry}

A propeller is composed of a hub and a rotating set of 
$B$ blades as schematically represented in
Figure~\ref{fig:cror_propeller_geometry}. The hub
is the part on which the blades are mounted.
We set the diameter of these blades being $D$
and their rotation speed being $\Omega$. 
In front of the propeller, there is a spinner which is
a conic element that conducts the 
inflow to the propeller blades.
The propeller can be seen as
a turbofan whose fan is not within a nacelle.
\begin{figure}[htp]
  \centering
  \includegraphics*[scale=0.30]{propeller_geometry.pdf}
  \caption{Geometry of a propeller.}
  \label{fig:cror_propeller_geometry}
\end{figure}
This absence implies that theoretically, the mass-flow can be
infinite. To quantify this, it is common for engines to
consider the bypass ratio. It is defined as the ratio of the
cold air (the fan exhaust)
divided by the hot air (the air that goes through the engine core).
To give an idea, one of the highest bypass ratio engine on today's aircraft is obtained
by the Pratt~\&~Whitney~PW1000G with a~12 bypass ratio. 
This number is representative of the mass-flow rate generated by the engine.
However, we have seen that mass-flow and the velocity
difference are the two parameters that can be used to increase
the thrust. Assuming that in a classical ducted turbofan, 
the bypass ratio is limited to~12, the only
way to further increase the thrust is to increase the 
velocity which deteriorates the propulsive efficiency.
For the sake of comparison, 
propellers are estimated to have a bypass ratio of~50. 
This explains why this architecture has
regained interest.

\subsection{Velocity triangle}
\label{sub:cror_propeller_velocity_triangle}
The velocity triangle applied to a propeller configuration
is shown in Figure~\ref{fig:cror_velocity_triangle_propeller}.
The aim of a propeller is to create thrust through an increase
of the axial velocity noted $\Delta V_x$ in the diagram. To do
so, the relative flow field is straighten up. This gives both
an increase in axial velocity but also in tangential velocity.
In fact, the inflow that was purely axial retrieves a tangential
component at the outlet. This is called the swirl and
is a lost energy as it cannot be used to produce thrust.
\begin{figure}[htp]
  \centering
  \includegraphics*[scale=0.55]{velocity_triangle_propeller.pdf}
  \caption{Velocity triangle applied to a propeller.}
  \label{fig:cror_velocity_triangle_propeller}
\end{figure}
Moreover, the relative velocity $W$ should be kept subsonic
otherwise the propulsive efficiency is reduced. This limits
the free-stream velocity $V_0$ of the aircraft and the size of 
the propeller as the rotation speed velocity depends on
the radius of the blades. This explains why propellers have
been limited so far to low-speed inflow conditions.

\subsection{Similarity coefficients}
\label{sub:similarity_coefficients}
To evaluate the performance of the propeller, four similarity
coefficients are commonly used:
the advance ratio $J$ that represents the operating point of the propeller,
the thrust $C_t$ and power $C_p$ coefficients and finally
the efficiency $\eta$
\begin{equation}
    J = \frac{V_0}{n D}, \quad
    C_T = \frac{F_x}{\rho n ^ 2  D ^ 4}, \quad
    C_P = \frac{M_x \Omega}{\rho n ^ 3 D ^ 5}, \quad
    \eta = J \frac{C_T}{C_P},
\end{equation}
where $V_0$ is the free-stream velocity 
as shown in Figure~\ref{fig:cror_propeller_geometry},
$\rho$ the free-stream density,
$n$ the rotation frequency ($n = \Omega / 2 \pi$),
$F_x$ the thrust and
$M_x$ the axial torque.
The efficiency defined here is actually the global propulsive efficiency
as it gives the ratio of the propulsive power over the mechanical power.

An estimation of the variation of the advance ratio $J$ and the 
efficiency $\eta$ depending on the flight conditions is 
given by~\citet{Bousquet2012}
\begin{alignat}{4}
    \text{(cruise)} \quad  0.8 &< \eta &< 0.95, \quad 1 &< J < 3.5 \\
    \text{(take-off and landing)} \quad  0.5 &< \eta &< 0.8, \quad J &< 1.
    \label{eq:estimation_sim_coeff}
\end{alignat}

\subsection{Main physical phenomena}
\label{sub:cror_propeller_physics}

The main physical phenomena that can be seen in a propeller are schematically represented
in Figure~\ref{fig:propeller_phys_phenomena}. Firstly, due to the presence of a boundary
layer on the pressure and suction sides of the blades, a wake is shed behind each blade, which involves a momentum deficit (Figure~\ref{fig:propeller_wakes}). 
It is mostly a two-dimensional
phenomenon seen at each radius. Secondly, 
in the tip region of the blade, the pressure difference between each 
side of the blade induces a vortex that is counter-rotating with respect to 
the rotation speed (Figure~\ref{fig:propeller_tip_vortices}). 
They are advected by the local relative velocity giving them
an helical path propagating downstream.
To reduce this phenomenon, one way is to modify the geometry of the tip
of the blades.
Finally, the propeller generates thrust through an acceleration of the fluid. Thus, the stream
tube is contracted (Figure~\ref{fig:propeller_stream_tube}). 
All of these phenomena are stationary in their relative frame of reference.
\begin{figure}[htp]
  \centering
  \subfigure[wakes]{
      \label{fig:propeller_wakes}
      \includegraphics[scale=.2]{propeller_wakes.pdf}}
  \quad\subfigure[tip vortices]{
      \label{fig:propeller_tip_vortices}
      \includegraphics[scale=.2]{propeller_tip_vortices.pdf}}
  \quad\subfigure[stream tube contraction]{
      \label{fig:propeller_stream_tube}
      \includegraphics[scale=.2]{propeller_stream_tube.pdf}}
  \caption{Main physical phenomena seen in a propeller.}
  \label{fig:propeller_phys_phenomena}
\end{figure}


\section{Contra-rotating open rotors}
\label{sec:cror_cror}
%!TEX root = ../../../adrien_gomar_phd.tex

\subsection{Geometry}
\label{sub:cror_geometry}

Figure~\ref{fig:cror_geometry} depicts the main
geometrical parameters of a CROR.
It is composed of two rotors, the first one is called
the front rotor and the second one is called the rear or aft rotor.
The two rotors do not have the same diameter $D$ and rotation speed
$\Omega$. Thus, subscript $F$ and $R$ denotes respectively,
the front and the rear parameter. The diameter is expressed in meters
while the rotation speed is expressed in radians per seconds.
As the rotors are contra-rotating, their rotation speed is opposed.
The absolute value of the rotation speed is not necessarily the same,
as it depends on the chosen architecture. Therefore not assumption
will be made on the relation between the front and the rear
rotation speed.
The difference of diameters is called the clipping or cropping
of the blades and is evaluated as through the non-dimensional parameter
$\kappa$:
\begin{equation}
    \kappa = \frac{D_F - D_R}{D_F}.
\end{equation}
This is done to avoid the interaction of the tip vortex generated
by the front rotor with the rear rotor.
Finally, the spacing between the rotors
is evaluated as the difference between the axial minimum of the
rear blade minus the maximum of the front blade. This parameter
helps reducing the noise produced by the interaction of
the rotors through the potential effects.
\begin{figure}[htbp]
  \centering
  \includegraphics*[scale=0.3]{cror_geometry.pdf}
  \caption{Contra-rotating open rotor geometrical parameters.}
  \label{fig:cror_geometry}
\end{figure}

The contra-rotating open rotor architecture is a classical engine
turbomachinery whose fan is not within a nacelle. As explain
above, this help increasing the mass-flow of the primary flow
which leads to a higher propulsive efficiency.
Two main architectures are retained. One based on a gearbox, the second
being build around a statorless low-pressure turbine.
\begin{figure}[htb]
  \centering
  \subfigure[Geared design]{\includegraphics[width=.4\textwidth]{geared_cror.pdf}}
  \subfigure[Statorless low-pressure turbine design]{\includegraphics[width=.4\textwidth]{stator_less_cror.pdf}}
  \caption{Contra-rotating open rotor architectures.}
  \label{fig:cror_architectures}
\end{figure}

\subsection{Velocity triangle}
\label{sub:cror_velocity_triangle}

The basic idea behind the CROR concept is that a propeller has a 
better propulsive efficiency than a turbofan. The problem is that
a residual tangential velocity is present behind a propeller.
In fact, applying the velocity triangle to a propeller configuration
as shown in Fig.~\ref{fig:velocity_triangle_propeller}, one can
observe that the outlet velocity (noted $V^{out}_x$) is not axial
leading a tangential velocity $\Delta V_{\theta}$. This tangential forms
the swirl observe behind a propeller. First, this is a lost energy and
second, it produces a torque that has an impact on the flight dynamics
of the aircraft. To alleviate this effect, one can use two propellers
that are counter-rotating as for instance the TP$400$ propellers
in the Airbus-A$400$M military airplane but this does not recover
the swirl energy that is lost.

To do so, a second contra-rotating rotor can be used.
\begin{figure}[htbp]
  \centering
  \includegraphics*[scale=0.40]{velocity_triangle_cror.pdf}
  \caption{Velocity triangle applied to a contra-rotating open rotor configuration.}
  \label{fig:velocity_triangle_cror}
\end{figure}
Figure~\ref{fig:velocity_triangle_cror} shows the application
of the velocity triangle to a CROR configuration. The swirl
energy that was lost in the propeller is now used to 
produce more thrust. Thus, a CROR has a better propulsive
efficiency than a propeller.


\subsection{Similarity coefficients}
\label{sub:cror_similarity_coeff}

In the case of a CROR configuration, two rotors are considered.
Two main ways exists to evaluate the global value of the
similarity coefficients. The first one, chosen by
\citet{Bechet2011} among others, is to consider
that the non-dimensioning parameter $D$, $n$ and $J$ are those
of the front rotor for both rotors. 
The second one uses the non-dimensioning parameter of the current rotor,
as done by \citet{Stuermer2008} and \citet{Zachariadis2011}.
The traction and power coefficients of the rear rotor is
computed using its own advance ratio, diameter and rotation frequency.
Nevertheless, computing the advance ratio of the rear rotor, as
the free-stream velocity should be updated to take into account
for the acceleration generated by the front rotor. The free-stream
velocity is chosen to be $V_0$, which is of course not true.
The efficiency is computed rotor per rotor and then
assembled through an arithmetic summation. This is the approach retained
in this thesis.


\section{Unsteadinesses}
\label{sec:cror_unsteady}
%!TEX root = ../../../adrien_gomar_phd.tex

\subsection{From steady to unsteady phenomena}
\label{sub:cror_from_steady_to_unsteady_phenomena}

The flow that is generated behind the front rotor
is steady in its frame of reference. Nevertheless,
due to the relative speed difference between the
front and the rear rotor, these steady flow distortions are
seen as unsteady features by the rear rotor. 
These unsteadiness are correlated to the Blade Passing Frequency (BPF):
\begin{equation}
	f = \frac{\Omega_{rel} B_{opp}}{2 \pi},
\end{equation}
where $\Omega_{rel}$ is the relative speed difference between
the current and the opposite row
and $B_{opp}$ the number of blades in the opposite row.
At first order, the unsteady effects presented here drive
most of the time-dependent field in a turbomachinery.

\subsection{Main unsteadinesses}
\label{sub:cror_main_unsteadinesses}

In sec.~\ref{sub:cror_propeller_physics}, the main physical phenomena
that appears in a propeller have been introduced. As seen above, due to
the relative speed difference between the two rotors, these phenomena
that were steady in their frame of reference are now seen as unsteady features
by the rear rotor.

\paragraph{Tip vortices}

As shown previously in Fig.~\ref{fig:propeller_tip_vortices}, tip vortices are shed in the
tip of the blades due to a pressure difference between each side of the blades.
If nothing particular is done, this low momentum perturbation can
hit the rear rotor and induce large unsteady fluctuations. To avoid this,
the rear rotor blades are clipped as mentioned earlier. 
This unsteadiness is correlated with the BPF.

\paragraph{Wakes and potential effects}

Compared to an isolated rotor, as for the case of a propeller,
the presence of an opposite rotor give rise to an unsteady
interaction through the potential effects.
This is added to the already present wake distortions. This is
schematically represented in Fig.~\ref{fig:cror_wakes_potential}.
\begin{figure}[htb]
  \centering
  \includegraphics*[width=0.30\textwidth]{cror_wakes_potential.pdf}
  \caption{Wakes and potential effects in a 
  contra-rotating open rotor configuration.}
  \label{fig:cror_wakes_potential}
\end{figure}
These two phenomena are correlated with the blade passing frequency.
In addition to this, the blades can exhibit vortex shedding whose frequency
is not known a priori.
Nevertheless,
vortex-shedding is likely to appear in large trailing edge blades.
This is not a common design in industrial configuration as large trailing edge
give larger drag.

\paragraph{Non-uniform inflow and installation effects}

In maneuver, the nacelle of the CROR is in incidence
which results in a non-uniform velocity triangle on the blades.
This leads to in-plane forces. This is an unsteady phenomenon that
whose frequency is correlated to the rotation frequency $\Omega / 2 \pi$.
The presence of a pylon (installation effect) give rises to an unsteady frequency
also correlated with the rotation frequency when a pusher CROR is considered. The presence of a pylon
is important as it changes the performances and flow behavior around the CROR.


\section{Challenges}
\label{sec:cror_challenges}
%!TEX root = ../../../adrien_gomar_phd.tex

\paragraph{Classification}
Figure~\ref{fig:cror_challenges} depicts the current challenges associated
with CROR configurations. Three main fields are involved: the aerodynamics, the
aeroelastics and the aeroacoustics fields.
\begin{figure}[htbp]
  \centering
  \includegraphics*[width=0.60\textwidth]{challenges.pdf}
  \caption{Challenges raised by contra-rotating open rotor configurations.}
  \label{fig:cror_challenges}
\end{figure}

\paragraph{Aerodynamic}
Theoretically, 
the CROR is meant to have a better propulsive efficiency than a turbofan and a
propeller. However, as it is a new architecture, studies need to be conducted
to understand the flow physic that develops within it. In particular,
the aerodynamic interaction between the two rotors needs to be understand.
\citet{Hendricks2011} developed an open-rotor cycle model based
on experimental performance characteristics made at NASA. This is 
an empiric approach that suffers from the impossibility to build new designs.
\citet{Peters2012} developed an equivalent code to design their CROR but the
final design is assessed for aeroacoustic purposes by full annulus unsteady simulations.
To improve this, \citet{Bechet2011} used a lifting-line code to
initialize a gradient optimization procedure based on mixing-plane
computations. This last step allows to gain almost a half point of
the CROR efficiency. This is more general than an empiric strategy
if the mixing-plane computations are reliable to assess the performance
parameters of CROR. This study has been performed by \citet{Zachariadis2011}.
They compared the performance prediction of mixing plane computations
to experimental data made on a open-rotor test case.
The agreement is fair for the thrust and power coefficients.
As small discrepancies are present, these are amplified when computing
the efficient which basically the ratio of the two coefficients.
\citet{Vion2011} and \citet{Stuermer2008} used unsteady
CFD computations to assess the unsteady performance and flow features.
\citet{Stuermer2008} \citet{Francois2013} demonstrated through a code to code comparison
that CFD was mature enough to estimate the in-plane forces.

\paragraph{Aeroacoustic}
Lot of research effort is put on the third challenge which
is aeroacoustic due to the absence of a duct. 
In fact, in the late eighties at NASA, \citet{Hager1988}
conducted a large project on innovative propulsion systems for the
next generation aircrafts. The potential of the CROR configuration
was identified but the noise emitted was so high that the solution
offered was to put noise limiter in the fuselage. It results in 
an increase of the weight. Together with the decrease of the price of the
barrel in the late eighties, the CROR never reached the commercial
aviation. This is why, today, a lot of research effort is put on the
understanding and mastering of noise sources in CRORs.
Several CFD studies have been performed in the literature.
\citet{Peters2012} showed that unsteady CFD simulation is able
to reproduce the aeroacoustic footprint of a CROR. They then optimized
their CROR and showed that optimized CROR design may be mature enough
for noise certification. \citet{Hoffer2012} and \citet{Ferrante2013}
developed an efficient CFD approach to simulation the aeroacoutic of CRORs.
It is based on time Fourier-based method. The method is able to
take into account for incidence effects.

\paragraph{Aeroelastic}
The aeroelastic challenge is less studied in the literature.
To the author knowledge, only whirl flutter has been investigated
in the CROR literature by \citet{Verley2013} and this study mainly
discusses the simulation tools needed to compute such a phenomena as
no experimental data are available.



\chconclu{The main flow physic that develop in contra-rotating open rotors
has been detailed. The unsteady phenomena that develops through this
engine are mostly correlated with the blade passing frequency, except for
the installation effects and the non-uniform inflow. The challenges
associated with this type of engine are recalled and it is highlighted 
that aeroelasticity of such systems remain unstudied. This is
why the present thesis will focus on aeroelasticity.}
