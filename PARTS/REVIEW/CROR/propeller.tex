%!TEX root = ../../../adrien_gomar_phd.tex

\subsection{Geometry}
\label{sub:cror_propeller_geometry}

A propeller is composed of one single rotor with blades
attached on it as shown in
Fig.~\ref{fig:cror_propeller_geometry}. 
The diameter of these blades is $D$
and their rotation speed is $\Omega$. 
A In front of the propeller, there is a spinner which helps
smoothing the air inflow.
The turboprop engine can be seen as
a turbofan whose fan is not within a nacelle.
\begin{figure}[htb]
  \centering
  \includegraphics*[width=0.40\textwidth]{propeller_geometry.pdf}
  \caption{Geometry of a propeller}
  \label{fig:cror_propeller_geometry}
\end{figure}
The absence of a nacelle implies that theoretically, the mass-flow can be
infinite. To quantify this, it is common in the engine field to
consider the bypass ratio. It is defined as the ratio between the
cold air (the un-combustioned air)
divided by the hot air (the air that goes through the engine core).
One of the highest bypass ratio engine on aircraft today is given
by the Pratt~\&~WhitneyPW1000G and is 12, while propellers are estimated
to have a bypass ratio of 50. This explains why this architecture has
regain interest.

\subsection{Velocity triangle}
\label{sub:cror_propeller_velocity_triangle}


