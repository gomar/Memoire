%!TEX root = ../../../adrien_gomar_phd.tex

\subsection{Geometry}
\label{sub:cror_propeller_geometry}

A propeller is composed of one single rotor with blades
attached on it as shown in
Fig.~\ref{fig:cror_propeller_geometry}. 
The diameter of these blades is $D$
and their rotation speed is $\Omega$. 
A In front of the propeller, there is a spinner which helps
smoothing the air inflow.
The turboprop engine can be seen as
a turbofan whose fan is not within a nacelle.
\begin{figure}[htb]
  \centering
  \includegraphics*[scale=0.30]{propeller_geometry.pdf}
  \caption{Geometry of a propeller.}
  \label{fig:cror_propeller_geometry}
\end{figure}
The absence of a nacelle implies that theoretically, the mass-flow can be
infinite. To quantify this, it is common in the engine field to
consider the bypass ratio. It is defined as the ratio between the
cold air (the un-combustioned air)
divided by the hot air (the air that goes through the engine core).
One of the highest bypass ratio engine on aircraft today is given
by the Pratt~\&~Whitney~PW1000G and is~12, while propellers are estimated
to have a bypass ratio of~50. This explains why this architecture has
regain interest.

\subsection{Velocity triangle}
\label{sub:cror_propeller_velocity_triangle}
The velocity triangle applied to a propeller configuration
is shown in Fig.~\ref{fig:cror_velocity_triangle_propeller}.
The aim of a propeller is to create a thrust through a increase
in the axial velocity note $\Delta V_x$ is the diagram. To do
so, the relative flow field is straighten up. This gives both
a increase in axial velocity but also in tangential velocity.
In fact, the inflow that was purely axial has a tangential
component at the outlet. This is called the swirl and
is a lost energy as it cannot be used to produce thrust.
\begin{figure}[htbp]
  \centering
  \includegraphics*[scale=0.40]{velocity_triangle_propeller.pdf}
  \caption{Velocity triangle applied to a propeller.}
  \label{fig:cror_velocity_triangle_propeller}
\end{figure}

\subsection{Similarity coefficients}
\label{sub:similarity_coefficients}
To estimate the characteristics of the propeller, four similarity
coefficients are used:
the advance ratio $J$ that represents the operating point of the propeller,
the thrust $C_t$ and power $C_p$ coefficients that estimate its performance and finally
the efficiency $\eta$:
\begin{equation}
    J = \frac{V_0}{n D}, \quad
    C_t = \frac{F_x}{\frac{1}{2} n ^ 2  D ^ 4}, \quad
    C_p = \frac{M_x \Omega}{\frac{1}{2} n ^ 3 D ^ 5}, \quad
    \eta = J \frac{C_t}{C_p},
\end{equation}
where $V_0$ is the free-stream velocity 
(depicted in Fig.~\ref{sub:cror_propeller_geometry}), 
$n$ the rotation frequency ($n = \Omega / 2 \pi$) and
$M_x$ the axial torque.
The efficiency defined here is the global propulsive efficiency
as it gives the ratio of the propulsive power over the mechanical power.

An estimation of the variation of the advance ratio $J$ and the 
efficiency $\eta$ depending on the flight conditions can be given as follow
\begin{alignat}{4}
    \text{(cruise)} \quad  0.8 &< \eta &< 0.95, \quad 1 &< J < 3.5 \\
    \text{(take-off)} \quad  0.5 &< \eta &< 0.8, \quad J &< 1.
\end{alignat}

\subsection{Main physical phenomena}
\label{sub:cror_propeller_physics}

\begin{figure}[htb]
  \centering
  \subfigure[Wakes]{
      \label{fig:propeller_wakes}
      \includegraphics[scale=.2]{propeller_wakes.pdf}}
  \quad\subfigure[Tip vortices]{
      \label{fig:propeller_tip_vortices}
      \includegraphics[scale=.2]{propeller_tip_vortices.pdf}}
  \quad\subfigure[Stream tube contraction]{
      \label{fig:propeller_stream_tube}
      \includegraphics[scale=.2]{propeller_stream_tube.pdf}}
  \caption{Main physical phenomena seen in a propeller.}
  \label{fig:propeller_phys_phenomena}
\end{figure}
The main physical phenomena that can be seen in a propeller are schematically represented
in Fig.~\ref{fig:propeller_phys_phenomena}. Firstly, due to the presence of a boundary
layer on the pressure side and suction side of the blades, a wake is generated.
This is a region of low velocity that is shed from the blades. This is mostly a two-dimensional
phenomenon seen at each radii. Secondly, in the tip of the blade, the pressure difference between 
the pressure side and the side induces the creation of a vortex counter-rotating compared to 
the rotation speed. The shapes of the blade is of prior importance to reduce this phenomena.
Finally, the propeller generates thrust through an acceleration of the fluid. Thus, the stream
tube is contracted. All of these phenomena are stationary in their frame of reference.


