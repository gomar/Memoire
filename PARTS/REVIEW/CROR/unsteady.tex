%!TEX root = ../../../adrien_gomar_phd.tex

\subsection{From steady to unsteady phenomena}
\label{sub:cror_from_steady_to_unsteady_phenomena}

The flow generated behind the front rotor
is steady in its frame of reference. Nevertheless,
due to the relative speed difference between the
front and the rear rotors, these steady flow distortions are
seen as unsteady features by the rear rotor. 
These unsteadinesses are correlated with the Blade Passing Frequency (BPF)
\begin{equation}
	f_{BPF} = \frac{\Omega_{rel} B_{opp}}{2 \pi},
\end{equation}
where $\Omega_{rel}$ is the relative speed difference between
the current and the opposite row
and $B_{opp}$ the number of blades in the opposite row.

\subsection{Main unsteadinesses}
\label{sub:cror_main_unsteadinesses}

The following section will be devoted to the classification
of unsteady phenomena that appear in CROR in 
order to choose an efficient strategy to simulate them.

\paragraph{Wakes and potential effects}

Compared to an isolated rotor, as for the case of a propeller,
the presence of the rear rotor gives rise to an unsteady
interaction by means of potential effects. In addition, wakes generated
behind the front rotor interact with the rear rotor.
This is schematically represented in Figure~\ref{fig:cror_wakes_potential}.
\begin{figure}[htp]
  \centering
  \includegraphics*[width=0.30\textwidth]{cror_wakes_potential.pdf}
  \caption{Wakes and potential effects in a 
  contra-rotating open rotor configuration.}
  \label{fig:cror_wakes_potential}
\end{figure}
These two phenomena are correlated with the blade passing frequency.
In addition to this, vortex shedding phenomena may occur behind the blades, 
whose frequency is not known \emph{a priori}.
This phenomenon is more likely to appear behind blades with a bluff trailing edge.
This is not a common design in industrial configurations as a bluff trailing edge
gives larger drag. Therefore, we can consider here that 
wake and potential effects are the driving unsteady phenomena
that are correlated with the blade passing frequency.

\paragraph{Non-uniform inflow and installation effects}

In maneuver, the CROR is in incidence with respect to the incoming flow
which results in a non-uniform velocity triangle on the blades.
This leads to in-plane forces, which represents an unsteady phenomenon
whose frequency is correlated with the rotation frequency $\Omega / 2 \pi$.
The presence of a pylon (installation effect) gives rise to an unsteady frequency
also correlated with the rotation frequency when a pusher CROR is considered.
It is important as it changes both performance and flow behavior around the CROR.
