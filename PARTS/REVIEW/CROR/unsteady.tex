%!TEX root = ../../../adrien_gomar_phd.tex

\subsection{From steady to unsteady phenomena}
\label{sub:cror_from_steady_to_unsteady_phenomena}

The flow that is generated behind the front rotor
is steady in its frame of reference. Nevertheless,
due to the relative speed difference between the
front and the rear rotor, these steady flow distortions are
seen as unsteady features by the rear rotor. 
These unsteadiness are correlated to the Blade Passing Frequency (BPF):
\begin{equation}
	f = \frac{\Omega_{rel} B_{opp}}{2 \pi},
\end{equation}
where $\Omega_{rel}$ is the relative speed difference between
the current and the opposite row
and $B_{opp}$ the number of blades in the opposite row.
At first order, the unsteady effects presented here drive
most of the time-dependent field in a turbomachinery.

\subsection{Main unsteadinesses}
\label{sub:cror_main_unsteadinesses}

In sec.~\ref{sub:cror_propeller_physics}, the main physical phenomena
that appears in a propeller have been introduced. As seen above, due to
the relative speed difference between the two rotors, these phenomena
that were steady in their frame of reference are now seen as unsteady features
by the rear rotor.

\paragraph{Tip vortices}

As shown previously in Fig.~\ref{fig:propeller_tip_vortices}, tip vortices are shed in the
tip of the blades due to a pressure difference between each side of the blades.
If nothing particular is done, this low momentum perturbation can
hit the rear rotor and induce large unsteady fluctuations. To avoid this,
the rear rotor blades are clipped as mentioned earlier. 
This unsteadiness is correlated with the BPF.

\paragraph{Wakes and potential effects}

Compared to an isolated rotor, as for the case of a propeller,
the presence of an opposite rotor give rise to an unsteady
interaction through the potential effects.
This is added to the already present wake distortions. This is
schematically represented in Fig.~\ref{fig:cror_wakes_potential}.
\begin{figure}[htb]
  \centering
  \includegraphics*[width=0.30\textwidth]{cror_wakes_potential.pdf}
  \caption{Wakes and potential effects in a 
  contra-rotating open rotor configuration.}
  \label{fig:cror_wakes_potential}
\end{figure}
These two phenomena are correlated with the blade passing frequency.
In addition to this, the blades can exhibit vortex shedding whose frequency
is not known a priori.
Nevertheless,
vortex-shedding is likely to appear in large trailing edge blades.
This is not a common design in industrial configuration as large trailing edge
give larger drag.

\paragraph{Non-uniform inflow and installation effects}

In maneuver, the nacelle of the CROR is in incidence
which results in a non-uniform velocity triangle on the blades.
This leads to in-plane forces. This is an unsteady phenomenon that
whose frequency is correlated to the rotation frequency $\Omega / 2 \pi$.
The presence of a pylon (installation effect) give rises to an unsteady frequency
also correlated with the rotation frequency when a pusher CROR is considered. The presence of a pylon
is important as it changes the performances and flow behavior around the CROR.
