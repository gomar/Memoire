%!TEX root = ../../../adrien_gomar_phd.tex

\subsection{From steady to unsteady phenomena}
\label{sub:cror_from_steady_to_unsteady_phenomena}

The flow that is generated behind the front rotor
is steady in its frame of reference. Nevertheless,
due to the relative speed difference between the
front and the rear rotor, these steady flow patterns are
seen as unsteady features by the rear rotor. Furthermore,
these unsteadiness are correlated to the Blade Passing Frequency (BPF):
\begin{equation}
	f = \frac{\Omega_{rel} B_{opp}}{2 \pi},
\end{equation}
where $\Omega_{rel}$ is the relative speed difference 
and $B_{opp}$ the number of blades in the opposite row.
At first order, the unsteady effects presented here drive
most of the time-dependent field in a turbomachinery.

\subsection{Main unsteadinesses}
\label{sub:cror_main_unsteadinesses}

\paragraph{Tip vortices}

\paragraph{Wakes and potential effects}

\paragraph{Non-uniform inflow}

\paragraph{Installation effects}

\paragraph{Gust and maneuvers}
