%!TEX root = ../../../adrien_gomar_phd.tex

\section{Introduction}
\label{sec:ca_introduction}

In the seventies, the two oil crisis showed the aeronautical industries its dependence
toward energy resources. To face this issue, the U.S. Senate directed NASA in $1975$
to look for every potential fuel-saving concept. The Advanced Turboprop
project was born~\cite{Hager1988}. The reflection led to the
concept of contra-rotating open rotor.

In an airplane engine, 
the thrust $F_x$ and the propulsive efficiency $\eta_{pr}$ are given by:
\begin{equation}
  \begin{split}
    F_x &= Q \Delta V, \\
    \eta_{pr} &= \displaystyle \frac{1}{1 + \displaystyle \frac{\Delta V}{2 V_0}},
  \end{split}
\end{equation}
where $Q$ is the mass-flow, $\Delta V$ is the inlet/outlet velocity difference and $V_0$
the inlet velocity. These equations means that the 
smaller the velocity difference $\Delta V$, the higher the propulsive efficiency.
Then, to increase the thrust $F_x$ which is the reason being of an engine,
the only way is to work on the mass-flow $Q$.
The current

In this way, a propeller is more efficient than a jet. In fact, a jet 
moves small mass of gas at high velocity while propellers moves 
large mass of air at low velocity, increasing the propulsive efficiency.

Only ten years later, the cost of 
the barrel decreased for twenty years to almost retrieve its original
value. 

In the beginning of the eighties, the cost of 
the barrel decreased for almost twenty years and then increased again.
Today, the cost of a barrel is almost at its maximum as shown
in Fig.~\ref{fig:crude_oil_price}.
\begin{figure}[htbp]
  \centering
  \includegraphics*[width=0.40\textwidth]{crude_oil_price.pdf}
  \caption{Evolution of the cost of a barel from $1861$ to $2012$ \cite{bpreview2013}}
  \label{fig:crude_oil_price}
\end{figure}

Nevertheless, Airbus forecast a doubled number of passengers in
$2031$. To allow a sustainable air transportation, the aeronautical
industry should reduce its environmental footprint. For instance,
the European Commission recently published the \emph{Flightpath $2050$}.
In this document, the aeronautical industries are set goals for $2050$
shown in Fig.~\ref{fig:flightpath_2050}.
\begin{figure}[htbp]
  \centering
  \includegraphics*[width=0.40\textwidth]{flightpath_2050.pdf}
  \caption{European Commission goals for the aeronautical industry. }
  \label{fig:flightpath_2050}
\end{figure}
The noise, CO2 and NOx emissions should be reduced of 
respectively $65\%$, $75\%$ and $80\%$.
These are ambitious goals, needing technological breakthrough.

The main source of pollutant emission is the engine. Two ways of reducing
these are through a better control of the combustion and better
efficiency of the device in global.

How to increase the mass-flow ??
HBR or propeller

\section{Integrated parameters: the similarity coefficients}
\label{sec:ca_similarity_coeff}

In the case of a CROR configuration, two rotors are considered.
Two main ways exists to evaluate the global value of the
similarity coefficients. The first one, chosen by
\citet{Bechet2011} among others, is to consider
that the non-dimensioning parameter $D$, $n$ and $J$ are those
of the front rotor for both rotors. 
The second one uses the non-dimensioning parameter of the current rotor,
as done by \citet{Stuermer2008} and \citet{Zachariadis2011}.
The traction and power coefficients of the rear rotor is
computed using its own advance ratio, diameter and rotation frequency.
Nevertheless, computing the advance ratio of the rear rotor, as
the freestream velocity should be updated to take into account
for the acceleration generated by the front rotor. The freestream
velocity is chosen to be $V_0$, which is of course not true.
The efficiency is computed rotor per rotor and then
assembled through an arithmetic addition. This is the approach retained
in this thesis.

An estimation of the variation of the advance ratio $J$ and the 
efficiency $\eta$ depending on the flight conditions can be given as follow
\begin{alignat}{4}
    \text{(cruise)} \quad  0.8 &< \eta &< 0.95, \quad 1 &< J < 3.5 \\
    \text{(take-off)} \quad  0.5 &< \eta &< 0.8, \quad J &< 1.
\end{alignat}

\section{Three-dimensional flow}
\label{sec:ca_3D_flow}



\section{The mean stationary flow}
\label{sec:ca_general_flow_physics}

Now that the average values of the integrated parameters are given,
one can focus on more local information. Here, we will first use
the azimuthal average variations. These are computed by extracting
an axial plane and azimuthally averaging the field. If $f(x, r, \theta)$
denotes the evolution of the field then its azimuthal variation at
an axial position $x_0$ is:
\begin{equation}
    f(x=x_0, r) = \int_{\theta} f(x=x_0, r, \theta) \diff \theta.
\end{equation}


% \subsection{General flow physics}
% \label{sub:ca_general_flow_physics}

% In this section, the general flow physic of a CROR configuration
% is detailed for four 

% The range of CROR engines is limited to a subsonic Mach number.
% In fact, the absence of a nacelle implies that the CROR
% will work under the airplane flight conditions. This is
% not the case of a turbofan whose nacelle helps controlling
% the Mach number seen by the fan. Thus the CROR range of operating
% condition is limited to a Mach number $M \leq 0.8$.
