%!TEX root = ../../../adrien_gomar_phd.tex

The convergence of the spectral operator depends on
the regularity of the approximated function. Consider a function
$u(t)$ that is continuous, periodic and bounded in $[0,T]$
and let $P_N \left(u(t)\right)$ denote its truncated Fourier series:
\begin{equation}
    P_N \left(u(t)\right) = \sum_{k=-N}^{N} \widehat{u}_k e^{i k\omega t}.
\end{equation}
The $\mathcal{L}2$-norm of the error writes:
\begin{equation}
   \| u \|_2 = \left(\int_0^T |u(t) - P_N \left(u(t)\right)|^2 \diff t \right)^{1/2}.
\end{equation}
If $u(t)$ is m-times continuously differentiable in $[0,T]$ ($m \geq 1$) 
and its $j$-th derivative is periodic on $[0,T]$ for all $j \leq m - 2$
then, it exists  $k_0 \in [1, N]$ such that for $k > k_0$:
\begin{equation}
    \widehat{u}_k = \mathcal{O} (k^{-m}),
\end{equation}
where $\widehat{u}_k$ is the k-th Fourier coefficient of $u(t)$.
This equation means that, the more regular the function is,
the faster the convergence rate of the Fourier
coefficients.
The property of the error to decay exponentially as soon as 
the function is approximated by a number of harmonics greater than $k0$ 
is called spectral accuracy~\cite{Canuto2006}. Note that
$k_0$ is not known but is rather essential for the analysis.
For $k$ below $k_0$, approximating the function $u(t)$ with its Fourier
series yields unacceptably high errors.