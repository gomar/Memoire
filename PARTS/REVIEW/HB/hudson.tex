%!TEX root = ../../../adrien_gomar_phd.tex

In this thesis, the final application is contra-rotating open rotor
configurations. To bound Fourier-based time methods for such
applications, we propose a classification of the
unsteady phenomena that can be computed using such approaches.

Inspired by \citet{Hodson1998},
a diagram presenting the unsteadinesses seen in 
a CROR is shown in Figure~\ref{fig:hudson}. A distinction
is made between unsteadinesses whose frequencies are
known or not.
\begin{figure}[htp]
  \centering
  \includegraphics*[scale=0.8]{hudson.pdf}
  \caption{Main unsteady phenomena seen by contra-rotating
  open rotors. Bold blue text highlights applications that can
  be treated with the harmonic balance implementation available in the
  current work and underlined red text shows additional applications
  made possible by extensions available in the literature.}
  \label{fig:hudson}
\end{figure}
From the bibliography presented previously, almost all
unsteady flows can be computed using Fourier-based time methods.
The current implementation of the HB method available for
this thesis is able to compute all the unsteady effects highlighted
by a bold text. In the literature, the presented work of 
\citet{Mavriplis2012} allows to compute transient unsteady flows
resulting from a change of operating point and/or a maneuver and
the work of \citet{McMullen2002} and \citet{Gopinath2006} allows
to capture periodic flows whose frequency is unknown. These two
kinds of unsteadiness are added
to the current panel of applications that can
be treated by Fourier-based time methods and are
highlighted by an underlined text. Note
that the gradient algorithm presented by \citet{McMullen2002}
and \citet{Gopinath2006} is only able to converge when a 
good approximate of the solution is given, meaning
that this strategy fails when one has no estimate
of the value of the frequency for the considered phenomenon.
It can also be inferred that using such an optimization algorithm
coupled with a Fourier-based time method
might require more computational time than a classical time-marching scheme,
which limits its applicability to academic configurations.

