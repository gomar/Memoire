%!TEX root = ../../../adrien_gomar_phd.tex

Inspired by \citet{Hodson1998},
a diagram presenting the unsteadiness seen in 
a CROR is depicted in Fig.~\ref{fig:hudson} and a distinction
is made between unsteadinesses whose frequencies are
known or not.
\begin{figure}[htbp]
  \centering
  \includegraphics*[width=0.8\textwidth]{hudson.pdf}
  \caption{Main unsteady phenomenon seen by a contra-rotating
  open rotor. Blue text highlights applications that can
  be treated with the harmonic balance implementation of the
  current thesis and red text shows additional applications
  available in the literature.}
  \label{fig:hudson}
\end{figure}
From the bibliography presented below, almost all
unsteady flows can be computed using Fourier-based methods.
The current implementation of the HB method available for
this thesis is able to compute all the unsteadinesses highlighted
by a bold text. In the literature, the presented work of 
\citet{Mavriplis2012} allows to compute transient unsteady flow
resulting from a change of operating point and/or a maneuver and
the work of \citet{McMullen2002} and \citet{Gopinath2006} allows
to capture periodic flows whose frequency is unknown. These two
kind of unsteadiness are thus highlighted by a bold and underlined
text and added
to the current panel of applications that can
be treated by Fourier-based methods. Let us note
that the gradient algorithm presented by \citet{McMullen2002}
and \citet{Gopinath2006} is only able to converge when a 
good approximate of the solution is given, meaning
that this strategy fails when one has no knowledge
of the value of the frequency for the considered phenomenon.
