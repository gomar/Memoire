%!TEX root = ../../../adrien_gomar_phd.tex

% The computational power available today in
% the research centers and in the industry
% is so big that large eddy simulation
% becomes possible for some industrial configurations.
% This is actually needed as high-fidelity
% simulations help the turbomachinery community understand
% the complex nature of flows that develop 
% within these components,
% allowing breakthrough ideas.
% However, even if high fidelity approaches
% are today within reach, it is still a challenge for
% multi-row configurations.
% Moreover, there will always be room for fast reliable
% computations. 
% As a matter of fact, on a daily basis, engineers need
% to run lots of simulations to test new designs.
% In this framework, large eddy simulation is
% too demanding to be used for design purposes.

% Today, in most companies, steady Reynolds-Averaged
% Navier--Stokes (RANS) based solver are used on a daily basis.
% For instance, this tool 
% helped engineering the $3$D blades shape of a CROR
% configuration as depicted in Fig.~\ref{fig:sm_leap}.
% \begin{figure}[htp]
%   \centering
%   \includegraphics*[width=0.40\textwidth]{HERA3_ISOLATED_TSM_N7_roe3_sa_film_0309_entropy.png}
%   \caption{$3$D blades shape of an industrial CROR configuration.}
%   \label{fig:sm_leap}
% \end{figure}
% Some further improvements are made possible by the use
% of the Unsteady RANS computations (U-RANS). In fact, such a
% method computes the unsteady fields and can thus take into account for
% the rotor interactions.
% However, in the industry, unsteady computations
% are still too expensive to be used on a daily basis.
% Based on a simple Fourier series decomposition, Fourier-based methods are 
% able to reproduce the unsteady field to engineering
% accuracy, for a cost of one order to two order of magnitude
% cheaper.

% In turbomachines, the relative speed motion between the blades
% give rise to inherent time-periodic phenomena.
% In fact, consider
% a CROR configuration as shown 
% in Fig.~\ref{fig:sm_unsteady_turbomachine}. 
% \begin{figure}[htp]
%   \centering
%   \includegraphics*[width=0.4\textwidth]{unsteady_turbomachine.pdf}
%   \caption{Main unsteady effects present in a turbomachinery stage. Here, a CROR
%   configuration is shown.}
%   \label{fig:sm_unsteady_turbomachine}
% \end{figure}
% Due to the
% viscosity effects acting on the upstream rotor blades, 
% a wake is shed behind it and 
% impinges the downstream rotor row. 
% In opposite, the flow field
% generated around this last can literally go back up
% to the upstream row. In fact
% the pressure fluctuations can go backwards in a subsonic flow, yielding
% the potential effects. Moreover, as the rows do not 
% show the same rotation speed,
% the field that is created in one row is perceived as unsteady in the opposite 
% row frame of reference. This unsteadiness can be
% correlated with the so-called Blade Passing Frequency (BPF) defined as:
% \begin{equation}
% 	f = \frac{\Omega_{rel} B_{opp}}{2 \pi},
% \end{equation}
% where $\Omega_{rel}$ is the relative speed difference 
% and $B_{opp}$ the number of blades in the opposite row.
% At first order, the unsteady effects presented here drive
% most of the time-dependent field in a turbomachine. This 
% is of course an approximation, but we will see at the end
% of this chapter that the range of unsteady periodic
% flow phenomenon in a CROR is large.

% The problem with classical time-marching scheme is 
% that it has no knowledge
% of the periodic nature of the field yielding a time-consuming
% transient. Thus, one idea is to build an efficient algorithm
% by taking advantage of this periodicity. 
% Hence the Fourier-based methods that are
% presented above.

There is a large variety of Fourier-based time methods in the
literature. 
The most important Fourier-based methods will be presented in this section.
In total four Fourier-based methods are presented, 
the \underline{L}inearized \underline{U}nsteady 
\underline{R}eynolds-averaged
Navier--Stokes method (LUR), 
the \underline{N}on-\underline{L}inear 
\underline{H}armonic method (NLH), the \underline{N}on-\underline{L}inear 
\underline{F}requency \underline{D}omain
method (NLFD) and the \underline{H}armonic \underline{B}alance 
method (HB).
These names are chosen here
for clarity but alternative appellations may be found sometimes. 
When this is the case, an effort will be made to synthesize
these appellations to give a good 
overview of the types of Fourier-based time methods in the literature
and their differences. Originally, these approaches have been developed
to efficiently simulate unsteady periodic phenomena.

For simplicity, the key features of the methods are described for
a simple non-linear partial differential equation, namely
the inviscid Burger's equation written in conservative form as:
\begin{equation}
  \frac{\partial u}{\partial t} + 
  \frac{\partial}{\partial x} \left( \frac{u^2}{2} \right) = 
  0.
  \label{eq:sm_nonlinear_convection_conservative}
\end{equation}
