%!TEX root = ../../../adrien_gomar_phd.tex

There is a large variety of Fourier-based time methods in the
literature. 
The most used ones in the turbomachinery community will be presented in this section.
In total, four Fourier-based methods are presented, 
the \underline{L}inearized \underline{U}nsteady 
\underline{R}eynolds-averaged
Navier--Stokes method (LUR), 
the \underline{N}on-\underline{L}inear 
\underline{H}armonic method (NLH), the \underline{N}on-\underline{L}inear 
\underline{F}requency \underline{D}omain
method (NLFD) and the \underline{H}armonic \underline{B}alance 
method (HB).
These names are chosen here
for clarity but alternative appellations may be found sometimes. 
When this is the case, an effort will be made to synthesize
these appellations to give a good 
overview of the types of Fourier-based time methods existing in the literature
and their differences. Originally, these approaches have been developed
to efficiently simulate unsteady periodic phenomena.

For simplicity, the key features of the methods are described for
a simple non-linear partial differential equation, namely
the inviscid Burger's equation written in conservative form as
\begin{equation}
  \frac{\partial u}{\partial t} + 
  \frac{\partial}{\partial x} \left( \frac{u^2}{2} \right) = 
  0.
  \label{eq:sm_nonlinear_convection_conservative}
\end{equation}
