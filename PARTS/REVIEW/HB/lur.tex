%!TEX root = ../../../adrien_gomar_phd.tex

\citet{Verdon1984} originally developed the unsteady linearized 
method in the framework of potential flows. Latter on, \citet{Hall1989}
extended the linearized method to Euler equations and
\citet{Clark2000} applied it on the Reynolds-Averaged Navier-Stokes equations
yielding the LUR method.
This method relies on a decomposition of the variables
into a steady part and a small-disturbance unsteady component:
\begin{equation}
	u = \overline{u} + u^\prime,
	\label{eq:sm_lur_decomposition}
\end{equation}
where $u^\prime$ is considered to be a small unsteady perturbation. 
In his thesis,
\citet{Hall1987} defines small to be less than $10\%$ of the
steady flow.
Injecting Eq.~\eqref{eq:sm_lur_decomposition} into 
Eq.~\eqref{eq:sm_nonlinear_convection_conservative} leads to:
\begin{equation}
	\frac{\partial u^\prime}{\partial t} + 
	\frac{1}{2}\frac{\partial}{\partial x} \left[
	\overline{u}^2 + 2 \overline{u} u^\prime + u^\prime u^\prime \right] = 
	0.
	\label{eq:sm_lur_step_1}
\end{equation}
By means of linearization, i.e. collecting terms
of equal order and neglecting terms of order superior than one, 
Eq.~\eqref{eq:sm_lur_step_1} can be split
into a steady equation:
\begin{equation}
	\frac{\partial \overline{u}^2}{\partial x} = 0,
	\label{eq:sm_lur_step_2}
\end{equation}
and an unsteady first order perturbation equation:
\begin{equation}
	\frac{\partial u^\prime}{\partial t} +
	\frac{\partial}{\partial x} \left[
	\overline{u} u^\prime \right] = 
	0.
	\label{eq:sm_lur_step_3}
\end{equation}
Note that Eq.~\eqref{eq:sm_lur_step_2} does not depends
on the value of the perturbation $u^\prime$, but the 
contrary does.
This means that we have a one-way coupling: the steady field
is first computed and then given as an input to the
perturbation equation to compute
the corresponding unsteady disturbance.

\subsection{Mono-frequential formulation}
As said before, Fourier-based methods have been developed to efficiently
capture periodic phenomena.
Hence, assuming that the velocity perturbation is harmonic at 
a pulsation $\omega$, one can write:
\begin{equation}
	u^\prime = \widehat{u}_1 e^{i \omega t} + \widehat{u}_{-1} e^{-i \omega t},
\end{equation}
with $u_1$ and $u_{-1}$ being opposite conjugates giving a
real value of the perturbation.
Injecting this definition into Eq.~\eqref{eq:sm_lur_step_3} and using
the orthogonality of the complex exponentials, leads
to:
\begin{equation}
	\begin{dcases}
		i \omega \widehat{u}_1 +
		\frac{\partial}{\partial x} \left[
		\overline{u} \widehat{u}_1 \right] &= 
		0, \\
		-i \omega \widehat{u}_{-1} +
		\frac{\partial}{\partial x} \left[
		\overline{u} \widehat{u}_{-1} \right] &= 
		0.
	\end{dcases}
	\label{eq:sm_lur_step_4}
\end{equation}
Finally a pseudo-time $\tau$ is added to time-march 
Eq.~\eqref{eq:sm_lur_step_2} and Eq.~\eqref{eq:sm_lur_step_4}
to the steady state, giving three equations in total:
\begin{equation}
	\fbox{$
	\begin{dcases}
		\frac{\partial \overline{u}}{\partial \tau} +
		\frac{\partial 
			\overline{u}^2}{\partial x} &= 0, \\
		\frac{\partial \widehat{u}_1}{\partial \tau} +
		i \omega \widehat{u}_1 +
			\frac{\partial}{\partial x} \left[
			\overline{u} \widehat{u}_1 \right] &= 
			0, \\
		\frac{\partial \widehat{u}_{-1}}{\partial \tau}
		-i \omega \widehat{u}_{-1} +
			\frac{\partial}{\partial x} \left[
			\overline{u} \widehat{u}_{-1} \right] &= 
			0.
	\end{dcases}
	$}
\end{equation}

\subsection{Extension to the Navier-Stokes equations}
To extend the LUR method to the Reynolds-Averaged
Navier-Stokes equations, one has to linearized these lasts. This
is not particularly difficult as only the first order terms should
be kept. The reader is referred to the paper of \citet{Clark2000} for
a detailed development of the LUR method on the Navier-Stokes
equations.

\subsection{Cost of the method}
As the method is made of three equations in total, one steady equation 
(namely a classical RANS equation) and two perturbation equations, 
if $\mathdollar_{\text{RANS}}$ 
denotes the CPU and memory cost of
one steady computation, then the cost of the LUR
method can be estimated as:
\begin{equation}
	\mathdollar_{\text{LUR}} = 3 \cdot \mathdollar_{\text{RANS}}.
\end{equation}
In practice, only two computations are done since the steady computation
is performed beforehand.
