%!TEX root = ../../../adrien_gomar_phd.tex

\citet{Verdon1984} originally developed the unsteady linearized 
method in the framework of potential flows. Latter on, \citet{Hall1989}
extended it to the Euler equations and
\citet{Clark2000} applied it to the Reynolds-Averaged Navier--Stokes equations,
yielding the LUR method.
This method relies on a decomposition of the variables
into a base part (generally the steady state) 
and a small-disturbance unsteady component
\begin{equation}
	u = \overline{u} + u^\prime,
	\label{eq:sm_lur_decomposition}
\end{equation}
where $u^\prime$ is considered to be a small unsteady perturbation.
In his PhD thesis,
\citet{Hall1987} defines small to be less than $10\%$ of the
steady flow.
Injecting Eq.~\eqref{eq:sm_lur_decomposition} into 
Eq.~\eqref{eq:sm_nonlinear_convection_conservative} yields
\begin{equation}
	\frac{\partial u^\prime}{\partial t} + 
	\frac{1}{2}\frac{\partial}{\partial x} \left[
	\overline{u}^2 + 2 \overline{u} u^\prime + u^\prime u^\prime \right] = 
	0.
	\label{eq:sm_lur_step_1}
\end{equation}
By means of linearization, \emph{i.e.} collecting terms
of equal order (equivalently $\overline{u^\prime} = 0$) 
and neglecting terms of order greater than one, 
Eq.~\eqref{eq:sm_lur_step_1} can be split
into a steady equation
\begin{equation}
	\frac{1}{2} \frac{\partial \overline{u}^2}{\partial x} = 0,
	\label{eq:sm_lur_step_2}
\end{equation}
and an unsteady first-order perturbation equation
\begin{equation}
	\frac{\partial u^\prime}{\partial t} +
	\frac{\partial}{\partial x} \left[
	\overline{u} u^\prime \right] = 
	0.
	\label{eq:sm_lur_step_3}
\end{equation}
There is a one-way coupling between the two equations:
the steady field
is first computed using Eq.~\eqref{eq:sm_lur_step_2}
and is secondly given as an input to the
perturbation equation to compute
the corresponding unsteady disturbance (Eq.~\eqref{eq:sm_lur_step_3}). 
However, the computed
perturbation is not used to update the steady solution.
Hence the one-way coupling.

\subsection{Mono-frequential formulation}
As mentioned before, Fourier-based time methods have been developed to efficiently
capture periodic phenomena.
Hence, assuming that the velocity perturbation is harmonic with 
angular frequency $\omega$, one can write
\begin{equation}
	u^\prime = \widehat{u}_1 e^{i \omega t} + \widehat{u}_{-1} e^{-i \omega t},
\end{equation}
with $\widehat{u}_1$ and $\widehat{u}_{-1}$ being complex conjugates giving a
real value for the perturbation.
Injecting this definition into Eq.~\eqref{eq:sm_lur_step_3} and using
the orthogonality property of the complex exponentials leads
to
\begin{equation}
	\begin{dcases}
		i \omega \widehat{u}_1 +
		\frac{\partial}{\partial x} \left[
		\overline{u} \widehat{u}_1 \right] &= 
		0, \\
		-i \omega \widehat{u}_{-1} +
		\frac{\partial}{\partial x} \left[
		\overline{u} \widehat{u}_{-1} \right] &= 
		0.
	\end{dcases}
	\label{eq:sm_lur_step_4}
\end{equation}
Finally a pseudo-time $\tau$ is added to time-march 
Eq.~\eqref{eq:sm_lur_step_2} and Eq.~\eqref{eq:sm_lur_step_4}
to the steady state, giving three equations in total
\begin{alignat}{2}
	\fbox{$
	\begin{dcases}
		\frac{\partial \overline{u}}{\partial \tau} +
		\frac{\partial 
			\overline{u}^2}{\partial x} &= 0, \\
		\frac{\partial \widehat{u}_1}{\partial \tau} +
		i \omega \widehat{u}_1 +
			\frac{\partial}{\partial x} \left[
			\overline{u} \widehat{u}_1 \right] &= 
			0, \\
		\frac{\partial \widehat{u}_{-1}}{\partial \tau}
		-i \omega \widehat{u}_{-1} +
			\frac{\partial}{\partial x} \left[
			\overline{u} \widehat{u}_{-1} \right] &= 
			0
	\end{dcases}
	$}
\end{alignat}

\subsection{Extension to the Navier--Stokes equations}
To extend the LUR method to the Reynolds-Averaged
Navier--Stokes equations, one has to consider
their linearized counterpart.
The reader is referred to the paper of \citet{Clark2000} for
a detailed development of the LUR method for the Navier--Stokes
equations.

\subsection{Numerical cost}
As the method is based on three equations in total, one steady equation 
(namely a classical RANS equation) and two perturbation equations, 
if $\mathdollar_{\text{RANS}}$ 
denotes the CPU and memory cost of
one steady computation, then the cost of the LUR
method can be estimated as
\begin{equation}
	\mathdollar_{\text{LUR}} = 3 \times \mathdollar_{\text{RANS}}.
\end{equation}
In practice, only two computations are performed since the steady computation
is usually available beforehand.
