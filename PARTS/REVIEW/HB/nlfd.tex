%!TEX root = ../../../adrien_gomar_phd.tex

\subsection{Mono-frequential formulation}

Originally proposed by \citet{McMullen2001}, the NLFD
method relies on a simple observation: to develop Fourier-based methods, and in
particular the NLH method, one has made use of the Fourier
series to efficiently represent an unsteady signal.
This representation has then be used to develop the unsteady
equation into $2N+1$ steady equations: one time-averaged equation
and $2N$ perturbation equations, 
where $N$ denotes the number
of harmonic kept to compute the solution.
The problem is that the equations need to be 
resolved in the frequency domain meaning
that all the numerical techniques should be adapted: the numerical schemes,
the turbulent models and so on. The smart idea 
proposed by \citet{McMullen2001} is to
make use of the fast Fourier Transform and its inverse to
allow an easy implementation of the method into a classical time-domain code.
To explain the development of this method, let us first 
write Eq.~\eqref{eq:sm_nonlinear_convection_conservative} 
in a more general form:
\begin{equation}
	(\ref{eq:sm_nonlinear_convection_conservative})
	\Leftrightarrow
	\frac{\partial u}{\partial t} + R = 0,
	\label{eq:sm_nonlinear_convection_residual}
\end{equation}
with
\begin{equation}
	R = \frac{\partial}{\partial x} \left( 
	\frac{u^2}{2} \right).
\end{equation}
Let us now consider that both $u$ and $R$ are periodic
in time with respect to period $T = 2 \pi / \omega$
and can be written using a Fourier series:
\begin{equation}
	\begin{split}
		u(t) &= \sum_{k=-\infty}^{\infty} \widehat{u}_k e^{i k \omega t}, \\
		R(t) &= \sum_{k=-\infty}^{\infty} \widehat{R}_k e^{i k \omega t}.
	\end{split}
\end{equation}
Note that decomposing $R(t)$ into a Fourier series is equivalent
to use the Fourier decomposition of $u(t)$ and express
$R(t)$ using the Fourier coefficients $\widehat{u}_k$ of $u$
since the cross-terms that may arise are also expressed 
using the same complex exponentials. This comes from the fact
that multiplying a complex exponential with a complex exponential
just forms a new complex exponential at the power of the sum of the
two.
Injecting these decompositions into 
Eq.~\eqref{eq:sm_nonlinear_convection_residual} and taking into account
for the orthogonality of the complex exponentials:
\begin{equation}
	i k \omega \widehat{u}_k + \widehat{R}_k = 0, \: k \in [-\infty, \infty].
\end{equation}
As previously, only a small number of harmonics $N$ is kept and 
a pseudo-time ($\tau$) derivative is added to march the equations
in pseudo-time to the steady-state solutions of all the harmonics:
\begin{equation}
	\fbox{$
	\displaystyle \frac{\partial \widehat{u}_k}{\partial \tau} + 
	i k \omega \widehat{u}_k + \widehat{R}_k = 0, \: k \in [-N, N].
	$}
	\label{eq:sm_nlfd_subset_eq}
\end{equation}
The fact that $R(t)$ is expressed using its own Fourier series 
makes it simpler to implement 
as it avoid developing its expression using 
the complex coefficients $\widehat{u}_k$. 
However, $\widehat{R}_k$ must be evaluated. To do so, as depicted
in Fig.~\ref{fig:nlfd_principle}, \citet{McMullen2001}
propose to use an Inverse Fast-Fourier Transform (IFFT) to get
$u(t)$ from $\widehat{u}_k$. Then the considered governing equations
are used to evaluate $R(t)$ which leads to $\widehat{R}_k$
through a Fast-Fourier Transform (FFT). Finally, the next iteration value 
$\widehat{u}_k$
is evaluated by adding $\widehat{R}_k$ and 
the corresponding temporal derivative $i k \omega \widehat{u}_k$. All
harmonics are coupled through the IFFT and FFT operations
that needs all of the former to compute the counterpart temporal signal,
hence the coupling. Moreover, 
in the non-viscous Burger's equation framework, 
the term $u^\prime u^\prime$ is not neglected anymore compared to the
NLH approach and the computation of the deterministic stress is encompass
by the FFT and IFFT operations.
\begin{figure}[htbp]
  \centering
  \includegraphics*[width=0.50\textwidth]{nlfd_principle.pdf}
  \caption{Simplified diagram of the computation of $\widehat{R}_k$ from $\widehat{u}_k$
  for the non-linear frequency domain method.}
  \label{fig:nlfd_principle}
\end{figure}

\subsection{Extensions}

\paragraph{Navier--Stokes equations}
The Navier--Stokes equations can be written in finite-volume,
semi-discrete form as:
\begin{equation}
	V \frac{\partial W}{\partial t} + R(W) = 0,
	\label{eq:navier_stokes_fv_sd}
\end{equation}
where $V$ is the volume of the cell and $W$
the vector of conservative variables.
This formulation is similar to
Eq.~\eqref{eq:sm_nonlinear_convection_residual} meaning that
nothing particular has to be made to derive this approach for
the Navier--Stokes equations. This is indeed attractive as the
method can be applied almost directly, except for the FFT and IFFT
step that should be added into the pseudo time loop.

\paragraph{Aeroelastic computations}
Since both the structural and the aerodynamic equations
are prone to time-periodic unsteadinesses,
\citet{Kachra2008} extended the NLFD approach to the strong coupling of
aeroelasticity within the two-dimensional Euler equations framework.
Both the fluid dynamics and the structural equations
are solved using the NLFD approach. They are coupled together 
every $15$ multigrid cycles.
A $2$D plunging and pitching airfoil is considered.
They demonstrate that with a one-harmonic NLFD computation, the
flutter boundary of a NACA$64A010$ airfoil is correctly predicted.
This leads to a gain of one order of magnitude compared to a classical
time-marching procedure. 

\paragraph{Gradient-based method to determine the frequency}
\citet{McMullen2001} applied the NLFD to a cylinder
vortex shedding. This could be done as the frequency of the
vortex shedding was known \emph{a priori} from experimental
and numerical data. Note that this suppose that the
numerical and the experimental vortex shedding frequencies
are equal, which is generally not true~\cite{Kato1991}.
However, for a given cylinder, it is generally not
possible to know this frequency \emph{a priori}. This is why
\citet{McMullen2002, McMullen2006a}
proposed a gradient based method for determining the frequency
of a periodic phenomena where the frequency is unknown
\emph{a priori}. They argue that the frequency domain formulation
helps forming a gradient operator to find the period $T$ based
on the minimization of the residual of the unsteady equations.
They applied their algorithm to find the frequency of the vortex
shedding around a cylinder, and found it with a $3.5\%$ accuracy
compared to experimental data. Nevertheless, as a gradient method is 
used, a good initial guess is needed for the algorithm to
converge. This limits the methods to well-known unsteady
problems.

\paragraph{Optimum shape design}
\citet{Nadarajah2003} compared an optimum shape design 
strategy for 
airfoils undergoing a change 
in angle of attack as a function of time 
using both a classical time-marching scheme
and the NLFD scheme within the Euler equations
framework. It is shown that the NLFD method
gives the same accuracy for the gradient and the optimum with only 
three time-instants (namely $N=1$)
compared to $23$ time-instants needed for 
the time-marching approach.
\citet{Nadarajah2007} extended it
to the three-dimensional Navier--Stokes equations.
A wing undergoing a change 
in angle of attack as a function of time is computed and
it is demonstrated that
five instants (namely $N=2$) is sufficient to provide
accurate results.
\citet{Tatossian2011} applied it
to the aerodynamic shape optimization of hovering rotor blades
in the Euler framework.
The capability of 
their shape optimization process
to redesign the Caradonna–Tung experimental 
blade is assessed and gives a proof
of concept.

\paragraph{Adaptive method}
The problem of Fourier-based methods is that the more
the number of harmonics computed, the higher the corresponding
CPU and memory cost. There is thus a need to optimize
the chosen number of harmonics.
\citet{Mosahebi2013} implemented an adaptive NLFD approach named
the p-NLFD. Based on the energy of the last mode compared
to the whole spectrum, the number of harmonics
is increased if a fixed threshold is not reached.
A speed-up of $2$ in terms of CPU cost and
memory reduction is observed for the case of a
vortex-shedding behind a cylinder.

\subsection{Cost of the method}
The NLFD method is close to the NLH approach in terms of number
of equation solved. However, at each time-step a fast Fourier transform
is performed to cast back the harmonics into the time-domain in order
to compute the residual $R(t)$. \citet{McMullen2006} argue
that the cost of the fast Fourier transform is less than the cost of 
the spatial derivatives. 
\citet{Kachra2008} quantitatively estimate it to be
approximately $2\%$ of the cost of one iteration, which is negligible.
Based on this affirmation, one can say that equivalent to the NLH
approach, if $\mathdollar_{\text{RANS}}$ 
denotes the CPU and memory cost of
one steady computation, the cost of the NLFD method can be 
approximated by:
\begin{equation}
	\mathdollar_{\text{NLFD}} = (2N+1) \times \mathdollar_{\text{RANS}}.
\end{equation}
This evaluation of the cost is confirmed on numerical
simulations by \citet{McMullen2002}.